%
% loesung.tex -- Lösung des Variationsproblems
%
% (c) 2020 Prof Dr Andreas Müller, Hochschule Rapperswil
%
% !TEX root = ../../buch.tex
% !TEX encoding = UTF-8
%
\section{Lösung des Variationsproblems
\label{widerstand:section:loesung}}
\kopfrechts{Lösung}
Das Minimalproblem~\ref{widerstand:aufgabe} ist ein Variationsproblem
mit einem Randterm.
In diesem Abschnitt lösen wird das Problem mit den im
Abschnitt~\ref{buch:nebenbedingungen:section:randfunktionen}
entwickelten Methoden.

%
% Die Euler-Lagrange-Differentialgleichung
%
\subsection{Die Euler-Lagrange-Differentialgleichung}
Die Lagrange-Funktion des Funktionals~\eqref{widerstand:eqn:Wy} ist 
\[
F(x,y')
=
\frac{2x}{1+y^{\prime 2}}
\]
mit den partiellen Ableitungen
\begin{align*}
\frac{\partial F}{\partial y}&=0
\\
\frac{\partial F}{\partial y'}
&=
\frac{4xy'}{(1+y^{\prime 2})^2}.
\end{align*}
Daraus ergibt sich die Euler-Lagrange-Differentialgleichung
\begin{align*}
0
=
\frac{d}{dx} \frac{\partial F}{\partial y'}
&=
\frac{d}{dx}
\frac{4xy'(x)}{(1+y'(x)^2)^2}.
\end{align*}
Statt diesen Ausdruck explizit abzuleiten schliessen wir, dass die rechte
Seite konstant sein muss, es gibt also eine Konstante $c$ derart, dass
\begin{equation}
\frac{ 4xy'(x) }{ (1+y'(x)^2)^2 } = c.
\label{widerstand:dgl:eqn:implizit}
\end{equation}
Die Konstante $c$ muss aus den Randbedingungen ermittelt werden.

%
% Lösung der Differentialgleichung
%
\subsection{Lösung der Differentialgleichung}
Die implizite Form der Differentialgleichung
\eqref{widerstand:dgl:eqn:implizit}
kann leider nicht auf einfache Art in explizite Form gebracht und damit
einer numerischen Lösung zugeführt werden.
Die rechte Seite von \eqref{widerstand:dgl:eqn:implizit} lässt sich
aber nach $x$ auflösen, womit das folgende Verfahren anwendbar ist.

%
% Lösung einer Differentialgleichung $F(x,y,y')=0$, die nach $x$ auflösbra ist
%
\subsubsection{Lösung einer Differentialgleichung $F(x,y,y')=0$,
die nach $x$ auflösbar ist}
Da die Funktion $F(x,y,y')$ in 
einer Umgebung der Anfangsbedinung $(x_0,y_0)$ nach $x$ auflösbar ist,
gibt es eine Funktion $x=H(y,p)$, wobei wir $p$ für die Variable schreiben,
für die die Ableitung $y'$ eingsetzt werden soll.
In einer Umgebung der Anfangsbedingung lässt sich $x$ durch die Variable $p$
ausdrücken, also als Funktion $x=x(p)$.
Die Ableitung von $y$ nach $x$ ist dann nach der Kettenregel
\begin{equation}
p
=
\frac{dy}{dx}
=
\frac{dy}{dp}\frac{dp}{dx}
\qquad\Rightarrow\qquad
\frac{dx}{dp}
=
\frac{1}{p}
\frac{dy}{dp}.
\label{widerstand:eqn:p}
\end{equation}
Damit ist die Ableitung on $x$ nach dem Parameter $p$ ist also festgelegt,
sobald die Ableitung $y$ nach dem Parameter $p$ bekannt ist.

Sei $y=u(x)$ ein Lösung, dann gilt $p=u'(x)$ oder
\[
x(p)
=
H(y(p),p).
\]
Durch Ableitung nach $p$ erhalten wir
\begin{align}
\frac{dx}{dp}
&=
\frac{\partial H}{\partial y}(y,p)
\frac{dy}{dp}
+
\frac{\partial H}{\partial p}(y,p)
\intertext{und unter Verwendung
\eqref{widerstand:eqn:p}}
\frac{1}{p}\frac{dy}{dp}
&=
\frac{\partial H}{\partial y}(y,p)
\frac{dy}{dp}
+
\frac{\partial H}{\partial p}(y,p)
\notag
\\
\frac{dy}{dp}
&=
p\frac{\partial H}{\partial y}(y(p),p)
\frac{dy}{dp}
+
p\frac{\partial H}{\partial p}(y(p),p)
\notag
\\
\frac{dy}{dp}
\biggl(
1
-
p\frac{\partial H}{\partial y}(y(p),p)
\biggr)
&=
p\frac{\partial H}{\partial p}(y(p),p)
\notag
\intertext{oder}
\frac{dy}{dp}
&=
-
\frac{
\displaystyle
p\frac{\partial H}{\partial p}(y(p),p)
}{
\displaystyle
p\frac{\partial H}{\partial y}(y(p),p)
-
1
}.
\label{widerstand:loesung:eqn:dglyp}
\end{align}
Dies ist eine Differentialgleichung für die Funktion $y(p)$ in expliziter
Form.

Zur Lösung der Differentialgleichung $F(x,y,y')=0$ 
muss also erst die Differentialgleichung
\eqref{widerstand:loesung:eqn:dglyp}
gelöst werden, um die Funktion $y(p)$ zu bestimmen.
Anschliessend kann $x(p)$ als Lösung der Differentialgleichung
\eqref{widerstand:eqn:p}
gefunden werden.

%
% Anwendung auf die Differentialgleichung ()
%
\subsubsection{Anwendung auf die Differentialgleichung
\eqref{widerstand:dgl:eqn:implizit}}
Die Differentialgleichung
\eqref{widerstand:dgl:eqn:implizit}
ist nach $x$ aufgelöst
\[
x
=
c
\frac{(1+y^{\prime 2})^2}{4y'}.
\]
Die Funktion $H(y,p)$ hängt also nicht von $y$ ab und ist nur
\[
H(p)
=
c
\frac{(1+p^2)^2}{4p}
=
c
\frac{p^4+2p^2+1}{4p}
.
\]
Da $H$ nicht von $y$ abhängt vereinfacht sich der Nenner der
Differentialgleichung \eqref{widerstand:loesung:eqn:dglyp}, es bleibt
nur noch die Ableitung von $H$ nach $p$ zu bestimmen:
\begin{align*}
\frac{\partial H}{\partial p}
&=
c\frac{3p^4+2p^2+1}{4p^2}.
\intertext{Daraus entstehen  aus \eqref{widerstand:loesung:eqn:dglyp}
und \eqref{widerstand:eqn:p} die Differentialgleichungen}
\frac{dy}{dp}
&=
p\frac{\partial H}{\partial p}(p)
=
c\frac{3p^4+2p^2-1}{4p}
&&\Rightarrow&
\frac{dx}{dp}
&=
c
\frac{3p^4+2p^2-1}{4p^2},
\intertext{die durch Quadratur gelöst werden können.
Die Lösungen sind}
y(p)
&=
c\frac{3p^4+4p^2}{16}
-\frac14\log |p|
+
C
&&\text{und}&
x(p)
&=
c
\frac{(1+p^2)^2}{4p}
+
D.
\end{align*}
Damit ist eine Parameterdarstellung der Lösungskurve gefunden.
Die Integrationskonstanten $C$ und $D$ müssen aus den Randbedingungen
ermittelt werden.
Da auch die Konstanten $c$ aus den Randbedingungen ermittelt werden muss,
sind zusätzliche Randbedingungen benötigt, die sich aus den Bedingungen
an die Randfunktionen ergeben.

%
% Die Randbedingungen
%
\subsection{Die Randbedingungen}
Am rechten Rand erfüllt die Lösung $y(x)$ die Randbedingung $y(R)=0$.
Für die Randbedingung am linken Rand muss $r$ so gewählt werden, dass
$\vec{f}_0\perp\vec{r}_0$ ist.
Da der Tangentialvektor $\vec{r}_0$ horizontal ist, muss der Vektor
$\vec{f}_0$ vertikal sein, seine $x$-Komponente muss daher verschwinden.
Aus \eqref{buch:nebenbedingungen:randfunktionen:eqn:fnew}
folgt daher, dass
\begin{align*}
0
=
\vec{f}_{0,x}
&=
F(x_0,y(x_0),y'(x_0))
-
y'(x_0)
\frac{\partial F}{\partial y'}(x_0,y(x_0),y'(x_0))
+
\frac{\partial g_0}{\partial x}(x_i,y(x_i))
\\
&=
F(r,L,y'(r))
-
y'(r)
\frac{\partial F}{\partial y'}(r,L,y'(r))
+
\frac{\partial g_0}{\partial x}(r,L)
\\
&=
\frac{2r}{1+y'(r)^2}
-
y'(r)
\frac{4ry'(r)}{(1+y'(r)^2)^2}
-
2r
\\
&=
2r
\frac{1+y'(r)^2 -2y'(r)^2
-
1
-2y'(r)^2
-y'(r)^4
}{(1+y'(r)^2)^2}
\\
&=
\frac{2r}{(1+y'(r)^2)^2}
y'(r)^2
(
-3
-y'(r)^2
)
\\
&=
-
\frac{2ry'(r)^2}{(1+y'(r)^2)^2}
(3+y'(r)^2).
\end{align*}





