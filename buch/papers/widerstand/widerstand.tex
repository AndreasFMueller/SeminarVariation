%
% widerstand.tex -- Beispiel-File für die Einleitung
%
% (c) 2020 Prof Dr Andreas Müller, Hochschule Rapperswil
%
% !TEX root = ../../buch.tex
% !TEX encoding = UTF-8
%
\section{Widerstand\label{widerstand:section:teil0}}
\kopfrechts{Widerstand}
Newton hat in seiner Untersuchung zum Strömungswiderstand angenommen,
dass dieser durch elastische Stösse mit einzelnen Teilchen entsteht.
Er nimmt an, dass das Teilchen wie ein Ping-pong-Ball auf der Oberfläche
reflektiert wird und der Widerstand durch den vom Geschoss auf das
Teilchen übertragenen Impuls entsteht.
Die Reibung zwischen Luft und Oberfläche wird also vollständig
vernachlässigt.
Newtons Überlegungen sind erst im Falle extrem geringer Dichte annähernd
erfüllt.
Für geringe Geschwindigkeiten und normale atmosphärische Dichte sind
andere Effekte viel grösser, erst recht bei Flüssigkeiten.
Newton war trotzdem der Meinung, dass seine Erkenntnisse für
den Schiffbau von Bedeutung sein würden.

%
% Widerstand eines Rotationskörpers
%
\subsection{Widerstand eines Rotationskörpers}
%
% koeper.tex -- Rotationskörper nach Newton
%
% (c) 2021 Prof Dr Andreas Müller, OST Ostschweizer Fachhochschule
%
\documentclass[tikz]{standalone}
\usepackage{amsmath}
\usepackage{times}
\usepackage{txfonts}
\usepackage{pgfplots}
\usepackage{csvsimple}
\usetikzlibrary{arrows,intersections,math,calc}
\definecolor{darkred}{rgb}{0.8,0,0}
\definecolor{darkgreen}{rgb}{0,0.6,0}
\begin{document}
\def\skala{1}
\begin{tikzpicture}[>=latex,thick,scale=\skala,
declare function={
	y(\t) = 5*(1-(\t/4)*(\t/4));
	yprime(\t) = -(5/8)*\t;
	nlength(\t) = sqrt(1 + yprime(\t)*yprime(\t));
	xn(\t) = -yprime(\t) / nlength(\t);
	yn(\t) = 1 / nlength(\t);
	rx(\t) = -2 * xn(\t) * yn(\t);
	ry(\t) = 1 - 2 * yn(\t) * yn(\t);
}]

\draw (-4,-0.05) -- (-4,0.05);
\draw (4,-0.05) -- (4,0.05);
\node at (4,-0.05) [below] {$R\mathstrut$};
\draw[line width=0.3pt] (1,0) -- (1,5);
\draw (1,-0.05) -- (1,0.05);
\node at (1,-0.05) [below] {$r\mathstrut$};
\node at (0,{y(1)}) [above left] {$y=L$};

\draw[color=darkred,line width=1.2pt]
	plot[domain=-4:-1] (\x,{y(\x)})
	--
	plot[domain=1:4] (\x,{y(\x)});

\def\x{2}
\draw[->,color=darkgreen,line width=1.4pt] (\x,{y(\x)}) -- +({xn(\x)},{yn(\x)});
\node[color=darkgreen] at ({\x+xn(\x)},{y(\x)+yn(\x)}) [above right] {$\vec{n}$};

\draw[color=blue,line width=0.3pt] (\x,0) -- (\x,5.5);
\draw[color=blue,line width=1pt]
	(\x,5.5)
	--
	(\x,{y(\x)})
	--
	+({-2*rx(\x)},{-2*ry(\x)})
	;

\draw[->,color=blue,line width=1.4pt] (\x,5.5) -- (\x,4.5);
\node[color=blue] at (\x,5) [left] {$\vec{e}$};
\fill[color=darkgreen] (\x,{y(\x)}) circle[radius=0.08];

\def\x{-2.5}
\draw[line width=0.3pt,color=darkred] (\x,0) -- (\x,{y(\x)});
\fill[color=darkred] (\x,{y(\x)}) circle[radius=0.08];
\node[color=darkred] at (\x,{y(\x)}) [above left] {$y(x)$};

\draw[->] (-4.1,0) -- (4.5,0) coordinate[label={$x$}];
\draw[->] (0,-0.1) -- (0,5.9) coordinate[label={right:$y$}];

\end{tikzpicture}
\end{document}


Wir beschreiben einen Rotationskörper durch die Meridiankurve
$y(x)$, wobei $x$ die Entfernung von der Achse ist.
Die Oberfläche des Rotationskörpers besteht also aus einem kreisförmigen
Teil mit Radius $r$ und 
einem Teil zwischen den Radien $x=r$ und $x=R$, der durch $y(x)$
beschrieben ist.

Die Teilchen, die auf den kreisförmigen Teil auftreffen, werden vollständig
reflektiert sie tragen daher $2\pi r^2$ zum Widerstand bei.

Für die Teilchen, die auf den schrägen Teil des Mantels auftreffen
muss zunächst die Normale auf die Mantelfläche berechnet werden.
Sie ist der Vektor
\[
\vec{n}(x)
=
\frac{1}{\sqrt{1+y'(x)^2}}
\begin{pmatrix}
-y'(x)\\
1
\end{pmatrix}
\]
Die Richtung des anströmenden Teilchens ist der Vektor
\[
\vec{e}
=
\begin{pmatrix}
 0\\
-1
\end{pmatrix}.
\]
Die reflektierte Richtung ist
\[
\vec{e}'
=
\vec{e}
-2\vec{n}\,\vec{n}\cdot\vec{e}
=
\begin{pmatrix} 0\\ -1 \end{pmatrix}
-2
\frac{-1}{1+y'(x)^2}
\begin{pmatrix}
-y'(x)\\
1
\end{pmatrix}
=
\begin{pmatrix}0\\-1\end{pmatrix}
+
\frac{2}{1+y'(x)^2}\begin{pmatrix}-y'(x)\\1\end{pmatrix}
\]
Für die Berechnung des Widerstands ist nur die Komponente der
Impulseänderung $\Delta p$ parallel zur Achse interessant.
Dies ist die zweite Komponente des zweiten Terms auf der rechten
Seite relevant, nämlich
\[
\Delta p
=
\frac{2}{1+y'(x)^2}
\]
Der Ring zwischen den Radien $x$ und $x+\Delta x$ hat den Flächeninhalt
\[
\pi ((x+\Delta x)^2 - x^2)
=
\pi (x^2 +2x\Delta x + (\Delta x)^2 - x^2)
=
2\pi x \Delta x + (\Delta x)^2.
\]
Die Mantelfläche trägt daher den Widerstand
\begin{equation*}
4\pi
\int_{r}^{R}
\frac{x}{1+y'(x)^2}\,dx
\end{equation*}
bei.
Damit bekommen wir für den gesamten Widerstand die Formel
\[
2\pi r^2 + 4\pi \int_r^R \frac{x}{1+y'(x)^2}\,dx.
\]
Der Einfachheit halber dividieren wir noch durch $2\pi$ und finden
damit das Funktional
\begin{equation}
W(y)
=
r^2 
+
\int_r^R \frac{2x}{1+y'(x)^2}\,dx
\end{equation}
mit der Lagrange-Funktion
\[
F(x,y')
=
\frac{2x}{1+y'(x)^2},
\]
welches zu varieren ist.

\begin{aufgabe}
\label{widerstand:aufgabe}
Finde $r$ und die Funktion $y(x)$ definiert auf dem Intervall $[r,R]$
derart, dass das
\begin{equation}
W(y)
=
r^2
+
\int_r^R \frac{2x}{1+y'(x)^2}\,dx
\label{widerstand:eqn:Wy}
\end{equation}
minimal wird mit den Randbedingungen
$y(r)=L$ und $y(R)=0$.
\end{aufgabe}

In der Notation von
Abschnitt~\ref{buch:nebenbedingungen:section:randfunktionen}
handelt es sich um ein Varaitionsproblem mit einer Randfunktion
\begin{equation}
g_0(x,y) = -x^2.
\label{widerstand:eqn:randfunktion}
\end{equation}



