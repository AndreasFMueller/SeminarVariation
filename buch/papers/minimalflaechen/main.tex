%
% main.tex -- Paper zum Thema <minimalflaechen>
%
% (c) 2020 Autor, OST Ostschweizer Fachhochschule
%
% !TEX root = ../../buch.tex
% !TEX encoding = UTF-8
%
\chapter{Minimalflächen\label{chapter:minimalflaechen}}
\kopflinks{Minimalflächen}
\begin{refsection}
	\chapterauthor{Ronja Allenfort und Ana Milivojevic}
	
	
	Unter einer Minimalfläche versteht man eine beliebige Fläche im Raum mit der Spezialität, dass der Flächeninhalt jeweils dem lokalen Minimum entspricht. 
	Gefunden werden solche Flächeninhalte über Variationsrechnungen unter Berücksichtigung der gegebenen Randbedingungen. 
	Ihre Anwendungen finden sie in den verschiedensten Bereichen der Mathematik oder Physik.
	
	Die Erforschung der Minimalflächen begann im 18. Jahrhundert mit den Arbeiten des französischen Mathematikers Joseph-Louis Lagrange. 
	1760 formulierte Lagrange das Problem der Minimalflächen als ein Variationsproblem.
	Er war einer der ersten Mathematiker, der systematisch die Bedingungen untersuchte, unter denen eine Fläche einen minimalen Flächeninhalt aufweist. 
	Diese frühen Arbeiten legten den Grundstein für die Variationsrechnung.
	
	Ein weiterer wichtiger Beitrag zur Theorie der Minimalflächen kam von Jean-Baptiste Meusnier im Jahr 1776. 
	Meusnier zeigte, dass Minimalflächen eine mittlere Krümmung von null haben.
	Er entdeckte, dass die mittlere Krümmung an jedem Punkt einer Minimalfläche null ist, was bedeutet, dass sich die Fläche lokal nicht weiter verkleinern lässt.
	Diese Erkenntnis war ein bedeutender Fortschritt in der Theorie der Minimalflächen und half, die mathematischen Grundlagen dieses Gebiets weiter zu festigen.
	
	Im 19. Jahrhundert führte der belgische Physiker Joseph Plateau Experimente mit Seifenfilmen durch.
	Plateau spannte Seifenfilme in Drahtschlingen und beobachtete, dass diese Filme immer eine Form annahmen, die eine Minimalfläche darstellt.
	Diese Experimente bestätigten viele theoretische Vorhersagen und zeigten, dass die mathematischen Modelle tatsächlich die physikalische Realität beschreiben.
	Plateaus Arbeiten haben wesentlich dazu beigetragen, das Verständnis von Minimalflächen zu vertiefen und ihre Bedeutung in der realen Welt aufzuzeigen.

%
% einleitung.tex -- Beispiel-File für die Einleitung
%
% (c) 2020 Prof Dr Andreas Müller, Hochschule Rapperswil
%
% !TEX root = ../../buch.tex
% !TEX encoding = UTF-8
%
\section{Teil 0\label{000template:section:teil0}}
\kopfrechts{Teil 0}
Lorem ipsum dolor sit amet, consetetur sadipscing elitr, sed diam
nonumy eirmod tempor invidunt ut labore et dolore magna aliquyam
erat, sed diam voluptua \cite{000template:bibtex}.
At vero eos et accusam et justo duo dolores et ea rebum.
Stet clita kasd gubergren, no sea takimata sanctus est Lorem ipsum
dolor sit amet.

Lorem ipsum dolor sit amet, consetetur sadipscing elitr, sed diam
nonumy eirmod tempor invidunt ut labore et dolore magna aliquyam
erat, sed diam voluptua.
At vero eos et accusam et justo duo dolores et ea rebum.  Stet clita
kasd gubergren, no sea takimata sanctus est Lorem ipsum dolor sit
amet.



%
% teil1.tex -- Beispiel-File für das Paper
%
% (c) 2020 Prof Dr Andreas Müller, Hochschule Rapperswil
%
% !TEX root = ../../buch.tex
% !TEX encoding = UTF-8
%
\section{Teil 1
\label{beispiel:section:teil1}}
\rhead{Problemstellung}
Sed ut perspiciatis unde omnis iste natus error sit voluptatem
accusantium doloremque laudantium, totam rem aperiam, eaque ipsa
quae ab illo inventore veritatis et quasi architecto beatae vitae
dicta sunt explicabo.
Nemo enim ipsam voluptatem quia voluptas sit aspernatur aut odit
aut fugit, sed quia consequuntur magni dolores eos qui ratione
voluptatem sequi nesciunt
\begin{equation}
\int_a^b x^2\, dx
=
\left[ \frac13 x^3 \right]_a^b
=
\frac{b^3-a^3}3.
\label{beispiel:equation1}
\end{equation}
Neque porro quisquam est, qui dolorem ipsum quia dolor sit amet,
consectetur, adipisci velit, sed quia non numquam eius modi tempora
incidunt ut labore et dolore magnam aliquam quaerat voluptatem.

Ut enim ad minima veniam, quis nostrum exercitationem ullam corporis
suscipit laboriosam, nisi ut aliquid ex ea commodi consequatur?
Quis autem vel eum iure reprehenderit qui in ea voluptate velit
esse quam nihil molestiae consequatur, vel illum qui dolorem eum
fugiat quo voluptas nulla pariatur?

\subsection{De finibus bonorum et malorum
\label{beispiel:subsection:finibus}}
At vero eos et accusamus et iusto odio dignissimos ducimus qui
blanditiis praesentium voluptatum deleniti atque corrupti quos
dolores et quas molestias excepturi sint occaecati cupiditate non
provident, similique sunt in culpa qui officia deserunt mollitia
animi, id est laborum et dolorum fuga \eqref{beispiel:equation1}.

Et harum quidem rerum facilis est et expedita distinctio
\ref{beispiel:section:teil2}.
Nam libero tempore, cum soluta nobis est eligendi optio cumque nihil
impedit quo minus id quod maxime placeat facere possimus, omnis
voluptas assumenda est, omnis dolor repellendus
\ref{beispiel:section:teil3}.
Temporibus autem quibusdam et aut officiis debitis aut rerum
necessitatibus saepe eveniet ut et voluptates repudiandae sint et
molestiae non recusandae.
Itaque earum rerum hic tenetur a sapiente delectus, ut aut reiciendis
voluptatibus maiores alias consequatur aut perferendis doloribus
asperiores repellat.




\printbibliography[heading=subbibliography]
\end{refsection}
