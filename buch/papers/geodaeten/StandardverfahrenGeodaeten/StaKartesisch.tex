%
% einleitung.tex -- Beispiel-File für die Einleitung
%
% (c) 2020 Prof Dr Andreas Müller, Hochschule Rapperswil
%
% !TEX root = ../../buch.tex
% !TEX encoding = UTF-8
%
\subsection{Kartesisch\label{geodaeten:section:StaKartesisch}}
\rhead{Standardverfahren Beispiele}

Lorem ipsum dolor sit amet, consetetur sadipscing elitr, sed diam
nonumy eirmod tempor invidunt ut labore et dolore magna aliquyam
erat, sed diam voluptua \cite{geodaeten:bibtex}.
At vero eos et accusam et justo duo dolores et ea rebum.
Stet clita kasd gubergren, no sea takimata sanctus est Lorem ipsum
dolor sit amet.

Lorem ipsum dolor sit amet, consetetur sadipscing elitr, sed diam
nonumy eirmod tempor invidunt ut labore et dolore magna aliquyam
erat, sed diam voluptua.
At vero eos et accusam et justo duo dolores et ea rebum.  Stet clita
kasd gubergren, no sea takimata sanctus est Lorem ipsum dolor sit
amet.

Da der metrische Tensor konstant ist, lässt sich aus Gleichung [\ref{geodaeten:equation:Standardverfahren:Christophsymbole}] herauslesen, dass die Christophsymbole des kartesischen Raumes alle gleich Null sind.
Dies lässt sich einfach erkennen, da die Christophsymbole einer Ableitung des metrischen Tensors entsprechen und die Ableitung einer Konstanten immer Null ist.
Denken wir an die Definition aus Abschnitt \ref{geodaeten:section:Standardverfahren}, macht dies durchaus Sinn.
Denn der Kartesische Raum ist nicht gekrümmt, weshalb keine Korrektur der Geraden notwendig ist.

