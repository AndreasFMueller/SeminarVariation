%
% teil1.tex -- Beispiel-File für das Paper
%
% (c) 2020 Prof Dr Andreas Müller, Hochschule Rapperswil
%
% !TEX root = ../../buch.tex
% !TEX encoding = UTF-8
%
\subsection{Zylinder\label{geodaeten:section:MetrischerTensor:Zylinder}}
\rhead{Metrischer Tensor Beispiele}

Für die Zylinderoberfläcjhe gilt $r$ als konstant und $u^1 = \phi$, $u^2 =z$ 
Vergleicht man die Einstein-Summe aus Gleichung \ref{geodaeten:equation:MetrischerTensor:Kartesisch:EinsteinSumme} mit dem Linienelement der Zylinderoberfläche

\begin{equation}
	\begin{aligned}
	ds^2 &= g_{11} \cdot du^1 \cdot du^1 + g_{12} \cdot du^1 \cdot du^2 + g_{21} \cdot du^2 \cdot du^1 + g_{22} \cdot du^2 \cdot du^2 \\
	&= r^2 \cdot \dot{\phi}^2 +\dot{z}^2
	\end{aligned}
\end{equation}
 
Kann mittels Koeffizienten Vergleich bestimmt werden, dass 

\begin{equation}
	\begin{alignedat}{3}
		g_{11} \cdot du^1 \cdot du^1 &= g_{11} \cdot \dot{\phi}^2 & &= r^2 \cdot \dot{\phi}^2 \\
		g_{22} \cdot du^2 \cdot du^2 &= g_{22} \cdot \dot{z}^2    & &= 1 \cdot \dot{z}^2 \\
		g_{12} \cdot du^1 \cdot du^2 &= g_{12} \cdot \dot{\phi} \cdot \dot{z} & &= 0 \cdot \dot{\phi} \cdot \dot{z} \\
		g_{21} \cdot du^2 \cdot du^2 &= g_{21} \cdot \dot{z} \cdot \dot{\phi} & &= 0 \cdot \dot{z} \cdot \dot{\phi} .
	\end{alignedat}
\end{equation}

Die Matrixeinträge entsprechen den Koeffizienten der Jeweiligen Ableitungsprodukte und damit lässt sich der metrische Tensor aufstellen als
\begin{equation}
	g_{ji} =\begin{pmatrix} g_{11} && g_{12} \\ g_{21} && g_{22} \end{pmatrix}= \begin{pmatrix} r^2 && 0 \\ 0 && 1 \end{pmatrix} .
\end{equation}


Ist $r$ nicht konstant, bedeutet dies, dass man sich in drei Dimensionen bewegen kann.
Deshalb muss eine $3 \times 3$-Matrix für den metrischen Tensor entstehen, damit jede Bewegung vollständig beschrieben ist. 

Die Einstein-Summ hat hier die Form

\begin{equation}
\begin{aligned}
	ds^2 = &\ g_{11} \cdot du^1 \cdot du^1 + g_{12} \cdot du^1 \cdot du^2 + g_{13} \cdot du^1 \cdot du^3 \nonumber \\
	&+ g_{21} \cdot du^2 \cdot du^1 + g_{22} \cdot du^2 \cdot du^2 + g_{23} \cdot du^2 \cdot du^3 \nonumber \\
	&+ g_{31} \cdot du^3 \cdot du^1 + g_{32} \cdot du^3 \cdot du^2 + g_{33} \cdot du^3 \cdot du^3  .
\end{aligned} 
	\label{geodaeten:equation:MetrischerTensor:Kartesisch:EinsteinSumme3D}
\end{equation}

Im Linienelement des Zylinders aus [\ref{geodaeten:equation:Linienelemente:Zylinder:Zylinder3D}] kann man herauslesen, dass keine Terme mit gemischten Ableitungsprodukte vorhanden sind.
Daher sind alle Einträge außer den Diagonalen Null.
Mit $u^1 = r$, $u^2 = \phi$ und $u^3 = z$  können die Diagonalen Elemente mit den Koeffizienten im Linienelement bestimmt werden. 
Somit gilt,

\begin{equation}
	\begin{aligned}
		g_{11}  &= 1  \\
		g_{22}  &= r^2 \\
		g_{33}  &= 1  
	\end{aligned}
\end{equation}

und damit ist der metrische Tensor im dreidimensionalen zylindrischen Raum

\begin{equation}
	T = \begin{pmatrix} 1 && 0 && 0 \\ 0 && r^2 && 0 \\ 0 && 0 && 1 \end{pmatrix} .
\end{equation}

Dieses Beispiel veranschaulicht, dass der metrische Tensor eine $n$x$n$ Matrix ist, wobei $n$ der Anzahl Dimensionen entspricht.
Dieser metrische Tensor beinhaltet eine Variable $r$ und ist somit nicht konstant. 
Daher existiert für die dreidimensionale Oberfläche des vierdimensionalen zylindrischen Raums eine Krümmung.
Diese Krümmung eines vierdimensionalen Raums ist allerdings schwer vorzustellen, weshalb die Krümmung des metrischen Tensors im Beispiel der Kugel (Abschnitt \ref{geodaeten:section:MetrischerTensor:Kugel}) genauer vorgestellt wird.
