%
% teil1.tex -- Beispiel-File für das Paper
%
% (c) 2020 Prof Dr Andreas Müller, Hochschule Rapperswil
%
% !TEX root = ../../buch.tex
% !TEX encoding = UTF-8
%
\subsection{Zylinder\label{geodaeten:section:MetrischerTensor:Zylinder}}
\rhead{Metrischer Tensor Beispiele}

Für die Zylinderoberfläche gilt $r$ als konstant und $u^1 = \phi$, $u^2 =z$.
Vergleicht man die Summe aus \eqref{geodaeten:equation:MetrischerTensor:Kartesisch:EinsteinSumme} mit dem Linienelement der Zylinderoberfläche
\begin{equation}
	ds^2 = r^2 \cdot d \phi^2 +dz^2 ,
\end{equation}
kann mittels Koeffizientenvergleich bestimmt werden, dass
\begin{equation}
	\begin{alignedat}{3}
		g_{11}  du^1 du^1 &= g_{11}  d \phi^2 & &= r^2 \cdot d \phi^2 \\
		g_{22}  du^2 du^2 &= g_{22}  dz^2    & &= 1 \cdot dz^2 \\
		2g_{12}  du^1 du^2 &= 2g_{12}  d \phi \, dz & &= 0 \cdot 2 \, d \phi \,  dz.
	\end{alignedat}
\end{equation}

Die Matrixeinträge entsprechen den Koeffizienten der jeweiligen Ableitungsprodukte und damit lässt sich der metrische Tensor aufstellen als
\begin{equation}
	g_{i\!j} =\begin{pmatrix} g_{11} && g_{12} \\ g_{21} && g_{22} \end{pmatrix}= \begin{pmatrix} r^2 && 0 \\ 0 && 1 \end{pmatrix} .
\end{equation}

Ist $r$ nicht konstant, bedeutet dies, dass man sich in drei Dimensionen bewegen kann.
Deshalb muss eine $3 \times 3$-Matrix für den metrischen Tensor entstehen, damit jede Bewegung vollständig beschrieben ist. 
Für den dreidimensionalen Fall, ergibt sich dank der Symmetrie des metrischen Tensors, das allgemeine Linienelement in der Summenform als
\begin{equation}
	\begin{aligned}
		ds^2 = &g_{11} \, du^1 \, du^1 + g_{22} \, du^2 \, du^2 + g_{33} \, du^3 \, du^3 \nonumber \\
		&+ 2 \cdot g_{12} \, du^1 \, du^2 + 2 \cdot g_{13} \, du^1 \, du^3 + 2 \cdot g_{23} \, du^2 \, du^3.
	\end{aligned}
	\label{geodaeten:equation:MetrischerTensor:Kartesisch:EinsteinSumme3D}
\end{equation}

Im Linienelement des Zylinders aus \eqref{geodaeten:equation:Linienelemente:Zylinder:Zylinder3D} kann man herauslesen, dass keine Terme mit gemischten Ableitungsprodukten vorhanden sind.
Daher sind alle Einträge ausser den Diagonalen Null.
Mit $u^1 = r$, $u^2 = \phi$ und $u^3 = z$  können die diagonalen Elemente mit den Koeffizienten im Linienelement bestimmt werden. 
Somit gilt
\begin{equation}
	\begin{aligned}
		g_{11}  &= 1  \\
		g_{22}  &= r^2 \\
		g_{33}  &= 1  
	\end{aligned}
\end{equation}
und damit ist der metrische Tensor im dreidimensionalen zylindrischen Raum
\begin{equation}
	G = \begin{pmatrix} 1 && 0 && 0 \\ 0 && r^2 && 0 \\ 0 && 0 && 1 \end{pmatrix} .
\end{equation}

Dieses Beispiel veranschaulicht, dass der metrische Tensor eine $n \times n$-Matrix ist, wobei $n$ der Anzahl Dimensionen entspricht.
Dieser metrische Tensor beinhaltet eine Variable $r$ und ist somit nicht konstant. 
Daher existiert für die dreidimensionale Oberfläche des vierdimensionalen zylindrischen Raums eine Krümmung.
Diese Krümmung eines vierdimensionalen Raums ist allerdings schwer vorzustellen, weshalb die Krümmung des metrischen Tensors in weiteren Beispielen genauer vorgestellt wird.
