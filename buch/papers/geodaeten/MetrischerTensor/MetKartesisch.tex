%
% teil1.tex -- Beispiel-File für das Paper
%
% (c) 2020 Prof Dr Andreas Müller, Hochschule Rapperswil
%
% !TEX root = ../../buch.tex
% !TEX encoding = UTF-8
%
\subsection{Kartesisch\label{geodaeten:section:MetKartesisch}}
\rhead{Metrischer Tensor Beispiele}

Der Metrische Tensor für einen zweidimensionalen kartesischen Raum kann aus der Gleichung [\ref{geodaeten:equation:Linienelemente:Kartesisch:equation1}] des Linienelements hergeleitet werden.
Gesucht ist eine Matrix $T$, welche multipliziert mit dem Vektor der Ableitungen aller Dimensionen im Quadrat das Linienelement ergibt. 
\begin{equation}
	\begin{aligned} 
	\mathbf{d\vec{s}}^2 &= \left| \begin{pmatrix} \dot{x}^2 \\ \dot{y}^2 \end{pmatrix} \right| \cdot dt^2 \\
	&= T \cdot \begin{pmatrix} \dot{x}^2 \\ \dot{y}^2 \end{pmatrix} dt^2
	\end{aligned}	
\end{equation}

Da das Linienelement in diesem einfachen Beispiel keine Koeffizienten vor den Ableitungen der Dimensionen hat, entspricht der metrische Tensor $T$ der Einheitsmatrix $I$
\begin{equation}
		T = I = \begin{pmatrix} 1 && 0 \\ 0 && 1 \end{pmatrix} .
\end{equation}

