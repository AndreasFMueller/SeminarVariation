%
% teil1.tex -- Beispiel-File für das Paper
%
% (c) 2020 Prof Dr Andreas Müller, Hochschule Rapperswil
%
% !TEX root = ../../buch.tex
% !TEX encoding = UTF-8
%
\subsection{Kartesisch\label{geodaeten:section:MetrischerTensor:Kartesisch}}
\rhead{Metrischer Tensor Beispiele}

Der Metrische Tensor für einen zweidimensionalen kartesischen Raum kann aus der Gleichung \eqref{geodaeten:equation:MetrischerTensor:AllgemeinesLinienelement} des allgemeinen Linienelements hergeleitet werden.
Schreiben wir die Summe für zwei Dimensionen aus, ergibt sich
\begin{equation}
	ds^2 = g_{11} \, du^1 \, du^1 + 2g_{12} \, du^1 \, du^2 + g_{22} \, du^2 \, du^2.
	\label{geodaeten:equation:MetrischerTensor:Kartesisch:EinsteinSumme}
\end{equation}
Für den kartesischen Raum gilt, $du^1 = dx$ und $du^2 = dy$, wobei zu beachten ist, dass bei der Einsteinschen Summenkonvention die Hochstele keinem Exponenten, sondern einem Index entspricht.

Aus Abschnitt \ref{geodaeten:section:Linienelemente:Kartesisch} kennen wir das Linienelement des Kartesischen Raums als
\begin{equation}
	ds^2 = dx^2 + dy^2 .
\end{equation}

Aus dem Linienelement können wir die Koeffizienten von 
\begin{equation}
	du^1 du^1 = dx^2 \quad \text{und} \quad du^2  du^2 = dy^2 
\end{equation}
als $1$ herauslesen. 
Die Koeffizienten für
\begin{equation}
du^1  du^2 = dx  dy \quad \text{und} \quad du^2  du^1 = dy  dx
\end{equation}
sind beide $0$.

In Gleichung \eqref{geodaeten:equation:MetrischerTensor:Kartesisch:EinsteinSumme} ist zu erkennen, dass diese Koeffizienten den Werten im metrischen Tensor $g_{ij}$ entsprechen.
An den richtigen Stellen eingesetzt ergibt sich der metrische Tensor des kartesischen Raums zu
\begin{equation}
	\begin{aligned}
		g_{11} &= \textcolor{red}{1} \\
		g_{22} &= \textcolor{magenta}{1} \\
		g_{12} &= g_{21} = \textcolor{blue}{0} \\
		g_{ij} &= \begin{pmatrix} \textcolor{red}{1} && \textcolor{blue}{0} \\ \textcolor{blue}{0} && \textcolor{magenta}{1} \end{pmatrix} .
	\end{aligned}
\end{equation}
