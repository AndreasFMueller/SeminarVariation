 %
% teil1.tex -- Beispiel-File für das Paper
%
% (c) 2020 Prof Dr Andreas Müller, Hochschule Rapperswil
%
% !TEX root = ../../buch.tex
% !TEX encoding = UTF-8
%
\section{Metrischer Tensor
\label{geodaeten:section:MetrischerTensor}}
\rhead{Metrischer Tensor}

Sed ut perspiciatis unde omnis iste natus error sit voluptatem
accusantium doloremque laudantium, totam rem aperiam, eaque ipsa
quae ab illo inventore veritatis et quasi architecto beatae vitae
dicta sunt explicabo.
Nemo enim ipsam voluptatem quia voluptas sit aspernatur aut odit
aut fugit, sed quia consequuntur magni dolores eos qui ratione
voluptatem sequi nesciunt

\begin{equation}
\int_a^b x^2\, dx
=
\left[ \frac13 x^3 \right]_a^b
=
\frac{b^3-a^3}3.
\label{geodaeten:equation1}
\end{equation}

Neque porro quisquam est, qui dolorem ipsum quia dolor sit amet,
consectetur, adipisci velit, sed quia non numquam eius modi tempora
incidunt ut labore et dolore magnam aliquam quaerat voluptatem.

Ut enim ad minima veniam, quis nostrum exercitationem ullam corporis
suscipit laboriosam, nisi ut aliquid ex ea commodi consequatur?
Quis autem vel eum iure reprehenderit qui in ea voluptate velit
esse quam nihil molestiae consequatur, vel illum qui dolorem eum
fugiat quo voluptas nulla pariatur?

\section{Beispiele zum metrischen Tensor}

%
% teil1.tex -- Beispiel-File für das Paper
%
% (c) 2020 Prof Dr Andreas Müller, Hochschule Rapperswil
%
% !TEX root = ../../buch.tex
% !TEX encoding = UTF-8
%
\subsection{Kartesisch\label{geodaeten:section:MetrischerTensor:Kartesisch}}
\rhead{Metrischer Tensor Beispiele}

Der Metrische Tensor für einen zweidimensionalen kartesischen Raum kann aus der Gleichung \eqref{geodaeten:equation:MetrischerTensor:AllgemeinesLinienelement} des allgemeinen Linienelements hergeleitet werden.
Schreiben wir die einsteinsche-Summe für zwei Dimensionen aus ergibt sich
\begin{equation}
	ds^2 = g_{11}  du^1  du^1 + g_{12}  du^1  du^2 + g_{21}  du^2  du^1 + g_{22}  du^2  du^2 .
	\label{geodaeten:equation:MetrischerTensor:Kartesisch:EinsteinSumme}
\end{equation}

In dem kartesischen Raum gilt, $du^1 = dx$ und $du^2 = dy$ wobei zu beachten ist, dass bei der Einsteinschen Summenkonvention die Hochstele keiner Potenz sondern eines Index entspricht.
Aus Abschnitt \ref{geodaeten:section:Linienelemente:Kartesisch} kennen wir das Linienelement des Kartesischen Raums als

\begin{equation}
	ds^2 = dx^2 + dy^2 .
\end{equation}
Aus dem Linienelement können wir die Koeffizienten von 

\begin{equation}
du^1 du^1 = dx^2 \quad \text{und} \quad du^2  du^2 = dy^2 
\end{equation}
als $1$ herauslesen.
Die Koeffizienten für

\begin{equation}
du^1 \cdot du^2 = dx \cdot dy \quad \text{und} \quad du^2 \cdot du^1 = dy \cdot dx
\end{equation}
sind beide $0$.
In Gleichung \ref{geodaeten:equation:MetrischerTensor:Kartesisch:EinsteinSumme} ist zu erkennen, dass diese Koeffizienten den Werten im metrischen Tensor $g_{ij}$ entsprechen.
An den richtigen Stellen eingesetzt ergibt sich der metrische Tensor des kartesischen Raums zu

\begin{equation}
	\begin{aligned}
		g_{11} &= \textcolor{red}{1} \\
		g_{12} &= \textcolor{blue}{0} \\
		g_{21} &= \textcolor{darkgreen}{0} \\
		g_{22} &= \textcolor{magenta}{1} \\
		g_{ij} &= \begin{pmatrix} \textcolor{red}{1} && \textcolor{blue}{0} \\ \textcolor{darkgreen}{0} && \textcolor{magenta}{1} \end{pmatrix} .
	\end{aligned}
\end{equation}


%
% teil1.tex -- Beispiel-File für das Paper
%
% (c) 2020 Prof Dr Andreas Müller, Hochschule Rapperswil
%
% !TEX root = ../../buch.tex
% !TEX encoding = UTF-8
%
\subsection{Zylinder\label{geodaeten:section:MetrischerTensor:Zylinder}}
\rhead{Metrischer Tensor Beispiele}

Für die Zylinderoberfläche gilt $r$ als konstant und $u^1 = \phi$, $u^2 =z$ 
Vergleicht man die Einstein-Summe aus Gleichung \eqref{geodaeten:equation:MetrischerTensor:Kartesisch:EinsteinSumme} mit dem Linienelement der Zylinderoberfläche

\begin{equation}
	\begin{aligned}
	ds^2 &= g_{11} \cdot du^1 \cdot du^1 + g_{12} \cdot du^1 \cdot du^2 + g_{21} \cdot du^2 \cdot du^1 + g_{22} \cdot du^2 \cdot du^2 \\
	&= r^2 \cdot d \phi^2 +dz^2
	\end{aligned}
\end{equation}
Kann mittels Koeffizienten Vergleich bestimmt werden, dass 

\begin{equation}
	\begin{alignedat}{3}
		g_{11} \cdot du^1 \cdot du^1 &= g_{11} \cdot d \phi^2 & &= r^2 \cdot d \phi^2 \\
		g_{22} \cdot du^2 \cdot du^2 &= g_{22} \cdot dz^2    & &= 1 \cdot dz^2 \\
		g_{12} \cdot du^1 \cdot du^2 &= g_{12} \cdot d \phi \cdot dz & &= 0 \cdot d \phi \cdot dz \\
		g_{21} \cdot du^2 \cdot du^2 &= g_{21} \cdot dz \cdot d \phi & &= 0 \cdot dz \cdot d \phi .
	\end{alignedat}
\end{equation}
Die Matrixeinträge entsprechen den Koeffizienten der jeweiligen Ableitungsprodukte und damit lässt sich der metrische Tensor aufstellen als
\begin{equation}
	g_{ji} =\begin{pmatrix} g_{11} && g_{12} \\ g_{21} && g_{22} \end{pmatrix}= \begin{pmatrix} r^2 && 0 \\ 0 && 1 \end{pmatrix} .
\end{equation}

Ist $r$ nicht konstant, bedeutet dies, dass man sich in drei Dimensionen bewegen kann.
Deshalb muss eine $3 \times 3$-Matrix für den metrischen Tensor entstehen, damit jede Bewegung vollständig beschrieben ist. 

Die Einstein-Summe hat hier die Form

\begin{equation}
\begin{aligned}
	ds^2 = &\ g_{11} \cdot du^1 \cdot du^1 + g_{12} \cdot du^1 \cdot du^2 + g_{13} \cdot du^1 \cdot du^3 \nonumber \\
	&+ g_{21} \cdot du^2 \cdot du^1 + g_{22} \cdot du^2 \cdot du^2 + g_{23} \cdot du^2 \cdot du^3 \nonumber \\
	&+ g_{31} \cdot du^3 \cdot du^1 + g_{32} \cdot du^3 \cdot du^2 + g_{33} \cdot du^3 \cdot du^3  .
\end{aligned} 
	\label{geodaeten:equation:MetrischerTensor:Kartesisch:EinsteinSumme3D}
\end{equation}

Im Linienelement des Zylinders aus \eqref{geodaeten:equation:Linienelemente:Zylinder:Zylinder3D} kann man herauslesen, dass keine Terme mit gemischten Ableitungsprodukte vorhanden sind.
Daher sind alle Einträge außer den Diagonalen Null.
Mit $u^1 = r$, $u^2 = \phi$ und $u^3 = z$  können die diagonalen Elemente mit den Koeffizienten im Linienelement bestimmt werden. 
Somit gilt,

\begin{equation}
	\begin{aligned}
		g_{11}  &= 1  \\
		g_{22}  &= r^2 \\
		g_{33}  &= 1  
	\end{aligned}
\end{equation}
und damit ist der metrische Tensor im dreidimensionalen zylindrischen Raum

\begin{equation}
	T = \begin{pmatrix} 1 && 0 && 0 \\ 0 && r^2 && 0 \\ 0 && 0 && 1 \end{pmatrix} .
\end{equation}

Dieses Beispiel veranschaulicht, dass der metrische Tensor eine $n \times n$-Matrix ist, wobei $n$ der Anzahl Dimensionen entspricht.
Dieser metrische Tensor beinhaltet eine Variable $r$ und ist somit nicht konstant. 
Daher existiert für die dreidimensionale Oberfläche des vierdimensionalen zylindrischen Raums eine Krümmung.
Diese Krümmung eines vierdimensionalen Raums ist allerdings schwer vorzustellen, weshalb die Krümmung des metrischen Tensors im Beispiel der Kugel (Abschnitt \ref{geodaeten:section:MetrischerTensor:Kugel}) genauer vorgestellt wird.

%
% teil1.tex -- Beispiel-File für das Paper
%
% (c) 2020 Prof Dr Andreas Müller, Hochschule Rapperswil
%
% !TEX root = ../../buch.tex
% !TEX encoding = UTF-8
%
\subsection{Kugel\label{geodaeten:section:MetrischerTensor:Kugel}}
\rhead{Metrischer Tensor Beispiele}

Sed ut perspiciatis unde omnis iste natus error sit voluptatem
accusantium doloremque laudantium, totam rem aperiam, eaque ipsa
quae ab illo inventore veritatis et quasi architecto beatae vitae
dicta sunt explicabo.
Nemo enim ipsam voluptatem quia voluptas sit aspernatur aut odit
aut fugit, sed quia consequuntur magni dolores eos qui ratione
voluptatem sequi nesciunt




