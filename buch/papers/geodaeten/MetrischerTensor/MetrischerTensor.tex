 %
% teil1.tex -- Beispiel-File für das Paper
%
% (c) 2020 Prof Dr Andreas Müller, Hochschule Rapperswil
%
% !TEX root = ../../buch.tex
% !TEX encoding = UTF-8
%
\section{Metrischer Tensor
\label{geodaeten:section:MetrischerTensor}}
\rhead{Metrischer Tensor}

Im vorherigen Kapitel wurde das Konzept des Linienelements erläutert.
Es wurde aufgezeigt, wie dieses mittels geometrischer und topologischer Analyse für verschiedene Räume und Koordinatensysteme berechnet werden kann.
Aus der Vektorgeometrie ist uns jedoch ein anderes nützliches Werkzeug bekannt, welches uns erlaubt, Abstände in einem Raum zu berechnen: Das Skalarprodukt.

Im euklidischen Raum ist das Skalarprodukt zwischen zwei Vektoren $\vec{u}$ und $\vec{v}$ definiert als
\begin{equation}
	\vec{u} \cdot \vec{v} = \sum_{i=1}^n u_i v_i,
\end{equation}
wobei $\vec{u} = (u_1, u_2, \ldots, u_n)$ und $\vec{v} = (v_1, v_2, \ldots, v_n)$ die Komponenten der Vektoren in einem $n$-dimensionalen Raum sind.

Die Länge eines Vektors $\vec{u}$ ergibt sich aus dem Skalarprodukt des Vektors mit sich selbst:
\begin{equation}
	\|\vec{u}\| = \sqrt{\vec{u} \cdot \vec{u}} = \sqrt{\sum_{i=1}^n u_i^2}. 
\end{equation}

Betrachten wir nun einen infinitesimal kleinen Vektor in einer Ebene mit kartesischen Koordinaten $(x, y)$, dann ergibt das Skalarprodukt dieses Vektors mit sich selbst
\begin{equation}
	ds^2 = dx \cdot dx + dy \cdot dy = dx^2 + dy^2,
\end{equation}
was die quadratische infinitesimale Länge des Vektors beschreibt und dem bereits bekannten Linienelement entspricht.

%Eventuell Einstieg Mannigfaltigkeit:
%Diese Definition des Skalarprodukts ist jedoch nicht allgemeingültig und gilt nur für die Berechnung von Abständen im euklidischen Raum.
%Was passiert jedoch in komplexeren Räumen mit speziellen Koordinatensystemen wie Zylinder- oder Kugelkoordinaten?
%
%Um diese Fragen zu beantworten, müssen wir zuerst die Konzepte der Riemannschen Mannigfaltigkeit und des metrischen Tensors verstehen.
%
%Eine Mannigfaltigkeit ist ein mathematisches Konstrukt, das dazu dient, komplizierte geometrische Objekte in eine einfachere, lokal verständliche Form zu bringen.
%Im Wesentlichen ist eine Mannigfaltigkeit eine Sammlung von Punkten, die in kleinen Bereichen (lokal) wie der euklidische Raum aussehen. 
%
%Ein einfaches Beispiel hierfür ist die Erdoberfläche.
%Steht ein "Flat-Earther" auf der Erdoberfläche, so erscheint ihm die Erde flach, weil der Bereich, den er betrachtet, sehr klein ist im Vergleich zur Gesamtgröße der Erde.
%Würde dieser "Flat-Earther" jedoch weiter zurücktreten und die gesamte Erde betrachten, sollte auch er erkennen können, dass sie in Wirklichkeit eine Kugel ist.
%
%Eine Mannigfaltigkeit ist also ein mathematisches Objekt, das lokal flach oder euklidisch erscheint, aber global eine kompliziertere Struktur haben kann, wie zum Beispiel die Oberfläche einer Kugel.
%
%Eine differenzierbare Mannigfaltigkeit setzt zusätzlich voraus, dass sie in der Umgebung jedes Punktes lokal differenzierbar ist.
%Differenzierbarkeit bedeutet, dass an jedem Punkt der Mannigfaltigkeit die Ableitung existiert, wodurch die Mannigfaltigkeit eine glatte Struktur erhält.
%Dies ermöglicht es, für jeden Punkt einer solchen Mannigfaltigkeit ein Skalarprodukt zu definieren, da die Mannigfaltigkeit lokal wie ein differenzierbarer euklidischer Raum erscheint.
%Die Funktion, die das Skalarprodukt für jeden Punkt einer solchen Mannigfaltigkeit definiert, wird als Metrik bezeichnet.
%
%Die Metrik ist somit das grundlegende Werkzeug, das es uns ermöglicht, Längen, Abstände und Winkel in einem Raum zu messen.
%In einfachen geometrischen Räumen, wie dem euklidischen Raum, kennen wir die Metrik bereits als die Formel für den Abstand zwischen zwei Punkten.
%Wir haben auch bereits die Metrik in verschiedenen Koordinatensystemen angewendet, um Linienelemente auf unterschiedlichen Oberflächen zu berechnen.
%
%Wenn wir jedoch von einem Koordinatensystem in ein anderes wechseln oder komplexere Räume wie gekrümmte Oberflächen betrachten, wird die Metrik durch den sogenannten metrischen Tensor $g_{ij}$ repräsentiert.
%Dieser Tensor ist eine symmetrische $n \times n$-Matrix, die die Metrik an jedem Punkt der Mannigfaltigkeit kodiert.
%Eine differenzierbare Mannigfaltigkeit, die zusätzlich durch einen metrischen Tensor ausgestattet ist, der an jedem Punkt der Mannigfaltigkeit definiert ist, wird als riemannsche Mannigfaltigkeit bezeichnet.

Diese Definition des Skalarprodukts ist jedoch nicht allgemeingültig und gilt nur für die Berechnung von Abständen im euklidischen Raum.
Was passiert jedoch in komplexeren Räumen mit speziellen Koordinatensystemen wie Zylinder- oder Kugelkoordinaten? 
Hier kommt der sogenannte metrische Tensor ins Spiel.

%Eventuell Einführung Metrik:
%Bevor wir den metrischen Tensor einführen, ist es jedoch wichtig, das Konzept der Metrik zu verstehen. 
%Eine Metrik ist eine Funktion, die den Abstand zwischen zwei Punkten in einem Raum definiert. 
%In einfachen geometrischen Räumen, wie dem euklidischen Raum, kennen wir die Metrik bereits als die Formel für den Abstand zwischen zwei Punkten. 
%In einem $n$-dimensionalen Raum beschreibt die Metrik die geometrischen Eigenschaften des Raumes, indem sie festlegt, wie Längen, Abstände und Winkel zwischen Vektoren berechnet werden.
%
%Wenn wir von einem Koordinatensystem in ein anderes wechseln oder komplexere Räume wie gekrümmte Oberflächen betrachten, wird die Metrik durch den sogenannten metrischen Tensor $g_{ij}$ repräsentiert.

Der metrische Tensor $g_{ij}$ ist ein fundamentales Werkzeug in der Differentialgeometrie und der allgemeinen Relativitätstheorie. 
Er ist eine symmetrische $n \times n$-Matrix, wobei $n$ der Dimension des Raumes entspricht. 
Diese Matrix enthält alle Informationen darüber, wie Abstände und Winkel in einem Raum berechnet werden.
Der metrische Tensor ist dabei raumabhängig, das heisst, er definiert für jeden Punkt im Raum, wie das Skalarprodukt berechnet wird.
Auf diese Weise beschreibt der metrische Tensor die geometrische Struktur des gesamten Raumes.

Mit dem metrischen Tensor lässt sich eine allgemeine Definition für das Skalarprodukt zweier Vektoren finden, die unabhängig vom Koordinatensystem ist:
\begin{equation}
	\begin{align}
		\vec{u} \cdot \vec{v} &= g_{ij} \, u^i \, v^j \\
		&= g_{11} \, u_1 \, v_1 + g_{12} \, u_1 \, v_2 + \dots + g_{nn} \, u_n \, v_n,
	\end{align}
\end{equation}
wobei die Notation des metrischen Tensors aus der Tensoralgebra stammt.
Die Einsteinsche Summenkonvention, die in der Tensoralgebra verwendet wird, impliziert hier eine Summierung über die Indizes $i$ und $j$, die die Dimensionen des Raumes durchlaufen.
Die Komponenten der Vektoren $u^i$ und $v^j$ werden dabei durch den metrischen Tensor $g_{ij}$ gewichtet.

Berechnet man nun mit dieser allgemeinen Definition das Skalarprodukt eines infinitesimalen Vektors mit sich selbst, erhält man:
\begin{equation}
	ds^2 = g_{ij} \, du^i \, du^j.
\end{equation}
Somit lässt sich das Linienelement in einer allgemeinen Form ausdrücken, die mithilfe des metrischen Tensors zu einer koordinatenunabhängigen Funktion wird.

Der metrische Tensor ist daher von zentraler Bedeutung für das Verständnis der Geometrie eines Raumes.
Er ermöglicht es uns, die Metrik eines Raumes in eine kompakte Schreibweise zu überführen und liefert die Grundlage für die Berechnung von Abständen und Winkeln in komplexeren geometrischen Strukturen.

Im nächsten Abschnitt werden wir uns ansehen, wie die Metrik eines Raumes verwendet wird, um den metrischen Tensor für verschiedene Koordinatensysteme herzuleiten.
Anhand von Beispielen wird erläutert, wie diese Metrik, die wir bereits in verschiedenen Koordinatensystemen angewendet haben, in die kompakte Form des metrischen Tensors überführt werden kann.


\section{Beispiele zum metrischen Tensor}

%
% teil1.tex -- Beispiel-File für das Paper
%
% (c) 2020 Prof Dr Andreas Müller, Hochschule Rapperswil
%
% !TEX root = ../../buch.tex
% !TEX encoding = UTF-8
%
\subsection{Kartesisch\label{geodaeten:section:MetrischerTensor:Kartesisch}}
\rhead{Metrischer Tensor Beispiele}

Der Metrische Tensor für einen zweidimensionalen kartesischen Raum kann aus der Gleichung \eqref{geodaeten:equation:MetrischerTensor:AllgemeinesLinienelement} des allgemeinen Linienelements hergeleitet werden.
Schreiben wir die einsteinsche-Summe für zwei Dimensionen aus ergibt sich
\begin{equation}
	ds^2 = g_{11}  du^1  du^1 + g_{12}  du^1  du^2 + g_{21}  du^2  du^1 + g_{22}  du^2  du^2 .
	\label{geodaeten:equation:MetrischerTensor:Kartesisch:EinsteinSumme}
\end{equation}

In dem kartesischen Raum gilt, $du^1 = dx$ und $du^2 = dy$ wobei zu beachten ist, dass bei der Einsteinschen Summenkonvention die Hochstele keiner Potenz sondern eines Index entspricht.
Aus Abschnitt \ref{geodaeten:section:Linienelemente:Kartesisch} kennen wir das Linienelement des Kartesischen Raums als

\begin{equation}
	ds^2 = dx^2 + dy^2 .
\end{equation}
Aus dem Linienelement können wir die Koeffizienten von 

\begin{equation}
du^1 du^1 = dx^2 \quad \text{und} \quad du^2  du^2 = dy^2 
\end{equation}
als $1$ herauslesen.
Die Koeffizienten für

\begin{equation}
du^1 \cdot du^2 = dx \cdot dy \quad \text{und} \quad du^2 \cdot du^1 = dy \cdot dx
\end{equation}
sind beide $0$.
In Gleichung \ref{geodaeten:equation:MetrischerTensor:Kartesisch:EinsteinSumme} ist zu erkennen, dass diese Koeffizienten den Werten im metrischen Tensor $g_{ij}$ entsprechen.
An den richtigen Stellen eingesetzt ergibt sich der metrische Tensor des kartesischen Raums zu

\begin{equation}
	\begin{aligned}
		g_{11} &= \textcolor{red}{1} \\
		g_{12} &= \textcolor{blue}{0} \\
		g_{21} &= \textcolor{darkgreen}{0} \\
		g_{22} &= \textcolor{magenta}{1} \\
		g_{ij} &= \begin{pmatrix} \textcolor{red}{1} && \textcolor{blue}{0} \\ \textcolor{darkgreen}{0} && \textcolor{magenta}{1} \end{pmatrix} .
	\end{aligned}
\end{equation}


%
% teil1.tex -- Beispiel-File für das Paper
%
% (c) 2020 Prof Dr Andreas Müller, Hochschule Rapperswil
%
% !TEX root = ../../buch.tex
% !TEX encoding = UTF-8
%
\subsection{Zylinder\label{geodaeten:section:MetrischerTensor:Zylinder}}
\rhead{Metrischer Tensor Beispiele}

Für die Zylinderoberfläche gilt $r$ als konstant und $u^1 = \phi$, $u^2 =z$ 
Vergleicht man die Einstein-Summe aus Gleichung \eqref{geodaeten:equation:MetrischerTensor:Kartesisch:EinsteinSumme} mit dem Linienelement der Zylinderoberfläche

\begin{equation}
	\begin{aligned}
	ds^2 &= g_{11} \cdot du^1 \cdot du^1 + g_{12} \cdot du^1 \cdot du^2 + g_{21} \cdot du^2 \cdot du^1 + g_{22} \cdot du^2 \cdot du^2 \\
	&= r^2 \cdot d \phi^2 +dz^2
	\end{aligned}
\end{equation}
Kann mittels Koeffizienten Vergleich bestimmt werden, dass 

\begin{equation}
	\begin{alignedat}{3}
		g_{11} \cdot du^1 \cdot du^1 &= g_{11} \cdot d \phi^2 & &= r^2 \cdot d \phi^2 \\
		g_{22} \cdot du^2 \cdot du^2 &= g_{22} \cdot dz^2    & &= 1 \cdot dz^2 \\
		g_{12} \cdot du^1 \cdot du^2 &= g_{12} \cdot d \phi \cdot dz & &= 0 \cdot d \phi \cdot dz \\
		g_{21} \cdot du^2 \cdot du^2 &= g_{21} \cdot dz \cdot d \phi & &= 0 \cdot dz \cdot d \phi .
	\end{alignedat}
\end{equation}
Die Matrixeinträge entsprechen den Koeffizienten der jeweiligen Ableitungsprodukte und damit lässt sich der metrische Tensor aufstellen als
\begin{equation}
	g_{ji} =\begin{pmatrix} g_{11} && g_{12} \\ g_{21} && g_{22} \end{pmatrix}= \begin{pmatrix} r^2 && 0 \\ 0 && 1 \end{pmatrix} .
\end{equation}

Ist $r$ nicht konstant, bedeutet dies, dass man sich in drei Dimensionen bewegen kann.
Deshalb muss eine $3 \times 3$-Matrix für den metrischen Tensor entstehen, damit jede Bewegung vollständig beschrieben ist. 

Die Einstein-Summe hat hier die Form

\begin{equation}
\begin{aligned}
	ds^2 = &\ g_{11} \cdot du^1 \cdot du^1 + g_{12} \cdot du^1 \cdot du^2 + g_{13} \cdot du^1 \cdot du^3 \nonumber \\
	&+ g_{21} \cdot du^2 \cdot du^1 + g_{22} \cdot du^2 \cdot du^2 + g_{23} \cdot du^2 \cdot du^3 \nonumber \\
	&+ g_{31} \cdot du^3 \cdot du^1 + g_{32} \cdot du^3 \cdot du^2 + g_{33} \cdot du^3 \cdot du^3  .
\end{aligned} 
	\label{geodaeten:equation:MetrischerTensor:Kartesisch:EinsteinSumme3D}
\end{equation}

Im Linienelement des Zylinders aus \eqref{geodaeten:equation:Linienelemente:Zylinder:Zylinder3D} kann man herauslesen, dass keine Terme mit gemischten Ableitungsprodukte vorhanden sind.
Daher sind alle Einträge außer den Diagonalen Null.
Mit $u^1 = r$, $u^2 = \phi$ und $u^3 = z$  können die diagonalen Elemente mit den Koeffizienten im Linienelement bestimmt werden. 
Somit gilt,

\begin{equation}
	\begin{aligned}
		g_{11}  &= 1  \\
		g_{22}  &= r^2 \\
		g_{33}  &= 1  
	\end{aligned}
\end{equation}
und damit ist der metrische Tensor im dreidimensionalen zylindrischen Raum

\begin{equation}
	T = \begin{pmatrix} 1 && 0 && 0 \\ 0 && r^2 && 0 \\ 0 && 0 && 1 \end{pmatrix} .
\end{equation}

Dieses Beispiel veranschaulicht, dass der metrische Tensor eine $n \times n$-Matrix ist, wobei $n$ der Anzahl Dimensionen entspricht.
Dieser metrische Tensor beinhaltet eine Variable $r$ und ist somit nicht konstant. 
Daher existiert für die dreidimensionale Oberfläche des vierdimensionalen zylindrischen Raums eine Krümmung.
Diese Krümmung eines vierdimensionalen Raums ist allerdings schwer vorzustellen, weshalb die Krümmung des metrischen Tensors im Beispiel der Kugel (Abschnitt \ref{geodaeten:section:MetrischerTensor:Kugel}) genauer vorgestellt wird.

%
% teil1.tex -- Beispiel-File für das Paper
%
% (c) 2020 Prof Dr Andreas Müller, Hochschule Rapperswil
%
% !TEX root = ../../buch.tex
% !TEX encoding = UTF-8
%
\subsection{Kugel\label{geodaeten:section:MetrischerTensor:Kugel}}
\rhead{Metrischer Tensor Beispiele}

Sed ut perspiciatis unde omnis iste natus error sit voluptatem
accusantium doloremque laudantium, totam rem aperiam, eaque ipsa
quae ab illo inventore veritatis et quasi architecto beatae vitae
dicta sunt explicabo.
Nemo enim ipsam voluptatem quia voluptas sit aspernatur aut odit
aut fugit, sed quia consequuntur magni dolores eos qui ratione
voluptatem sequi nesciunt




