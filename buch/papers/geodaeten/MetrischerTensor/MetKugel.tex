%
% teil1.tex -- Beispiel-File für das Paper
%
% (c) 2020 Prof Dr Andreas Müller, Hochschule Rapperswil
%
% !TEX root = ../../buch.tex
% !TEX encoding = UTF-8
%
\subsection{Kugel\label{geodaeten:section:MetrischerTensor:Kugel}}
\rhead{Metrischer Tensor Beispiele}

In diesem Abschnitt werden wir den metrischen Tensor für die Oberfläche einer Kugel herleiten.
Wir beginnen mit dem bereits bekannten Linienelement für die Kugeloberfläche und zeigen, wie der metrische Tensor daraus abgeleitet wird.

Das Linienelement für die Oberfläche einer Kugel mit Radius $r$ in Kugelkoordinaten $(\theta, \phi)$ ist gegeben durch:
\begin{equation}
	ds^2 = r^2 \left( d\theta^2 + \sin^2\theta \, d\phi^2 \right).
\end{equation}

Um den metrischen Tensor $g_{ij}$ für die Kugeloberfläche zu bestimmen, drücken wir das Linienelement in der allgemeinen Form
\begin{equation}
	ds^2 = g_{ij} \, du^i \, du^j
\end{equation}
aus, wobei $u^1 = \theta$ und $u^2 = \phi$ die Koordinaten auf der Kugeloberfläche darstellen.

Vergleichen wir nun die beiden Ausdrücke für das Linienelement:
\begin{equation}
	ds^2 = r^2 \, d\theta^2 + r^2 \sin^2\theta \, d\phi^2
\end{equation}
und
\begin{equation}
	ds^2 = g_{11} \, (d\theta)^2 + g_{22} \, (d\phi)^2 + 2g_{12} \, d\theta \, d\phi.
\end{equation}

Durch den Vergleich der Terme sehen wir, dass:
\begin{equation}
	g_{11} = r^2, \quad g_{22} = r^2 \sin^2\theta, \quad \text{und} \quad g_{12} = g_{21} = 0.
\end{equation}

Daraus ergibt sich der metrische Tensor für die Kugeloberfläche in Matrixform:
\begin{equation}
	g_{ij} = r^2 \begin{pmatrix}
		1 & 0 \\
		0 & \sin^2\theta
	\end{pmatrix}.
\end{equation}

Hier ist ersichtlich, dass der Radius $r$ lediglich ein Skalierungsfaktor ist und für die geometrischen Eigenschaften der Kugeloberfläche keine entscheidende Rolle spielt. 
Die wesentliche Geometrie wird durch die Winkelabhängigkeit der Metrik bestimmt.
Dieser Tensor beschreibt somit die Geometrie der Kugeloberfläche und ermöglicht die Berechnung von Abständen, Winkeln und anderen geometrischen Größen.





