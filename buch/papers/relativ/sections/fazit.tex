
\section{Fazit 
\label{relativ:section:fazit}}
\kopfrechts{Fazit}

Die Rechnungen in Abschnitt~\ref{relativ:section:teilchen-konst-e-feld}
haben gezeigt, dass man mittels der Variationsrechnung
die Bewegungsgleichung von Teilchen unter relativistischen
Bedingungen erhalten kann.
Doch was genau ist der Vorteil an dieser Methode?

Dieser liegt darin, dass man eine
Koordinatensystem-unabhängige Grösse zur Berechnung
der Lösung verwendet, nämlich den in Abschnitt~\ref{relativ:section:abstand}
eingeführten Abstand.
Unter Verwendung der korrekten kovarianten Grösse,
\index{kovariant}%
dem vierdimensionale Feld gegeben durch \((\varphi, \bm{A})\),
kann man somit erwarten, dass die Lösung ebenfalls kovariant ist.
Wechselt man also vom System \(K\) ins System \(K'\) und
rechnet den Vierervektor des elektromagnetischen Feldes entsprechend
von \(A^i\) zu \({A^i}'\) um, so stimmt die erhaltene Lösung
überein mit der Transformierten der Lösung im ursprünglichen System.

Mit dem Variationsprinzip hat man also eine Methode,
die auch für grosse Geschwindigkeitsbereiche automatisch
auf die richtigen Bewegungsgleichungen führt, selbst wenn uns
hierbei die Intuitionen und Methoden der klassischen Mechanik fehlen.
