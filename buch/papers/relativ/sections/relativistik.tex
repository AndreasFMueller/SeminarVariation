
\section{Das Relativitätsprinzip 
\label{relativ:section:relativistik}}
\rhead{Das Relativitätsprinzip}

Um relativistische Mechanik verstehen zu können,
muss natürlich zuerst auf das Relativitätsprinzip eingegangen werden.
Dabei gibt es zwei wesentliche Unterschiede zur klassischen Mechanik.

Der erste wichtige Unterschied ist, dass die Zeit unter relativistischer Betrachtung keine absolute Grösse mehr ist.
Geschehenes kann also nicht einfach anhand eines starren Zeitstrahls erklärt werden, sondern die Zeit ist abhängig vom Betrachter.
Man geht über vom dreidimensionalen Raum in die vierdimensionale Raumzeit, in welcher die Welt relativistisch beschrieben wird.

Der zweite Unterschied ist,
dass die Geschwindigkeit der Wirkungsausbreitung begrenzt ist,
und zwar durch die Lichtgeschwindigkeit \(c=\qty[per-mode=fraction]{299792458}{\metre\per\second}\).
Diese ist zwar auch in der klassischen Mechanik eine Konstante,
jedoch ist die Geschwindigkeit der Wirkungsausbreitung dort unbegrenzt.
Ein einfaches Beispiel, um dies zu veranschaulichen,
ist die Betrachtung von starren Körpern in der klassischen Mechanik.
Wird ein starrer Körper beispielsweise an einem Punkt angestossen,
so muss sich, gemäss Definition eines starren Körpers,
jeder Teil dieses Körpers augenblicklich und zeitgleich in Bewegung setzen.
Dies bedeutet also, dass sich die Wirkung (Anstossen des Körpers)
vom Punkt aus, in dem dieser angestossen wurde,
mit unendlicher Geschwindigkeit in alle Teile des Körpers ausbreitet.


\subsection{Koordinatentransformationen 
\label{relativ:section:koordtrafo}}

In der klassischen Mechanik gibt es die folgenden,
intuitiven Transformationen, um zwischen verschiedenen Bezugssystemen
zu vergleichen:

\begin{aligned}
    &\text{Translation in der Zeit: } && t \rightarrow t + b \\
    &\text{Translation im Raum: } && \vec{r} \rightarrow \vec{r} + \vec{a} \\
    &\text{Orthogonale Drehung: } && \vec{r} \rightarrow A \vec{r} \\
    &\text{Transformation auf ein Bezugssystem mit Relativgeschwindigkeit: } && \vec{r} \rightarrow \vec{r} + \vec{v} \cdot t
\end{aligned}.

Eine Kombination aus diesen Transformationen wird als Galilei-Transformation bezeichnet.

In der relativistischen Mechanik muss man sich hingegen der Lorentz-Transformation bedienen,
welche eine Erweiterung der Galilei-Transformation darstellt
und wesentlich kompliziertere Formeln annehmen kann. Besipielsweise beschreibt
\begin{equation}
    t = \frac{t' + \frac{V}{c^2}x'}{\sqrt{1-\frac{V^2}{c^2}}}
\end{equation}
die Umrechnung der Zeit bei einer Bewegung mit konstanter Geschwindigkeit \(V\).\todo{Diese Aussage überprüfen.}


\subsection{Der Abstand 
\label{relativ:section:abstand}}

Ebenfalls muss der Begriff des Abstands für die relativistische Mechanik umformuliert werden.
In der klassischen Mechanik ist der euklidische Abstand
\begin{equation}
    \de{s}=\sqrt{\de{x}^2 + \de{y}^2 + \de{z}^2}
\end{equation}
für alle Bezugssysteme identisch.

Analog dazu gibt es in der relativistischen Mechanik den erweiterten Abstand
\begin{equation}
    \de{s} = \sqrt{c^2\de{t}^2 - \de{x}^2 - \de{y}^2 - \de{z}^2},
\end{equation}
welcher abenfalls für verschiedene Bezugssysteme identisch ist.


\subsection{Die Eigenzeit 
\label{relativ:section:eigenzeit}}

Angenommen, wir beobachten eine Uhr, welche sich um die Strecke
\(\sqrt{\de{x}^2 + \de{y}^2 + \de{z}^2}\)
bewegt.
Im Koordinatensystem der Uhr gilt
\(\de{x'} = \de{y'} = \de{z'} = 0\).
Gemäss der Invarianz des Abstands gilt somit
\begin{equation*}
    \underbrace{\de{s}^2 = c^2\de{t}^2 - \de{x}^2 - \de{y}^2 - \de{z}^2}_{\text{Unser Koordinatensystem}}
        = \underbrace{\de{s'}^2 = c^2\de{t'}^2}_{\text{Koordinatensystem der Uhr}} .
\end{equation*}
\todo{Diesen Abschnitt vollenden.}
