
\section{Ladungen im Elektromagnetischen Feld 
\label{relativ:section:em_feld}}
\rhead{Ladungen im Elektromagnetischen Feld}


\subsection{Elementarteilchen in der Relativitätstheorie 
\label{relativ:section:elementarteilchen}}

Wie bereits in der Einleitung zu Abschnitt~\ref{relativ:section:relativistik} erwähnt,
kann es gemäss der Relativitätstheorie keine starren Körper geben.
Angemessener ist daher die Betrachtung \emph{punktförmiger Elementarteilchen}.
Der Zustand eines solchen Elementarteilchens ist dabei vollständig definiert durch
die drei Raumkoordinaten und die drei zugehörigen Geschwindigkeitskomponenten.


\subsection{Wirkungsintegral 
\label{relativ:section:wirkungsintegral}}

Das Wirkungsintegral für ein Elementarteilchen im elektromagnetischen Feld ist
\begin{equation}
    S = \int_a^b \left(-mc\,ds + \frac{e}{c} \mathbf{A}\,d\mathbf{r} - e\varphi\,dt\right),
\label{relativ:eqn:wirk-int-em-feld}
\end{equation}
oder als Integral über die Zeit geschrieben,
\begin{equation}
    S = \int_{t_1}^{t_2} \left( -mc^2 \sqrt{1-\frac{v^2}{c^2}} + \frac{e}{c} \mathbf{A} \mathbf{v} - e \varphi \right) \, dt,
\label{relativ:eqn:wirk-int-em-feld-zeit}
\end{equation}
wobei \(\displaystyle \mathbf{v} = \frac{d\mathbf{r}}{dt}\) der Geschwindigkeitsvektor der drei räumlichen Dimensionen ist.
Der Integrand in~\ref{relativ:eqn:wirk-int-em-feld-zeit} ist gerade die Lagrange-Funktion
\begin{equation}
    L = -mc^2 \sqrt{1-\frac{v^2}{c^2}} + \frac{e}{c} \mathbf{A} \mathbf{v} - e \varphi
\label{relativ:eqn:lagrange-em-feld}
\end{equation}


\subsection{Bewegungsgleichung 
\label{relativ:section:bewegungsgleichung}}

Bei der Berechnung der Bahn eines Elementarteilchens
in einem elektromagnetischen Feld geht man von
der vereinfachenden aber angemessenen Annahme aus,
dass die Rückwirkung des Teilchens auf das Feld vernachlässigt werden kann.
\footnote{Konkret muss für diese Annahme beispielsweise
\(H \ll \frac{m^2c^4}{e^3}\) erfüllt sein.
Die rechte Seite ergibt dabei für ein Elektron
eine magnetische Feldstärke von
\(\frac{m_e^2c^4}{e^3} \approx
\frac{(\qty{9e-31}{\kilogram})^2 (\qty{3e8}{\metre\per\second})^4}{(\qty{1.6e-19}{\ampere\second})^3}
\approx \qty[per-mode=fraction]{1.6e30}{\ampere\per\metre}\),
was im Vakuum einer magnetischen Flussdichte von
\(\mu_0 \cdot \qty[per-mode=fraction]{1.6e30}{\ampere\per\metre} \approx
\qty[per-mode=fraction]{1.25e-6}{\tesla\metre\per\ampere} \cdot
\qty[per-mode=fraction]{1.6e30}{\ampere\per\metre} =
\qty{2e24}{\tesla}\)
entspricht, einem Wert weitaus grösser als alle bekannten physikalischen Phänomene.}

Die Bewegungsgleichung ergibt sich aus der Variation des Wirkungsintegrals
mit der Euler-Lagrange-Differentialgleichung
\begin{equation}
    \frac{d}{dt} \frac{\partial L}{\partial \mathbf{v}} - \frac{\partial L}{\partial \mathbf{r}} = 0,
\label{realtiv:eqn:euler-lagrange-em-feld}
\end{equation}
wobei \(L\) in~\ref{relativ:eqn:lagrange-em-feld} gegeben ist.
Die Ableitung nach dem Geschwindigkeitsvektor \(\mathbf{v}\) ergibt
\begin{equation}
    \frac{\partial L}{\partial \mathbf{v}} =
    mc^2 \frac{-\frac{2}{c^2}\mathbf{v}}{-2\sqrt{1-\frac{v^2}{c^2}}}
    + \frac{e}{c} \mathbf{A} - 0
    = \frac{m \mathbf{v}}{\sqrt{1-\frac{v^2}{c^2}}} + \frac{e}{c} \mathbf{A}.
\label{realtiv:eqn:part-diff-v}
\end{equation}
Dies wird auch als verallgemeinerter Teilchenimpuls \(\mathbf{P}\) bezeichnet.
Die partielle Ableitung nach \(\mathbf{r}\) ergibt dann
\begin{equation}
    \frac{\partial L}{\partial \mathbf{r}} = \nabla L
    = 0 + \frac{e}{c} \operatorname{grad} \mathbf{Av} - e \operatorname{grad} \varphi
\label{realtiv:eqn:part-diff-r}
\end{equation}
\todo{Rechnung fortsetzen}
