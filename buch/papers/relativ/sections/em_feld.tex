
\section{Ladungen im Elektromagnetischen Feld 
\label{relativ:section:em_feld}}
\rhead{Ladungen im Elektromagnetischen Feld}

\subsection{Elementarteilchen in der Relativitätstheorie 
\label{relativ:section:elementarteilchen}}

Wie bereits in der Einleitung zu Abschnitt~\ref{relativ:section:relativistik} erwähnt,
kann gemäss der Relativitätstheorie keine starren Körper geben.
Angemessener ist daher die Betrachtung \emph{punktförmiger Elementarteilchen}.
Der Zustand eines solchen Elementarteilchens ist dabei vollständig definiert durch
die drei Raumkooridnaten und die drei zugehörigen Geschwindigkeitskomponenten.

\subsection{ Wirkungsintegral 
\label{relativ:section:wirkungsintegral}}

Das Wirkungsintegral für ein Elementarteilchen im elektromagnetischen Feld ist
\begin{equation}
    S = - \int_{t_1}^{t_2} \left( -mc^2 \sqrt{1-\frac{v^2}{c^2}} + \frac{e}{c} \mathbf{A} \mathbf{v} - e \varphi \right) \, dt
\end{equation}
