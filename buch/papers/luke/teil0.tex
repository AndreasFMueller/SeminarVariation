%
% einleitung.tex -- Beispiel-File für die Einleitung
%
% (c) 2020 Prof Dr Andreas Müller, Hochschule Rapperswil
%
% !TEX root = ../../buch.tex
% !TEX encoding = UTF-8
%
\section{Luke's Lagrangian\label{luke:section:teil0}}
\rhead{Teil 0}

1967 veröffentlichte J.C. Luke (HIER CITE) den Luke`s Lagrangian mit welchem über das Variationsprinzip die Bewegungen von Oberflächenwellen auf einer freien Fluid Oberfläche unter Einwirkung der Schwerkraft berechnet werden kann.
Genau gesagt lässt sich durch die Minimierung des Lagrangian über das Variationsprinzip, Gleichungen für Oberflächengravitationswellen eines Potentialflusses mit einer undurchlässigen freien Oberfläche sowie Boden, bei konstanten Druck, hergeleitet.

Die Formel \eqref{luke:Luke_Variation_Formel} zeigt Luke´s Variationsformel in welcher Luke´s Lagrangian \eqref{luke:Luke_Lagrangian_Allgemein} vorkommt.

\begin{equation}
	\mathcal{L} 
	=
	\int_{t_0}^{t_1} \int_{\Omega}^{} \mathscr{L} \rho \, d^2\bm{x}\, dt
	\label{luke:Luke_Variation_Formel}
\end{equation}

\begin{equation}
	\mathscr{L}
	=
	-\int_{-h}^{\eta} g z + \frac{\partial\phi}{\partial t}+ \frac{1}{2}(\nabla\phi)^2 + \frac{1}{2}(\frac{\partial\phi}{\partial z})^2\, dz
	\label{luke:Luke_Lagrangian_Allgemein}
\end{equation}

\begin{itemize}
	\item
	$\phi(\bm{x},z,t)$ ist das Geschwindigkeitspotential
	\item
	$\rho$ Dichte der Flüssigkeit
	\item
	$g$ Erdbeschleunigung
	\item
	$t$ Zeit
	\item 
	$\bm{x}$ horizontale Koordinatenvektor mit $x$ und $y$ als Komponenten
	\item 
	$z$ vertikale Koordinate
	\item 
	$\Omega$ horizontale Achse 
	\item 
	$\nabla$ horizontaler Gradient
	\item 
	$\eta(\bm{x},t)$ freie Oberfläche der Flüssigkeit
	\item 
	$h(\bm{x},t)$ Gewässergrund 
	
\end{itemize}

\subsection{Erweitern des Luke`s Lagrangian
	\label{luke:subsection:Erweitern}}
Im sinne der Vereinfachung wird die Oberflächenspannung vernachlässigt sowie die Flüssigkeitsdichte $\rho$ konstant auf $\rho = 1\frac{g}{ml}$ gesetzt. Das entspricht der Dichte von Wasser und sollte somit nicht die weiteren Berechnungen beeinflussen.

Das Integral \eqref{luke:Luke_Lagrangian_Allgemein} kann umgeschrieben und Integriert werden. Dabei wird das Integral teilweise integriert.
 \begin{equation}
 	\mathscr{L}
 	=
 	-\int_{-h}^{\eta} g z\, dz - \int_{-h}^{\eta} \frac{\partial\phi}{\partial t}\, dz -\int_{-h}^{\eta} \frac{1}{2}(\nabla\phi)^2 + \frac{1}{2}(\frac{\partial\phi}{\partial z})^2 dz
 \end{equation}
 \begin{equation}
 	\mathscr{L}
 	=
 	-\frac{1}{2}g\eta^2 + \frac{1}{2}h^2g +\tilde{\phi} \frac{\partial\eta}{\partial t} + \check{\phi} \frac{\partial h}{\partial t} -\int_{-h}^{\eta} \frac{1}{2}(\nabla\phi)^2 + \frac{1}{2}(\frac{\partial\phi}{\partial z})^2 dz
 	\label{luke:Luke_Lagrangian_allgemein_vereinfacht}
 \end{equation}
Dabei ist $\tilde{\phi} = \phi(\bm{x},\eta,t)$ das Geschwindigkeitspotential auf der Oberfläche und $\check{\phi} = \phi(\bm{x},-h,t)$ das Geschwindigkeitspotential am Gewässergrund.

Beim Variationsprinziep wird die Veränderung der Funktion unter Variation ihrer Parameter betrachtet.
Da aber die Tiefe $h$ vorgegeben ist, verändert diese sich nicht während des Minimierungsprozesses.
Daher hat der dieser Term $\frac{1}{2}h^2g$ keinen Einfluss auf die Variation und kann weggelassen werden.

Um denn Freiheitsgrad der Variation zu erhöhen, wird die horizontale Geschwindikeit $\bm{u} = \nabla\phi $ entspricht $\frac{\partial \phi}{\partial x}, \frac{\partial \phi}{\partial y}$ und die vertikale Geschwindigkeit $v = \frac{\partial \phi}{\partial z}$ eingeführt. 
Luke`s Lagrangian setzt vorraus das die approximation rotationsfrei ist.
Das implementiert, dass bei der Wahl des Ansatzes für $\phi$ rotationsfrei sein muss.
Weil es sich bei der Rotationsfreiheit des Geschwindigkeitspotentials um eine Nebenbedingungen im Variationsproblem handelt, müssen die Lagrange-Multiplikatoren $\bm{\mu}$ und $\upsilon$ für $\bm{u}$ und $v$ eingesetzt werden.
Somit kann der Lagrangian umformuliert werden.

\begin{equation}
	\mathscr{L}
	=
	-
	\frac{1}{2} g \eta^2
	+
	\widetilde{\phi} \frac{\partial\eta}{\partial t}
	+
	\check{\phi} \frac{\partial h}{\partial t}
	-
	\int_{-h}^{\eta} \left[ \frac{1}{2} (\bm{u}^2 + v^2) + \bm{\mu} \cdot (\nabla\phi - \bm{u}) + \upsilon \cdot (\frac{\partial\phi}{\partial z} - v) \right] dz
	\label{luke:Luke_Lagrangian_mit_Multi}
\end{equation}

Die Lagrange-Multiplikatoren müssen über Variation bestimmt werden.
Dafür betrachten wir die Variation der Lagrangefunktion \eqref{luke:Luke_Lagrangian_mit_Multi} bezogen auf die Geschwindigkeiten $\bm{u}$ und $v$.
Diese Variation liefert uns die Gleichungen um die stationären Punkte der Lagrangefunktion zu charakterisieren. 
Die Variation von \(\mathscr{L}\) nach $\bm{u}$ und $v$ ergibt:

\begin{equation}
	\frac{\partial \mathscr{L}}{\partial \bm{u}} = 0	
\end{equation}
\begin{equation}
	\frac{\partial \mathscr{L}}{\partial v} = 0	
\end{equation}
Setzen wir $\mathscr{L}$ ein und führen die Variation durch:

\begin{equation}
	\frac{\partial}{\partial u} \left( -\frac{1}{2} g \eta^2 + \widetilde{\phi} \frac{\partial\eta}{\partial t} + \check{\phi} \frac{\partial h}{\partial t} - \int_{-h}^{\eta} \left[ \frac{1}{2} (\bm{u}^2 + v^2) + \bm{\mu} \cdot (\nabla\phi - \bm{u}) + \upsilon \cdot \left(\frac{\partial\phi}{\partial z} - v\right) \right] dz \right) = 0	
\end{equation}
\begin{equation}
	\frac{\partial}{\partial v} \left( -\frac{1}{2} g \eta^2 + \widetilde{\phi} \frac{\partial\eta}{\partial t} + \check{\phi} \frac{\partial h}{\partial t} - \int_{-h}^{\eta} \left[ \frac{1}{2} (\bm{u}^2 + v^2) + \bm{\mu} \cdot (\nabla\phi - \bm{u}) + \upsilon \cdot \left(\frac{\partial\phi}{\partial z} - v\right) \right] dz \right) = 0
\end{equation}

Um $\bm{\mu}$ und $\upsilon$ zu bestimmen, damit die Lagrangefunktion ein Minimum oder Maximum annimmt, werden die Variationen darauf umgeformt. Dabei wird berechnet, dass $\bm{\mu} = \bm{u}$ und $\upsilon = v$ sein muss.
Diese erkenntniss kann wiederum in \eqref{luke:Luke_Lagrangian_mit_Multi} eingesetzt werden. Damit ergibt sich folgende Lagrangefunktion:

\begin{equation}
	\mathscr{L}
	=
	-
	\frac{1}{2} g \eta^2
	+
	\widetilde{\phi} \frac{\partial\eta}{\partial t}
	+
	\check{\phi} \frac{\partial h}{\partial t}
	-
	\int_{\eta}^{-d} \left[ \frac{1}{2} u^2 + \frac{1}{2} v^2 - u \cdot \nabla \phi - v \phi_y \right] dy 
	\label{luke:Luke_Lagrangian_mit_Multi_verkuerzt}
\end{equation}

Somit haben wir drei Lagrangefunktionen mit unterschiedlich vielen Variablen.
Die Ursprüngliche Luke`s Lagrangian \eqref{luke:Luke_Lagrangian_allgemein_vereinfacht} hat zwei Variablen $(\phi, \eta)$.
Die angepasste Lagrangefunktion mit den Lagrange-Multiplikatoren \eqref{luke:Luke_Lagrangian_mit_Multi} hat sechs Variablen $(\phi, \eta, \bm{u}, v, \bm{\mu}, \upsilon)$ und die vereinfachte Form davon \eqref{luke:Luke_Lagrangian_mit_Multi_verkuerzt} hat vier Variablen $(\phi, \eta, \bm{u}, v)$.
Diese zusätzlichen Variablen bringen zusätzliche Freiheitsgrade. Damit kann bei der Konstruktion von Approximationen für verschiedener Oberflächenwasserwellen mehrere untergeordnete Beziehungen erfüllt werden.
















