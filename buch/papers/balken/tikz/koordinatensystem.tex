%
% koordinatensystem.tex -- Koordinatensystem
%
% (c) 2021 Prof Dr Andreas Müller, OST Ostschweizer Fachhochschule
%
\documentclass[tikz]{standalone}
\usepackage{amsmath}
\usepackage{times}
\usepackage{txfonts}
\usepackage{pgfplots}
\usepackage{csvsimple}
\usetikzlibrary{arrows,intersections,math,calc}
\definecolor{darkred}{rgb}{0.8,0,0}
\begin{document}
\def\skala{1}
\def\r{0.7}
\def\wone{-20}
\def\wtwo{50}
\def\w{20}
\begin{tikzpicture}[>=latex,thick,scale=\skala]

\coordinate (p1) at (1.5,-2.8);
\coordinate (p2) at (6.5,-1.2);

\draw[color=darkred] (p1) to[out=\wone,in={-180+\wtwo}] (p2);

\coordinate (q) at (4.2,-2.87);
\draw[color=orange] ($(q)+({-180+\w}:1)$) -- ++(\w:2);

\fill[color=orange] (q) circle[radius=0.08];
\node[color=orange] at (q) [below right] {Tangente};

\fill[color=blue] (p1) circle[radius=0.08];
\fill[color=blue] (p2) circle[radius=0.08];

\draw[->,color=blue] ($(p2)+({-90+\wtwo}:\r)$) arc({-90+\wtwo}:{90+\wtwo}:\r);
\draw[->,color=blue] ($(p1)+({270+\wone}:\r)$) arc({270+\wone}:{90+\wone}:\r);

\node[color=blue] at ($(p1)+({180+\wone}:\r)$) [left] {$M$};
\node[color=blue] at ($(p2)+({\wtwo}:\r)$) [above right] {$M$};

\draw[->] (0,0) -- (8,0) coordinate[label={$x$}];
\draw[->] (0,0) -- (0,-4) coordinate[label={left:$z,\,w(x)$}];

\end{tikzpicture}
\end{document}

