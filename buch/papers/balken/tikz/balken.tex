%
% balken.tex -- Balkengleichung, Herleitung mit Momenten
%
% (c) 2021 Prof Dr Andreas Müller, OST Ostschweizer Fachhochschule
%
\documentclass[tikz]{standalone}
\usepackage{amsmath}
\usepackage{times}
\usepackage{txfonts}
\usepackage{pgfplots}
\usepackage{csvsimple}
\usetikzlibrary{arrows,intersections,math}
\definecolor{darkred}{rgb}{0.8,0,0}
\definecolor{darkgreen}{rgb}{0,0.6,0}
\def\l{4}
\def\s{0.8}
\pgfmathparse{\s*sqrt(3)/2}
\xdef\h{\pgfmathresult}
\pgfmathparse{\s*0.8}
\xdef\S{\pgfmathresult}
\def\balken{
	\fill[color=brown!20] (-\l,0) rectangle (\l,0.4);
	\draw[color=brown] (-\l,0) rectangle (\l,0.4);
}
\def\last{
	\fill[color=blue!20] (-\l,0.0) rectangle ++({2*\l},-0.4);
	\draw[color=blue] (-\l,0.0) rectangle ++({2*\l},-0.4);
	\node[color=blue] at (\l,-0.2) [right] {$q_0\mathstrut$};
	\foreach \x in {-3.5,-2.5,...,3.5}{
		\draw[->,color=blue] (\x,0.0) -- ++(0,-0.4);
	}
}
\def\bemassung{
	\draw[line width=0.3pt] (-\l,0.0) -- +(0,0.7);
	\draw[line width=0.3pt] (\l,0.0) -- +(0,0.7);
	\draw[<->] (-\l,0.6) -- (\l,0.6);
	\node at (0,0.6) [above] {$l$};
}
\def\auflageA{
	\fill[color=gray!40] ({-\l-0.8},{-\h}) rectangle +(1.6,-0.3);
	\draw (-\l,0) -- ++(-60:\s) -- ++({-\s},0) -- cycle;
	\draw ({-\l-0.8},{-\h}) -- +(1.6,0);
	\node[color=darkgreen] at (-\l,{-\h-0.5}) [below] {$A$};
	\draw[->,color=darkgreen] (-\l,{-\h-0.5}) -- (-\l,-0.1);
}
\def\auflageB{
	\fill[color=gray!40] ({\l-0.8},{-\h}) rectangle +(1.6,-0.3);
	\draw (\l,0) -- ++(-60:\S) -- ++({-\S},0) -- cycle;
	\draw ({\l-0.8},{-\h}) -- +(1.6,0);
%\draw ({-\l-0.8},{-\s*sqrt(3)/2}) -- +(1.6,0);
	\node[color=darkgreen] at (\l,{-\h-0.5}) [below] {$B$};
	\draw[->,color=darkgreen] (\l,{-\h-0.5}) -- (\l,-0.1);
}

\begin{document}
\def\skala{1}
\begin{tikzpicture}[>=latex,thick,scale=\skala]

\begin{scope}[yshift=1cm]
	\bemassung
	\last
\end{scope}

\balken

\draw[color=darkred] plot[domain=-1:1,samples=100] ({\x*\l},{-0.6+0.8*\x*\x});
\node[color=darkred] at (0,-0.6) [below] {$w(x)$};

\auflageA
\auflageB

\begin{scope}[yshift=-3.8cm]
	\begin{scope}[yshift=1cm]
		\last
	\end{scope}
	\auflageA
	\balken
	\fill[color=white,opacity=0.8] (-1,-1) rectangle (5,1.2);
	\draw (-1,0) -- ++(0,-0.4) -- ++(-0.1,0);
	\draw (-1,0) -- ++(0,1.4) -- ++(-0.1,0);
	\draw[->] (-0.8,-0.20) arc (-60:60:0.8);
	\node at (-0.3,0.4) [right] {$M_y(x)$};
	\draw[->,line width=0.3pt] (-\l,1.3) -- ++(3,0);
	\node at ({-0.5*\l},1.3) [above] {$x$};
\end{scope}

\end{tikzpicture}
\end{document}

