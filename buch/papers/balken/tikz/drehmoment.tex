%
% drehmoment.tex -- template for standalon tikz images
%
% (c) 2021 Prof Dr Andreas Müller, OST Ostschweizer Fachhochschule
%
\documentclass[tikz]{standalone}
\usepackage{amsmath}
\usepackage{times}
\usepackage{txfonts}
\usepackage{pgfplots}
\usepackage{csvsimple}
\usetikzlibrary{arrows,intersections,math,calc}
\definecolor{darkred}{rgb}{0.8,0,0}
\begin{document}
\def\skala{1}
\begin{tikzpicture}[>=latex,thick,scale=\skala]

%
% common.tex -- gemeinsame definition
%
% (c) 2017 Prof Dr Andreas Müller, Hochschule Rapperswil
%
%
% packages.tex -- packages required by the paper planet
%
% (c) 2019 Prof Dr Andreas Müller, Hochschule Rapperswil
%

% if your paper needs special packages, add package commands as in the
% following example
%\usepackage{packagename}

\usepackage{cleveref}
%
% common.tex -- gemeinsame definition
%
% (c) 2017 Prof Dr Andreas Müller, Hochschule Rapperswil
%
%
% packages.tex -- packages required by the paper planet
%
% (c) 2019 Prof Dr Andreas Müller, Hochschule Rapperswil
%

% if your paper needs special packages, add package commands as in the
% following example
%\usepackage{packagename}

\usepackage{cleveref}
%
% common.tex -- gemeinsame definition
%
% (c) 2017 Prof Dr Andreas Müller, Hochschule Rapperswil
%
\input{../common/packages.tex}
\input{../common/common.tex}
\mode<beamer>{%
\usetheme[hideothersubsections,hidetitle]{Hannover}
}
\beamertemplatenavigationsymbolsempty
\title[NB]{Nebenbedingungen}
\author[A.~Müller]{Prof. Dr. Andreas Müller}
\date[]{}
\newboolean{presentation}


\mode<beamer>{%
\usetheme[hideothersubsections,hidetitle]{Hannover}
}
\beamertemplatenavigationsymbolsempty
\title[NB]{Nebenbedingungen}
\author[A.~Müller]{Prof. Dr. Andreas Müller}
\date[]{}
\newboolean{presentation}


\mode<beamer>{%
\usetheme[hideothersubsections,hidetitle]{Hannover}
}
\beamertemplatenavigationsymbolsempty
\title[NB]{Nebenbedingungen}
\author[A.~Müller]{Prof. Dr. Andreas Müller}
\date[]{}
\newboolean{presentation}



\fill[color=blue] (p1) circle[radius=0.08];

\draw[line width=0.3pt] ($(p1)+(0,1.4)$) -- ++({\l/2},0);
\draw[line width=0.3pt] ($(p1)+(0,1.4)+(-135:0.1)$) -- ++(45:0.2);
\draw[line width=0.3pt] ($(p1)+({\l/2},1.4)+(-135:0.1)$) -- ++(45:0.2);
\node at ($(p1)+({\l/4},1.4)$) [above] {$s\mathstrut$};
\draw[line width=0.3pt] ($(p1)+(0,0.5)$) -- ++(0,1.2);

\draw[->,color=darkred] (15:{0.4*\l}) arc (15:-15:{0.4*\l});
\node[color=darkred] at (-15:{0.4*\l}) [right] {$M = F\cdot s$};

\end{tikzpicture}
\end{document}

