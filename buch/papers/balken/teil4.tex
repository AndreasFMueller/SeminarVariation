%
% einleitung.tex -- Beispiel-File für die Einleitung
%
% (c) 2020 Prof Dr Andreas Müller, Hochschule Rapperswil
%
% !TEX root = ../../buch.tex
% !TEX encoding = UTF-8
%
\section{Fazit\label{balken:section:teil4}}

In dieser Arbeit haben wir die umfassende Herleitung und Anwendung der Balkengleichung untersucht. Dabei wurde deutlich, wie die Variationsprinzipien zur Formulierung dieser fundamentalen Differentialgleichung beitragen. Ein besonderes Augenmerk lag auf der praktischen Anwendung der Balkengleichung im Ingenieurwesen, wobei verschiedene Szenarien analysiert wurden, um wichtige Parameter für die Strukturanalyse zu bestimmen. Dies ermöglicht Ingenieuren, fundierte Entscheidungen im Entwurfs- und Konstruktionsprozess zu treffen und die Sicherheit sowie Effizienz von Strukturen zu verbessern.

Die Herleitung der Balkengleichung, wie sie im Kapitel \cite{balken:Balkentheorie} sehen können, zeigt die Integration der Belastungs- und Verformungskomponenten und bestätigt die theoretischen Annahmen mit realen Anwendungsszenarien. Die Gleichung (13.62), fasst die Beziehung zwischen den aufgebrachten Lasten und der resultierenden Durchbiegung des Balkens zusammen und bietet eine präzise mathematische Grundlage für weitere strukturelle Analysen.

Durch die Verbindung von theoretischen Grundlagen und praktischen Anwendungen haben wir ein tieferes Verständnis für das Verhalten von Strukturen unter Belastung gewonnen. Die Balkengleichung bleibt somit ein unverzichtbares Werkzeug in der Strukturanalyse und im Bauingenieurwesen.


