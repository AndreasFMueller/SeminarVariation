%
% einleitung.tex -- Beispiel-File für die Einleitung
%
% (c) 2020 Prof Dr Andreas Müller, Hochschule Rapperswil
%
% !TEX root = ../../buch.tex
% !TEX encoding = UTF-8
%
\section{Einleitung
\label{buch:paper:section:introduction}}
\rhead{Einleitung}
Die Idee war es, das Variationsprinzip der Analysis mit der Informatik zu kombinieren. 
Durch Recherche und Unterstützung von ChatGPT entstand folgendes Thema:

\begin{quote}
    Genetische Algorithmen, welche das Variationsprinzip nutzen, um eine Population 
    möglicher Lösungen für ein Problem zu entwickeln und zu verfeinern. 
    Die grundlegende Idee besteht darin, aus einer Vielzahl von möglichen Lösungen 
    die optimale herauszufiltern und kontinuierlich zu verbessern.
\end{quote}

\section{Ziel der Arbeit
\label{buch:paper:varalg:section:goal}} 
\rhead{Ziel der Arbeit}
Das Ziel dieser Arbeit ist es, das angedeutete Variationsprinzip anzuwenden und dabei 
zu untersuchen, ob dasselbe Variationsprinzip gemeint ist, wie es in der Analysis 
verwendet wird.
