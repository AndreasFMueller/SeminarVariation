%
% teil1.tex -- Beispiel-File für das Paper
%
% (c) 2020 Prof Dr Andreas Müller, Hochschule Rapperswil
%
% !TEX root = ../../buch.tex
% !TEX encoding = UTF-8
%
\section{Algorithmus auf ein Problem anwenden
\label{buch:paper:varalg:section:find_problem}}
\rhead{Algorithmus auf ein Problem anwenden}
Um den Algorithmus zu verstehen, benötigt man ein Problem, das 
viele Lösungen bietet, von denen eine optimal ist. Die verschiedenen 
Lösungen sollen kleine Abweichungen aufweisen, was auf die Nachbarlösungen hinweist, 
und das Problem soll mit der Grösse aufwändiger werden, um eine 
Lösung zu finden. Dabei kam das Travelling Salesman Problem (TSP) 
in Betracht, welches ein NP-vollständiges Problem darstellt. Es 
erfüllt die Anforderungen mit vielen Variationen von Lösungen, wobei
irgendeines das Optimum ist.

\subsection{Travelling Salesman Problem
\label{buch:paper:varalg:subsection:tsp}}
\rhead{Travelling Salesman Problem}
Das Travelling Salesman Problem (TSP) ist ein klassisches Problem 
aus der kombinatorischen Optimierung und Informatik. Ziel ist es, 
für einen Händler die kürzeste Rundreise zu finden, welche eine 
gegebene Anzahl von Städten umfasst, wobei jede Stadt 
genau einmal besucht wird, bevor der Händler genau zum Ausgangspunkt 
zurückkehrt. 

