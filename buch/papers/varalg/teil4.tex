%
% teil3.tex -- Beispiel-File für Teil 3
%
% (c) 2020 Prof Dr Andreas Müller, Hochschule Rapperswil
%
% !TEX root = ../../buch.tex
% !TEX encoding = UTF-8
%
\section{Variationsprinzip der Analysis und Algorithmus
\label{buch:paper:varalg:section:variations_analysis_algorithm_result}}
\rhead{Variationsprinzip Mathematik und Algorithmus}
Im vorherigen Kapitel \ref{buch:paper:varalg:section:genetic_algorithm_process} wurde erklärt, 
wie der Algorithmus funktioniert, der das Variationsprinzip anwendet. Dies 
passt jedoch nicht genau zu dem Variationsprinzip, das in der Analysis 
verwendet wird. In diesem Kapitel wird genauer darauf eingegangen.
\begin{table}
   \centering
   \caption{Woran unterscheiden sich die beiden Prinzipien?}
   \begin{tabularx}{\textwidth}{|X|X|X|}
      \hline
       & Analysis 
       & Algorithmus 
       \\ \hline
      Ziele  
       & Wie im Kapitel Funktionale \ref{buch:variation:problem:subsection:funktionale}

      beschrieben, ist das Ziel der Variationsrechnung die Optimierung von Funktionalen 
      durch die Findung einer optimalen Funktion.
       & Im Algorithmus bedeutet Variation, dass es eine Menge möglicher Lösungen gibt, 
      aus denen die besten Lösungen ausgewählt und weiterverarbeitet werden, in der 
      Hoffnung, dass die neuen Lösungen besser sind.
      \\ \hline
      Techniken  
       & Hier werden analytische Techniken wie die Euler-Lagrange-Gleichung verwendet, 
      um Optimierungsprobleme zu lösen. (Verschiedene Techniken sind in den Kapiteln 
      2-10 beschrieben)
       & Im Algorithmus werden Mechanismen verwendet, die stochastische 
      \footnote{
         Im Fall des Algorithmus sind stochastische Methoden gemeint, bei denen 
         eine Anzahl an Zufallsereignissen oder -kombinationen erstellt und 
         diese anschließend ausgewertet oder weiterverarbeitet werden.
      }
      Methoden wie Kreuzung und Mutation beinhalten, um Vielfalt zu erzeugen und aufrechtzuerhalten.
      \\ \hline
      Nachbarlösungen
       & Es wird das Konzept der Richtungsableitung in Richtung einer Funktion \(f(x)\)
       genutzt. Diese Funktion kann durch kleine Änderungen des Parameters \(x\) beliebig
       nahe beeinanderliegen. Aufgrund dieser Eigenschaft kann man die Ableitung nach dem
       Parameter \(x\) bilden, woraus sich die Möglichkeit die Euler-Lagrange-Differentialgleichung
       abzuleiten. Das Bedeutet das die Nachbarlösungen beliebig nahe sein können.  
       & Die Nachbarlösungen im Algorithmus werden die Variationen direkt untersucht,
      gekreuzt und mutiert, um die besten Lösungen zu finden. Jede Nachbarlösung 
      ist eine endliche Distanz von der aktuellen Lösung entfernt und kann nicht beliebig 
      nahe sein. Da so keine kontinuierlich Variation möglich ist, gibt es keine Ableitungen
      und es müssen Methoden dazu genutzt werden, wie im Kapitel 3 
      \ref{buch:paper:varalg:section:genetic_algorithm_process} beschrieben.
      \\ \hline
      Lösung
       & Die Lösung ist eine Funktion, mit der das Optimum errechnet werden kann.
       & Die Lösung am Ende könnte das Optimum sein oder nur sehr nah dran.
      \\ \hline
   \end{tabularx}
   \caption{Vergleich auf den Begriff Variationen}
   \label{tab:variation_comparison}
\end{table}
Die Aussage, dass im Algorithmus das Variationsprinzip verwendet ist so nicht
korrekt. Diese Unterscheiden sich in einigen Punkten, wie in der Tabelle 
\ref{tab:variation_comparison} aufgezeigt.
Beide suchen zwar das Optimum, aber in der Analysis sucht 
man eine Funktion, mit der das Optimum berechnet werden kann, wie z.B. 
die Durchbiegung einer Kette. Im Algorithmus wird versucht, das Optimum 
durch Zufall und Variationen (neue Kombinationen) zu finden. Am Ende 
hat man eine Lösung, die das Optimum sein könnte, sehr nahe dran ist 
oder komplett daneben liegt.
