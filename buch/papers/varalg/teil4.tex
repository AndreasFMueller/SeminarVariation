%
% teil3.tex -- Beispiel-File für Teil 3
%
% (c) 2020 Prof Dr Andreas Müller, Hochschule Rapperswil
%
% !TEX root = ../../buch.tex
% !TEX encoding = UTF-8
%
\section{Variationsprinzip der Analysis und genetischer Algorithmus
\label{buch:paper:varalg:section:variations_analysis_algorithm_result}}
\rhead{Variationsprinzip Analysis und genetischer Algorithmus}
Im vorherigen Abschnitt \ref{buch:paper:varalg:section:genetic_algorithm_process} 
wurde dargelegt, aus welchen Komponenten ein genetischer Algorithmus besteht und wie 
solcher aufgebaut werden kann. 
Obwohl in beiden Fällen das Ziel darin besteht, das Optimum zu erreichen, unterscheidet
sich das Variationsprinzip der Analysis grundlegend von dem, was in einem genetischen 
Algorithmus zur Anwendung kommmt. In diesem Abschnitt werden die Unterschiede in der 
Tabelle \ref{tab:variation_comparison} aufgeführt und beschrieben.

Ein wesentlicher Unterschied ist, was man aus der Analysis und genetischer Algorithmus
erhält. In der Analysis ist ein wesentliches Konzept die Deformation in eine bestimmte
Richtung, das Differential, was in der Richtungsableitung/Variation steckt. Die Richtungsableitung
beschreibt, wie sich eine Funktion in eine bestimmte Richtung ändert. Die Variation in 
der Variationsrechnung untersucht die Änderung eines Funktionals. Das bedeutet, durch die Ableitung
erhalten wir eine Funktion, mit welcher das Optimum berechnen werden kann, wie z.B. die 
Durchbiegung einer Kette. Im Algorithmus wird versucht, das Optimum durch Zufall von  
Variationen, durch neue Kombinationen der genetischens Strings, zu finden. Am
Ende erhält man für die gestellt Problemstellung eine Lösung, die das Optimum sein könnte, 
sehr nahe dran ist oder komplett daneben liegt. Ändert sich die Ausgangslage, muss der Algorithmus
erneut ausgeführt werden, um eine neue mögliche Lösung zu finden.

\begin{table}
   \centering
   \caption{Woran unterscheiden sich die beiden Prinzipien?}
   \begin{tabularx}{\textwidth}{|L|X|X|}
      \hline
       & Analysis 
       & Genetischer Algorithmus 
       \\ \hline
      Ziel
       & Wie im Abschnitt Funktionale \ref{buch:variation:problem:subsection:funktionale} und
       im Abschnittn \ref{buch:paper:varalg:section:variations_analysis_algorithm_result}
      beschrieben, ist das Ziel der Variationsrechnung die Optimierung von Funktionalen, aus der
      Ableitung erhählt man am Ende eine Funktion, mit welcher das Optimum berechnet werden kann. 
      Ändern sich nun die Gegebenheiten, wie z.B. der Startpunkt liegt neu höher, werden die Parameter 
      angepasst und die Funktion neu ausgerechnet, ohne die Ableitung noch einmal zu machen.
       & Im Algorithmus bedeuten die Variation, dass es eine Menge möglicher Lösungen gibt, 
      aus denen die besten ausgewählt und weiterverarbeitet werden, in der 
      Hoffnung, dass die neuen Lösungen besser sind. Die erhaltene Lösung ist ein Endresultat, für
      genau die gestellte Problemstellung. Ändert sich die Ausgangslage, muss der Algorithmus
      erneut ausgeführt werden, um eine neue mögliche Lösung zu finden. Ein weiteres Ziel ist es,
      ein genügend gutes Ergebnis zu erhalten, mit einer annehmbaren Laufzeit.
      \\ \hline
      Techniken  
       & Hier werden analytische Techniken wie die Euler-Lagrange-Gleichung verwendet, 
      um Optimierungsprobleme zu lösen. Die verschiedene Techniken sind ab diesem Kapitel
      \ref{buch:chapter:variation} beschrieben. Aus den Techniken erhalten wir eine Funktion,
      mit welcher das Optimum berechnet werden kann.
       & Im Algorithmus werden Mechanismen verwendet, die stochastische
      \footnote{
         Im Fall des Algorithmus sind stochastische Methoden gemeint, bei denen 
         eine Anzahl an Zufallsereignissen oder -kombinationen erstellt und 
         diese anschließend ausgewertet oder weiterverarbeitet werden.
      }
      Methoden wie Kreuzung und Mutation beinhalten, um Vielfalt zu erzeugen und aufrechtzuerhalten.
      Die Techniken sollen die Wahrscheinlickeit erhöhen, dass die besten Lösungen gefunden werden.
      Zusätzlich sollen sie den Zeit- und Speicheraufwand reduzieren.
      Die Techniken sind im Abschnitt 3 \ref{buch:paper:varalg:section:genetic_algorithm_process}
      \\ \hline
      Nachbarlösungen
       & Es wird das Konzept der Richtungsableitung in Richtung einer Funktion \(f(x)\)
       genutzt. Diese Funktion kann durch kleine Änderungen des Parameters \(x\) beliebig
       nahe beeinanderliegen. Aufgrund dieser Eigenschaft kann man die Ableitung nach dem
       Parameter \(x\) bilden, woraus sich die Möglichkeit ergibt, die Euler-Lagrange-Differentialgleichung
       abzuleiten. Das bedeutet, dass die Nachbarlösungen beliebig nahe sein können.  
       & Bei den Nachbarlösungen im Algorithmus werden die Variationen direkt untersucht,
      gekreuzt und mutiert, um die besten Lösungen zu finden. Jede Nachbarlösung 
      ist eine endliche Distanz von der aktuellen Lösung entfernt und kann nicht beliebig 
      nahe sein. Da so keine kontinuierlich Variation möglich ist, gibt es keine Ableitungen
      und es müssen Methoden dazu genutzt werden, wie im Abschnitt 
      \ref{buch:paper:varalg:section:genetic_algorithm_process} beschrieben.
      \\ \hline
      Lösung
       & Die Lösung ist eine Funktion, mit der das Optimum errechnet werden kann.
       & Die Lösung am Ende könnte das Optimum sein oder nur sehr nah dran.
      \\ \hline
      Änderung der Parameter
       & Die Ableitung muss nicht noch einmal durchfeführt werden, sondern man berechnet die 
       Funktion mit den neuen Parametern, um das Optimum zu erhalten.
       & Ändern sich die Parameter, muss der Algorithmus erneut ausgeführt werden, um eine
      neue mögliche Lösung zu finden.
      \\ \hline
   \end{tabularx}
   \label{tab:variation_comparison}
\end{table}

