%
% teil3.tex -- Beispiel-File für Teil 3
%
% (c) 2020 Prof Dr Andreas Müller, Hochschule Rapperswil
%
% !TEX root = ../../buch.tex
% !TEX encoding = UTF-8
%
\subsection{Ersetzen
\label{varalgbuch:paper:varalg:subsection:replacement}}
\index{Ersetzen}%
Der Ersetzungsschritt macht, was er aussagt. Meistens wird die ganze 
Population durch die neue ersetzt, sodass mit einer neuen Generation gearbeitet wird.
Es gibt aber auch die Möglichkeit, einen Teil zu erhalten. Dazu setzt man eine 
Grenze fest und dieser wird in die neue Population übertragen. Man erhofft sich so, 
dass aus den besten Lösungen beider Generationen noch bessere entstehen.
Wichtig ist, dass die Gesamtgrösse der Population gleich bleibt.

\subsubsection{Ersetzen auf das TSP angepasst
\label{buch:paper:varalg:subsection:replacement_tsp}}
Dieser Schritt kann auf das TSP angewendet werden, ohne weitere
Anpassungen vorzunehmen.
