%
% teil3.tex -- Beispiel-File für Teil 3
%
% (c) 2020 Prof Dr Andreas Müller, Hochschule Rapperswil
%
% !TEX root = ../../buch.tex
% !TEX encoding = UTF-8
%
\subsection{Ersetzen
\label{varalgbuch:paper:varalg:subsection:replacement}}
\rhead{Ersetzen}
Der Ersetz schritt macht was er aussagt. Meistens wird die ganze 
Population durch die neue Ersetzt. Je nach Strategie macht es Sinn, dass 
die besten Population behalten werden. Wichtig ist das man die gesammt 
grösse der Generation nicht vergrössert, also von den neuen nur die besten 
halten bis der Satz voll ist.

Der Ersetzungsschritt macht genau das, was der Name impliziert. In der 
Regel wird die gesamte Population durch die neue ersetzt. Je nach 
Strategie kann es jedoch sinnvoll sein, die besten Individuen der 
alten Population zu behalten. Wichtig ist, dass die Gesamtgröße der 
Generation nicht vergrößert oder verkleinert wird. Es werden  
nur die besten Individuen der neuen Generation behalten, bis die 
ursprüngliche Populationsgröße wieder erreicht ist.
