%
% 2_erweiterung_nD.tex -- Erweiterung der eindimensionalen Idee auf n Dimensionen.
%
% (c) 2024 Flurin Brechbühler, OST - Ostschweizer Fachhochschule Rapperswil
%
% !TEX root = ../../buch.tex
% !TEX encoding = UTF-8
%
\section{Erweiterung nD\label{fem:erweiterung_nD}}
\kopfrechts{Erweiterung nD}
Um das in einer Dimension erarbeitete Prinzip auf höhere Dimensionen auszuweiten, muss zum einen stat über eine Länge über den n-Dimensionalen Raum $\Omega$ integriert werden -
aus 
\begin{equation}
    \int_{l} u(x) \cdot v(x) \diff x
\end{equation}
wird also
\begin{equation}
    \int_{\Omega} u(\vec{x}) \cdot v(\vec{x}) \diff \vec{x}.
\end{equation}

Zum zum anderen werden die Ableitungen durch Gradienten ersetzt - 
\begin{equation}
    \int_{l} f'(x) \cdot v'(x) \diff x
\end{equation}
wird zu
\begin{equation}
    \int_{\Omega} \Delta f(\vec{x}) \cdot \Delta v(\vec{x}) \diff \vec{x}.
\end{equation}


\subsection{Matrix befüllen}
Das zu lösende Problem wird erneut diskretisiert, es werden also einzelne Punkte aus dem Definitionsbereich ausgewählt.
Diese Punkte werden Elemente genannt und jeweils mit den nächstgelegenen Nachbaren verbunden und nummeriert. 
Bei der Nummerierung der Elemente sollte darauf geachtet werden, dass benachbarte Elemente auch in der Nummerierung benachbart sind, da so die Matrix "möglichst diagonal" wird. % TODO: Tönt schlecht... Gibt sicher ein Name für dieses Problem -> Fussnote
Es sollte ein Bild ähnlich des zweidimensionalen Beispiels in Abbildung TODO: \ref{} resultieren. % TODO: Bild einfügen

Zum befüllen der Matrizen wird, analog zum eindimensionalen Fall, jeder Eintrag mit der zugehörigen Formel ausgewertet.
Für $\mathbf{L}$ gilt
\begin{equation}
    l_{ij} = \int_{\Omega} a_i(\vec{x}) \cdot a_j(\vec{x}) \diff \vec{x},
\end{equation}
während für $\mathbf{M}$ 
\begin{equation}
    m_{ij} = \int_{\Omega} \Delta a_i(\vec{x}) \cdot \Delta a_j(\vec{x}) \diff \vec{x},
\end{equation}
eingesetzt werden muss.

Es werden auch hier nur die Einträge ungleich null sein, deren Elemente sich berühren.
Der Eintrag in der Zeile $a$ und Spalte $b$ ist also nur dann ungleich Null, wenn Element $a$ und Elelment $b$ im Graphen TODO: \ref{} verbunden sind.
Werden also nun die Matrizen für den skizzierten Graphen befüllt, so sind all die Elemente undleich null, die in TODO: \ref{} schwarz markiert sind.


\subsection{Interpolationsfunktionen in höheren Dimensionen}
Meist wird das zu evaluierende Gebiet in Dreiecke, Tetraeder oder höherdimensionale euklidische Simplexe unterteilt. 
Entlang den Dreieckskanten werden die selben Formfunktionen, die im Kapitel \ref{fem:1d:diskretisieren} beschrieben wurden, verwendet.
Das Gebiet zwischen zwei Elementverbindungen wird wie in der Grafik TODO: \ref{} ausgefüllt.
