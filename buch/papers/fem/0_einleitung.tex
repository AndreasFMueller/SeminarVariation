%
% 0_einleitung.tex -- Einleitung zur Methode der Finiten Elemente
%
% (c) 2024 Flurin Brechbühler, OST - Ostschweizer Fachhochschule Rapperswil
%
% !TEX root = ../../buch.tex
% !TEX encoding = UTF-8
%
%\section{Einleitung\label{fem:section:einleitung}}
\newcommand{\myforall}{\quad \forall}
%\newcommand{\diff}{\ \differential}
\newcommand{\diff}{\, d}


\kopfrechts{Einleitung}

Differenzialgleichungen sind mit Computern eher schwer zu lösen. 
Dies ist speziell auch bei mehrdimensionalen Problemen mit komplexen Randbedingungen --- wie sie bei Minimalproblemen oft auftreten --- der Fall. 
Zum Lösen solcher Probleme werden oftmals numerische Methoden eingesetzt. 
Das Rayleigh-Ritz-Verfahren, wie es im Kapitel \ref{antennen:ritzAnw} angewandt wird, ist eine solche Methode.
Sie beschränkt dabei die Menge der möglichen Lösungsfunktionen, um das unendlich-dimensionale Problem auf ein endlich dimensionales herunterzubrechen, sodass zur Lösung Computer eingesetzt werden können. 

Die Methode der finiten Elemente, auch FEM (aus dem Englischen {\em Finite Element Method}), ist eine weitere numerische Methode.
Sie begrenzt jedoch nicht die Menge der berücksichtigten Lösungsfunktionen, sondern stellt eine Lösung aus mehreren kleinen Funktionsstücken --- den Elementen --- zusammen.
Weiter ist interessant, dass die FEM nicht nur beim Lösen von Minimalproblemen oft Anwendung findet, sondern man bei deren Herleitung Nutzen aus der Kenntnis der ersten Variation ziehen kann. 

In diesem Paper wird die Methode der finiten Elemente anhand eines eindimensionalen Problems hergeleitet. 
Anschliessend soll gezeigt werden, wie dieser Ansatz auf mehrere Dimensionen ausgeweitet werden könnte. 
Die Methode wird dann angewandt, um die Differenzialgleichung eines einfachen Pendels zu lösen. 
Zum Abschluss sollen kurz einige Gebiete, in denen die Methode der finiten Elemente zur Anwendung kommt, vorgestellt werden.
