%
% linear_interpoliert.tex -- Linear interpolierte Funktion
%
% (c) 2024 Flurin Brechbühler
%
\documentclass[tikz]{standalone}
\usepackage{amsmath}
\usepackage{times}
\usepackage{txfonts}
\usepackage{pgfplots}
\usepackage{csvsimple}

\usetikzlibrary{arrows,intersections,math}
\definecolor{darkred}{rgb}{0.8,0,0}
\definecolor{darkpurple}{rgb}{0.6,0,0.6}
\definecolor{darkblue}{rgb}{0,0,0.8}
\definecolor{darkgreen}{rgb}{0,0.6,0}
\definecolor{darkyellow}{rgb}{0.2,0.4,0}
\definecolor{darkorange}{rgb}{0.9,0.4,0}
\definecolor{darkgrey}{rgb}{0.3,0.3,0.3}

\begin{document}
\def\skala{0.65}

\begin{tikzpicture}[>=latex,thick,scale=\skala]
    % Beispielfunktion
    \def\bspFkt(#1){
        (sin(deg(1.4*(#1))) + (#1)/3 + 1/3)
    }
    \def\bspFktDiff(#1){
        (1.4*cos(deg(1.4*(#1))) + 1/3)
    }

    \def\interpFkt(#1){
        % a_0
        and(#1 >= 0-1, #1 < 0) * (\bspFkt(0)/2*((1-2*(\x-0))*((\x-0)+1)^2)) +
        and(#1 >= 0, #1 < 0+1) * (\bspFkt(0)/2*((1+2*(\x-0))*(1-(\x-0))^2)) +
        % a_1
        and(#1 >= 1-1, #1 < 1) * (\bspFkt(1)/2*((1-2*(\x-1))*((\x-1)+1)^2)) +
        and(#1 >= 1, #1 < 1+1) * (\bspFkt(1)/2*((1+2*(\x-1))*(1-(\x-1))^2)) +
        % a_2
        and(#1 >= 2-1, #1 < 2) * (\bspFkt(2)/2*((1-2*(\x-2))*((\x-2)+1)^2)) +
        and(#1 >= 2, #1 < 2+1) * (\bspFkt(2)/2*((1+2*(\x-2))*(1-(\x-2))^2)) +
        % a_3
        and(#1 >= 3-1, #1 < 3) * (\bspFkt(3)/2*((1-2*(\x-3))*((\x-3)+1)^2)) +
        and(#1 >= 3, #1 < 3+1) * (\bspFkt(3)/2*((1+2*(\x-3))*(1-(\x-3))^2)) +
        % a_4
        and(#1 >= 4-1, #1 < 4) * (\bspFkt(4)/2*((1-2*(\x-4))*((\x-4)+1)^2)) +
        and(#1 >= 4, #1 < 4+1) * (\bspFkt(4)/2*((1+2*(\x-4))*(1-(\x-4))^2)) +
        %% a_\n
        %and(#1 >= \n-1, #1 < \n) * (\bspFkt(\n)/2*((1-2*(\x-\n))*((\x-\n)+1)^2)) +
        %and(#1 >= \n, #1 < \n+1) * (\bspFkt(\n)/2*((1+2*(\x-\n))*(1-(\x-\n))^2)) +
        % b_0
        and(#1 >= 0-1, #1 < 0) * (\bspFktDiff(0)/2*((\x-0)*((\x-0)+1)^2)) +
        and(#1 >= 0, #1 < 0+1) * (\bspFktDiff(0)/2*((\x-0)*(1-(\x-0))^2)) +
        % b_1
        and(#1 >= 1-1, #1 < 1) * (\bspFktDiff(1)/2*((\x-1)*((\x-1)+1)^2)) +
        and(#1 >= 1, #1 < 1+1) * (\bspFktDiff(1)/2*((\x-1)*(1-(\x-1))^2)) +
        % b_2
        and(#1 >= 2-1, #1 < 2) * (\bspFktDiff(2)/2*((\x-2)*((\x-2)+1)^2)) +
        and(#1 >= 2, #1 < 2+1) * (\bspFktDiff(2)/2*((\x-2)*(1-(\x-2))^2)) +
        % b_3
        and(#1 >= 3-1, #1 < 3) * (\bspFktDiff(3)/2*((\x-3)*((\x-3)+1)^2)) +
        and(#1 >= 3, #1 < 3+1) * (\bspFktDiff(3)/2*((\x-3)*(1-(\x-3))^2)) +
        % b_4
        and(#1 >= 4-1, #1 < 4) * (\bspFktDiff(4)/2*((\x-4)*((\x-4)+1)^2)) +
        and(#1 >= 4, #1 < 4+1) * (\bspFktDiff(4)/2*((\x-4)*(1-(\x-4))^2)) +
        %% b_\n
        %and(#1 >= \n-1, #1 < \n) * (\bspFktDiff(\n)/2*((\x-\n)*((\x-\n)+1)^2)) +
        %and(#1 >= \n, #1 < \n+1) * (\bspFktDiff(\n)/2*((\x-\n)*(1-(\x-\n))^2)) +
        0
    }

    % Plots
        % Beispielfunktion
    \draw[color=darkgrey,line width=1.4pt] 
        plot[domain=0:4, scale=4, smooth]({\x}, {\bspFkt(\x)/2});
    \node at (2cm, 3cm) [color=darkgrey] {$f(x)$};
    
        % Interpolierte Funktion
    \draw[color=darkred,line width=1.4pt] 
        plot[domain=0:4, scale=4, samples=200]({\x}, {\interpFkt(\x)/2});
    \node at (5cm,2cm) [color=darkred] {$f_{\text{kubisch}}(x) = \sum_{n=0}^{4}a_{n}(x)$};
    \node at (5cm,1.2cm) [color=darkred] {$+ \sum_{n=0}^{4}b_{n}(x)$};

        % 100 * Fehler
    %\draw[color=darkgreen,line width=1.4pt] 
    %    plot[domain=0:4, scale=4, samples=200]({\x}, {(\interpFkt(\x)-\bspFkt(\x))/2});
    %\node at (5cm,-1cm) [color=darkgreen] {$100 \cdot (f_{\text{kubisch}}(x) - f(x))$};

    % x-Achse
    \draw[->] (-0.1,0) -- (16.4,0) coordinate[label={$x$}];
    \foreach \x in {0,...,4}{
        \draw (\x*4,-0.1) -- (\x*4,0.1);
        \node at (\x*4,0) [below] {$x_\x$};
    }

    % y-Achse
    \draw[->] (0,{-0.1}) -- (0,{4.3})
    coordinate[label={right:$y$}];
    \node at (0,0) [left] {$0$};
    \draw (-0.1,4) -- (0.1,4);
    \node at (0,4) [left] {$3$};
\end{tikzpicture}

\end{document}