%
% linear_interpoliert.tex -- Linear interpolierte Funktion
%
% (c) 2024 Flurin Brechbühler
%
\documentclass[tikz]{standalone}
\usepackage{amsmath}
\usepackage{times}
\usepackage{txfonts}
\usepackage{pgfplots}
\usepackage{csvsimple}

\usetikzlibrary{arrows,intersections,math}
\definecolor{darkred}{rgb}{0.8,0,0}
\definecolor{darkpurple}{rgb}{0.6,0,0.6}
\definecolor{darkblue}{rgb}{0,0,0.8}
\definecolor{darkgreen}{rgb}{0,0.6,0}
\definecolor{darkyellow}{rgb}{0.2,0.4,0}
\definecolor{darkorange}{rgb}{0.9,0.4,0}
\definecolor{darkgrey}{rgb}{0.3,0.3,0.3}

\begin{document}
\def\skala{0.65}

\begin{tikzpicture}[>=latex,thick,scale=\skala]
    % Beispielfunktion
    \def\bspFkt(#1){
        (sin(deg(1.4*(#1))) + (#1)/3 + 1/3)
    }

    % Plots
        % Beispielfunktion
    \draw[color=darkgrey,line width=1.4pt] 
        plot[domain=0:4, scale=4, smooth]({\x}, {\bspFkt(\x)/2});
    \node at (2cm, 3cm) [color=darkgrey] {$f(x)$};
    
        % Interpolierte Funktion
    \foreach \x in {1, 2, 3, 4}{
        \draw[color=darkred,line width=1.4pt] (\x*4-4, {2*\bspFkt(\x-1)}) -- (\x*4, {2*\bspFkt(\x)});
    }
    \node at (5cm,2cm) [color=darkred] {$f_{linear}(x) = \sum_{n=0}^{4}a_n(x)$};

    % x-Achse
    \draw[->] (-0.1,0) -- (16.4,0) coordinate[label={$x$}];
    \foreach \x in {0,...,4}{
        \draw (\x*4,-0.1) -- (\x*4,0.1);
        \node at (\x*4,0) [below] {$x_\x$};
    }

    % y-Achse
    \draw[->] (0,{-0.1}) -- (0,{4.3})
    coordinate[label={right:$y$}];
    \node at (0,0) [left] {$0$};
    \draw (-0.1,4) -- (0.1,4);
    \node at (0,4) [left] {$3$};
\end{tikzpicture}

\end{document}