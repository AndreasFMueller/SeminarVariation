%
% linear_fromfkt_skaliert.tex -- Lineare Formfunktionen skaliert
%
% (c) 2024 Flurin Brechbühler
%
\documentclass[tikz]{standalone}
\usepackage{amsmath}
\usepackage{times}
\usepackage{txfonts}
\usepackage{pgfplots}
\usepackage{csvsimple}

\usetikzlibrary{arrows,intersections,math}
\usetikzlibrary{shapes.misc, positioning}

\definecolor{darkred}{rgb}{0.8,0,0}
\definecolor{darkpurple}{rgb}{0.6,0,0.6}
\definecolor{darkblue}{rgb}{0,0,0.8}
\definecolor{darkgreen}{rgb}{0,0.6,0}
\definecolor{darkyellow}{rgb}{0.2,0.4,0}
\definecolor{darkorange}{rgb}{0.9,0.4,0}
\definecolor{darkgrey}{rgb}{0.3,0.3,0.3}

\begin{document}
\def\skala{0.65}

\begin{tikzpicture}[>=latex,thick,scale=\skala]
    % Beispielfunktion
    \def\bspFkt(#1){
        (sin(deg(1.4*(#1))) + (#1)/3 + 1/3)
    }

    % Plots
        % Beispielfunktion
    \draw[color=darkgrey,line width=1.4pt] 
        plot[domain=0:4, scale=2, smooth]({\x}, {\bspFkt(\x)});
    \draw[color=darkgrey]
        (3,{2*\bspFkt(1.5)}) -- (4, 4)
        node [above right, color=darkgrey] {$f(x)$};
    
        % Formfunktionen
    \foreach \x/\col in {1/darkpurple, 2/darkblue, 3/darkgreen, 4/darkorange}{
        \draw[color=\col,line width=1.4pt] (0, 0) -- (\x*2-2,0) -- (\x*2, {2*\bspFkt(\x)});
    }
    \foreach \x/\col in {0/darkred, 1/darkpurple, 2/darkblue, 3/darkgreen}{
        \draw[color=\col,line width=1.4pt] (8, 0) -- (\x*2+2,0) -- (\x*2, {2*\bspFkt(\x)});
    }
    

    % x-Achse
    \draw[->] (-0.1,0) -- (8.4,0) coordinate[label={$x$}];
    \foreach \x in {0,...,4}{
        \draw (\x*2,-0.1) -- (\x*2,0.1);
        \node at (\x*2,0) [below] {$x_\x$};
    }

    % y-Achse
    \draw[->] (0,{-0.1}) -- (0,{4.5})
    coordinate[label={right:$y$}];
    \node at (0,0) [left] {$0$};
    \draw (-0.1,4) -- (0.1,4);
    \node at (0,4) [left] {$3$};


    % Beschriftungen der Formfunktionen
    \foreach \x/\col in {0/darkred, 1/darkpurple, 2/darkblue, 3/darkgreen, 4/darkorange}{
        \node at (\x*2, {2*\bspFkt(\x)}) [yshift = 0.3cm, color=\col, inner sep=1pt, fill=white, draw, rounded rectangle, stroke=none] {${a_\x(x)}$};
    }
\end{tikzpicture}

\end{document}