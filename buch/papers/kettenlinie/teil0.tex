%
% einleitung.tex -- Beispiel-File für die Einleitung
%
% (c) 2020 Prof Dr Andreas Müller, Hochschule Rapperswil
%
% !TEX root = ../../buch.tex
% !TEX encoding = UTF-8
%
\section{Einleitung\label{kettenlinie:section:Einleitung}}
\rhead{Einleitung}
Die Kettenlinie, auch als Katenoide bekannt, beschreibt die natürliche Form einer idealen, homogenen und flexiblen Kette, die unter ihrem eigenen Gewicht hängt und an beiden Enden befestigt ist.
Diese Kurve hat in der Mathematik und Physik eine besondere Bedeutung, da sie als Lösung eines Variationsproblems auftritt und die Minimalfläche zwischen zwei Punkten darstellt.
Die Kettenlinie stellt ein Variationsproblem dar, weil sie die Form einer Kurve beschreibt, die eine bestimmte physikalische Bedingung erfüllt: die Minimierung der potenziellen Energie. 

