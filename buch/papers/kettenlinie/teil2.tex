%
% teil2.tex -- Beispiel-File für teil2 
%
% (c) 2020 Prof Dr Andreas Müller, Hochschule Rapperswil
%
% !TEX root = ../../buch.tex
% !TEX encoding = UTF-8
%
\section{Herleitung der Lagrange-Funktionen\label{kettenlinie:section:Herleitung der Lagrange-Funktionen}}
\kopfrechts{Herleitung der Lagrange-Funktionen}
Nachdem die theoretischen Grundlagen und Eigenschaften der Kettenlinie behandelt wurden, geht es nun um die konkreten Berechnungen.
In diesem Abschnitt werden die wesentlichen Formeln und deren Herleitungen vorgestellt, die zur Bestimmung der Kettenlinie notwendig sind.
Ziel ist es, den mathematischen Prozess verständlich darzustellen und die Anwendung der Formeln an konkreten Beispielen zu demonstrieren.

\subsection{Kettenlinien als Variationsproblem mit Nebenbedingung
\label{kettenlinie:subsection:Kettenlinien als Variationsproblem mit Nebenbedingung}}
Die Kettenlinie beschreibt die Form einer hängenden Kette, die unter ihrem eigenen Gewicht hängt und dabei eine Form annimmt, welche die potentielle Energie minimiert.
Eine der wesentlichen Bedingungen ist die vorgegebene Länge der Kette.
Diese Anforderung der konstanten Kettenlänge stellt die Nebenbedingung für das Variationsproblem dar.

Man erkennt daraus, dass es sich um ein klassisches Beispiel für ein Variationsproblem handelt.
In unserem Problem suchen wir nämlich nach einer Funktion, in diesem Fall die Form der Kette mit vorgegebener Länge, die einen Integral, hier die potentielle Energie, minimiert.

\subsection{Bogenlänge
\label{kettenlinie:subsection:Bogenlänge}}
Wie wir nun wissen, soll die potentielle Energie minimiert werden.
Es wird nun aber kein Objekt als Ganzes betrachtet, sondern die Summe der potentiellen Energie aller Teilchen der Kette zusammen.
\index{Energie!potentiell}%
Deshalb wird für die kommenden Berechnungen öfters die Bogenlänge benötigt, weshalb sie hier nochmals erläutert und ins Gedächtnis gerufen wird.
\index{Bogenlänge}%

Bei einem unendlich kleinen Teil einer beliebigen Kurve ist die Bogenlänge ungefähr
\begin{equation}
	ds
	=
	\sqrt{dx^2 + dy^2}
	\label{kettenlinie:equation1},
\end{equation}
was dem Pythagoras entspricht.
\index{Pythagoras}%
Hier kann man \(dx^2\) ausklammern, damit man auf die Formel
\begin{equation}
	ds
	=
	\sqrt{dx^2 + dy^2}
	=
	\sqrt{dx^2 \biggl( 1 + \biggl( \frac{dy}{dx} \biggr)^2 \biggr)}
	=
	\sqrt{1 + \biggl( \frac{dy}{dx} \biggr)^2} \, dx
\end{equation}
kommt.
In der Folge kann man \(\frac{dy}{dx}\) durch die Ableitung von \(y(x)\) ersetzen.
\(y\) ist die Form des Bogens, die wir suchen.
Bei unendlich kleinen Teilen ergibt sich dann der Term
\begin{equation}
	ds
	=
	\sqrt{1 + y'^2} \, dx.
\end{equation}
Um die Gesamtlänge der Kurve zu berechnen, summiert man die Längen der unendlich kleinen Abschnitte entlang der Kurve auf.
Dieser Prozess wird durch Integration erreicht, bei der jeder kleine Abschnitt durch das Differential \(ds\) repräsentiert wird.
Das Integral
\begin{equation}
	l
	=
	\int_{x_1}^{x_2} \sqrt{1 + y'^2} \, dx
\end{equation}
ermöglicht es, die Gesamtlänge der Kurve von einem Startpunkt \(x_1\) bis zu einem Endpunkt \(x_2\) zu bestimmen.

\subsection{Minimum der potentiellen Energie
\label{kettenlinie:subsection:Minimum der potentiellen Energie}}
Für die Herleitung der Kettenlinie brauchen wir zudem die Formel der potentiellen Energie für die Minimierung. Diese setzt sich zusammen aus der Masse \(m\), der Erdbeschleunigung \(g\) und der Höhe \(h\):
\begin{equation}
	E_{\text{pot}}
	=
	mgh.
\end{equation}
Auch benötigen wir die Formel 
\begin{equation}
	l
	=
	\int_{x_1}^{x_2} \sqrt{1 + y'^2} \, dx
\end{equation}
der Bogenlänge, welche wir eben erläutert haben.
Um die Nebenbedingung für die Kettenlänge zu erfüllen, brauchen wir eine zweite Lagrange-Funktion.
Dafür benötigen wir die Formel
\begin{equation}
	m = \mu \cdot ds = \mu \sqrt{1 + y'^2} \, dx
\end{equation}
für die Masse eines Kettenstücks, wobei \(\mu\) die Massendichte ist.
\index{Massendichte}%
Die Höhe des Kettenstücks ist definiert durch
\begin{equation}
	h = y(x).
\end{equation}
Die potentielle Energie des Kettenstücks ist dann:
\begin{equation}
	dE_{\text{pot}} = m \cdot g \cdot h = \mu \sqrt{1 + y'^2} \cdot g \cdot y(x) \, dx.
\end{equation}
Wie bereits erwähnt, setzt sich die potentielle Energie aus der Summe der potentiellen Energien aller Teilchen der Kette zusammen:
\begin{equation}
	E_{\text{pot}} = \int dE_{\text{pot}} = \int_{x_1}^{x_2} \mu g \sqrt{1 + y'^2} y  \, dx.
\end{equation}
Die Konstanten \(\mu\) und \(g\) können ignoriert werden.
Wir erhalten
\begin{equation}
	\int_{x_1}^{x_2} \sqrt{1 + y'^2} \, y \, dx.
\end{equation}
Diese muss extremal sein, in unserem Fall minimal.

\subsection{Einsatz der Euler-Lagrange-Gleichung
\label{kettenlinie:subsection:Einsatz der Euler-Lagrange-Gleichung}}
In diesem Fall kommt nun die Euler-Lagrange-Gleichung zum Einsatz.
Wir rufen Sie hier nochmals ins Gedächtnis:
\begin{equation}
	\int_{x_1}^{x_2} L(x, y, y') \, dx \, \text{ist extremal}
\quad\Leftrightarrow\quad
\frac{\partial L}{\partial y}(x, y, y') - \frac{d}{dx} \frac{\partial L}{\partial y'}(x, y, y') = 0.
\end{equation}
Zuerst muss die Lagrange-Funktion \(L\) bestimmt werden, welche in diesem Fall durch
\begin{equation}
	L(x, y, y')
	=
	y \sqrt{1+y'^2} + \lambda \sqrt{1+y'^2}
	=
	(y + \lambda) \sqrt{1+y'^2}
\end{equation}
gegeben ist.
Darin ist \(\lambda\) der Lagrange-Multiplikator, der zur Berücksichtigung der Nebenbedingung nötig ist.

Nun setzen wir die Lagrange-Funktion in die Euler-Lagrange-Gleichung ein, um die gewünschte Differentialgleichung zu erhalten. Es muss also folgende Differentialgleichung gelten:

\begin{align*}
	&                 &
	\frac{\partial}{\partial y} \biggl( \sqrt{1 + y'^2} \, (y + \lambda) \biggr) - \frac{d}{dx} \frac{\partial}{\partial y'} \biggl( \sqrt{1 + y'^2} \, (y + \lambda) \biggr) 
	&=
	0
	\\
	& \Leftrightarrow &
	\sqrt{1 + y'^2} - \frac{d}{dx} (y + \lambda) \dfrac{y'}{\sqrt{1 + y'^2}}
	&=
	0
	\\
	& \Leftrightarrow &
	\sqrt{1 + y'^2} - \frac{y'^2}{\sqrt{1 + y'^2}} - (y + \lambda) \frac{d}{dx} \frac{y'}{\sqrt{1 + y'^2}}
	&=
	0
	\\
	& \Leftrightarrow &
	\frac{1 + y'^2 - y'^2}{\sqrt{1 + y'^2}} - (y + \lambda) \frac{y'' \sqrt{1 + y'^2} - y' \frac{y' y''}{\sqrt{1 + y'^2}}}{(1 + y'^2)} 
	&=
	0
	\\
	& \Leftrightarrow &
	\frac{1}{\sqrt{1 + y'^2}} - (y + \lambda) \frac{y'' (1 + y'^2) - y''y'^2}{(1 + y'^2)^ \frac{3}{2}} 
	&=
	0
	\\
	& \Leftrightarrow &
	\frac{1}{\sqrt{1 + y'^2}} - (y + \lambda) \frac{y''}{\sqrt{1 + y'^2}^3} 
	&=
	0
	\\
	& \Leftrightarrow &
	\frac{1 + y'^2 - (y + \lambda) y''} {\sqrt{1 + y'^2}^3}
	&=
	0
	\\
	& \Rightarrow &
	1 + y'^2 - (y + \lambda) y''
	&=
	0 \label{kettenlinie:equation_1} \tag{\theequation} \stepcounter{equation}.
\end{align*}

\subsection{Lösung der Differentialgleichung
\label{kettenlinie:subsection:Lösung der Differentialgleichung}}
Wir haben in \eqref{kettenlinie:equation_1} die Differentialgleichung
\begin{equation}
	1 + y'^2 - (y + \lambda) y''
	=
	0
\end{equation}
bekommen.
Hier ist \(\lambda\) die Vertikalverschiebung, die wir in die Funktion \(y\) integrieren können:
\begin{equation}
	y
	\rightarrow
	y - \lambda ,
\end{equation}
wobei die Differentialgleichung gleich bleibt.

Die Differentialgleichung
\begin{equation}
	1 + y'^2 - y y''
	=
	0
\end{equation}
lässt sich entsprechend umstellen und man erhält
\begin{equation}
	1
	=
	y y'' - y^2.
\end{equation}
Diese kann man nun ohne Rechnen lösen, wenn man weiss, dass die Hyperbelfunktionen die Gleichungen
\index{Hyperbelfunktionen}%
\begin{align*}
	1 &= \cosh(x)^2 - \sinh(x)^2 \\
	\cosh'(x) &= \sinh(x) \\
	\cosh''(x) &= \cosh(x)
\end{align*}
erfüllen. Somit ist
\begin{equation}
	y
	=
	\cosh(x), \quad
	y'
	=
	\sinh(x)
\end{equation}
eine Lösung der Differentialgleichung.
\begin{equation}
	y(x)
	=
	a \cosh\left(\frac{x}{a}\right)
\end{equation}
erfüllt die Differentialgleichung ebenfalls. Ausserdem ist 
\begin{equation}
	y'(x)
	=
	\sinh\left(\frac{x}{a}\right)
\end{equation}
und
\begin{equation}
	y''(x)
	=
	\frac{1}{a} \cosh\left(\frac{x}{a}\right).
\end{equation}
Somit wurden alle Lösungen gefunden.

Nach der Herleitung der grundlegenden Differentialgleichung und der Gleichung der Kettenlinie, können wir nun die horizontale Verschiebung \(x_0\) und die vertikale Verschiebung \(y_0\) einführen, um die allgemeinste Form der Kettenlinie zu formulieren.
Diese erlaubt es, die Kurve entlang der \(x\)-Achse zu verschieben und vertikal zu justieren, was in praktischen Anwendungen wie dem Bau von Hängebrücken oder der Modellierung von Kabeln unter realen Bedingungen nützlich ist.

Die allgemeine Form der Kettenlinie lässt sich ausdrücken als:
\begin{equation}
	y(x)
	=
	a \cosh\left(\frac{x - x_0}{a}\right) + y_0.
\end{equation}
Hierbei ist:

\begin{itemize}
	\item \(a\) ein Skalierungsfaktor, der die Krümmung bestimmt,
	\item \(x_0\) die horizontale Verschiebung, die den Scheitelpunkt der Kurve bestimmt,
	\item \(y_0\) die vertikale Verschiebung, die die minimale Höhe der Kurve angibt.
\end{itemize}
