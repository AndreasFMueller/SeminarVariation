%
% teil2.tex -- Beispiel-File für teil2 
%
% (c) 2020 Prof Dr Andreas Müller, Hochschule Rapperswil
%
% !TEX root = ../../buch.tex
% !TEX encoding = UTF-8
%
\section{Herleitung der Funktionen\label{kettenlinie:section:Herleitung der Funktionen}}
\rhead{Herleitung der Funktionen}
Nachdem die theoretischen Grundlagen und Eigenschaften der Kettenlinie behandelt wurden, geht es nun um die konkreten Berechnungen.
In diesem Abschnitt werden die wesentlichen Formeln und deren Herleitungen vorgestellt, die zur Bestimmung der Kettenlinie notwendig sind.
Ziel ist es, den mathematischen Prozess verständlich darzustellen und die Anwendung der Formeln an konkreten Beispielen zu demonstrieren.

\subsection{Kettenlinien als Variationsproblem
\label{kettenlinie:subsection:Kettenlinien als Variationsproblem}}
Die Kettenlinie nimmt die Form einer hängenden Kette an, welche unter ihrem eigenen Gewicht hängt.
Dabei nimmt sie eine Form ein, die die potenzielle Energie minimiert.

Man erkennt daraus, dass es sich um ein klassisches Beispiel für ein Variationsproblem handelt.
In unserem Problem suchen wir nämlich nach einer Funktion, in diesem Fall die Form der Kette, die einen Integral, hier die potentielle Energie, minimiert.

\subsection{Bogenlänge
\label{kettenlinie:subsection:Bogenlänge}}
Wie wir nun wissen, soll die potenzielle Energie minimiert werden.
Es wird nun aber kein Objekt als Ganzes betrachtet, sondern die Summe der potenziellen Energie aller Teilchen der Kette zusammen.
Deshalb wird für die kommenden Berechnungen öfters die Bogenlänge benötigt, weshalb sie hier nochmals erläutert und ins Gedächtnis gerufen wird.

Bei einem unendlich kleinen Teil einer beliebigen Kurve ist die Bogenlänge ungefähr der Pythagoras.
\begin{equation}
	ds
	=
	\sqrt{dx^2 + dy^2}
	\label{kettenlinie:equation1}
\end{equation}
\(dx^2\) kann man ausklammern.
\begin{equation}
	ds
	=
	\sqrt{dx^2 + dy^2}
	=
	\sqrt{dx^2 \left( 1 + \left( \frac{dy}{dx} \right)^2 \right)}
	=
	\sqrt{1 + \left( \frac{dy}{dx} \right)^2} \, dx
\end{equation}
In der Folge kann man \(\frac{dy}{dx}\) mit der Ableitung von \(y\) ersetzen.
\(y\) ist die Form des Bogens, die wir suchen.
Bei unendlich kleinen Teilen ergibt sich dann folgender Term:
\begin{equation}
	ds
	=
	\sqrt{1 + y'^2} \, dx
\end{equation}
Um die Gesamtlänge der Kurve zu berechnen, summiert man die Längen der unendlich kleinen Abschnitte entlang der Kurve auf.
Dieser Prozess wird durch Integration erreicht, bei der jeder kleine Abschnitt durch das Differential \(ds\) repräsentiert wird.
Das Integral ermöglicht es, die Gesamtlänge der Kurve von einem Startpunkt \(x_1\) bis zu einem Endpunkt \(x_2\) zu bestimmen.
\begin{equation}
	L
	=
	\int_{x_1}^{x_2} \sqrt{1 + y'^2} \, dx
\end{equation}

\subsection{Minimum der potenziellen Energie
\label{kettenlinie:subsection:Minimum der potenziellen Energie}}
Für die Herleitung der Kettenlinie brauchen wir zudem die Formell der potentiellen Energie, welche sich wie folgt aus der Masse m, der Erdbeschleunigung g und der Höhe h zusammensetzt.
\begin{equation}
	E_{\text{pot}}
	=
	mgh
\end{equation}
Auch benötigen wir die Formell der Bogenlänge, welche wir eben erläutert haben.
\begin{equation}
	L
	=
	\int_{x_1}^{x_2} \sqrt{1 + y'^2} \, dx
\end{equation}
Wie bereits erwähnt, setzt sich die potenzielle Energie aus der Summe der pozentiellen Energien aller Teilchen der Kette zusammen, es folgt folgende Formel, wobei \(g\) die Gravitationskonstante und \(\mu\) die Massendichte ist.
\begin{equation}
	E_{\text{pot}}
	=
	\int_{a}^{b} g \mu \sqrt{1 + y'^2} \, y \, dx
\end{equation}
Die Konstanten \(\mu\) und \(g\) können ignoriert werden.
Wir erhalten den folgenden Term:
\begin{equation}
	\int_{a}^{b} (\sqrt{1 + y'^2} \, y) \, dx
\end{equation}
Dieser muss extremal sein, in unserem Fall minimal.

\subsection{Einsatz der Euler-Lagrange-Gleichung
\label{kettenlinie:subsection:Einsatz der Euler-Lagrange-Gleichung}}
In diesem Fall kommt nun die Euler-Lagrange-Gleichung zum Einsatz. Wir rufen Sie hier nochmals ins Gedächtnis.
\begin{equation}
	\int_{a}^{b} L(x, y, y') \, dx \, \text{ist extremal} \iff \frac{\partial L}{\partial y}(x, y, y') - \frac{d}{dx} \frac{\partial L}{\partial y'}(x, y, y') = 0
\end{equation}
Zuerst muss die Lagrange-Funktion \(L\) bestimmt werden.
\begin{equation}
	L(x, y, y')
	=
	\sqrt{1 + y'^2} \, y
\end{equation}
Nun setzen wir die Lagrange-Funktion in die Euler-Lagrange-Gleichung ein, um die gewünschte Differentialgleichung zu erhalten. Es muss also folgende Differentialgleichung gelten:
\begin{equation}
\frac{\partial}{\partial y} \left( \sqrt{1 + y'^2} \, y \right) - \frac{d}{dx} \frac{\partial}{\partial y'} \left( \sqrt{1 + y'^2} \, y \right) = 0
\end{equation} 