%
% 2_Multipol.tex -- Beispiel-File für teil2 
%
% (c) 2020 Prof Dr Andreas Müller, Hochschule Rapperswil
%
% !TEX root = ../../buch.tex
% !TEX encoding = UTF-8
%
\section{Multipolmoment
\label{planet:section:multipol}}
\rhead{Multipolmoment}
In der Physik ist die Multipolentwicklung ein Verfaheren um die Passion-Gleichung im Dreidimensionalen Raum zu lösen, die daraus entsteht eine Laurent-Reihe.
Die Entwicklungskoeffizienten dieser Reihe heissen Multipolmomente.
Diese Methode wird hauptsächlich in der Elektrostatik und Magnetostatik verwendet, kann aber auf jeder Gebiet der Physik verallgemeinert werden, in dem die Poisson-Gleichung vorhanden ist.

Die Motivation der Multipolentwicklung liegt darin, das Verhalten von beliebigen Potentialen (Gravitationspotential) in grosser Entfernung zu betrachten.

\subsection{Grundlagen
\label{planet:subsection:grundlagen}}

Die Poisson-Gleichung lässt sich allgemein als

\begin{equation}
\Delta \phi (\vec{r}) = - f (\vec{r})
\end{equation}

\noindent
schreiben, wobei \(\Delta\) der Laplace-Operator, \(f\) eine Dichte und \(\phi\) ein Potential ist (das Minus ist Konvention).
Die formale Lösung dieser Gleichung ist:

\begin{equation}
\phi (\vec{r}) = \frac{1}{4\pi} \int d^3 \vec{r}\,' \frac{f (\vec{r}\,')} {\left| \vec{r} - \vec{r}\,' \right|}
\end{equation}

\noindent
Ist \(f (\vec{r})\) in einem Volumen lokalisiert, kann für Orte \(\vec{r}\), die weit außerhalb dieses Volumens liegen, \(r \gg r\,'\), der Bruch in einer Taylor-Reihe in \(\vec{r}\,'\) um \(\vec{r}\,' = 0\) entwickelt werden:

\begin{equation}
\frac{1}{\left| \vec{r} - \vec{r}\,' \right|} = \sum_{n=0}^{\infty} \frac{1}{n!} \left( \vec{r}\,' \cdot \vec{\nabla}\,' \right)^n \left. \frac{1}{\left| \vec{r} - \vec{r}\,' \right|} \right|_{\vec{r}\,'=0}
\end{equation}

\noindent
Dabei bedeutet \(\vec{\nabla}\,'\), dass der Nablaoperator \(\vec{\nabla}\) nur auf die gestrichenen Koordinaten \(\vec{r}\,'\) und nicht auf \(\vec{r}\) wirkt.
Nach Bilden der Ableitungen wird diese an der Stelle \(\vec{r}\,' = 0\) ausgewertet.
Durch Umformen erhält man:

\begin{equation}
\frac{1}{\left| \vec{r} - \vec{r}\,' \right|} = \sum_{n=0}^{\infty} \frac{1}{n!} \left( - \vec{r}\,' \cdot \vec{\nabla} \right)^n \frac{1}{r}
\end{equation}

\noindent
Die genaue Form der Entwicklung und der Multipole hängt vom jeweiligen Koordinatensystem ab.

\subsection{Karthesische Multpolentwicklung
\label{planet:subsection:kartentwicklung}}

Die karthesische Multipolentwicklung wird im karthesischen Koordinatensystem durchgeführt.
Dort ist

\begin{equation}
\vec{r}\,' \cdot \vec{\nabla} = r\,'_{i} \partial_{i},
\end{equation}

\noindent
wobei Einsteinsche Summenkonvention verwendet wird.
Dann muss bei einem Summanden \(n\)-ter Ordnung ein Tensor \(n\)-ter Stufe, nämlich \(\textstyle \prod_{k=1}^{n} \partial_{i_{k}} \frac{1}{r}\) berechnet werden:

\begin{equation}
\begin{aligned}
\frac{1}{\left| \vec{r} - \vec{r}\,' \right|} &= \frac{1}{r} - r\,'_{i} \partial_{i} \frac{1}{r} + \frac{1}{2} r\,'_{i} r\,'_{j} \partial_{i} \partial_{j} \frac{1}{r} + \mathcal{O}(r\,'^{3}) \\
&= \frac{1}{r} + r\,'_{i} \frac{r_{i}}{r^{3}} + \frac{1}{2} r\,'_{i} r\,'_{j} \frac{3 r_{i} r_{j} - r^{2} \delta_{ij}}{r^{5}} + \mathcal{O}(r\,'^{3}).
\end{aligned}
\end{equation}

\noindent
Das Symbol \(\delta_{ij}\) repräsentiert das sogenannte Kronecker-Delta.

\noindent
Die formale Lösung \(\phi (\vec{r})\) der Poisson-Gleichung ist unter Verwendung der Identität \(r\,'_{i} r\,'_{j} r^{2} \delta_{ij} = r\,'^{2} r_{i} r_{j} \delta_{ij}\) wie folgt darstellbar:

\begin{equation}
\begin{aligned}
\phi (\vec{r}) &= \frac{1}{4\pi} \left[ \frac{1}{r} \underbrace{\int f (\vec{r}\,') \, d^3 \vec{r}\,'}_{\text{Monopol-}} + \frac{r_{i}}{r^{3}} \underbrace{\int d^3 \vec{r}\,' \, r\,'_{i} f (\vec{r}\,')}_{\text{Dipol-}} + \frac{1}{2} \frac{r_{i} r_{j}}{r^{5}} \underbrace{\int d^3 \vec{r}\,' \left( 3 r\,'_{i} r\,'_{j} - r\,'^{2} \delta_{ij} \right) f (\vec{r}\,')}_{\text{Quadrupolmoment}} + \dots \right] \\
&= \frac{1}{4\pi} \left[ \frac{1}{r} q + \frac{r_{i}}{r^{3}} p_{i} + \frac{1}{2} \frac{r_{i} r_{j}}{r^{5}} Q_{ij} + \dots \right]
\end{aligned}
\end{equation}

In der entwickelten Reihe kommen die Faktoren \(r\) im Nenner mit steigendem Exponenten.
Dabei werden die höheren Ordnungen der Multipolmomente, mit zuhnemendem Abstand zum beobachteten Volumen immer weiter vernachlässigbar.
Es wurde einfachheitshalber für die Berechnungen der Kugelfrom in diesem Kapitel als Approximation das erste Moment verwendet und die höheren vernachlässigt.



\subsection{Anwendung
\label{planet:subsection:anwendung}}

Die Gravitation hat im Gegensatz zum Elekrtomagnetismus nur positive Ladungen, die Massen.
Es können dennoch gravitative Multipole definiert werden.
Aus der Poisson-Gleichung des Newtonschen Gravitationsgesetzes

\begin{equation}
    \Delta \Phi = -4\pi G\rho
\end{equation}

\noindent
Mit der Gravitationskonstante \(G\) und der Massendichte \(\rho\) ist der gravitative Monopol die Gesamtmasse \(M\).

Da Massen nur positiv sind lässt sich, aus den Ladungen kein Dipolmoment bilden.
Aus diesem Grund ist der gravitative Quadrupol nicht über zwei Dipole definiert.
Dennoch haben Massen verteilungen Quadrupole.
\cref{Quadroplanet} zeigt ein solches wobei, die blau gefärbten Abweichungen, das fehlen der Masse als negative Masse gesehen wird und die rot gefärbten Abweichungen, die überschüssigen Massen als positive Massen betrachtet wird.  

\beispiel{Die Erde ist durch ihre Rotation keine perfekte Kugel, sie ist leicht eingedellt an den Polen und am Äquator bildet sich ein Wulst \cref{Quadroplanet}.
Desweiteren sind durch bestimmte Dichteunterschiede weitere Abweichungen der Kugelfrom, in \cref{geoid} ersichtlich.}

Einige himmelsmechanische Phänomene, wie die dynamische Entwicklung der Bahnelemente von Satelliten, lassen sich mit dem daraus resultierenden Quadrupolmoment beschreiben.
