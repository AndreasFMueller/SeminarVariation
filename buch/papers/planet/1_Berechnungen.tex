%
% teil1.tex -- Beispiel-File für das Paper
%
% (c) 2020 Prof Dr Andreas Müller, Hochschule Rapperswil
%
% !TEX root = ../../buch.tex
% !TEX encoding = UTF-8
%
\section{Herleitung hydrostatische Gleichgewichtsform
\label{planet:section:teil1}}
\rhead{Problemstellung}
Damit die hydrostatische Gleichgewichtsform einfacher berechnet werden kann, werden ein paar Annahmen getroffen.
\begin{itemize}
	\item Unser Himmelskörper besteht aus einer idealen Flüssigkeit, das heisst sie ist Inkompressibel und hat keine innere Reibung.
	\item Die Massenverteilungsdichte \(\sigma\) für 2D und die Dichte \(\rho\) für 3D der Flüssigkeit sind konstant.
	Dies ist eine Idealisierung der Bedingung 2 für die Definition eines Planeten.
\end{itemize}

Konkret sieht unsere Problemstellung folgendermassen aus.
\begin{itemize}
	\item Die potenzielle Energie \(U\) soll minimal werden.
	\item Die Fläche \(A\) bzw. das Volumen \(V\) ist gegeben.
	\item Da man eine runde Form erwartet, werden die Berechnungen in Polarkoordinaten bzw. Kugelkoordinaten durchgeführt. Somit suchen wir die Funktion \(r(\varphi)\) in 2D und \(r(\varphi,\theta)\) in 3D.
\end{itemize}

\subsection{Gravitation}

Die Kraft, welche Teilchen auf andere Teilchen ausüben, nennt man Gravitation.
Sie ist eine der vier Grundkräfte der Physik.
Die Gravitation kann man sich als Feld vorstellen.
Für die Berechnung der Kraft, welche auf ein Teilchen im Gravitationsfeld wirkt, benötigt man das Gravitationspotential \(\Phi\).
Es ist definiert als
\begin{equation}
	\Phi(r) = -\frac{GM}{r}.
	\label{planet:equ:gravpot}
\end{equation}
\(G\) ist die Gravitationskonstante \((G \approx 6.6743 \cdot 10^{-11} \text{m}^3 \text{kg}^{-1} \text{s}^{-2})\).
\(M\) ist die Masse des Körpers, welcher das Gravitationsfeld erzeugt, \(r\) der Abstand vom Ursprung.

Damit man die Kraft, welche auf ein Teilchen wirkt, berechnen kann, benötigt man den Gradienten des Gravitaionspotenzials, welcher definiert ist als
\begin{equation*}
	\nabla \Phi (r) = \frac{\partial \Phi}{\partial r} = \frac{\partial}{\partial r} \biggl( -\frac{GM}{r} \biggr) = \frac{GM}{r^2}.
\end{equation*}

Um nun die Kraft auf ein Teilchen zu berechnen muss man den Gradient des Gravitationspotenzials mit der Masse des Teilchens multiplizieren.
Somit lautet die Formel für die Berechnung der Kraft welche auf ein Teilchen wirkt
\begin{equation*}
	F = -m\nabla \Phi = -\frac{GMm}{\vec{r}^2}.
\end{equation*}
Die Kraft ist negativ, da sie zum Zentrum der Masse zeigen muss.


\subsection{Hydrostatische Gleichgewichtsform in 2D}
\begin{figure}
	\centering
	\includegraphics{papers/planet/pictures/Flächenelement.pdf}
	\caption{Visualisierung Polarkoordinaten}
	\label{planet:fig:2d}
\end{figure}
Für die Berechnung der hydrostatische Gleichgewichtsform soll nun die potenzielle Energie \(U\) minimal werden.
Für die Berechnung der Fläche in 2D muss man wissen, dass \(dA = r \, dr \, d\varphi\).
Dies sieht man in Abbildung \ref{planet:fig:2d}.
Als Nebenbedingung hat man die Fläche \(A\), welche mit 
\begin{equation}
	A = \int_{A}^{} dA
	\label{planet:equ:A}
\end{equation}
berechnet wird.
Die Formel für die Berechung der potenziellen Energie ist
\begin{equation}
	U = \int_{A} \sigma \, \Phi (r) \, dA.
	\label{planet:equ:U}
\end{equation}
Wenn man nun Gleichung \eqref{planet:equ:gravpot} in Gleichung \eqref{planet:equ:U} ein erhält man
\begin{equation*}
	U = \int_{0}^{2\pi}\int_{0}^{r(\varphi)} \frac{-GM}{r} \, \sigma r \, dr \, d\varphi.
\end{equation*}
Das erste Integral kann man auflösen und die Konstante aus dem Integral heraus nehmen, es resultiert 
\begin{equation}
	U =-GM\sigma \int_{0}^{2\pi} r(\varphi) \, d\varphi .
\end{equation}
Aus dieser Gleichung entnimmt man die Lagrange-Funktion
\begin{equation}
	L(\varphi ,r,r' = r(\varphi).
\end{equation}
Für die zweite Lagrange-Funktion schreiben wir die Gleichung der Nebenbedingung \eqref{planet:equ:A} aus zu
\begin{equation*}
	A = \int_{0}^{2\pi}\int_{0}^{r(\varphi)} r \, dr \, d\varphi.
\end{equation*}
Das innere Integral kann man auflösen, sodass die Gleichung zu
\begin{equation*}
	A = \int_{0}^{2\pi}\frac{r(\varphi)^2}{2} \, d\varphi
\end{equation*}
vereinfacht wird.
Aus diesem Integral entnimmt man die Lagrange-Funktion der Nebenbedingung
\begin{equation*}
	N(\varphi ,r,r' = \frac{r^2}{2}.
\end{equation*}

Um die gesuchte Lösung zu finden, nutzt man die Methode der Lagrange-Muliplikatoren.
Genaueres über die Methode der Lagrange-Muliplikatoren findet man in Kapitel \ref{buch:nebenbedingungen:lagrangemult:subsection:einzeln}.
Die wichtigsten Punkte, welche man für die Lösung dieser Problemstellung braucht, sind, dass man die Euler-Lagrange-Differentialgleichung als Gradienten ansehen kann.
Mit diesen Gradienten kann man anschliessend ein Gleichungssystem aufstellen und die gesuchte Funktion finden.
Damit man die gesuchte Funktion finden kann, muss man den Gradienten bzw. die Euler-Lagrange-Differentialgleichung der jeweiligen Lagrange-Funktionen berechnen und anschliessend in
\begin{equation}
	\nabla N \cdot \lambda = \nabla L
	\label{planet:equ:lamdagleichung}
\end{equation}
einsetzen.
Für den "'Gradienten"' der Lagrange-Funktionen benötigt man die jeweiligen partiellen Ableitungen
\begin{equation*}
	L(\varphi ,r(\varphi),r(\varphi)_\varphi) = r(\varphi),
	\frac{\partial L}{\partial r} = 1,
	\frac{\partial L}{\partial r'} = 0,
\end{equation*}
und
\begin{equation*}
	N(\varphi ,r(\varphi),r(\varphi)_\varphi) = \frac{r^2}{2} ,
	\frac{\partial N}{\partial r} = r,
	\frac{\partial N}{\partial r'} = 0.
\end{equation*}

Die "'Gradienten"' der Lagrange-Funktionen sind definiert als 
\begin{equation*}
	\nabla L = 
	\frac{\partial L}{\partial r}-  \frac{d}{d\varphi}\left( \frac{\partial L}{\partial r} \right)
\end{equation*}
und
\begin{equation*}
	\nabla N = \frac{\partial N}{\partial r} - \frac{d}{d\varphi}\left(\frac{\partial N}{\partial r'}\right).
\end{equation*}

Wenn man nun die Ableitungen der Lagrange-Funktionen in die "'Gradienten"' einsetzen
\begin{equation*}
	\left(\frac{\partial N}{\partial r} - \frac{d}{d\varphi}\left(\frac{\partial N}{\partial r'}\right)\right)\cdot \lambda = \frac{\partial L}{\partial r}-  \frac{d}{d\varphi}\left( \frac{\partial L}{\partial r'} \right)
\end{equation*} 
und anschliessend das Ganze in die Gleichung \eqref{planet:equ:lamdagleichung}
einsetzt und erhält man
\begin{equation*}
	r = \frac{1}{\lambda}.
\end{equation*}
Was sagt nun dieses Ergebnis?
Da \(\lambda\) ist eine Konstatnte muss \(r\) ebenfalls eine Konstante sein.
Da die Berechungen in Polarkoordinaten durchgeführt wurden, bedeutet ein konstanter Radius, dass die gesuchte Form ein Kreis ist.

\subsection{Hydrostatische Gleichgewichtsform in 3D}
\begin{figure}
	\centering
	\includegraphics{papers/planet/pictures/Flächenelement_spherical.pdf}
	\caption{Visualisierung Kugelkoordinaten}
	\label{planet:fig:3d}
\end{figure}
Für die Berechnung im 3-dimensionalen Raum geht man ähnlich vor wie im 2-dimensionalen Raum.
Der Unterschied ist, dass man eine weitere Variable \(\theta\) habt.
Da man nun mehr als eine Variable hat, muss man die Aufgabe mit der Euler-Ostrogadski-Differentialgleichung lösen.
Für die Berechnung der hydrostatische Gleichgewichtsform soll nun wieder die potenzielle Energie \(U\) minimal werden.
Da die Berechnungen in 3D durchgeführt werden, benötigt man für die Berechung die Formel für ein Volumenteilchen \(dV = r^2 \sin (\theta) \, dr \, d\theta \, d\varphi \).
Zu sehen ist dies in Abbildung \ref{planet:fig:3d}
Als Nebenbedingung hat man in 3D das Volumen \(V\) welches mit 
\begin{equation}
	V = \int_{V}^{} dV
	\label{planet:equ:V}
\end{equation}
berechnet wird.
Die Formel für die Berechnung der potenziellen Energie ist
\begin{equation}
	U = \int_{V} \sigma \,  \Phi (r)\, dV.
	\label{planet:equ:U3d}
\end{equation}
Nun setzt man wiederrum die Gleichung \eqref{planet:equ:gravpot} in Gleichung \eqref{planet:equ:U3d} erhält man
\begin{equation*}
	U = \int_{0}^{2\pi}
	\int_{0}^{\pi}
	\int_{0}^{r(\varphi,\theta)}
	\frac{-GM}{r}\, \sigma\, r^2 \sin (\theta) \,
	dr \, d\theta \, d\varphi.
\end{equation*}
Mit ein wenig Umstellen und dem Auflösen des ersten Integrals erhält man
\begin{equation*}
	U =-GM\sigma \int_{0}^{2\pi}\int_{0}^{\pi}\frac{r(\varphi,\theta)^2}{2}  \sin (\theta) \, d\theta \, d\varphi.
\end{equation*}
Wiederrum entnimmt man aus diesem Interal die Lagrange-Funktion
\begin{equation*}
	L(\varphi,\theta ,r,r_\varphi,r_\theta) = \frac{r^2}{2}  \sin (\theta).
\end{equation*}
Mit dem Wissen, dass \(dV = r^2 \sin (\theta) \, dr \, d\theta \, d\varphi \) kann man \eqref{planet:equ:V} als
\begin{equation*}
	V = \int_{0}^{2\pi}\int_{0}^{\pi}\int_{0}^{r(\varphi,\theta)} r^2 \sin (\theta) \, dr \, d\theta \, d\varphi
\end{equation*}
ausschreiben.
Hier löst man wiederrum das innere Integral, auf schreibt die Konstanten vor das Integral und erhält
\begin{equation*}
	V = \int_{0}^{2\pi}\int_{0}^{\pi}\frac{r(\varphi,\theta)^3}{3} \sin (\theta) \, d\theta \, d\varphi.
\end{equation*}
Aus dieser Gleichung entnimmt man die Lagrange-Funktion der Nebenbedingung
\begin{equation*}
	N(\varphi,\theta ,r,r_\varphi,r_\theta) = \frac{r^3}{3} \sin (\theta).
\end{equation*}
Wie in 2D findet man die gesuchte Funktion wenn man die Gleichung \eqref{planet:equ:lamdagleichung} löst.
Da man in der Berechung im 3D zwei Variablen hat (\(\varphi,\Theta\)), muss man die partiellen Ableitungen nach beiden Variablen berechnen.
Konkret benutzt man die Euler-Ostrogradski-Differentialgleichung als Gradient wenn die gesuchte Funktion von mehr als einer Varbiable abhängig ist.
Die Ableitungen sehen folgendermassen aus:
\begin{align*}
	\frac{\partial L}{\partial r} &= r  \sin (\theta) \\
	\frac{\partial L}{\partial r_\varphi} &= 0 \\
	\frac{\partial L}{\partial r_\theta} &= 0 \\
\intertext{und}
	\frac{\partial N}{\partial r} &= r^2\sin (\theta) \\
	\frac{\partial N}{\partial r_\varphi} &= 0 \\
	\frac{\partial N}{\partial r_\theta} &= 0
\end{align*}
Glücklicherweise werden viele der Ableitungen 0.
Die "'Gradienten"' sind definiert als
\begin{equation*}
	\nabla L =  \frac{\partial L}{\partial r} 
	-\frac{d}{d\varphi}\left( \frac{\partial L}{\partial r_\varphi} \right)
	-\frac{d}{d\theta}\left( \frac{\partial L}{\partial r_\theta} \right)
\end{equation*}
und
\begin{equation*}
	\nabla N=  \frac{\partial N}{\partial r} 
	-\frac{d}{d\varphi}\left( \frac{\partial N}{\partial r_\varphi} \right)
	-\frac{d}{d\theta}\left( \frac{\partial N}{\partial r_\theta} \right).
\end{equation*}
Somit können wir die "'Gradienten"' in die Gleichung \eqref{planet:equ:lamdagleichung} einsetzen und erhalten
\begin{equation*}
	r^2\sin (\theta) \cdot \lambda = r \sin (\theta).
\end{equation*}
Wenn man diese Gleichung weiter vereinfacht erhält man
\begin{equation*}
	r = \frac{1}{\lambda}.
\end{equation*}
Man sieht, das Ergebniss ist das selbe wie man in 2D erhält.
In 3D bedeutet ein Konstanter Radius, das die gesuchte Form eine Kugel ist.

\subsection{Interpretation der Ergebnisse}
Sowohl in 2D als auch in 3D erhält man das Ergebniss, dass die Form in welcher die potenzielle Energie eines Planeten minimiert wird, ein Kreis bzw. eine Kugel ist.
Diese Feststellung wird durch die Realität bestätigt, in der grössere Himmelskörper, welche ihre starren Körperkräfte überwinden, kugelähnliche Formen annehmen.

