%
% 0_Einleitung.tex
%
% (c) 2020 Prof Dr Andreas Müller, Hochschule Rapperswil
%
% !TEX root = ../../buch.tex
% !TEX encoding = UTF-8
%
\section{Einleitung\label{planet:section:einleitung}}
\rhead{Einleitung}
Was ist eigentlich ein Planet?
Wie unterscheidet man einen Planeten von einem Zwergplaneten?
Genau diese Fragen stellten sich Wissenschaftler im Jahr 2006 \cite{planet:iaub5}.
Die Definition sieht folgendermassen aus:
Ein Planet ist ein Himmelskörper welcher
\begin{enumerate}
	\item einen Orbit um eine Sonne hat,
	\item ausreichende Masse hat, damit seine Eigengravitation die Kräfte eines starren Körpers überwinden kann, sodass er eine hydrostatische Gleichgewichtsform (nahezu rund) annimmt, und
	\item die Umgebung seiner Umlaufbahn freigeräumt hat.
\end{enumerate}

\noindent
In diesem Kapitel wird die hydrostatische Gleichgewichtsform behandelt.
Sie wird in 2D und in 3D hergeleitet.


