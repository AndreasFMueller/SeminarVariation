%
% einleitung.tex -- Beispiel-File für die Einleitung
%
% (c) 2020 Prof Dr Andreas Müller, Hochschule Rapperswil
%
% !TEX root = ../../buch.tex
% !TEX encoding = UTF-8
%
\section{Teil 0\label{planet:section:teil0}}
\rhead{Teil 0}
Was ist eigentlich ein Planet?
Wie Unterscheidet man einen Planet von einem Zwergplanet?
Genau diese Fragen stellten sich Wissenschaftler im Jahr 2006 \cite{planet:iaub5}.
Die Definition auf welche sich geeinigt wurde sieht folgendermassen aus.
Ein Planet ist ein Himmelskörper welcher:
\begin{itemize}
	\item einen Orbit um eine Sonne hat,
	\item hat ausreichende Masse, damit seine Eigengravitation die Kräfte eines starren Körpers überwinden kann, sodass es eine hydrostatische Gleichgewichtsform (nahezu rund) annimmt, und
	\item hat die Umgebung seiner Umlaufbahn freigeräumt.
\end{itemize}

In diesem Kapitel wird sich mit der hydrostatische Gleichgewichtsform befasst.
Sie wird in 2D und in 3D hergeleitet


