%
% teil1.tex -- Beispiel-File für das Paper
%
% (c) 2020 Prof Dr Andreas Müller, Hochschule Rapperswil
%
% !TEX root = ../../buch.tex
% !TEX encoding = UTF-8
%
\section{Herleitung hydrostatische Gleichgewichtsform
\label{planet:section:teil1}}
\rhead{Problemstellung}
Damit man die hydrostatische Gleichgewichtsform einfacher berechnen kann werden ein paar Annahmen getroffen.
\begin{itemize}
	\item Unser Himmelskörper besteht aus einer idealen Flüssigkeit, das heisst sie ist Inkompressibel und hat keine inneren Reibungen.
	\item Die Massenverteilungsdichte \(\sigma\) und die Dichte \(\rho\) der flüssigkeit sind konstant.	
\end{itemize}

Konkret sieht unsere Problemstellung folgendermassen aus.
\begin{itemize}
	\item Die potenzielle Energie \(U\) soll Minimal werden.
	\item Die Fläsche \(A\) bzw. das Volumen \(V\) ist gegeben.
	\item Da man eine runde Form erwartet, werden die Berechnungen in Polarkoordinaten bzw. Kugelkoordinaten durchgeführt. Somit suchen wir die Funktion \(r(\phi)\) in 2D und \(r(\phi,\theta)\) in 3D.
\end{itemize}

\subsection{Gravitation}

Die Kraft welche Teilchen auf andere Teilchen ausüben nennt man Gravitation.
Sie ist eine der vier Grundkräfte der Physik.
Die Gravitation kann man sich als Feld vorstellen.
Für die berechnung der Kraft, welche auf ein Teilchen im Gravitationsfeld wirkt benötigen man das Gravitaionspotential \(\Phi\).
Das Gravitaionspotenzial ist definiert als
\begin{equation}
	\Phi(r) = -\frac{GM}{r}.
	\label{planet:equ:gravpot}
\end{equation}
\(G\) ist die Gravitationskonstante \((G \approx 6.6743 \cdot 10^{-11} m^3 kg^-1 s^-2)\).
\(M\) ist die Masse des Körper, welcher das Gravitaionsfeld erzeugt.
\(r\) der Abstand vom Ursprung.

Damit man die Kraft welche auf ein Teilchen wirkt berechnen kann benötigt man den Gradienten des Gravitaionspotenzials, welcher definiert ist als
\begin{equation*}
	\nabla \Phi (r) = \frac{\partial \phi}{\partial r} = \frac{\partial}{\partial r} (-\frac{GM}{r^2}) = \frac{GM}{r^2}
\end{equation*}

Um nun die Kraft auf ein Teilchen zu berechnen muss man den Gradient des Gravitaionspotenzials mit der Masse des Teilchens multiplizieren.
\begin{equation*}
	F = -m\nabla \phi = -\frac{GMm}{r^2}
\end{equation*}
Die Kraft ist negativ, da sie zum Zentrum der Masse zeigen muss.


\subsection{Hydrostatische Gleichgewichtsform in 2-D}
\begin{figure}
	\centering
	\includegraphics{papers/planet/pictures/Flächenelement.pdf}
	\caption{Visualisierung Polarkoordinaten}
\end{figure}
Für die Berechnung der hydrostatische Gleichgewichtsform soll nun die potenzielle Energie \(U\) minimal werden.
Für die Berechnung in 2-D muss man wissen, dass \(dA = r \, dr \, d\phi\).
Als nebenbedingung hat man die Fläche \(A\) welche mit 
\begin{equation}
	A = \int_{A}^{} dA
	\label{planet:equ:A}
\end{equation}
berechnet wird.
Die Formel für die Berechung der potenziellen Energie ist
\begin{equation*}
	U = \int_{A} \sigma  \Phi (r) \, dA.
	\label{planet:equ:U}
\end{equation*}
Wenn man nun Gleichung \ref{planet:equ:gravpot} in Gleichung \ref{planet:equ:U} ein erhält man
\begin{equation*}
	U = \int_{0}^{2\pi}\int_{0}^{r} \frac{-GM}{r} \, \sigma r \, dr \, d\varphi.
\end{equation*}
Wenn man nun das erste Integral auflösen und die Konstante aus dem Integral raus nehmen resultiert 
\begin{equation}
	U =-GM\sigma \int_{0}^{2\pi} r(\varphi) \, d\varphi .
\end{equation}
Aus dieser Gleichung entnimmt man das erste Funktional
\begin{equation}
	L(\varphi ,r(\varphi),r(\varphi)_\varphi) = r(\varphi).
\end{equation}
Für das zweite Funktional schreiben wir die Nebenbedigung aus.
\begin{equation*}
	A = \int_{0}^{2\pi}\int_{0}^{r} r \, dr \, d\varphi
\end{equation*}
Das innere Integral kann man auflösen, sodass die Gleichung zu
\begin{equation*}
	A = \int_{0}^{2\pi}\frac{r^2}{2} \, d\varphi
\end{equation*}
vereinfacht wird.
Aus diesem Integral entnimmt man das Funktional der Nebenbedingung
\begin{equation*}
	N(\varphi ,r(\varphi),r(\varphi)_\varphi) = \frac{r^2}{2}.
\end{equation*}

Damit man die Gesuchte Funktion finden muss man den Gradienten des jewiligen Funktionals berechnen und anschliessend in
\begin{equation}
	\nabla N \cdot \lambda = \nabla L
	\label{planet:equ:lamdagleichung}
\end{equation}
einsetzen.
Für den Gradienten der Funktionale benötigt man die jeweiligen partiellen Ableitungen
\begin{equation*}
	L(\varphi ,r(\varphi),r(\varphi)_\varphi) = r(\varphi),
	\frac{\partial L}{\partial r(\varphi)} = 1,
	\frac{\partial L}{\partial r(\varphi)_\varphi} = 0,
\end{equation*}
und
\begin{equation*}
	N(\varphi ,r(\varphi),r(\varphi)_\varphi) = \frac{r^2}{2} ,
	\frac{\partial N}{\partial r(\varphi)} = r,
	\frac{\partial N}{\partial r(\varphi)_\varphi} = 0.
\end{equation*}

Die Gradienten der Funtionale sind definiert als 
\begin{equation*}
	\nabla L = 
	\frac{\partial L}{\partial r(\varphi)}-  \frac{d}{d\varphi}\left( \frac{\partial L}{\partial r(\varphi)} \right)
\end{equation*}
und
\begin{equation*}
	\nabla N = \frac{\partial N}{\partial r(\varphi)} - \frac{d}{d\varphi}\left(\frac{\partial N}{\partial r(\varphi)_\varphi}\right).
\end{equation*}

Wenn man nun die Ableitungen der Funktionale in die Gradienten einsetzen
\begin{equation*}
	\left(\frac{\partial N}{\partial r(\varphi)} - \frac{d}{d\varphi}\left(\frac{\partial N}{\partial r(\varphi)_\varphi}\right)\right)\cdot \lambda = \frac{\partial L}{\partial r(\varphi)}-  \frac{d}{d\varphi}\left( \frac{\partial L}{\partial r(\varphi)} \right)
\end{equation*} 
und anschliessend das ganze in die Gleichung \ref{planet:equ:lamdagleichung}
einsetzt und erhält man
\begin{equation*}
	r = \frac{1}{\lambda}.
\end{equation*}
Was sagt nun dieses Ergebniss.
\(\lambda\) ist eine Konstatnte.
Folglich muss \(r\) ebenfalls eine Konstante sein.
Da die Berechungen in Polarkoordinaten durchgeführt wurden, bedeutet ein konstanter Raduis das die zu suchende Form ein Kreis ist.

\subsection{Hydrostatische Gleichgewichtsform in 3-D}
\begin{figure}
	\centering
	\includegraphics{papers/planet/pictures/Flächenelement_spherical.pdf}
	\caption{Visualisierung Kugelkoordinaten}
\end{figure}
Für die Berechnung im 3 - Dimensionalen Raum geht man ähnlich vor wie im 2 - Dimensionalen Raum.
Der Unterschied ist, das man eine weitere Variable \(\Theta\) habt.
Da man nun mehr als 1 Varbiable habt muss man die Aufgabe mit der Euler-Ostrogadski Diffrentialgleichung lösen.
Für die Berechnung der hydrostatische Gleichgewichtsform soll nun wieder die potenzielle Energie \(U\) minimal werden.
Da die Berechenungen in 3-D durchgeführt werden benötigt man für die Berechung die Formel für ein Volumenteilchen \(dV = r^2 \sin (\theta) \, dr \, d\theta \, d\varphi \).
Als Nebenbedingung hat man in 3-D das Volumen \(V\) welches mit 
\begin{equation}
	V = \int_{V}^{} dV
	\label{planet:equ:V}
\end{equation}
berechnet wird.
Die Formel für die Berechnung der potenziellen Energie ist
\begin{equation}
	U = \int_{V} \sigma  \phi (r)\, dV.
	\label{planet:equ:U3d}
\end{equation}
Nun setzt man wiederrum die Gleichung \ref{planet:equ:gravpot} in Gleichung \ref{planet:equ:U3d} erhält man
\begin{equation*}
	U = \int_{0}^{2\pi}\int_{0}^{\pi}\int_{0}^{r} \frac{-GM}{r}\, \sigma\, r^2 \sin (\theta) \, dr \, d\theta \, d\varphi.
\end{equation*}
Mit ein wenig umstelle und dem auflösen des ersten Integrals erhält man
\begin{equation*}
	U =-GM\sigma \int_{0}^{2\pi}\int_{0}^{\pi}\frac{r^2}{2}  \sin (\theta) \, d\theta \, d\varphi.
\end{equation*}
Wiederrum entnimmt man aus diesem Intergal das erste Funktional
\begin{equation*}
	L(\varphi,\theta ,r(\varphi,\theta),r(\varphi,\theta)_\varphi,r(\varphi,\theta)_\theta) = \frac{r^2}{2}  \sin (\theta).
\end{equation*}
Mit dem Wissen, dass \(dV = r^2 \sin (\theta) \, dr \, d\theta \, d\varphi \) kann man \ref{planet:equ:V} zu
\begin{equation*}
	V = \int_{0}^{2\pi}\int_{0}^{\pi}\int_{0}^{r} r^2 \sin (\theta) \, dr \, d\theta \, d\varphi.
\end{equation*}
ausschreiben.
Hier löst man wiederrum das innere Integral auf schreibt die Konstanten vor das Integral und erhält
\begin{equation*}
	V = \int_{0}^{2\pi}\int_{0}^{\pi}\frac{r^3}{3} \sin (\theta) \, d\theta \, d\varphi.
\end{equation*}
Aus dieser Gleichung entnimmt man das Funktional der Nebenbedingung
\begin{equation*}
	N(\varphi,\theta ,r(\varphi,\theta),r(\varphi,\theta)_\varphi,r(\varphi,\theta)_\theta) = \frac{r^3}{3} \sin (\theta).
\end{equation*}
Wie im 2 Dimensionale findet man die gesuchte Funktion wenn man die Gleichung \ref{planet:equ:lamdagleichung} löst.
Da man in der Berechung im 3 Dimensionalen 2 Variablan hat (\(\Phi,\Theta\)) muss man die partiellen Ableitungen nach beiden Variablen berechnen.
Diese sehen folgendermassen aus:
\begin{equation*}
	\frac{\partial L}{\partial r} = r  \sin (\theta) ,
	\frac{\partial L}{\partial r_\varphi} = 0 ,
	\frac{\partial L}{\partial r_\theta} = 0
\end{equation*}
und
\begin{equation*}
	\frac{\partial N}{\partial r} = r^2\sin (\theta) ,
	\frac{\partial N}{\partial r_\varphi} = 0 ,
	\frac{\partial N}{\partial r_\theta} = 0.
\end{equation*}
Glücklicherweise sind werden viele der Ableitungen zu 0.
Die Gradienten sind definiert als
\begin{equation*}
	\nabla L =  \frac{\partial L}{\partial r(\varphi)} 
	-\frac{d}{d\varphi}\left( \frac{\partial L}{\partial r_\varphi} \right)
	-\frac{d}{d\theta}\left( \frac{\partial L}{\partial r_\theta} \right)
\end{equation*}
und
\begin{equation*}
	\nabla N=  \frac{\partial N}{\partial r(\varphi)} 
	-\frac{d}{d\varphi}\left( \frac{\partial N}{\partial r_\varphi} \right)
	-\frac{d}{d\theta}\left( \frac{\partial N}{\partial r_\theta} \right).
\end{equation*}
Somit können wir die Gradienten in die Gleichung \ref{planet:equ:lamdagleichung} einsetzten und erhalten
\begin{equation*}
	r^2\sin (\theta) \cdot \lambda = r \sin (\theta).
\end{equation*}
Wenn man diese Gleichung weiter vereinfacht erhält man
\begin{equation*}
	r = \frac{1}{\lambda}.
\end{equation*}
Man sieht, das Ergebniss ist das selbe wie man in 2-D erhält.
In 3-D bedeutet ein Konstanter Radius, das die gesuchte Form eine Kugel ist.

\subsection{Interpretation der Ergebnisse}
Sowohl in 2-D als auch in 3-D erhält man das Ergebniss, dass die Form in welcher die Potenzielle Energie eines Planeten minimiert wird, ein Kreis bzw. eine Kugel ist.
Diese Feststellung wird durch die Realität bestätigt in der grössere Himmelskörper, welche ihre starren Körperkräfte überwinden, kugelähnliche Formen annehmen.

