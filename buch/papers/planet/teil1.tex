%
% teil1.tex -- Beispiel-File für das Paper
%
% (c) 2020 Prof Dr Andreas Müller, Hochschule Rapperswil
%
% !TEX root = ../../buch.tex
% !TEX encoding = UTF-8
%
\section{Teil 1
\label{planet:section:teil1}}
\rhead{Problemstellung}
Damit wir die hydrostatische Gleichgewichtsform werden ein paar annahmen getroffen.
\begin{itemize}
	\item Unser Himmelskörper besteht aus einer idealen Flüssigkeit, das heisst sie ist Inkompressibel und hat keine inneren Reibungen.
	\item Die Massenverteilungsdichte \(\sigma\) und die Dichte \(\rho\) der flüssigkeit sind konstant.	
\end{itemize}

Konkret sieht unsere Problemstellung folgendermassen aus.
\begin{itemize}
	\item Die potenzielle Energie \(U\) soll Minimal werden.
	\item Die Fläsche \(A\) bzw. das Volumen \(V\) ist gegeben.
	\item Da man eine runde Form erwarten werden die Berechnungen in Polarkoordinaten bzw. Kugelkoordinaten durchgeführt. Somit suchen wir die Funktion \(r(\phi)\) in 2D und \(r(\phi,\theta)\) ind 3D.
\end{itemize}

\subsection{Gravitation}

Die Kraft welche Teilchen auf andere Teilchen ausüben nennt man Gravitation.
Sie ist eine der vier Grundkräfte der Physik.
Die Gravitation kann man sich als Feld vorstellen.
Für die berechnung der Kraft, welche auf ein Teilchen im Gravitationsfeld wirk benötigen wir das Gravitaionspotential \(\Phi\).
Das Gravitaionspotenzial ist definiert als
\begin{equation}
	\Phi(r) = -\frac{GM}{r}.
	\label{planet:equ:gravpot}
\end{equation}
\(G\) ist die Gravitationskonstante \((G \approx 6.6743 \cdot 10^{-11} m^3 kg^-1 s^-2)\).
\(M\) ist die Masse des Körper, welcher das Gravitaionsfeld erzeugt.
\(r\) der Abstand vom Ursprung.

Damit man die Kraft welche auf ein Teilchen wirkt berechnen kann benötigt man den Gradienten des Gravitaionspotenzials, welcher definiert ist als
\begin{equation}
	\nabla \Phi (r) = \frac{\partial \phi}{\partial r} = \frac{\partial}{\partial r} (-\frac{GM}{r^2}) = \frac{GM}{r^2}
\end{equation}

Um nun die Kraft auf ein Teilchen zu berechnen muss man den Gradient des Gravitaionspotenzials mit der Masse des Teilchens multiplizieren.
\begin{equation}
	F = -m\nabla \phi = -\frac{GMm}{r^2}
\end{equation}
Die Kraft ist negativ, da sie zum Zentrum der Masse zeigen muss.


\subsection{Hydrostatische Gleichgewichtsform in 2-D}
Für die Berechnung der hydrostatische Gleichgewichtsform soll nun die potenzielle Energie \(U\) minimal werden.
Als nebenbedingung haben wir die Fläche \(A\) welche mit 
\begin{equation}
	A = \int_{A}^{} dA
	\label{planet:equ:A}
\end{equation}
berechnet wird.
Die Formel ür die Berechung der potenziellen Energie ist
\begin{equation*}
	U = \int_{A} \sigma  \Phi (r) \, dA.
	\label{planet:equ:U}
\end{equation*}
Wenn wir nun Gleichung \ref{planet:equ:gravpot} in Gleichung \ref{planet:equ:U} ein erhalten wir
\begin{equation*}
	U = \int_{0}^{2\pi}\int_{0}^{r} \frac{-GM}{r} \, \sigma r \, dr \, d\varphi.
\end{equation*}
Wenn wir nun das erste Integral auflösen und die Konstante aus dem Integral raus nehmen erhalten wir 
\begin{equation}
	U =-GM\sigma \int_{0}^{2\pi} r(\varphi) \, d\varphi .
\end{equation}
Aus diers Gleichung nehmen wir unser erstes Funktional
\begin{equation}
	L(\varphi ,r(\varphi),r(\varphi)_\varphi) = r(\varphi).
\end{equation}
Für das zweite Funktional muss man wissen, dass \(dA = r \, dr \, d\phi\).
Somit können wir das Integral der Gleichung \ref{planet:equ:A} einfügen und erhalten
\begin{equation*}
	A = \int_{0}^{2\pi}\int_{0}^{r} r \, dr \, d\varphi
\end{equation*}
Das innere Integral können wir auflösen, sodass das Integral zu
\begin{equation*}
	A = \int_{0}^{2\pi}\frac{r^2}{2} \, d\varphi
\end{equation*}
vereinfacht.
Aus diesem Integral entnehmen wir das Funktional der Nebenbedingung
\begin{equation*}
	N(\varphi ,r(\varphi),r(\varphi)_\varphi) = \frac{r^2}{2} 
\end{equation*}

Damit wir die Gesuchte Funktion finden müssen wir den Gradienten des Jewiligen Funktionals berechnen und anschliessend in
\begin{equation*}
	\nabla N \cdot \lambda = \nabla L
\end{equation*}
einsetzen.
Für den Gradienten der Funktionale benötigen wir die jeweiligen partiellen Ableitungen
\begin{equation*}
	L(\varphi ,r(\varphi),r(\varphi)_\varphi) = r(\varphi),
	\frac{\partial L}{\partial r(\varphi)} = 1,
	\frac{\partial L}{\partial r(\varphi)_\varphi} = 0
\end{equation*}









