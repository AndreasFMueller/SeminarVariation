%
% teil1.tex -- Beispiel-File für das Paper
%
% (c) 2020 Prof Dr Andreas Müller, Hochschule Rapperswil
%
% !TEX root = ../../buch.tex
% !TEX encoding = UTF-8
%
\section{Teil 1
\label{planet:section:teil1}}
\rhead{Problemstellung}
Damit wir die hydrostatische Gleichgewichtsform werden ein paar annahmen getroffen.
\begin{itemize}
	\item Unser Himmelskörper besteht aus einer idealen Flüssigkeit, das heisst sie ist Inkompressibel und hat keine inneren Reibungen.
	\item Die Massenverteilungsdichte \(\sigma\) und die Dichte \(\rho\) der flüssigkeit sind konstant.	
\end{itemize}

Konkret sieht unsere Problemstellung folgendermassen aus.
\begin{itemize}
	\item Die potenzielle Energie \(U\) soll Minimal werden.
	\item Die Fläsche \(A\) bzw. das Volumen \(V\) ist gegeben.
	\item Da man eine runde Form erwarten werden die Berechnungen in Polarkoordinaten bzw. Kugelkoordinaten durchgeführt. Somit suchen wir die Funktion \(r(\phi)\) in 2D und \(r(\phi,\theta)\) ind 3D.
\end{itemize}

Die Kraft welche Teilchen auf andere Teilchen ausüben nennt man Gravitation.
Die Gravitation kann man sich als Feld vorstellen.
Für die berechnung der Kraft, welche auf ein Teilchen im Gravitationsfeld wirk benötigen wir das Gravitaionspotential \(\Phi\).
Das Gravitaionspotenzial ist definiert als
\begin{equation}
	\Phi(r) = -\frac{GM}{\abs{r-r'}}.
\end{equation}
\(G\) ist die Gravitationskonstante.
\(M\) ist die Masse des Körper, welcher das Gravitaionsfeld erzeugt.
\(r\) der Ort ist, an dem das Potential berechnet wird.
\(r'\)ist der Ort der Masse.


Sed ut perspiciatis unde omnis iste natus error sit voluptatem
accusantium doloremque laudantium, totam rem aperiam, eaque ipsa
quae ab illo inventore veritatis et quasi architecto beatae vitae
dicta sunt explicabo.
Nemo enim ipsam voluptatem quia voluptas sit aspernatur aut odit
aut fugit, sed quia consequuntur magni dolores eos qui ratione
voluptatem sequi nesciunt
\begin{equation}
\int_a^b x^2\, dx
=
\left[ \frac13 x^3 \right]_a^b
=
\frac{b^3-a^3}3.
\label{planet:equation1}
\end{equation}
Neque porro quisquam est, qui dolorem ipsum quia dolor sit amet,
consectetur, adipisci velit, sed quia non numquam eius modi tempora
incidunt ut labore et dolore magnam aliquam quaerat voluptatem.

Ut enim ad minima veniam, quis nostrum exercitationem ullam corporis
suscipit laboriosam, nisi ut aliquid ex ea commodi consequatur?
Quis autem vel eum iure reprehenderit qui in ea voluptate velit
esse quam nihil molestiae consequatur, vel illum qui dolorem eum
fugiat quo voluptas nulla pariatur?

\subsection{De finibus bonorum et malorum
\label{planet:subsection:finibus}}
At vero eos et accusamus et iusto odio dignissimos ducimus qui
blanditiis praesentium voluptatum deleniti atque corrupti quos
dolores et quas molestias excepturi sint occaecati cupiditate non
provident, similique sunt in culpa qui officia deserunt mollitia
animi, id est laborum et dolorum fuga \eqref{planet:equation1}.

Et harum quidem rerum facilis est et expedita distinctio
\ref{planet:section:teil2}.
Nam libero tempore, cum soluta nobis est eligendi optio cumque nihil
impedit quo minus id quod maxime placeat facere possimus, omnis
voluptas assumenda est, omnis dolor repellendus
\ref{planet:section:teil3}.
Temporibus autem quibusdam et aut officiis debitis aut rerum
necessitatibus saepe eveniet ut et voluptates repudiandae sint et
molestiae non recusandae.
Itaque earum rerum hic tenetur a sapiente delectus, ut aut reiciendis
voluptatibus maiores alias consequatur aut perferendis doloribus
asperiores repellat.


