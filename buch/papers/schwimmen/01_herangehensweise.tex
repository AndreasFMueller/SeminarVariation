%
% einleitung.tex -- Beispiel-File für die Einleitung
%
% (c) 2020 Prof Dr Andreas Müller, Hochschule Rapperswil
%
% !TEX root = ../../buch.tex
% !TEX encoding = UTF-8
%
\section{Herangehensweise\label{schwimmen:section:teil0}}
\rhead{Herangehensweise}

Es soll ermittelt werden, welche die energieeffizienteste Methode ist, um den Fluss zu überqueren. Da die Energie \(E\) mittels der Formel \[E = P \cdot t\] berechnet werden kann, wobei \(P\) für die Leistung und \(t\) für die Zeit steht, stellt sich heraus, dass die Energie für die Flussüberquerung direkt von der Zeit abhängt. Ergo sollte es möglich sein, nicht die Energie zu optimieren, sondern die Zeit.

Dies vereinfacht vieles, allerdings muss nun die Leistung, mit der geschwommen werden kann, begrenzt werden. Eine schwimmende Person kann nicht unbegrenzt schnell schwimmen und somit nicht in keiner Zeit am anderen Ufer ankommen.

\subsection{Variationsprinzip}
Das Variationsprinzip ist ein Konzept, das besagt, dass die Natur in vielen Fällen den optimalen Weg wählt. In diesem Fall ist es das Prinzip der kleinsten Wirkung, also des kleinsten Energieaufwand, der für die Flussüberquerung benötigt wird. Die Euler-Lagrange-Differentialgleichung ist das Werkzeug, um das Variationsprinzip zu berechnen. Das Lagrange-Integral gibt das Minimum der Wirkung zurück.



% Es soll ermitelt werden was die energieefizientiste Methode ist um den Fluss zu überqueren. Da die Energie \(E\) mittels der Formel \[E=P\cdot t\] berechnet werden kann wobei \(P\) für die Leistung und \(t\) für die Zeit steht, stehlt sich heraus das die Energie für die Flussüberquerung direkt von der Zeit abhängig ist. Ergo sollte es möglich sein nicht eine Optimierung der Energie sondern eine der Zeit zu machen. 
% Das vereinfacht vieles, jetzt muss man aber die Leistung mit der man schwimmen kann beschränken. Die schwimmende Person kann nicht unbeschränkt schnell schwimmen und dadurch in keiner Zeit am anderen Ufer ankommen.

% \subsection{Variationsprinzip}
% Das Variationsprinzip ist ein Konzept, das besagt, dass die Natur in vielen Fällen den optimalen Weg wählt. In diesem Fall ist es das Prinzip der kleinsten Wrikung, der kleinste Energieaufwand der für die Flussüberquerung gebraucht wird.
% Die Lagrange Formel ist das Werkzueg um das Variationsprinzip zu berechnen. Das Lagrange-Integral gibt das minimum der Wirkung zurück.









% Lorem ipsum dolor sit amet, consetetur sadipscing elitr, sed diam
% nonumy eirmod tempor invidunt ut labore et dolore magna aliquyam
% erat, sed diam voluptua \cite{schwimmen:bibtex}.
% At vero eos et accusam et justo duo dolores et ea rebum.
% Stet clita kasd gubergren, no sea takimata sanctus est Lorem ipsum
% dolor sit amet.

% Lorem ipsum dolor sit amet, consetetur sadipscing elitr, sed diam
% nonumy eirmod tempor invidunt ut labore et dolore magna aliquyam
% erat, sed diam voluptua.
% At vero eos et accusam et justo duo dolores et ea rebum.  Stet clita
% kasd gubergren, no sea takimata sanctus est Lorem ipsum dolor sit
% amet.


