%
% o1_herangehensweise.tex -- Beispiel-File für die Einleitung
%
% (c) 2020 Prof Dr Andreas Müller, Hochschule Rapperswil
%
% !TEX root = ../../buch.tex
% !TEX encoding = UTF-8
%
\section{Herangehensweise\label{schwimmen:section:teil0}}
\kopfrechts{Herangehensweise}

Es soll ermittelt werden, welche die energieeffizienteste Methode
ist, um den Fluss zu überqueren. Da die Energie \(E\) mittels der
Formel \[E = P \cdot t\] berechnet werden kann, wobei \(P\) für die
Leistung und \(t\) für die Zeit steht, stellt sich heraus, dass die
Energie für die Flussüberquerung direkt von der Zeit abhängt. Ergo
sollte es möglich sein, nicht die Energie zu optimieren, sondern
die Zeit.

Dies vereinfacht vieles, allerdings muss nun die Leistung, mit der
geschwommen werden kann, begrenzt werden. Eine schwimmende Person
kann nicht unbegrenzt schnell schwimmen und somit nicht in beliebig
kurzer Zeit am anderen Ufer ankommen.

\subsection{Variationsprinzip}
Das Variationsprinzip ist ein Konzept, das besagt, dass die Natur
in vielen Fällen den optimalen Weg wählt. In diesem Fall ist es das
Prinzip der kleinsten Wirkung, also des kleinsten Energieaufwand,
der für die Flussüberquerung benötigt wird. Die
Euler-Lagrange-Differentialgleichung ist das Werkzeug, um das
Variationsprinzip zu berechnen. Das Integral der Lagrange-Funktion
gibt das Minimum der Wirkung zurück.


