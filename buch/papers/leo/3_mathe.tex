%
% teil2.tex -- Beispiel-File für teil2 
%
% (c) 2020 Prof Dr Andreas Müller, Hochschule Rapperswil
%
% !TEX root = ../../buch.tex
% !TEX encoding = UTF-8
%
\section{Mathe
\label{leo:section:mathe}}
\rhead{Mathe}
\todo{Optimale Steuerung ausführen}
Das Problem:
\begin{itemize}
	\item $\mathbf{x(t)}$ und $\mathbf{u(t)}$ nicht unabhängig!
	\item Die DGL $\dot{x} = f\left(x,u\right)$ müsste zuerst gelöst werden, jedoch ist der Steuerungsparameter in der DGL enthalten.
\end{itemize}
\vspace{20pt}
Die Lösung:
\begin{itemize}
	\item Die DGL $\mathbf{\dot{x}} = f\left(\mathbf{x},\mathbf{u}\right)$ als Nebenbedingung betrachten und in das Problem einbauen.
	\item Wie: Lagrange-Multiplikator-Funktion $p(t)$
\end{itemize}
\vspace{10pt}
Das führt zum neuen Funktional:
\[F\left(\mathbf{x},\mathbf{u}, p\right) = L\left(\mathbf{x},\mathbf{u}\right) + p \cdot \left(f\left(\mathbf{x},\mathbf{u}\right) - \mathbf{\dot{x}}\right)\]


\subsubsection{Steuerungs-Hamiltonfunktion}
Steuerungs-Hamiltonfunktion \( H(\mathbf{x},\mathbf{u}, p) = p \cdot f(\mathbf{x},\mathbf{u}) + L(\mathbf{x},\mathbf{u}) \) lösst die Bewegungsgleichungen für $\mathbf{x}$ und $p$.


\subsubsection{Minimum-Prinzip von Pontrjagyn}
Minimum-Prinzip von Pontrjagyn kann die Steuerungsvariable $u$ eindeutig bestimmen.


