%
% teil2.tex -- Beispiel-File für teil2 
%
% (c) 2020 Prof Dr Andreas Müller, Hochschule Rapperswil
%
% !TEX root = ../../buch.tex
% !TEX encoding = UTF-8
%
\section{Aufstiegsgleichung 
\label{leo:section:aufstiegsgleichung}}
\rhead{Aufstiegsgleichung}
Das Ziel ist die Minimierung des Energieverbrauchs,
\begin{equation}
	J = \int_{0}^{t_f} F \, dt
\end{equation}
ausgedrückt durch den Treibstoffverbrauch.
Dazu muss die Endgeschwindigkeit genug hoch sein um im Orbit zu bleiben.
Das lässt sich Zusammenfassen in der Aufstiegsgleichung
\begin{equation}
	v_f = \underbrace{v_* \ln \left(\frac{m_0}{m_f}\right)}_{\text{Raketengleichung}} 
	- \underbrace{2F_* \int_0^{t_f} \frac{\sin^2\left(\frac{\alpha}{2}\right)}{m} \, dt }_{\text{Steuerverluste}}
	- \underbrace{\frac{1}{H} \int_0^{t_f} k_Dv^2 e^{-\frac{h}{H}} \, dt }_{\text{Strömungsverluste}}
	- \underbrace{\int_0^{t_f} g \sin \left(\gamma\right) \, dt}_{\text{Schwerkraftverluste}}
	\label{leo:aufstiegsgleichung}
\end{equation}
durch zusammen setzen von  Formel\,\eqref{leo:raketengleichung},  Formel\,\eqref{leo:raketenluftwiderstand}, Formel\,\eqref{leo:strömungsverlust} und Formel\,\eqref{leo:schwerkraftswiderstand}.
Die Formel\,\eqref{leo:aufstiegsgleichung} gilt es zu minimieren.
%Sie stammt aus der Bewegungsgleichung
%\begin{equation}
%	\dot{v} = \frac{F_*}{m} \cos \alpha - \frac{D}{m} - g \sin \gamma
%\end{equation}
%und wird daraus hergeleitet.



