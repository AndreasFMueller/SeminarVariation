%
% resultat.tex 
%
% 
%
% !TEX root = ../../buch.tex
% !TEX encoding = UTF-8
%

\section{Resultat\label{antennen:resultat}}

Das Verfahren nach Ritz hat ergeben, dass die Antenne eine kreisförmige Abrundung an den Ecken 
zur besten Effizienz führen, da sie die Gleichung \eqref{antennen:Verhältnis} optimieren. 
Es muss jedoch noch bestimmt werden, wie gross der Radius dieser Abrundungen ist und somit wo
genau dieser sich befindet. 

\subsection{Parametrisierung Dreieck\label{antennen:param3eck}}
Die Länge $l$, hierbei der Umfang 
des abgerundeten Dreiecks, sowie dessen Fläche $A$ kann mit den Formeln
\definecolor{clrGreen}{RGB}{0, 117, 18}
\begin{align}
	l &= \textcolor{blue}{2 \cdot \pi \cdot r} + \textcolor{orange}{3 \cdot s - 6 \cdot \sqrt{3} \cdot r} \tag{20.24} \label{antennen:Länge} \\
	A &= \textcolor{clrGreen}{r^2 \cdot \pi} + \textcolor{black}{3 \cdot r \cdot (s - 2 \cdot \sqrt{3} \cdot r)} + \textcolor{red}{\frac{\sqrt{3} \cdot (s - 2 \cdot \sqrt{3} \cdot r)^2}{4}} \tag{20.25} \label{antennen:Fläche}
\end{align}
\setcounter{equation}{25}
berechnet werden.
Der Wirkungsgrad ist nun zu einem Problem geworden, das nur noch abhängig von 
der Seitenlänge s des Dreiecks und des Radius r der Kreise ist. Bildlich ist dies 
in Abbildung \ref{antennen:tikzdreieckAufteilung} veranschaulicht.

%TODO Erklären der Formeln l und A mittels Grafik (Formelabschnitte einfärben??)
\begin{figure}
	\centering
	\begin{tikzpicture}
		\definecolor{clrGreen}{RGB}{0, 117, 18}

		\def\sidelength{3.14}
		
		\pgfmathsetmacro{\triangleheight}{sqrt(3)/2*\sidelength}

		\draw[fill=white] (0,0) -- (\sidelength,0) -- (0.5*\sidelength, \triangleheight) -- cycle;
		\coordinate (A) at (0,0);
		\coordinate (B) at (2.51,0);
		\coordinate (C) at (2.51/2,1.73/2*2.51);

		\coordinate (As) at ($(A) + (9pt,5pt)$);
		\coordinate (Bs) at ($(B) + (9pt,5pt)$);
		\coordinate (Cs) at ($(C) + (9pt,5pt)$);

		\node[circle, inner sep=0pt, minimum size=9pt, fill=clrGreen, draw=blue, line width=1pt] at (As) {};
		\node[circle, inner sep=0pt, minimum size=9pt, fill=clrGreen, draw=blue, line width=1pt] at (Bs) {};
		\node[circle, inner sep=0pt, minimum size=9pt, fill=clrGreen, draw=blue, line width=1pt] at (Cs) {};
		
		\draw[line width=10pt, orange] (As) -- (Bs) (Bs) -- (Cs) (Cs) -- (As);
		\draw[line width=8pt, black] (As) -- (Bs) (Bs) -- (Cs) (Cs) -- (As);
		
		\path[draw=black, fill=red] (As) -- (Bs) -- (Cs) -- (As);

		\draw[->] (0,0) -- (4,0) node[below right] {};
		\draw[->] (0,0) -- (0,4) node[left] {};

		\foreach \x in {0, 1,2,3}
		\draw (\x,1pt) -- (\x,-1pt) node[below] {\x};
		\foreach \y in {1, 2, 3}
		\draw (1pt,\y) -- (-1pt,\y) node[left] {\y};
	\end{tikzpicture}
	\caption{Aufteilung des Dreiecks}
	\label{antennen:tikzdreieckAufteilung}
\end{figure}

Durch Ableiten und Null-setzten
\begin{equation}
	\frac{d}{dr} \bigg(\frac{l}{A^2}\bigg)=0
	\label{antennen:Ableitung}
\end{equation}
wird für einen gegebenen Parameter $s$ ein Radius $r$ als Lösung erhalten. 
Die Ableitung ergibt die Gleichung 
\begin{equation}
	\frac{\left(- 4 \pi r + 12 \sqrt{3} r\right) \left(- 6 \sqrt{3} r + 2 \pi r + 3 s\right)}{\left(\pi r^{2} + 3 r \left(- 2 \sqrt{3} r + s\right) + \frac{\sqrt{3} \left(- 2 \sqrt{3} r + s\right)^{2}}{4}\right)^{3}} + \frac{- 6 \sqrt{3} + 2 \pi}{\left(\pi r^{2} + 3 r \left(- 2 \sqrt{3} r + s\right) + \frac{\sqrt{3} \left(- 2 \sqrt{3} r + s\right)^{2}}{4}\right)^{2}}=0
	\label{antennen:Ableitunggelöst}
\end{equation}
in der $s$ mit der gewünschten Seitenlänge parametrisiert werden kann. Nach Lösen des Gleichungssystems resultieren nun die möglichen und auch unmöglichen Werte für den Radius $r$.
\subsection{Fazit\label{antennen:fazit}}
In diesem Kapitel wurde eine Antennenform ermittelt, welche die optimale Effizienz aufweist. Mittels dem Variationsprinzip von Ritz wurde dargelegt, dass eine Antenne in Form eines gleichseitigen Dreiecks für eine Effizienzsteigerung abgerundete Ecken benötigt. Die Gleichung \eqref{antennen:Ableitunggelöst} entspricht nun einer Formel für das Design einer optimalen, dreieckigen Loop-Antenne. 
