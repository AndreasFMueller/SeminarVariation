%
% unserRitz.tex 
%
% 
%
% !TEX root = ../../buch.tex
% !TEX encoding = UTF-8
%

\section{Ritz Verfahren Grundsätzlich\label{antennen:ritzGrundsätzlich}}

Wie schon sehr oft in diesem Buch erwähnt, gibt es das Problem, dass man für ein Funktional
\begin{equation}
I(y)
=
\int_{x_1}^{x_2}L(x,y(x),y'(x))\,dx
\label{antennen:normalesFunktional}
\end{equation}
eine Funktion $y(x)$ finden muss, welche das Funktional extremal macht.
Solch eine Funktion ist die Lösung der
\begin{equation}
\frac{\partial L}{\partial y} - \frac{d}{dx} \left( \frac{\partial L}{\partial y'} \right) = 0
\label{antennen:el-DGL}
\end{equation}
Euler-Lagrange Differentialgleichung.

Wenn diese Gleichung \eqref{antennen:el-DGL} jedoch analytisch unlösbar ist 
kommt das Verfahren nach Ritz ins Spiel.

\subsection{Approximationsfunktion nach Ritz\label{antennen:approxFunkt}}

Eine kleine Erinnerung der Fourier-Theorie, aus dieser weiss man dass 
periodische Funktionen $f(t)$ als Linearkombination von anderen Funktionen, 
im Sinne der Fourierreihe als
\begin{equation}
FR[f(t)]
=
\frac{a_0}{2}+\sum_{n=1}^{\infty}\left[a_n\cdot\cos\left
(n\omega_f t\right)+b_n\cdot\sin\left(n\omega_f t\right)\right]
\label{antennen:fourier}
\end{equation}
beschrieben werden können.

Eine ähnliche Idee wird beim Verfahren nach Ritz angewendet.
Es wird eine Approximationsfunktion in der Form
\begin{equation}
y(x)=\sum_{k=1}^n a_k \psi_k(x)
\label{antennen:ritzFunkt}
\end{equation}
definiert. Die Funktionen $\psi_k(x)$ und Koeffizienten $a_k$, oder
wie in der Welt des Verfahrens nach Ritz,  
\em Koordinatenfunktionen $\psi_k(x)$ \em und \em Koordinaten $a_k$ \em.
Im weiteren Verlauf jedoch behalten wir die intuitiveren, zuerst genannten Namen.

\subsection{Mögliche Approximationsfunktionen\label{antennen:approxBsp}}

Je nach Problemstellung gibt es bessere oder schlechtere Approximationsfunktionen.
Mögliche ausgeschriebene Funktionen können so aussehen:
\begin{equation}
	\begin{aligned}
		\text{Exponential-Entwicklung: }
		y(x)
		&=
		a_0+a_1 x+a_2 x^2+\cdots+a_n x^n \\
		\text{Fourier-Entwicklung: } 
		y(x)
		&=
		a_0+a_1\cos(x)+b_1\sin(x)+\cdots+a_n\cos(n x)+b_n\sin(n x)\\
		\text{Hyperbolische-Entwicklung: } 
		y(x)
		&=
		a_1 e^{\lambda x}+a_2 e^{-\lambda x}+\cdots+a_n e^{(-1)^n \lambda}
	\end{aligned}
\label{antennen:approxFunktBsp}
\end{equation}

Das Ziel ist es schlussendlich das Problem so genau wie nötig, mit so wenig
Koeffizienten wie möglich auszudrücken.


\subsection{Unsere Approximationsfunktion\label{antennen:unsereApproxFunkt}}

Bei unserem Problem ist zu erwarten, dass die Lösung eine glatte und symmetrische
Funktion sein wird, somit konnten wir uns auf
\begin{equation}
y_n(x)
= 
\sum_{k=1}^n a_k\sin((2k-1)x)
\label{antennen:unserRitz}
\end{equation}
als Approximationsfunktion einigen. 

Diese Entwicklung wir auch Sinus-Fourier Reihe genannt
und wenn man diese mal für $n=2$ ausschreibt ergibt sich
\begin{equation}
y_n(x)
=
a_1\sin(x)+a_2\sin(3x)
\label{antennen:approxFunktn2}
\end{equation}
als Approximationsfunktion. Im weiteren Verlauf ist $y_n(x)$ \eqref{antennen:approxFunktn2} 
unsere endgültige Approximationsfunktion, mit welcher wir die Problemstellung angehen werden.




