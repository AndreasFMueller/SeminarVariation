%
% unserRitz.tex 
%
% 
%
% !TEX root = ../../buch.tex
% !TEX encoding = UTF-8
%

\section{Ritz Verfahren Grundsätzlich\label{antennen:ritzGrundsätzlich}}

Wie schon sehr oft in diesem Buch erwähnt, gibt es das Problem, dass man für ein Funktional
\begin{equation}
I(y)=\int_{x_1}^{x_2}L(x,y(x),y'(x))\,dx
\label{antennen:normalesFunktional}
\end{equation}
eine Funktion $y(x)$ finden muss, welche das Funktional extremal macht.
Solch eine Funktion ist die Lösung der
\begin{equation}
\frac{\partial L}{\partial y} - \frac{d}{dx} \left( \frac{\partial L}{\partial y'} \right) = 0
\label{antennen:el-DGL}
\end{equation}
Euler-Lagrange Differentialgleichung.

Wenn diese Gleichung \eqref{antennen:el-DGL} jedoch analytisch unlösbar ist kommt das Verfahren nach Ritz ins Spiel.

\subsection{Approximations-Funktion nach Ritz\label{antennen:approxFunkt}}

Eine kleine Erinnerung der Fourier-Theorie, aus dieser weiss man dass periodische Funktionen $f(t)$ als Linearkombination von anderen Funktionen, im Sinne der Fourierreihe als
\begin{equation}
F R[f(t)]=\frac{a_0}{2}+\sum_{n=1}^{\infty}\left[a_n \cdot \cos \left(n \omega_f t\right)+b_n \cdot \sin \left(n \omega_f t\right)\right]
\label{antenne:fourier}
\end{equation}
beschrieben werden können.

Eine ähnliche Idee wird beim Verfahren nach Ritz angewendet.
Es wird eine approximations-Funktion in der Form
\begin{equation}
y(x)=\sum_{k=1}^n a_k \psi_k(x)
\end{equation}

definiert. Diese hat die Funktionen $\psi_k(x)$ und Koeffizienten $a_k$. 
In der Welt des Verfahrens nach Ritz heissen die Funktionen \em Koordinatenfunktionen \em und die Koeffizienten \em Koordinaten \em.

\subsection{Mögliche Approximations-Funktionen\label{antennen:approxBsp}}

Je nach Problemstellung gibt es bessere oder schlechtere approximations-Funktionen. Mögliche ausgeschriebene Funktionen können so aussehen:

\begin{equation}
	\begin{aligned}
		\text{Exponential-Entwicklung: }
		y(x)&=
		a_0+a_1 x+a_2 x^2+\cdots+a_n x^n \\
		\text{Fourier-Entwicklung: } 
		y(x)&=
		a_0+a_1\cos(x)+b_1\sin(x)+\cdots+a_n\cos(n x)+b_n\sin(n x)\\
		\text{Hyperbolische-Entwicklung: } 
		y(x)&=
		a_1 e^{\lambda x}+a_2 e^{-\lambda x}+\cdots+a_n e^{(-1)^n \lambda}
	\end{aligned}
\end{equation}

Das Ziel ist es schlussendlich das Problem so genau wie nötig mit so wenig wie möglich Koeffizienten auszudrücken.


\subsection{Unsere Approxiamtions-Funtion}







