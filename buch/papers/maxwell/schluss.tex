%
% einleitung.tex -- Beispiel-File für die Einleitung
%
% (c) 2020 Prof Dr Andreas Müller, Hochschule Rapperswil
%
% !TEX root = ../../buch.tex
% !TEX encoding = UTF-8
%
%Konklusion
\section{Konklusion}
\rhead{Konklusion}
In dieser Arbeit haben wir die Maxwell-Gleichungen im Kontext der Variationsrechnung untersucht und gezeigt, dass diese als Ergebnis der Minimierung einer geeigneten Lagrange-Funktion resultieren. Wir haben die Energien eines Systems zusammengetragen und dieses Funktional mittels der Variationsrechnung minimiert.
Diese Untersuchung verdeutlicht einmal mehr, wie grundlegende physikalische Gesetze aus extremal Prinzipien folgen. 
Insbesondere konnten wir zeigen, dass gerade auch die fundamentalen Gleichungen der Elektrodynamik, die Maxwell-Gleichungen, aus einem Variationsprinzip hergeleitet werden können. Dieses Resultat unterstreicht jene physikalische Theorie erneut. 

Des weiteren könnte diese Arbeit als Ausgangspunkt genommen werden, um weitere Felder und ihre Wechselwirkungen zu berücksichtigen und in die Lagrange-Funktion miteinzubeziehen. Durch eine Erweiterung würde man erwarten, dass andere fundamentale Gesetze resultieren, die bis hin zur Lagrange-Funktion des gegenwärtig umfassendsten Modell der Physik, dem Standardmodell reichen.
