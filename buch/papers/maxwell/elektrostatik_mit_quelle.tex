%
% einleitung.tex -- Beispiel-File für die Einleitung
%
% (c) 2020 Prof Dr Andreas Müller, Hochschule Rapperswil
%
% !TEX root = ../../buch.tex
% !TEX encoding = UTF-8
%
%erste Maxwell Gleichung mit Quelle
\subsection{Gausssches Gesetz
\label{maxwell:section:elektrostatik_mit_quelle}}
\rhead{Problemstellung}
Nun betrachten wir einen luftleeren, dreidimensionalen Raum $V\subset\mathbb{R}^3$, in dem ein elektrisches Potentialfeld $\varphi(x,y,z)$ und eine Ladungsdichte $\varrho(x,y,z)$ existieren.
Auch für diesen Raum möchten wir mittels Variationsrechnung eine Gleichung finden, die das Verhalten des elektrischen Potentialfeldes beschreibt.

\subsubsection{Ansatz}
\rhead{Ansatz}
Es ist naheliegend, dass auch in diesem Szenario die Energie im System minimiert werden muss.
Wie auch in Gleichung \eqref{maxwell:section:energieintegral_quellenfrei} ist die Energie im elektrischen Feld
\[
W_{\text{e}}
=
\iiint_V \frac{1}{2}\,\varepsilon_0\,(\varphi_x^2 + \varphi_y^2 + \varphi_z^2)\, dV.
\]
Jedoch ist dies nicht die einzige Komponente der Gesamtenergie des Systems.
Laut \eqref{maxwell:section:definition_elektrischespotentialfeld} ist das elektrische Potential eine auf die Ladung normierte potentielle Energie.
Somit haben wir die fehlende Komponente gefunden.
Damit wir die durch die Ladung
\begin{equation}
q
=
\iiint_V \varrho(x,y,z)\, dV
\label{maxwell:ladung}
\end{equation}
verursachte potentielle Energie $W_{\text{q}}$ im System mit der Ladungsdichte $\varrho$ ausdrücken können, müssen wir untersuchen, was eine infinitesimale potentielle Energie verursacht.
Wenn wir diese infinitesimal kleine potentielle Energie
\(
dW_{\text{q}}
=
\varphi\, dq
\)
unter die Lupe nehmen und für die infinitesimal kleine Ladung
\(
dq
=
\varrho\, dV
\)
einsetzen, erhalten wir
\(
dW_{\text{q}}
=
\varphi\,\varrho\, dV.
\)
Jetzt müssen die infinitesimalen potentiellen Energien im Raum $V$ zusammengezählt werden und es resultiert
\begin{equation}
W_{\text{q}}
=
\iiint_V \varphi\,\varrho\, dV.
\label{maxwell:section:potenzielle_energie_ladung}
\end{equation}
Mit der Gesamtenergie
\[
W_{\text{tot}}
=
W_{\text{e}} - W_{\text{q}}
=
\iiint_V \frac{1}{2}\,\varepsilon_0\left(\varphi_x^2 + \varphi_y^2 + \varphi_z^2\right) - \varphi\,\varrho\, dV
\]
des Systems haben wir unser zu minimierendes Integral gefunden.
Daraus können wir wieder die Lagrange-Funktion
\begin{equation}
L(x,y,z,\varphi,\varphi_x,\varphi_y,\varphi_z)
=
\frac{1}{2}\,\varepsilon_0\left(\varphi_x^2 + \varphi_y^2 + \varphi_z^2\right) - \varphi\,\varrho
\label{maxwell:section:lagrangefunktion_mit_quelle}
\end{equation}
% TODO: Müller fragen wegen kopplungsterme
ablesen.
Man bemerkt, dass die Lagrange-Funktion einen zusätzlichen additiven Term erhalten hat im Verlgeich zur Lagrange-Funktion \eqref{maxwell:section:lagrangefunktion_quellenfrei}.
Solche Terme nennt man Kopplungsterme.
Sie werden verwendet, um die Wechselwirkung zwischen Termen in der Lagrange-Funktion zu berücksichtigen.
Der Kopplungsterm in unserem Fall beschreibt die Wechselwirkung zwischen dem Feld und der Ladungsdichte.
Nun können wir diese Lagrangefunktion in die Euler-Ostrogradski-Differentialgleichung einsetzen, um zu untersuchen, wie sich das elektrische Potentialfeld verhält.

\subsubsection{Einsetzen in die Euler-Ostrogradski-Differentialgleichung}
Nun wollen wir die in \eqref{maxwell:section:lagrangefunktion_mit_quelle} gefundene Lagrangefunktion in die Euler-Ostrogradski-Differentialgleichung einsetzten.
Um die Rechnung übersichtlicher zu gestalten, machen wir uns die Linearität der Euler-Ostrogradski-Differentialgleichung
\begin{equation}
F\left\{W_{\text{tot}}\right\}
=
F\left\{W_{\text{e}} - W_{\text{q}}\right\}
=
F\left\{W_{\text{e}}\right\} + F\left\{-W_{\text{q}}\right\}
\label{maxwell:section:linearität_von_DGL}
\end{equation}
zu nutze.
Die Lösung von $F\left\{W_{\text{e}}\right\}$ haben wir in Gleichung \eqref{maxwell:section:laplace_gleichung_1} bereits gefunden.
Durch einsetzen von $-W_{\text{q}}$ in die Euler-Ostrogradski-Differentialgleichung erhalten wir
\[
\frac{\partial}{\partial\varphi}\left(-\varrho\,\varphi\right) - \underbrace{\frac{\partial}{\partial x}\frac{\partial}{\partial\varphi_x}\left(\varrho\,\varphi\right)}_{=0} - \underbrace{\frac{\partial}{\partial y}\frac{\partial}{\partial\varphi_y}\left(\varrho\,\varphi\right)}_{=0} - \underbrace{\frac{\partial}{\partial z}\frac{\partial}{\partial\varphi_z}\left(\varrho\,\varphi\right)}_{=0}
=
0.
\]
Da alle partiellen Ableitungen nach $\varphi_x, \varphi_y, \varphi_z$ null ergeben, müssen wir nur noch die partielle Ableitung nach $\varphi$ erledigen.
Somit ist
\(
-\varrho
=
0.
\)
Durch Addition unserer zwei Teillösungen erhalten wir die Gleichung
\begin{align*}
-\varepsilon_0\,\Delta\varphi - \varrho
&=
0
\\
\Leftrightarrow \qquad \varepsilon_0\,\Delta\varphi
&=
-\varrho.
\end{align*}
Schlussendlich erhalten wir
\begin{equation}
\Delta\varphi
=
-\frac{\varrho}{\varepsilon_0}.
\label{maxwell:section:erste_maxwellgleichung_1}
\end{equation}
% TODO: definiton des laplace-operators suchen
Durch gleiches Vorgehen wie in Gleichung \eqref{maxwell:section:laplace_gleichung_3} erhalten wir
\[
\nabla\cdot\underbrace{\nabla\varphi}_{\displaystyle-\vec{E}}
=
-\frac{\varrho}{\varepsilon_0}.
\]
Somit können wir erneut die Schlussdifferentialgleichung mit dem elektrischen Feld ausdrücken. Dadurch muss
\begin{equation}
\nabla\cdot\vec{E}
=
\frac{\varrho}{\varepsilon_0}
\label{maxwell:section:erste_maxwellgleichung_2}
\end{equation}
gelten.
Wir sehen, dass dies dem gaussschen Gesetz entspricht.

\subsubsection{Interpretation des Resultates}
Das gausssche Gesetz \eqref{maxwell:section:erste_maxwellgleichung_2} besagt, dass die Quelle des elektrischen Feldes eine Ladungsdichte $\varrho$ ist.
Dies bedeutet, dass elektrische Feldlinien auf Ladungen enden können.
In anderen Worten wird das elektrostatische Feld durch Ladungen erzeugt!
Da es sowohl positive wie auch negative Ladungen gibt, gilt dasselbe für die Ladungsdichten.
Wenn die Ladungsdichte positiv ist, sagt uns die erste Maxwell-Gleichung, dass das elektrische Feld eine Quelle besitzt.
Das heisst, salopp gesagt, die Feldvektoren zeigen weg von der Ladungsdichte.
Wenn die Ladungsdichte jedoch negativ ist, besitzt das elektrische Feld eine Senke.
Das heisst in diesem Fall, dass die Feldvektoren auf die Ladungsdichte zeigen. Dies ist in Abbildung \ref{maxwell:section:E-Feld_punktladung} veranschaulicht.

\subsubsection{Exkurs zur Poisson-Gleichung}
% Darf auch weggelassen werden.
Die Poisson-Gleichung
\[
\Delta\varphi
=
f,
\]
wobei $\phi$ ein Potentialfeld und $f$ eine Quelle ist, findet in vielen Teilen der Physik ihre Anwendungen.
Die Quelle $f$ kann wie in Gleichung \eqref{maxwell:section:erste_maxwellgleichung_1} eine Funktion des Raumes und/oder eine Funkiton der Zeit sein.
Ein Potentialfeld, dass die Poisson-Gleichung erfüllt, führt zu einem Gradientenfeld $\nabla\varphi$, dass die Quelle $f$ besitzt und Rotationsfrei ist.
Die homogene Gleichung, also wo $f = 0$ ist, führt uns zur Laplace-Gleichung.







