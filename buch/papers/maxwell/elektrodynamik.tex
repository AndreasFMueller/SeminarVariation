%
% einleitung.tex -- Beispiel-File für die Einleitung
%
% (c) 2020 Prof Dr Andreas Müller, Hochschule Rapperswil
%
% !TEX root = ../../buch.tex
% !TEX encoding = UTF-8
%
%Elektrodynmaik
\section{Elektrodynamik\label{section:maxwell:elektrodynmaik}}
\rhead{Elektrodynamik}
In der Elektrodynamik erlauben wir es, dass
\[
\frac{\partial f}{\partial t}
\neq
0
\]
für die Funktionen, die wir betrachten, gelten darf.
Zur Folge dessen berücksichtigen wir nun zeitabhängige skalar- und Vektorfelder.
Wie sich später herausstellen wird, führt diese Zeitabhängigkeit dazu, dass sich das magnetische und elektrische Feld gegenseitig beeinflussen.
Daraus folgen einige spannende Tatsachen, auf die wir später noch zu Sprechen kommen.
Das Ziel dieses Abschnittes ist einen groben Weg zur Lagrange-Funktion aufzuzeigen und danach die daraus entstehenden Gleichungen zu interpretieren.
