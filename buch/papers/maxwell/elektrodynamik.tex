%
% einleitung.tex -- Beispiel-File für die Einleitung
%
% (c) 2020 Prof Dr Andreas Müller, Hochschule Rapperswil
%
% !TEX root = ../../buch.tex
% !TEX encoding = UTF-8
%
%Elektrodynmaik
\section{Elektrodynamik\label{section:maxwell:elektrodynmaik}}
\rhead{Elektrodynamik}
In der Elektrodynamik erlauben wir es, dass
\[
\frac{\partial f}{\partial t}
\neq
0
\]
für die Funktionen, die wir betrachten, gelten darf.
Das heisst Ladungen dürfen sich nun mit einer beliebigen Geschwindigkeit $\vec{v}$ bewegen und wir berücksichtigen zeitabhängige Skalar- und Vektorfelder.
Bei den statischen Maxwell-Gleichungen fällt auf, dass das elektrische und magnetische Feld keinerlei Abhängigkeiten voneinander haben.
Wie sich später herausstellen wird, ändert sich dies, denn die Zeitabhängigkeit führt dazu, dass sich das elektrische und magnetische Feld gegenseitig beeinflussen.
Daraus folgen einige spannende Tatsachen, auf die wir später noch zu Sprechen kommen.
 
Im folgenden werden wir unseren Raum definieren und bekannte Vektorfelder in ihrer Definition anpassen.

\subsubsection{Der vierdimensionale Raum}
Den Raum in dem wir uns nun bewegen werden wir definieren als
\begin{equation}
	\Omega
	=
	\begin{pmatrix}
		ct\\
		x\\
		y\\
		z
	\end{pmatrix}
	=
	\Omega^{\mu} \subset \mathbb{R}^4,
\end{equation}
wobei $\Omega^0$ die Zeit $t$ multipliziert mit der Lichtgeschwindigkeit
\[
c
=
\frac{1}{\sqrt{\varepsilon_0\,\mu_0}}
\]
ist.
Damit wir uns später ein wenig Schreibaufwand sparen können, schreiben wir die Ableitungen der Raumkomponenten als
\[
\frac{\partial}{\partial \Omega^{\mu}}
=
\partial^{\mu}.
\]

\subsubsection{Elektrisches Feld dynamisch}
Das elektrische Feld
\(
\vec{E}: \mathbb{R}^4 \rightarrow \mathbb{R}^3
\)
wird in der Elektrodynamik definiert als
\begin{equation}
	\vec{E}(t,x,y,z)
	=
	- \nabla\varphi(t,x,y,z) - \frac{\partial \vec{A}}{\partial t}(t,x,y,z).
	\label{maxwell:section:definiton_dynamisch_elektrischesFeld}
\end{equation}
Es fällt auf, dass das elektrische Feld, das elektrische Potentialfeld und das magnetische Vektorpotential nun einen vierdimensionalen Inputvektor besitzen, wobei die zusätzliche Dimension die Zeit ist. 

\subsubsection{Magnetisches Feld dynamisch}
Das magnetische Feld
\(
\vec{B}: \mathbb{R}^4 \rightarrow \mathbb{R}^3
\)
wird in der Elektrodynamik ähnlich definiert wie in der Magnetostatik. Nämlich ist
\begin{equation}
	\vec{B}(t,x,y,z)
	=
	\nabla \times \vec{A}(t,x,y,z).
	\label{maxwell:section:definition_dynamisch_magnetischesFeld}
\end{equation}
Der Unterschied liegt hier einzig in der zusätzlichen Zeitkomponente im Inputvektor.

\subsection{Dynamische Maxwell-Gleichungen}
Das Ziel dieses Abschnittes ist ein Variationsprinzip zu formulieren, welches uns die Verhaltensgleichungen der Elektrodynamik liefert. 
Wir betrachten nun den luftleeren, vierdimensionalen Raum $\Omega$.
In diesem Raum existieren nun das elektrische Feld $\vec{E}(t,x,y,z)$, das magnetische Feld $\vec{B}(t,x,y,z)$, eine Stromdichte $\vec{\jmath}(t,x,y,z)$ und eine statische Ladungsdichte $\varrho(x,y,z)$.

\subsubsection{Ansatz}
Wir wählen einen ähnlichen Ansatz wie bislang, nämlich ein Integral über die Zeit der Energie im Raum zu minimieren.
Das Integral über die Zeit ist notwendig, da wir über alle Inputgrössen integrieren müssen.
Diese neue Grösse werden wir mit $D$ bezeichnen.
Durch superponierung unserer bisher gefundenen Energien erhalten wir
\begin{align*}
	D
	&=
	\int_{(ct)_0}^{(ct)_1} W_{\text{tot}}\,d(ct)
	=
	\int_{t_0}^{t_1} \left(W_{\text{e}} - W_{\text{q}} + W_{\text{p}} - W_{\text{m}}\right)c\,dt
	\\
	&= \int_{\Omega} \frac{1}{2}\,\varepsilon_0\,\vec{E}\,^2 - \varphi\,\varrho 
	+ \vec{A}\cdot\vec{\jmath} - \frac{1}{2\mu_0}\vec{B}^2 \,d\Omega.
\end{align*}
Man sieht hier sehr schön, wie die Feldenergie aus dem Term
\[
w_{\text{e}} - w_{\text{m}}
=
\frac{1}{2}\,\varepsilon_0\,\vec{E}^2 - \frac{1}{2\,\mu_0}\vec{B}^2
\]
besteht und die Kopplungsterme 
\(
\varphi\,\varrho
\)
und
\(
\vec{A}\cdot\vec{\jmath}
\)
sind.
Hiermit ist es uns noch nicht gelungen ein Energiefunktional zu formulieren, das von der Form
\[
I(f) = \int_{t_0}^{t_1} \int_{z_0}^{z_1} \int_{y_0}^{y_1} \int_{x_0}^{x_1} L(t,x,y,z,f,f_t,f_x,f_y,f_z)\,dx\,dy\,dz\,dt 
\]
ist, weil das elektrische und magnetische Feld von unterschiedlichen Potentialen abhängig sind.

\subsubsection{Elektromagnetisches 4er Potential}
An dieser Stelle führen wir ein neues Vektorfeld ein, welches die Potentiale des elektrischen und magnetischen Feldes beinhaltet.
Dieses neue Vektorfeld heisst elektromagnetisches 4er Potential
\(
A:\mathbb{R}^4 \rightarrow \mathbb{R}^4
\)
und wird definiert als
\begin{equation}
	A
	=
	\begin{pmatrix}
		\varphi / c\\
		A_x\\
		A_y\\
		A_z
	\end{pmatrix}
	=
	A^{\mu},
\end{equation}
wobei $A_x$, $A_y$, $A_z$ die Komponenten des magnetischen Vektorpotentiales $\vec{A}$ sind.
Mittels dieser Grösse können wir nun das elektrische und magnetische Feld ausdrücken.
Also
\begin{equation}
\frac{1}{c}\, \vec{E}
=
-\frac{1}{c}\, \nabla\,\varphi - \frac{1}{c}\,\frac{\partial \vec{A}}{\partial t}
=
\begin{pmatrix}
	-\partial^1 A^0 - \partial^0 A^1\\
	-\partial^2 A^0 - \partial^0 A^2\\
	-\partial^3 A^0 - \partial^0 A^3
\end{pmatrix}
=
\begin{pmatrix}
	E_x / c\\
	E_y / c\\
	E_z / c
\end{pmatrix}
\end{equation}
und
\begin{equation}
\vec{B}
=
\nabla \times \vec{A}
=
\begin{pmatrix}
	\partial^2 A^3 - \partial^3 A^2\\
	\partial^3 A^1 - \partial^1 A^3\\
	\partial^1 A^2 - \partial^2 A^1
\end{pmatrix}
=
\begin{pmatrix}
	B_x\\
	B_y\\
	B_z
\end{pmatrix}.
\end{equation}
Daraus Ergibt sich für die Feldenergiedichten
\[
w_{\text{e}}
=
\frac{1}{2}\underbrace{\varepsilon_0\,c^2}_{\displaystyle=1/\mu_0} \biggl(\left(\partial^0 A^1 + \partial^1 A^0\right)^2 + \left(\partial^0 A^2 + \partial^2 A^0\right)^2 + 
\left(\partial^0 A^3 + \partial^3 A^0\right)^2\biggr)
\]
und 
\[
w_{\text{m}}
=
\frac{1}{2\mu_0}\,\biggl(\left(\partial^2 A^3 - \partial^3 A^2\right)^2 + \left(\partial^3 A^1 - \partial^1 A^3\right)^2 + 
\left(\partial^1 A^2 - \partial^2 A^1\right)^2\biggr).
\]

\subsubsection{4er Stromdichte}
Damit wir nun auch die Kopplungsterme mit dem elektromagnetischen 4er Potential ausdrücken können führen wir die 4er Stromdichte
\(
J:\mathbb{R}^4 \rightarrow \mathbb{R}^4
\)
ein. Wir definieren sie als
\begin{equation}
J
=
\begin{pmatrix}
	-c\varrho\\
	j_x\\
	j_y\\
	j_z
\end{pmatrix}
=
J^{\mu},
\end{equation}
wobei $j_x$, $j_y$ und $j_z$ die Komponenten der Stromdichte $\vec{\jmath}$ sind.
Somit können wir unsere beiden Kopplungsterme schön kompakt  als
\begin{equation}
	w_{\text{p}} - w_{\text{q}}
	=
	J\cdot A
\end{equation}
schreiben.

\subsubsection{Formulierung der Lagrange-Funktion}
Uns ist es nun gelungen die Grösse $D$ mit einer Funktion und ihren partiellen Ableitungen zu formulieren. Nämlich ist 
\begin{align*}
D
=
\int_{\Omega}
\frac{1}{2\mu_0}\biggl(\left(\partial^0 A^1 + \partial^1 A^0\right)^2 + \left(\partial^0 A^2 + \partial^2 A^0\right)^2 + 
\left(\partial^0 A^3 + \partial^3 A^0\right)^2\biggr) \\
-  \frac{1}{2\mu_0}\,\biggl(\left(\partial^2 A^3 - \partial^3 A^2\right)^2 + \left(\partial^3 A^1 - \partial^1 A^3\right)^2 + 
\left(\partial^1 A^2 - \partial^2 A^1\right)^2\biggr)\\
+ J^0 A^0 + J^1 A^1 + J^2 A^2 + J^3 A^3 \,d\Omega
\end{align*}
und somit die Lagrange-Funktion
\begin{align*}
L(\Omega^0,...,\Omega^3, A^0,...,A^3, \partial^0 A^i,...,\partial^3 A^i)
=
\frac{1}{2\mu_0}\biggl(\left(\partial^0 A^1 + \partial^1 A^0\right)^2 + \left(\partial^0 A^2 + \partial^2 A^0\right)^2 + 
\left(\partial^0 A^3 + \partial^3 A^0\right)^2\biggr) \\
-  \frac{1}{2\mu_0}\,\biggl(\left(\partial^2 A^3 - \partial^3 A^2\right)^2 + \left(\partial^3 A^1 - \partial^1 A^3\right)^2 + 
\left(\partial^1 A^2 - \partial^2 A^1\right)^2\biggr)\\
+ J^0 A^0 + J^1 A^1 + J^2 A^2 + J^3 A^3.
\end{align*}

\subsubsection{Einsetzen in die Euler-Ostrogradski-Differentialgleichung}
Da unsere Lagrange-Funktion von einer Vektorgrösse abhängig ist erhalten wir eine Euler-Ostrogradski-Differentialgleichung von der Form
\[
\frac{\partial L}{\partial A^i} 
- \frac{\partial}{\partial \Omega^0}\frac{\partial L}{\partial(\partial^0 A^i)}
- \frac{\partial}{\partial \Omega^1}\frac{\partial L}{\partial(\partial^1 A^i)}
- \frac{\partial}{\partial \Omega^2}\frac{\partial L}{\partial(\partial^2 A^i)}
- \frac{\partial}{\partial \Omega^3}\frac{\partial L}{\partial(\partial^3 A^i)}
= 0 \qquad \text{für } i=0,1,2,3 \,.
\]

{\larger\textcircled{\smaller[2]1}} $i = 0$
\[
J^0  -\frac{1}{\mu_0}\biggl(\partial^1\underbrace{\left(\partial^0 A^1 + \partial^1 A^0\right)}_{\displaystyle=-E_x/c}
+ \partial^2\underbrace{\left(\partial^0 A^2 + \partial^2 A^0\right)}_{\displaystyle=-E_y/c}
+ \partial^3\underbrace{\left(\partial^0 A^3 + \partial^3 A^0\right)}_{\displaystyle=-E_z/c}\biggr)
=
0
\]

{\larger\textcircled{\smaller[2]2}} $i = 1$
\[
J^1  -\frac{1}{\mu_0}\biggl(\partial^0\underbrace{\left(\partial^0 A^1 + \partial^1 A^0\right)}_{\displaystyle=-E_x/c}
+ \underbrace{\partial^2\left(\partial^1 A^2 + \partial^2 A^1\right)
	- \partial^3\left(\partial^3 A^1 + \partial^1 A^3\right)}_{\displaystyle=(\nabla\times\vec{B})_x}\biggr)
=
0
\]

{\larger\textcircled{\smaller[2]3}} $i = 2$
\[
J^2  -\frac{1}{\mu_0}\biggl(\partial^0\underbrace{\left(\partial^0 A^2 + \partial^2 A^0\right)}_{\displaystyle=-E_y/c}
+ \underbrace{\partial^3\left(\partial^2 A^3 + \partial^3 A^2\right)
	- \partial^1\left(\partial^1 A^2 + \partial^2 A^1\right)}_{\displaystyle=(\nabla\times\vec{B})_y}\biggr)
=
0
\]

{\larger\textcircled{\smaller[2]4}} $i = 3$
\[
J^3  -\frac{1}{\mu_0}\biggl(\partial^0\underbrace{\left(\partial^0 A^3 + \partial^3 A^0\right)}_{\displaystyle=-E_z/c}
+ \underbrace{\partial^1\left(\partial^3 A^1 + \partial^1 A^3\right)
	- \partial^2\left(\partial^2 A^3 + \partial^3 A^2\right)}_{\displaystyle=(\nabla\times\vec{B})_z}\biggr)
=
0
\]

Für $i=0$ resultiert die partielle Differentialgleichung
\begin{align}
-c\varrho + \frac{1}{\mu_0\,c}\biggl(\frac{\partial E_x}{\partial x}
+ \frac{\partial E_y}{\partial y} + \frac{\partial E_z}{\partial z}\biggr)
&=
0\\[1em]
\Leftrightarrow \qquad \nabla\cdot\vec{E}
&=
\frac{\varrho}{\varepsilon_0},
\label{maxwell:section:gauss_dynamisch}
\end{align}
was dem dynamischen gaussschen Gesetz entspricht.
Für $i=1,2,3$ resultiert das partielle Differentialgleichungssystem
\begin{align}
\vec{\jmath} + \frac{1}{\mu_0\,c^2}\frac{\partial \vec{E}}{\partial t}
- \frac{1}{\mu_0}\nabla\times\vec{B}
&=
0\\[1em]
\Leftrightarrow \qquad \varepsilon_0\,\mu_0\frac{\partial \vec{E}}{\partial t} + \mu_0\vec{\jmath}
&=
\nabla\times\vec{B},
\label{maxwell:section:ampere_dynamisch}
\end{align}
was dem dynamischen ampereschen Gesetz entspricht.


