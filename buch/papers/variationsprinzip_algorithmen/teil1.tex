%
% teil1.tex -- Beispiel-File für das Paper
%
% (c) 2020 Prof Dr Andreas Müller, Hochschule Rapperswil
%
% !TEX root = ../../buch.tex
% !TEX encoding = UTF-8
%
\section{Algorithmus auf ein Problem anwenden}
Um den Algorithmus zu verstehen, brauchen es ein Problem, 
welches viele Lösungen gibt und eines davon sollte das Optimum 
sein. Dabei kamm das Travelling Salesman Problem auf, welches ein
NP-vollständiges Problem ist. Es erfüllt die Anfoderungen das 
aus vielen Variationen von Lösungen, das Optimum gefunden werden 
kann.


\subsection{Travelling Salesman Problem}
Das Travelling Salesman Problem (TSP) ist ein klassisches Problem 
aus der Kombinatorischen Optimierung und Informatik. Ziel ist es 
für den Händler eine kürzeste Rundreise zu finden, welche eine 
gegebene Anzahl von Städten zu besuchen und dabei wird jede Stadt 
genau einmal passiert, bevor der Händler genau zum Ausgangspunkt 
zurückkehrt. 

