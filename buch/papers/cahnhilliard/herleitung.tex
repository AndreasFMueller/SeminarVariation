%
% cahnhillard.tex -- Herleitung der Cahn-Hilliard-Gleichung
%
% (c) 2024 Patrik Müller, Hochschule Rapperswil
%
% !TeX root = ../../buch.tex
% !TeX encoding = UTF-8
% !TeX spellcheck = de_CH
%
\section{Herleitung\label{cahnhilliard:section:herleitung}}
\rhead{Herleitung}

\begin{enumerate}
\item Aufstellen des Variationsproblems mit \eqref{cahnhilliard:energy}
\item Anwenden Euler-Ostrogradsky-Gleichung
\item Koppeln mit Massenerhaltung
\end{enumerate}

Aus \eqref{eq:energy} Skalierungsfaktor und höher gradige Terme entfernen
\begin{align}
I(c)
&=
\int_\Omega L(x, c, \nabla c) \di{x}
\nonumber
\\
&=
\int_\Omega F(c) + \frac{\epsilon^2}{2} \abs{\nabla c}^2 \di{x}
\end{align}

Anwenden von Euler-Ostrogradsky-Gleichung
\begin{align*}
\pderiv{I}{c}
&=
\deriv{F}{c}
\\
\pderiv{I}{(\nabla c)}
&=
% \pderiv{}{(\nabla c)} \frac{\epsilon^2}{2} \abs{\nabla c}^2
% &=
\epsilon^2 \sum_{i=1}^3 \pderiv{c}{x_i}
\\
\pderiv{}{x_i}\pderiv{I}{(\nabla c)}
&=
\epsilon^2 \sum_{i=1}^3 \pderiv[2]{c}{x_i}
=
\epsilon^2 \Delta c
% \pderiv{}{(\nabla c)} \left[
% \sum_{i=1}^3 \left( \pderiv{I}{c_i} \right)
% \right]
% \\
% &&&=
% 2 \sum_{i=1}
\end{align*}

Somit ergibt sich das Funktional
\begin{align*}
\frac{\delta I}{\delta c}
&=
\deriv{F}{c} -  \epsilon^2 \Delta c
\equiv
\mu
\end{align*}

Koppeln mit Massenerhaltung
\begin{alignat*}{2}
\pderiv{c}{t}
&=
- \nabla \cdot \mathcal{J}
,\quad&
x &\in \Omega
\\
\nabla c \cdot n
&=
0
,&
x &\in \partial\Omega
\\
\flux \cdot n
&=
0
,&
x &\in \partial\Omega
\end{alignat*}

Was könnte der Fluss in unserem Problem sein?
\begin{align*}
\flux
=
- M \nabla \mu
,\quad \text{wobei } M > 0
\end{align*}

Freie Energie im System kann nicht zunehmen (1. Gesetz der Thermodynamik)
\begin{align*}
\deriv{}{t} I
&\leq
0
\end{align*}
\begin{align*}
\deriv{}{t} I
&=
\int_\Omega \deriv{F}{c} \pderiv{c}{t} + \epsilon^2 \nabla c \cdot \nabla \pderiv{c}{t} \di{x}
\\
&=
\int_\Omega \deriv{F}{c} \pderiv{c}{t} - \epsilon^2 \Delta c \pderiv{c}{t} \di{x}
+ \int_{\partial\Omega} \epsilon^2 \nabla c \pderiv{c}{t} \cdot n \di{s}
\\
&=
\int_\Omega \mu \pderiv{c}{t} \di{x}
\\
&=
\int_\Omega \mu \nabla \cdot (M \nabla \mu) \di{x}
\\
&=
\int_{\partial\Omega} \mu M \nabla \mu \cdot n \di{s} - \int_\Omega \nabla \mu \cdot (M \nabla \mu) \di{x}
\\
&=
-\int_\Omega M \abs{\nabla \mu}^2 \di{x}
\end{align*}

Die Masse im System kann sich nicht ändern
\begin{align*}
0
&=
\deriv{}{t} \int_\Omega c \di{x}
\\
&=
\int_\Omega \pderiv{c}{t} \di{x}
\\
&=
\int_\Omega M \Delta \mu \di{x}
\\
&=
\int_{\partial\Omega} \underbrace{M \nabla \mu}_\flux \cdot n \di{s}
\end{align*}


\begin{align}
\begin{aligned}
\pderiv{c}{t}
&=
\nabla \cdot (M \nabla \mu)
,\quad&
x &\in \Omega
\\
\mu
&=
\deriv{F}{c} -  \epsilon^2 \Delta c
,&
x &\in \Omega
\\
\nabla c \cdot n
&=
0
,&
x &\in \partial\Omega
\\
M \nabla \mu \cdot n
&=
0
,&
x &\in \partial\Omega
\end{aligned}
\label{cahnhilliard:cheq}
\end{align}
