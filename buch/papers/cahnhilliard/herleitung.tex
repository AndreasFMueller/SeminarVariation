%
% cahnhillard.tex -- Herleitung der Cahn-Hilliard-Gleichung
%
% (c) 2024 Patrik Müller, Hochschule Rapperswil
%
% !TeX root = ../../buch.tex
% !TeX encoding = UTF-8
% !TeX spellcheck = de_CH
%
\section{Herleitung der Cahn-Hilliard-Gleichung\label{cahnhilliard:section:herleitung}}
\kopfrechts{Herleitung der Cahn-Hilliard-Gleichung}
Mit \eqref{cahnhilliard:energy} haben wir nun einen Ausdruck, auf den wir ein Variationsprinzip anwenden können.
In den folgenden Abschnitten werden wir das Variationsproblem aufstellen und
das Resultat an die Kontinuitätsgleichung koppeln.
Daraus werden wir die Cahn-Hilliard-Gleichung erhalten.

\subsection{Anwenden der Euler-Ostrogradski-Gleichung}
Bevor wir die Euler-Ostrogradski-Gleichung anwenden,
entfernen wir den Skalierungsfaktor $N_V$ aus \eqref{cahnhilliard:energy}.
Wir nehmen zudem an,
dass die Änderungen der Konzentration über Distanzen,
die im Vergleich zum intermolekularen Abstand gross sind,
nur geringfügig sind.
Damit können wir die Abhängigkeiten von höheren Ableitungen vernachlässigen.
Auf dieser Grundlage ergibt sich das Funktional
\begin{align}
I(c)
&=
\int_\Omega L(x, c, \nabla c) \di{x}
\nonumber
\\
&=
\int_\Omega F(c) + \frac{\varepsilon^2}{2} |\nabla c|^2 \di{x}.
\end{align}
Berechnen wir nun die Ableitungen des Funktionals nach den Feldvariablen erhalten wir
\begin{align*}
\pderiv{I}{c}
&=
\deriv{F}{c}
\\
\pderiv{I}{(\nabla c)}
&=
% \pderiv{}{(\nabla c)} \frac{\varepsilon^2}{2} |\nabla c|^2
% &=
\varepsilon^2 \sum_{i=1}^3 \pderiv{c}{x_i}
\\
\pderiv{}{x_i}\pderiv{I}{(\nabla c)}
&=
\varepsilon^2 \sum_{i=1}^3 \pderiv[2]{c}{x_i}
=
\varepsilon^2 \Delta c
% \pderiv{}{(\nabla c)} \left[
% \sum_{i=1}^3 \left( \pderiv{I}{c_i} \right)
% \right]
% \\
% &&&=
% 2 \sum_{i=1}
.
\end{align*}
Somit können wir nun die Euler-Ostrogradski-Gleichung anwenden:
\begin{align*}
\frac{\delta I}{\delta c}
&=
\deriv{F}{c} -  \varepsilon^2 \Delta c
\equiv
\mu
.
\end{align*}
Der erhaltene Ausdruck $\mu$ wird in der Literatur häufig als
\emph{chemisches Potential} bezeichnet.
\index{chemisches Potential}%
\index{Potential!chemisch}%
Im Gleichgewichtszustand gilt $\mu = 0$.
Wir sind jedoch insbesondere daran interessiert,
wie sich unser Gemisch über die Zeit verhält und entmischt.
Daher betrachten wir im folgenden Abschnitt die Dynamik des Systems,
indem wir die zeitliche Entwicklung der Konzentration
$c$ analysieren.

\subsection{Koppeln mit Kontinuitätsgleichung}
Nachdem wir das chemische Potential $\mu$ bestimmt haben,
welches die treibende Kraft für die Veränderung der Konzentration darstellt,
ist der nächste Schritt,
die zeitliche Entwicklung der Konzentration $c(x,t)$ zu untersuchen.
Diese Entwicklung wird durch die Kontinuitätsgleichung
\eqref{buch:symmetrien:felder:eqn:kontinuitaet} gesteuert,
da die Gesamtmasse der Komponenten in einem abgeschlossenen System konstant bleiben muss.
Für das Salatsaucen-Problem gelten folgende Differentialgleichung mit Randbedingungen:
\begin{align}
\begin{aligned}
\pderiv{c}{t}
&=
- \nabla \cdot \flux
,\quad&
x &\in \Omega
\\
\nabla c \cdot n
&=
0
,&
x &\in \partial\Omega
\\
\flux \cdot n
&=
0
,&
x &\in \partial\Omega
.
\end{aligned}
\label{cahnhilliard:continuity}
\end{align}
Nun stellt sich die allerdings die Frage,
was der Fluss $\flux$ \;in unserem Problem genau sein könnte?
Eine mögliche Inspiration bietet die Wärmeleitungsgleichung,
wo der Fluss proportional zum Gradienten des Temperaturfeldes ist.
Dies führt uns zu der Hypothese, dass auch in unserem Fall der Fluss
durch den Gradienten eines Potentials bestimmt wird.
Daher vermuten wir:
\begin{align*}
\flux
=
- M \nabla \mu
,\quad\text{wobei } M > 0
.
\end{align*}
In der Literatur wird $M$ als die Mobilitätsfunktion bezeichnet,
\index{Mobilitätsfunktion}%
die häufig als konstant angenommen wird.
Der negative Vorzeichen in der Flussgleichung impliziert,
dass die Bewegung der Teilchen in Richtung des abnehmenden chemischen Potentials erfolgt,
um die freie Energie des Systems zu minimieren.

Im folgenden Abschnitt werden wir untersuchen,
ob der vermutete Fluss $\flux{}$ \;wirklich die freie Energie minimiert.
Dies wird es uns ermöglichen,
die Konsistenz unseres Modells zu bestätigen und
die Dynamik der Entmischung in der Salatsauce präzise zu beschreiben.

\subsubsection{Beweis der Energieminimierung}
Da es sich bei unserem Problem um ein Variationsproblem handelt,
sollte die freie Energie minimiert werden.
Dies bedeutet, dass die zeitliche Ableitung des Funktionals  $I$
nicht positiv sein sollte:
\begin{align*}
\deriv{}{t} I
&\leq
0
.
\end{align*}
Leiten wir also nun das Funktional $I$ nach der Zeit ab,
erhalten wir
\begin{align*}
\deriv{}{t} I
&=
\int_\Omega \deriv{F}{c} \pderiv{c}{t}
+ \varepsilon^2 \nabla c \cdot \nabla \pderiv{c}{t} \di{x}
.
\end{align*}
Mit diesem Ausdruck lässt sich jedoch noch keine Aussage treffen,
ob die Bedingung der Energieminimierung erfüllt wird.
Um den Ausdruck weiter zu vereinfach integrieren wir den zweiten Term partiell
und wenden noch einige Umformungen an.
Dann resultiert:
\begin{align*}
\deriv{}{t} I
&=
\int_\Omega \deriv{F}{c} \pderiv{c}{t} - \varepsilon^2 \Delta c \pderiv{c}{t} \di{x}
+ \int_{\partial\Omega} \varepsilon^2 \pderiv{c}{t} \underbrace{\nabla c\cdot n}_{\displaystyle =0} \di{s}
\\
&=
\int_\Omega \mu \pderiv{c}{t} \di{x}
\\
&=
\int_\Omega \mu \nabla \cdot (M \nabla \mu) \di{x}
.
\end{align*}
Durch erneute partielle Integration erhalten wir
\begin{align*}
\deriv{}{t} I
&=
\int_{\partial\Omega} \mu M \nabla \mu \cdot n \di{s} - \int_\Omega \nabla \mu \cdot (M \nabla \mu) \di{x}
.
\end{align*}
Gemäss der Randbedingungen von \eqref{cahnhilliard:continuity}
verschwindet das Oberflächenintegral.
Final erhalten wir also
\begin{align*}
\deriv{}{t} I
&=
-\int_\Omega M |\nabla \mu|^2 \di{x}
,
\end{align*}
wobei alle Terme im Integral positiv sind.
Somit haben wir bewiesen,
dass unser vermuteter Fluss alle Bedingungen erfüllt.

\subsubsection{Cahn-Hilliard-Gleichung}
Nun haben wir also endlich alle Zutaten zusammen,
um die Cahn-Hilliard-Gleichung aufzustellen:
\begin{align}
\begin{aligned}
\pderiv{c}{t}
&=
\nabla \cdot (M \nabla \mu)
,\quad&
x &\in \Omega
\\
\mu
&=
\deriv{F}{c} -  \varepsilon^2 \Delta c
,&
x &\in \Omega
\\
\nabla c \cdot n
&=
0
,&
x &\in \partial\Omega
\\
M \nabla \mu \cdot n
&=
0
,&
x &\in \partial\Omega
.
\end{aligned}
\label{cahnhilliard:cheq}
\end{align}
Die Cahn-Hilliard-Gleichung beschreibt somit
die zeitliche Entwicklung der Konzentration $c(x,t)$
in Abhängigkeit vom chemischen Potential $\mu$,
und stellt sicher,
dass die freie Energie des Systems minimiert wird,
während die Massenerhaltung gewährleistet ist.
