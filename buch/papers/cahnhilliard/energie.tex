%
% einleitung.tex -- Beispiel-File für die Einleitung
%
% (c) 2024 Patrik Müller, Hochschule Rapperswil
%
% !TeX root = ../../buch.tex
% !TeX encoding = UTF-8
%
\section{Freie Energie\label{cahnhilliard:section:energie}}
\kopfrechts{Freie Energie}

Die freie Energie ist ein grundlegendes Konzept in der Thermodynamik.
\index{freie Energie}%
\index{Energie!frei}%
\index{Thermodynamik}%
Sie beschreibt,
wie viel Energie in einem System verfügbar ist,
um Arbeit zu leisten.
Diese Zustandsgrösse leitet sich aus der inneren Energie,
\index{innere Energie}%
\index{Energie!innere}%
der Temperatur und der Entropie des Systems ab.
\index{Entropie}%
Für ein einfaches,
abgeschlossenes System bei konstanter Temperatur $T$
\index{Temperatur}%
und konstantem Volumen $V$ wird die Helmholtz-Energie $F$ verwendet.
\index{Helmholtz-Energie}%
\index{Energie!Helmholtz-}%
Sie ist definiert als
\begin{align*}
F
& =
E - TS,
\end{align*}
wobei $E$ die innere Energie und $S$ die Entropie des Systems sind.
Die Helmholtz-Energie ist besonders nützlich in Systemen mit konstantem Volumen,
wie in festen Körpern oder Flüssigkeiten in starren Behältern.
Aus diesem Grund eignet sie sich besonders gut für unser Salatsaucen-Problem.

In den folgenden Abschnitten betrachten wir $F$ als eine gegebene Funktion
für ein homogenes Gemisch zweier Stoffe $A$ und $B$.
\index{Stoff}%
Dabei interessiert uns nur das lokale Mischungsverhältnis der beiden Stoffe,
das wir als
\begin{align*}
c
& =
\frac{N_B}{N_A + N_B}
\end{align*}
definieren.
Dabei bezeichen $N_A$ und $N_B$ die lokale Anzahl der Atome von Stoff $A$
bzw. Stoff $B$.

\subsection{Freie Energie eines heterogenen Systems}
Um von der gegebenen, inneren Energiefunktion $F(c)$ eines homogenen Systems
\index{homogen}%
auf die innere Energiefunktion $f(c)$ für ein heterogenes System zu gelangen,
kann eine multivariate Taylor-Reihe verwendet werden.
\index{Taylor-Reihe}%
In einem homogenen System ist die Konzentration $c$ überall gleich,
sodass die freie Energie $F(c)$ ausschliesslich von dieser konstanten Konzentration abhängt.
In einem heterogenen System hingegen variiert die Konzentration $c(x)$ räumlich.
Somit ist $f$ abhängig von $\nabla^i c$ für alle $i \in \mathbb{N}$.
Da im homogenen Fall die räumlichen Ableitungen $0$ sind,
wird die Taylor-Reihe um den Punkt
\begin{align*}
\mathbf{c_0}
=
(c, 0, 0, \ldots)
\end{align*}
erstellt.
Somit ergibt sich
\begin{align*}
f(c, \nabla c, \nabla^2 c, \ldots)
=
& F(c)
+ \sum_{i=1}^3 \pderiv{f(\mathbf{c_0})}{c_i} c_i
+ \sum_{i,j=1}^3 \pderiv{f(\mathbf{c_0})}{c_{ij}} c_{ij}
+ \frac{1}{2} \sum_{i,j=1}^3 \frac{\partial^2 f(\mathbf{c_0})}{%
\partial c_i \partial c_j} c_{i} c_{j}
+ \ldots
,
\end{align*}
wobei gilt
\begin{align*}
\quad
c_i
=
\pderiv{c}{x_i}
,\quad
c_{ij}
=
\frac{\partial^2 c}{\partial x_i \partial x_j}
.
\end{align*}
Um den Ausdruck weiter zu vereinfachen,
nehmen wir an,
dass die Materialeigenschaften unseres Gemisches isotrop sind.
\index{Materialeigenschaft}%
\index{isotrop}%
Isotropie bedeutet,
dass die Eigenschaften des Materials unabhängig von der Richtung sind,
das heisst,
das Material verhält sich in alle Richtungen gleich.
Dies vereinfacht die Analyse,
da die Einflüsse der räumlichen Ableitungen der Konzentration
in alle Richtungen gleich sein müssen.
Insbesondere bedeutet dies,
dass $f$ gegenüber Spiegelungen ($x_i \rightarrow -x_i$) und
Permutationen ($x_i \rightarrow x_j)$ invariant sein muss.
Aus diesen Eigenschaften lässt sich erkennen,
dass
\begin{align*}
\pderiv{f(\mathbf{c_0})}{c_i}
&=
0,
\\
\pderiv{f(\mathbf{c_0})}{c_{ij}}
&=
\frac{\partial^2 f(\mathbf{c_0})}{\partial c_i \partial c_j}
=
0
\quad \forall i \neq j,
\\
\pderiv{f(\mathbf{c_0})}{c_{ii}}
&=
\kappa_1,
\\
 \pderiv[2]{f(\mathbf{c_0})}{c_i}
&=
\kappa_2
.
\end{align*}
Damit können wir nun die totale innere Energie unseres Systems ausdrücken als
\begin{align}
\mathcal{E}(c)
=
&N_V \int_\Omega f\di{x}
=
N_V \int_\Omega \left[
F(c) + \kappa_1 \Delta c + \frac{\kappa_2}{2} |\nabla c|^2  + \ldots
\right]\di{x}
\label{cahnhilliard:energylong}
,
\end{align}
dabei bezeichnet $N_V$ die Anzahl Moleküle im Gebiet $\Omega$.

\subsubsection{Energieausdruck in angenehmere Form bringen}
Auf \eqref{cahnhilliard:energylong} könnten wir bereits ein Variationsprinzip anwenden.
Jedoch können wir denn Ausdruck zuerst noch weiter vereinfachen.
Wenden wir partielle Integration auf den Laplace-Term von \eqref{cahnhilliard:energylong} an,
erhalten wir
\begin{align*}
\int_\Omega \kappa_1 \Delta c \di{x}
&=
\int_{\partial\Omega} \kappa_1 \nabla c \cdot n \di{s}
- \int_\Omega \nabla \kappa_1 \cdot \nabla c \di{x}
.
\end{align*}
Das Oberflächenintegral können wir gleich $0$ setzen,
da wir ausschliessen,
dass das Gefäss das Mischverhältnis der Salatsauce beeinflusst.
Somit können wir weiter umformen zu
\begin{align*}
\int_\Omega \kappa_1 \Delta c \di{x}
&=-\int_\Omega \sum_{i=1}^3 \pderiv{\kappa_1}{x_i} \pderiv{c}{x_i} \di{x}
\\
&=-\int_\Omega \sum_{i=1}^3 \pderiv{\kappa_1}{c} \pderiv{c}{x_i} \pderiv{c}{x_i} \di{x}
\\
&=
-\int_\Omega \pderiv{\kappa_1}{c} |\nabla c|^2 \di{x}
.
\end{align*}
Wird der erhaltene Ausdruck in \eqref{cahnhilliard:energylong} eingesetzt,
ergibt sich
\begin{align}
\mathcal{E}(c)
&=
N_V \int_\Omega \biggl[
  F(c) + \underbrace{\left( \frac{\kappa_2}{2} - \pderiv{\kappa_1}{c} \right)}_{\displaystyle\frac{1}{2}\varepsilon^2} |\nabla c|^2  + \ldots
\biggr]\di{x}
\nonumber
\\
&=
N_V \int_\Omega \left[
  F(c) + \frac{\varepsilon^2}{2} |\nabla c|^2  + \ldots
\right]\di{x}
,
\label{cahnhilliard:energy}
\end{align}
wobei $\varepsilon$ in der Praxis gemäss \cite{cahnhilliard:freeenergy}
und \cite{cahnhilliard:deriv} häufig als konstant angenommen wird.
