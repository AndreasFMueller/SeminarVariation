%
% einleitung.tex -- Beispiel-File für die Einleitung
%
% (c) 2024 Patrik Müller, Hochschule Rapperswil
%
% !TeX root = ../../buch.tex
% !TeX encoding = UTF-8
%
\section{Freie Energie\label{cahnhilliard:section:energie}}
\rhead{Freie Energie}

\begin{align*}
\mathbf{c_0} = (c, 0, 0, \ldots)
\end{align*}
\begin{align*}
f(c, \nabla c, \nabla^2 c, \ldots)
=
& F(c)
+ \sum_{i=1}^3 \pderiv{f(\mathbf{c_0})}{c_i} c_i
+ \sum_{i,j=1}^3 \pderiv{f(\mathbf{c_0})}{c_{ij}} c_{ij}
+ \frac{1}{2} \sum_{i,j=1}^3 \frac{\partial^2 f(\mathbf{c_0})}{%
\partial c_i \partial c_j} c_{i} c_{j}
% \\
% & + \frac{1}{2} \sum_{i,j,k=1}^3 \frac{\partial^2 f(c_0)}{\partial c_k \partial c_{ij}} c_k c_{ij}
% + \frac{1}{2} \sum_{i,j,k,l=1}^3 \frac{\partial^2 f(c_0)}{\partial c_{ij} \partial c_{kl}} c_{ij} c_{kl}
+ \ldots
\end{align*}
wobei gilt
\begin{align*}
\quad
c_i
=
\pderiv{c}{x_i}
,\quad
c_{ij}
=
\frac{\partial^2 c}{\partial x_i \partial x_j}
\end{align*}

% TODO: Definition isotrop

\begin{align*}
\pderiv{f(\mathbf{c_0})}{c_i}
&=
0,
\\
\pderiv{f(\mathbf{c_0})}{c_{ij}}
&=
\frac{\partial^2 f(\mathbf{c_0})}{\partial c_i \partial c_j}
=
0
\quad \forall i \neq j,
\\
\pderiv{f(\mathbf{c_0})}{c_{ii}}
&=
\kappa_1,
\\
 \pderiv[2]{f(\mathbf{c_0})}{c_i}
&=
\kappa_2
\end{align*}

\begin{align*}
\mathcal{E}(c)
=
&N_V \int_\Omega f\di{x}
=
N_V \int_\Omega \left[
F(c) + \kappa_1 \Delta c + \frac{\kappa_2}{2} \abs{\nabla c}^2  + \ldots
\right]\di{x}
\\
\end{align*}
wobei $N_V$ die Anzahl Moleküle im Gebiet $\Omega$ sind.

\begin{align*}
\int_\Omega \kappa_1 \Delta c \di{x}
&=
\int_{\partial\Omega} \kappa_1 \nabla c \cdot n \di{s}
- \int_\Omega \nabla \kappa_1 \cdot \nabla c \di{x}
\\
&=-\int_\Omega \sum_{i=1}^3 \pderiv{\kappa_1}{x_i} \pderiv{c}{x_i} \di{x}
\\
&=-\int_\Omega \sum_{i=1}^3 \pderiv{\kappa_1}{c} \pderiv{c}{x_i} \pderiv{c}{x_i} \di{x}
\\
&=
-\int_\Omega \pderiv{\kappa_1}{c} \abs{\nabla c}^2 \di{x}
\end{align*}

\begin{align}
\mathcal{E}(c)
&=
N_V \int_\Omega \left[
F(c) + \kappa_1 \Delta c + \frac{\kappa_2}{2} \abs{\nabla c}^2  + \ldots
\right]\di{x}
\nonumber
\\
&=
N_V \int_\Omega \left[
  F(c) + \underbrace{\left( \frac{\kappa_2}{2} - \pderiv{\kappa_1}{c} \right)}_{\frac{1}{2}\epsilon^2} \abs{\nabla c}^2  + \ldots
\right]\di{x}
\nonumber
\\
&=
N_V \int_\Omega \left[
  F(c) + \frac{\epsilon^2}{2} \abs{\nabla c}^2  + \ldots
\right]\di{x}
,\quad
\text{wobei $\epsilon$ konstant}
\label{cahnhilliard:energy}
\end{align}
