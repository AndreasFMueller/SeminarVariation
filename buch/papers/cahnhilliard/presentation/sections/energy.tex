% !TeX spellcheck = de_CH
% !TeX encoding = UTF-8
% !TeX root = ../presentation.tex

\section{Freie Energie}

\begin{frame}{Annahmen}
\begin{itemize}
\item Die Temperatur ist konstant
\item $\nabla c$ ist sehr gross im Verhältnis zur intermolekularen Distanz
\item $c$ und dessen Ableitungen sind unabhängige Variablen
\end{itemize}
\end{frame}

\begin{frame}{Freie Energy für heterogene Gemische}
\uncover<+->{
\textbf{Vermutung:}
Die lokale freie Energie in einem heterogenen Gemisch ist abhängig
von der lokalen Konzentration und
von der Konzentration der anliegenden Umgebung.
}

$ $

\uncover<+->{
\textbf{Idee:} $f(c, \nabla c, \nabla^2 c, \ldots)$ multivariable Taylorreihe
um den Punkt $\mathbf{c_0} = (c, 0, 0, \ldots)$
\begin{align*}
f(c, \nabla c, \nabla^2 c, \ldots)
=
\uncover<+->{
& F(c)
}
\uncover<+->{
+ \sum_{i=1}^3 \pderiv{f(\mathbf{c_0})}{c_i} c_i
}
\uncover<+->{
+ \sum_{i,j=1}^3 \pderiv{f(\mathbf{c_0})}{c_{ij}} c_{ij}
}
\uncover<+->{
+ \frac{1}{2} \sum_{i,j=1}^3 \frac{\partial^2 f(\mathbf{c_0})}{\partial c_i \partial c_j} c_{i} c_{j}
% \\
% & + \frac{1}{2} \sum_{i,j,k=1}^3 \frac{\partial^2 f(c_0)}{\partial c_k \partial c_{ij}} c_k c_{ij}
% + \frac{1}{2} \sum_{i,j,k,l=1}^3 \frac{\partial^2 f(c_0)}{\partial c_{ij} \partial c_{kl}} c_{ij} c_{kl}
+ \ldots
}
\end{align*}
\begin{align*}
\text{wobei}
\quad
c_i
=
\pderiv{c}{x_i}
,\quad
c_{ij}
=
\frac{\partial^2 c}{\partial x_i \partial x_j}
\end{align*}
}
\end{frame}

\begin{frame}{Vereinfachung}
\uncover<+->{
Wir betrachten nur isotrope Medien
}
\uncover<+->{
\begin{block}{Isotropie}
Isotropie bezeichnet die Unabhängigeit einer Eigenschaft von der Richtung
\end{block}
}
\uncover<+->{
$\Rightarrow$ invariant in Bezug auf Spiegelung ($x_i \rightarrow -x_i$)  und Permutation ($x_i \rightarrow x_j$)
}
\begin{align*}
\uncover<+->{
\pderiv{f(\mathbf{c_0})}{c_i}
&=
0
,&
}
\uncover<+->{
\pderiv{f(\mathbf{c_0})}{c_{ij}}
&=
\frac{\partial^2 f(\mathbf{c_0})}{\partial c_i \partial c_j}
=
0
\quad \forall i \neq j
\\
}
\uncover<+->{
\pderiv{f(\mathbf{c_0})}{c_{ii}}
&=
\kappa_1
,&
}
\uncover<+->{
 \pderiv[2]{f(\mathbf{c_0})}{c_i}
&=
\kappa_2
}
\end{align*}
% \begin{align*}
% f(c, \nabla c, \nabla^2 c, \ldots)
% &=
% F(c) + \kappa_1 \Delta c + \frac{\kappa_2}{2} \abs{\nabla c}^2  + \ldots
% \end{align*}
\end{frame}

\begin{frame}{}
\begin{align*}
\mathcal{E}(c)
=
&N_V \int_\Omega f\di{x}
=
N_V \int_\Omega \left[
F(c) + \kappa_1 \Delta c + \frac{\kappa_2}{2} \abs{\nabla c}^2  + \ldots
\right]\di{x}
\\
&\text{wobei $N_V$ die Anzahl Moleküle im Gebiet $\Omega$ sind}
\end{align*}
\end{frame}

\begin{frame}
\uncover<+->{
$\Delta c$-Term in eine angenehmere Form bringen
% $\pderiv{c}{n}$ verschwindet am Rand
}
\begin{align*}
\int_\Omega \kappa_1 \Delta c \di{x}
&=
\uncover<+->{
\int_{\partial\Omega} \kappa_1 \pderiv{c}{n} \di{s}
- \int_\Omega \nabla \kappa_1 \cdot \nabla c \di{x}
\\
}
\uncover<+->{
&=-\int_\Omega \sum_{i=1}^3 \pderiv{\kappa_1}{x_i} \pderiv{c}{x_i} \di{x}
\\
}
\uncover<+->{
&=-\int_\Omega \sum_{i=1}^3 \pderiv{\kappa_1}{c} \pderiv{c}{x_i} \pderiv{c}{x_i} \di{x}
\\
}
\uncover<+->{
&=
-\int_\Omega \pderiv{\kappa_1}{c} \abs{\nabla c}^2 \di{x}
}
\end{align*}
\end{frame}

\begin{frame}{Totale freie Energie}
\begin{align}
\uncover<+->{
\mathcal{E}(c)
&=
N_V \int_\Omega \left[
F(c) + \kappa_1 \Delta c + \frac{\kappa_2}{2} \abs{\nabla c}^2  + \ldots
\right]\di{x}
\nonumber
\\
}
\uncover<+->{
&=
N_V \int_\Omega \left[
  F(c) + \underbrace{\left( \frac{\kappa_2}{2} - \pderiv{\kappa_1}{c} \right)}_{\frac{1}{2}\epsilon^2} \abs{\nabla c}^2  + \ldots
\right]\di{x}
\nonumber
\\
}
\uncover<+->{
&=
N_V \int_\Omega \left[
  F(c) + \frac{\epsilon^2}{2} \abs{\nabla c}^2  + \ldots
\right]\di{x}
,\quad
\text{wobei $\epsilon$ konstant}
\label{eq:energy}
}
\end{align}
\end{frame}

