
\section{Doppelpendel als chaotisches System}
Das Doppelpendel ist ein klassisches Beispiel für einen chaotischen Prozess.
Dabei handelt es sich um ein gewöhnliches Pendel mit einem weiteren Pendel
befestigt am Ende des Ersten.
In unserem Fall handelt es sich um ein mathematisches Pendel, dass heisst
die Verbindungsstangen des Pendels sind masselos.
Das Verhalten dieses System ist sehr stark abhängig von deren Anfangsbedingungen.
Die kleinste Änderung kann zu einem komplett anderen Ausgang führen.
Die Bewegungsgleichungen des Doppelpendels werden in folgenden Kapiteln
mithilfe des Lagrange-Formalismus hergeleitet werden.