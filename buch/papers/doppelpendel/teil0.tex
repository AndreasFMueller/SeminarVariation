
\section{Einleitung}
Das Doppelpendel ist ein klassisches Beispiel für ein nichtlineares dynamisches System.(Zitat)
Dabei handelt es sich um ein gewöhnliches Pendel mit einem weiteren Pendel
befestigt am Ende des Ersten.
In unserem Fall handelt es sich um ein mathematisches Pendel, dass heisst
die Verbindungsstangen des Pendels sind masselos.
Das Verhalten dieses System ist sehr stark abhängig von deren Anfangsbedingungen.
Die kleinste Änderung darin kann zu einem vollständig anderen Ausgang führen.
Dies macht das Modell zu einem chaotischen Prozess.
Man kann den Übergang ins Chaos schön als sogenanntes Bifurkationsdiagramm darstellen,(Zitat)
worauf wir hier in diesem Paper nicht weiter eingehen.
Im Verlauf des Kapitels werden die Bewegungsgleichungen des Doppelpendels
hergeleitet mithilfe der Lagrange Mechanik.
Zudem wird erläutert weshalb die Lagrange Mechanik angewandt wird.