
\section{Einleitung}
Das Doppelpendel ist ein klassisches Beispiel für ein nichtlineares dynamisches System.
Dabei ist die Rede von einem gewöhnlichen Pendel mit einem weiteren Pendel
befestigt am Ende des Ersten.
In unserem Fall handelt es sich um ein mathematisches Pendel, dass heisst
die Verbindungsstangen des Pendels sind masselos und starr.
Im Verlauf des Kapitels werden die Bewegungsgleichungen des Doppelpendels
mithilfe der Lagrange-Mechanik hergeleitet.
Zudem wird erläutert, weshalb die Lagrange-Mechanik angewandt wird.