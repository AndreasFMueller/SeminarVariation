\section{Chaotisches Verhalten}
Schon die Tatsache, dass sich für die Differentialgleichungen keine 
analytischen Lösungen bestimmen lassen, weisen auf das chaotische Verhalten
des Systems hin.
Wie im vorherigen Kapitel angetönt, liess ich mithilfe der Gleichungen eine Simulation
erstellen, um genau dieses Verhalten näher zu analysieren.
Die dadurch gewonnen Erkenntnisse werden in diesem Abschnitt erläutert.

\subsection{Physikalisches Chaos}
In der Physik wird das Chaos als aperiodisches Langzeitverhalten in einem deterministischen
System bezeichnet, mit empfindlicher Abhängigkeit der Anfangsbedingungen.
Aperiodisches Langzeitverhalten bedeutet, wenn sich das System nach langer zeitlicher Betrachtung
nicht in sich wiederholendes Muster verfällt.
Das einfache Pendel weist ein periodisches Langzeitverhalten auf, denn es schwingt immer im gleichen 
Muster.
Deterministisch heisst, dass das Verhalten des Systems vorherbestimmbar ist.
Dies mag zunächst verwirrend erscheinen, da chaotisches Verhalten grundsätzlich nicht
berechenbar ist.
Was aber gemeint ist, man kann vorhersagen ob das System in eine chaotischen Zustand 
verfällt oder stabil bleibt.

\subsection{Simulationsanalyse}
Das erste Experiment (Abbildung ) wird aufzeigen, wie stark sich die Trajektoren zweier Doppelpendel 
bei kleinsten Unterschieden in den Anfangsbedingungen unterscheiden.
Beim Wiederholen dieses Experiment lässt sich feststellen, dass für exakt dieselben Anfangsbedingungen
das gleiche Ergebnis resultiert und daher deterministisch ist.

Im zweiten Experiment (Abbildung ) haben wir drei Doppelpendel mit jeweils anderen Anfangsbedingungen.
Das bedeutet auch mit unterschiedlicher potentieller Energie.
Es lässt bereits vermuten, dass die Pendel mit hoher potentieller Energie eine stärkere Tendenz haben
in den chaotischen Zustand überzugehen.
Das zeigen auch die Trajektoren aus diesem Versuch als Ergebnis.

Nun was wir hier für drei verschiedene Anfangsbedingungen durchgeführt haben, kann man natürlich
für etliche Winkel durchrechnen lassen und beeurteilen ob bei gegebenem Winkel das System chaotisch wird.
Genau das wurde durch einen Schüler im <<Schweizer Jugend forscht>> mithilfe eines neuronalen Netzes
untersucht. %Zitat Davide
Das Resultat wurde als ein sogenanntes Bifurkationsdiagramm ausgegeben siehe Abbildung.
