\hyphenation{Minimal-eigen-schaften}
\section{Lagrange-Formalismus}
Das Prinzip der kleinsten Wirkung besagt, dass die Natur immer Trajektoren
mit Minimaleigenschaften wählt.
Beispielsweise wählt das Licht durch jedes Medium den Weg mit der kürzesten Zeit, was zur Folge hat,
dass das Licht gebrochen wird wenn es ein Medium verlässt und in ein anderes übergeht.
Dieses Phänomen wird auch im Brechungsgesetz von Snellius beschrieben
(siehe Abschnitt \ref{buch:variation:problem:satz:snellius}).

Die Wirkung \(S\) wird definiert als
\begin{align*}
    S = \int_{t_0}^{t_1} L(t,q_i,\dot{q}_i) \: dt.
\end{align*}
Das Ziel ist es dieses sogenannte Funktional zu minimieren.
Der Integrand des Funktionals ist die Lagrange-Funktion.
Der Lagrange-Formalismus verwendet \(q_i\) als verallgemeinerte Koordinaten
und \(\dot{q}_i\) als ihre Ableitungen.
Damit werden die Bewegungen eines Systems beschrieben.
Der Vorteil besteht darin, dass man beliebige Koordinaten verwenden kann,
sofern die Energie im System in den Koordinaten ausgedrückt werden kann.
Die Lagrange-Funktion \(L\) ist definiert als
\begin{equation}
    L = T - V.
    \label{eq:lagrange} 
\end{equation}
Dies gilt aber nur für Systeme mit verallgemeinertem Potential und holonomen Zwangsbedingungen.
Holonome Zwangsbedingungen sind solche, die durch eine algebraische Gleichung
ausgedrückt werden können.

Um die Lagrange-Funktion anschliessend noch zu minimieren, muss sie in die
Euler-Lagrange-Differentialgleichung
\begin{equation*}
    \frac{d}{dt} \left( \frac{\partial L}{\partial \dot{q}_i} \right) 
    - \frac{\partial L}{\partial q_i} = 0
\end{equation*}
eingesetzt werden
(siehe Abschnitt \ref{buch:variation:section:eulerlagrange}).




