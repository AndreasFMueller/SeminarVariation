%
% 6-homogen.tex
%
% (c) 2024 Prof Dr Andreas Müller
%
\section{Homogene Funktionen
\label{buch:fuvar:section:homogen}}
Die kinetische Energie eines Teilchens, welches sich auf der 
Bahn mit den Koordinaten $x_k(t)$ bewegt, ist
\begin{equation}
E_{\text{kin}}
=
\frac12 m (\dot{x}_1(t)^2 + \dots + \dot{x}_n(t)^2).
=
f(\dot{x}_1(t),\dots,\dot{x}_n(t))
\label{buch:fuvar:homogen:eqn:fbeispiel}
\end{equation}
Die Funktion $f(x)$ hat die Eigenschaft
\begin{align*}
f(\lambda \dot{x}(t))
&=
f(\lambda \dot{x}_1(t),\dots,\lambda\dot{x}_2(t))
=
(\lambda\dot{x}_1(t))^2
+\ldots+
(\lambda\dot{x}_1(t))^2
=
\lambda^2(\dot{x}_1(t)^1+\ldots+\dot{x}_n(t)^2)
\\
&=
\lambda^2f(\dot{x}(t)).
\end{align*}
Die Ableitungen von $x_k(t)$ gehen also nur in den zweiten Potenzen
in die Funktion $f$ ein, es gibt keine Terme mit anderen Exponenten.
Diese Eigenschaft findet man sehr häufig in Funktionen in der Physik.
Der Wert der Funktion $f$ von \eqref{buch:fuvar:homogen:eqn:fbeispiel}
muss die Dimension einer Energie haben, was nur mit dem Produkt einer
Masse mit dem Quadrat der Geschwindigkeit möglich ist.
Andere Potenzen der Geschwindigkeit führen nur auf eine Energie, wenn
ein zusätzlicher Faktor mit der Dimension einer passenden Potenz der
Geschwindigkeit auftritt, der die Masseinheiten korrigiert.

\begin{definition}[homogene Funktion]
Eine Funktion $f\colon \mathbb{R}^n\to\mathbb{R}$ heisst {\em homogen
vom Grad $m$} wenn für jedes $\lambda\in\mathbb{R}$ und jedes
$x\in\mathbb{R}^n$ gilt $f(\lambda x) = \lambda^m f(x)$.
$f$ heisst {\em positiv homogen vom Grad $m$}, wenn
$f(\lambda x) = |\lambda|^m f(x)$ gilt.
\end{definition}

Die kinetische Energie \eqref{buch:fuvar:homogen:eqn:fbeispiel} ist
also ein homogene Funktion vom Grad 2 der Geschwindigkeit.

