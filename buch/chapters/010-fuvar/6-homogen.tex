%
% 6-homogen.tex
%
% (c) 2024 Prof Dr Andreas Müller
%
\section{Homogene Funktionen
\label{buch:fuvar:section:homogen}}
Die kinetische Energie eines Teilchens, welches sich auf der 
Bahn mit den Koordinaten $x_k(t)$ bewegt, ist
\begin{equation}
E_{\text{kin}}
=
\frac12 m (\dot{x}_1(t)^2 + \dots + \dot{x}_n(t)^2).
=
f(\dot{x}_1(t),\dots,\dot{x}_n(t))
\label{buch:fuvar:homogen:eqn:fbeispiel}
\end{equation}
Die Funktion $f(x)$ hat die Eigenschaft
\begin{align*}
f(\lambda \dot{x}(t))
&=
f(\lambda \dot{x}_1(t),\dots,\lambda\dot{x}_2(t))
=
(\lambda\dot{x}_1(t))^2
+\ldots+
(\lambda\dot{x}_1(t))^2
=
\lambda^2(\dot{x}_1(t)^1+\ldots+\dot{x}_n(t)^2)
\\
&=
\lambda^2f(\dot{x}(t)).
\end{align*}
Die Ableitungen von $x_k(t)$ gehen also nur in den zweiten Potenzen
in die Funktion $f$ ein, es gibt keine Terme mit anderen Exponenten.
Diese Eigenschaft findet man sehr häufig in Funktionen in der Physik.
Der Wert der Funktion $f$ von \eqref{buch:fuvar:homogen:eqn:fbeispiel}
muss die Dimension einer Energie haben, was nur mit dem Produkt einer
Masse mit dem Quadrat der Geschwindigkeit möglich ist.
Andere Potenzen der Geschwindigkeit führen nur auf eine Energie, wenn
ein zusätzlicher Faktor mit der Dimension einer passenden Potenz der
Geschwindigkeit auftritt, der die Masseinheiten korrigiert.

%
% Definition
%
\subsection{Definition und Beispiele}
Homogene Funktionen verallgemeinern die Eigenschaft einer Funktion,
dass nur Terme vorkommen dürfen, die die gleiche Masseinheit haben,
mit Hilfe der folgenden Definition.

\begin{definition}[homogene Funktion]
\index{homogene Funktion}%
Eine Funktion $f\colon \mathbb{R}^n\to\mathbb{R}$ heisst {\em homogen
vom Grad $m$} wenn für jedes $\lambda\in\mathbb{R}$ und jedes
$x\in\mathbb{R}^n$ gilt $f(\lambda x) = \lambda^m f(x)$.
$f$ heisst {\em positiv homogen vom Grad $m$}, wenn
$f(\lambda x) = |\lambda|^m f(x)$ gilt.
\end{definition}

Die kinetische Energie \eqref{buch:fuvar:homogen:eqn:fbeispiel} ist
also ein homogene Funktion vom Grad 2 der Geschwindigkeit.
Eine homogene Funktion vom Grad $m=0$ hat die Eigenschaft
$f(tx)=f(x)$ für alle $t$, sie hängt also nur von der Richtung, nicht
vom Betrag ab.

\begin{satz}
\label{buch:fuvar:homogen:satz:produktquotient}
Sei $f$ und $g$ homogene Funktionen vom Grad $m_f$ und $m_g$, dann ist
$fg$ eine homogene Funktion vom Grad $m_f+m_g$ und $f/g$ 
eine homogene Funktion vom Grad $m_f-m_g$, sofern der Quotient
definiert ist.
\end{satz}

\begin{proof}
Durch Rechnung ergibt sich
\begin{align*}
f(tx)g(tx)
&=
t^{m_f}f(x) t^{m_g}g(x)
=
t^{m_f+m_g}f(x)g(x)
\\
\frac{f(tx)}{g(tx)}
&=
\frac{t^{m_f}f(x)}{t^{m_g}g(x)}
=
t^{m_f-m_g}\frac{f(x)}{g(x)}\qquad \text{falls $g(x)\ne 0$},
\end{align*}
was die Behauptung beweist.
\end{proof}

\begin{satz}
\label{buch:fuvar:homogen:satz:polynom}
Die Monome $m_{\mathbf{l}}(x)=x^\mathbf{l}$ sind homogene Funktionen
vom Grad $|\mathbf{l}|$.
Für $l_k>0$ ist die partielle Ableitung $m_{\mathbf{l}}(x)$ eine
homogene Funktion vom Grad $|\mathbf{l}|-1$.
Das Polynom
\begin{equation}
f(x) = \sum_{|\mathbf{l}| = m} a_{\mathbf{l}} x^{\mathbf{l}}
\label{buch:fuvar:homogen:polynom:summe}
\end{equation}
ist homogen vom Grad $m$.
Falls $f(x)$ von $x_k$ abhängt ist die partielle Ableitung von
$f$ nach $x_k$ homogen vom Grad $m$.
\end{satz}

\begin{proof}
Die partiellen Ableitungen der Monome sind
\[
\frac{\partial m_{\mathbf{l}}}{\partial x_k}(x)
=
\begin{cases}
l_kx^{\mathbf{l}-e_k}=l_km_{\mathbf{l}-e_k}(x)&\qquad l_k > 0     \\
0                                             &\qquad \text{sonst}
\end{cases}
\]
Die Ableitung ist also genau dann homogen vom Grad $|\mathbf{l}|-1$,
wenn $x_k$ tatsächlich vorkommt.
Wenn das Polynom $f$ von $x_k$ abhängt, dann gibt es Terme in der
Summe~\eqref{buch:fuvar:homogen:polynom:summe}, die $x_k$ enthalten.
Die partiellen Ableitungen dieser Terme nach $x_k$ sind homogen,
die anderen Terme verschwinden.
Somit sind die partiellen Ableitungen homogen vom Grad $|\mathbf{l}|-1$.
\end{proof}

\begin{beispiel}
Die Funktion
\[
f
\colon
\mathbb{R}^2\to\mathbb{R}
:
(x,y)
\mapsto
f(x,y)
=
\begin{cases}
\displaystyle \frac{x^3y-xy^3}{x^2+y^2}&\qquad (x,y)\ne (0,0)\\
(0,0)&\qquad\text{sonst},
\end{cases}
\]
hat als Beispiel~\ref{buch:fuvar:richtungsableitung:beispiel:schwarz}
zum Satz von Schwarz gedient.
Sie hat gezeigt, dass die partiellen Ableitungen nicht vertauschen müssen,
wenn die zweiten Ableitungen nicht mehr stetig sind.
Der Zählen von $f$ ist homogen vom Grad 4 und der Nenner ist homogen vom
Grad 2, daher ist $f$ homogen vom Grad 2.
Zu einer vorgegebenen Richtung ist $t\mapsto f(tx)$ daher immer eine
Parabel.

Nach der Quotientenregel sind die ersten partiellen Ableitungen von $f/g$
\begin{equation}
\frac{\partial g/h}{\partial x_k}(x)
=
\frac{\displaystyle
\frac{\partial g}{\partial x_k}(x)h(x)-g(x)\frac{\partial h}{\partial x_k}(x)
}{h(x)^2}.
\label{buch:fuvar:homogen:eqn:quotientenregel}
\end{equation}
Da Zähler und Nenner die Form wie in Satz~\ref{buch:fuvar:homogen:satz:polynom}
haben, sind Zähler und Nenner von
\eqref{buch:fuvar:homogen:eqn:quotientenregel}
homogen und damit sind die partiellen Ableitungen von $f=g/h$ homogen
vom Grad 1.
Die Abbildung~\ref{buch:fuvar:richtungsableitung:fig:schwarz} zeigt,
dass die Graphen der Funktionen
$t\mapsto (\frac{\partial f}{\partial x})(tx,ty)$
Geraden sind.

Das gleiche Argument zeigt auch, dass die zweiten Ableitungen 
homogenen Funktionen vom Grad $0$ sind.
Tatsächlich zeigt 
die Abbildung~\ref{buch:fuvar:richtungsableitung:fig:schwarz} auch,
dass die zweite Ableitung nur von der Richtung abhängt.
\end{beispiel}

Die partielle Ableitung des Monoms $x^{\mathbf{l}}$ nach $x_k$
verschwindet nicht, falls $l_k>0$ ist.
In diesem Fall lassen sich alle Exponenten $\mathbf{l}$ aus der
partiellen Ableitung ablesen.
Auch geht genau eine Potenz von $x_k$ verloren, es sollte also
möglich sein, durch Multiplizieren mit $x_k$ das Monom im wesentlichen
wiederherzustellen.
Tatsächlich ist
\begin{align*}
\sum_{k=1}^n
x_k
\frac{\partial x^{\mathbf{l}}}{\partial x_k}
&=
\sum_{k=1}^n
l_k
x_k
x^{\mathbf{l}-e_k}
(1-\delta_{l_k0})
=
\biggl(
\sum_{k=1}^n
l_k
(1-\delta_{l_k0})
\biggr)
x^{\mathbf{l}}
=
\biggl(
\sum_{k=1}^n
l_k
-\underbrace{\sum_{k=1}^n
l_k\delta_{l_k0}}_{\displaystyle=0}
\biggr)
x^{\mathbf{l}}
=
|\mathbf{l}|
x^{\mathbf{l}}
\\
&=
m\,
x^{\mathbf{l}}.
\end{align*}
Damit ist das Monom aus den Ableitungen rekonstruiert.
Für ein beliebiges homogenes Polynom $p(x)$ vom Grad $m$ gilt daher auch
\[
\sum_{k=1}^n
x_k
\frac{\partial p}{\partial x_k}(x)
=
mp(x).
\]
Diese Formel gilt aber nicht nur für homogene Polynome.
Der Satz~\ref{buch:fuvar:homogen:satz:euler} von im nächsten Abschnitt
zeigt, dass er für beliebige homogene Funktionen gilt.

%
% Der Satz von Euler
%
\subsection{Der Satz von Euler}
Der Satz von Euler besagt, dass sich eine homogene Funktion aus den
partiellen Ableitungen rekonstruieren lässt.

\begin{satz}[Euler]
\label{buch:fuvar:homogen:satz:euler}
Ist $f$ eine differenzierbare homogene Funktion
$\mathbb{R}^n\to\mathbb{R}$ vom Grad $m$, dann gilt
\[
\sum_{k=1}^n x_k \frac{\partial f}{\partial x_k}(x)
=
x\cdot \nabla f(x)
=
m f(x)
\]
für alle $x\in\mathbb{R}^n$.
\end{satz}

\begin{proof}
Nach der Kettenregel ist
\[
\frac{df}{dt}(tx)\bigg|_{t=1}
=
\sum_{k=1}^n
\frac{\partial f}{\partial x_k}(tx)\,
\frac{d(tx_k)}{dt} \bigg|_{t=1}
=
\sum_{k=1}^n
\frac{\partial f}{\partial x_k}(x)\,x_k.
\]
Andererseits ist $f(tx)=t^mf(x)$ und daher
\[
\frac{df}{dt}(tx) \bigg|_{t=1}
=
\frac{d}{dt} t^m f(x)\bigg|_{t=1}
=
mt^{m-1}
\bigg|_{t=1}
f(x)
=
m f(x).
\]
Damit ist der Satz bewiesen.
\end{proof}

\begin{beispiel}
Eine homogene Funktion vom Grad $1$ hat also die Eigenschaft, dass
\[
f(x)
=
\sum_{k=1}^n
\frac{\partial f}{\partial x_k}(x)\, x_k.
\]
Die Ableitung nach $x_l$ ist daher
\begin{align*}
\frac{\partial f}{\partial x_l}(x)
&=
\sum_{k=1}^n
\frac{\partial f}{\partial x_k}(x)
\underbrace{\frac{\partial x_k}{\partial x_l}}_{\displaystyle\delta_{kl}}
+
\sum_{k=1}^n
\frac{\partial^2 f}{\partial x_l\,\partial x_k}(x)\, x_k
\\
&=
\frac{\partial f}{\partial x_l}(x)
+
\sum_{k=1}^n
\frac{\partial^2 f}{\partial x_l\,\partial x_k}(x)\, x_k
\qquad\Rightarrow\qquad
\sum_{k=1}^n 
\frac{\partial^2 f}{\partial x_l\,\partial x_k}(x)\, x_k
=
0.
\qedhere
\end{align*}
\end{beispiel}

%
% Die elementarsymmetrischen Polynome
%
\subsection{Die elementarsymmetrischen Polynome}
Die elementarsymmetrischen Polynome sind eine Familie besonders interessanter
homogener Funktionen.

\begin{definition}[elementarsymmetrische Polynome]
Der Koeffizient $\sigma_k(x)$ der Potenz $X^k$ im Polynom
\[
p(X)
=
\prod_{k=1}^n
(X+x_k)
=
X^n + \sigma_1(x) X^{n-1} + \sigma_2(x) X^{n-2} + \ldots + \sigma_{n-1}(x) X + \sigma_n(x)
\]
heisst {elementarsymmetrisches Polynom} vom Grad $k$.
\end{definition}

Durch Ausrechnen findet man sofort
\begin{equation}
\begin{aligned}
\sigma_1(x)
&=
x_1+\ldots+x_n
\\
\sigma_2(x)
&=
x_1x_2+\ldots+x_1x_n+x_2x_3+\ldots x_2x_n+\ldots +\ldots x_{n-1}x_n
\\[-3pt]
&\;\vdots
\\
\sigma_k(x)
&=
\sum_{i_1<i_2<\ldots<i_k}
x_{i_1}x_{i_2}\cdots x_{i_k}
\\
&\;\vdots
\\[-3pt]
\sigma_n(x)
&=
x_1x_2\cdots x_n
\end{aligned}
\label{buch:fuvar:homogen:eqn:sigmadef}
\end{equation}
Der Vollständigkeit halber schreiben wir $\sigma_0(x)=1$.

Mit den elementarsymmetrischen Polynomen lassen sich die Koeffizienten
eines Polynoms mit Hilfe der Nullstellen ausdrücken.
Sind $x_1,\dots,x_n$ die Nullstellen des Polynoms $p(x)$, dann ist
\begin{align*}
p(x)
&=
(x-x_1)(x-x_2)\cdot\ldots\cdot(x-x_n)
%\\
%&=
=
\sum_{k=0}^n (-1)^{n-k}\sigma_k(x_1,\dots,x_n) x^k.
\end{align*}

Die Darstellung \eqref{buch:fuvar:homogen:eqn:sigmadef} der
elementarsymmetrischen Polynome zeigt, dass $\sigma_k(x)$ eine Summe
von Monomen vom Grad $k$ ist, es folgt daher der folgende Satz.

\begin{satz}
\label{buch:fuvar:homogen:satz:elementarsymmetrischhomogen}
Das elementarsymmetrische Polynom $\sigma_k(x)$ ist homogen vom Grad $k$.
\end{satz}

Die elementarsymmetrischen Polynome haben die zusätzliche Eigenschaft,
dass sie sich nicht ändern, wenn die Zahlen $x_1,\dots,x_n$ permutiert
werden.
Ist $\pi\in S_n$ eine Permutation der Zahlen $1,\dots,n$, dann ist
\[
\sigma_k(x_{\pi(1)},\dots,x_{\pi(n)})
=
\sigma_k(x_1,\dots,x_n)
\]
Dasselbe gilt für jeden Ausdruck, der nur von $\sigma_0(x),\dots,\sigma_n(x)$ 
abhängt.

\begin{definition}[symmetrische Funktion]
Eine Funktion $f\colon\mathbb{R}^n\to\mathbb{R}$ heisst {\em symmetrisch},
wenn für jede Permutation $\pi\in S_n$ gilt, dass
$f(x_{\pi(1)},\dots,x_{\pi(n)})=f(x_1,\dots,x_n)$
ist.
Der Wert von $f(x)$ ändert unter beliebigen Permutation der Variablen
$x_1,\dots,x_n$ nicht.
\end{definition}

Das Polynom $p(x)=x_1^2+x_2^2+\dots+x_n^2$ ist homogen vom Grad 2 und
ausserdem symmetrisch.
Das Polynom $p(x)$ ist keine elementarsymmetrische Funktion, kann
aber durch elementarsymmetrische Polynome ausgedrückt werden:
\begin{align*}
x_1^2 + \dots + x_n^2
&=
(x_1+\dots+x_n)^2 - 2(x_1x_2+\dots+x_{n-1}x_n)
=
\sigma_1(x)^2 -2 \sigma_2(x)
\\
x_1^3+\dots+x_n^3
&=
\sigma_1(x)^3
-3\sigma_1(x)\sigma_2(x)
+3\sigma_3(x)
\end{align*}
In der Tat gilt der folgende Satz von Lagrange, der auch schon
Newton bekannt war:

\begin{satz}[Lagrange]
Ist $p(x)$ ein symmetrisches Polynom in den Variablen $x_1,\dots,x_n$,
dann gibt es genau ein Polynom $q(s_1,\dots,s_n)$ derart, dass
$p(x)=q(\sigma_1(x),\dots,\sigma_n(x))$.
\end{satz}

\begin{proof}
Sei $m$ der Grad des symmetrischen Polynoms $p(x)$.
Es kann zerlegt werden in eine Summe
\[
p(x)
=
\sum_{k=0}^m
p_k(x)
\]
von homogenen Polynomen $p_k(x)$ vom Grad $k\le m$. $p_k(x)$ besteht
aus den Monomen vom Grad $k$ in $p(x)$.
Da auch die elementarsymmetrischen Polynome
homogen sind, genügt es daher zu zeigen, dass sich jedes der homogenen
Polynome $p_k(x)$ eindeutig durch die elementarsymmetrischen Polynome
ausdrücken lässt.

\url{https://en.wikipedia.org/wiki/Elementary_symmetric_polynomial}
\end{proof}

