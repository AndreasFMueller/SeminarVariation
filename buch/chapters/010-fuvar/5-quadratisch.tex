%
% 5-quadratische.tex
%
% (c) 2023 Prof Dr Andreas Müller
%
\section{Quadratische Minimalprobleme
\label{buch:fuvar:section:quadratisch}}
\kopfrechts{Quadratische Minimalproblem}
Die quadratische Funktion
\begin{equation}
y
=
ax^2 + bx + c
\label{buch:fuvar:quadratisch:eqn:1dim}
\end{equation}
ist eines der ersten Extremalprobleme, das der Analysis-Student mit
Hilfe der Ableitung zu lösen lernt.
Dabei ist die Maschinerie der Differentialrechnung gar nicht nötig,
denn quadratisches Ergänzen, wie man es zur Herleitung der Lösungsformel
für die quadratische Gleichung ebenfalls braucht, erlaubt, die Funktion
in der Form
\[
y
=
a\biggl(x+\frac{b}{2a}\biggr)^2
+c
-\frac{b^2}{4a^2}
\]
zu schreiben.
Aus dieser Form wird durch Lösen der linearen Gleichung
\[
x+\frac{b}{2a} = 0
\qquad\Rightarrow\qquad
x = -\frac{b}{2a}
\]
klar, dass das Extremum bei $-b/2a$ liegen muss.
Diese besonders einfache Lösung des quadratischen Minimalproblems
ist auch auf ein Minimalprobleme mit mehreren Variablen
verallgemeinerungsfähig, dies soll in diesem Abschnitt durchgeführt
werden.

%
% Quadratische Formen
%
\subsection{Quadratische Formen}
Ein quadratisches Minimalproblem ist die Aufgabe, das Minimum der
Funktion
\begin{equation}
f
\colon
\mathbb{R}^n \to \mathbb{R}
:
x\mapsto \transpose{x} A x + \transpose{b} x + c
\label{buch:fuvar:quadratisch:eqn:ndim}
\end{equation}
mit $A\in M_{n\times n}(\mathbb{R})$, $b\in\mathbb{R}^n$ und $c\in\mathbb{R}$
zu bestimmen.

%
% Symmetrische Matrizen
%
\subsubsection{Symmetrische Matrizen}
Als Skalar ändert der Ausdruck $g(x) = \transpose{x}Ax$  nicht, wenn man ihn
transponiert.
Es ist also
\[
g(x)
=
\transpose{x}Ax
=
\transpose{(\transpose{x}Ax)}
=
\transpose{x}\transpose{A}x.
\]
Der Mittelwert hat daher ebenfalls den Wert
\[
g(x)
=
\frac12(\transpose{x}Ax+\transpose{x}\transpose{A}x)
=
\transpose{x}\frac12(A+\transpose{A})x.
\]
Die Matrix
\[
S=\frac12(A+\transpose{A})
\]
ist symmetrisch.
Die Funktion $f(x)$ von \eqref{buch:fuvar:quadratisch:eqn:ndim}
ändert also nicht, wenn man die Matrix $A$ durch die {\em symmetrisierte
Matrix} $S$ ersetzt.

Ist die Matrix $A$ antisymmetrisch, also $\transpose{A}=-A$, dann folgt
\[
g(x)
=
\transpose{g(x)}
=
\transpose{(\transpose{x}Ax)}
=
\transpose{x}\transpose{A} x
=
-\transpose{x}Ax
=
-g(x),
\]
was nur möglich ist, wenn $g(x)=0$ ist.
Für eine antisymmetrische Matrix ist die quadratische Form $g(x)$
identisch gleich $0$.

Eine Matrix $A$ kann immer in der Form
\[
A
=
\frac12(A+\transpose{A})
+
\frac12(A-\transpose{A})
=
S
+
T
\]
geschrieben werden, wobei $S$ symmetrisch und $T$ antisymmetrisch ist.
Wegen
\[
\transpose{x}Ax
=
\transpose{x}
\frac12(A+\transpose{A})
x
+
\transpose{x}
\frac12(A-\transpose{A})
x
=
\transpose{x}Sx
\]
spielt nur der symmetrische Teil $S$ der Matrix $A$ in der quadratischen
Form eine Rolle.
Im Folgenden werden wir daher annehmen, dass die Matrix $A$ symmetrisch
ist.

%
% Quadratisches Ergänzen
%
\subsubsection{Quadratisches Ergänzen}
Beim quadratischen Ergänzen sucht man nach einem Wert $x_S$, der das
quadratische Polynom auf der rechten Seite von
\eqref{buch:fuvar:quadratisch:eqn:1dim} zu einem reinen Quadrat der Form
\[
a(x-x_S)^2 + C
\]
macht.
Durch Ausmultiplizieren findet man eine lineare Gleichung, deren
Lösung $x_S=b/2a$ ist.
Diese Idee lässt sich auch für die quadratische Funktion $f(x)$ mehrere
Variablen in \eqref{buch:fuvar:quadratisch:eqn:ndim} durchführen.
Gesucht ist ein Vektor $x_S$ derart, dass 
\begin{align*}
f(x)
&=
\transpose{(x-x_S)}A(x-x_S) + C
\\
\transpose{x}Ax + \transpose{b}x + c
&=
\transpose{x}Ax
-\transpose{x_S}Ax - \transpose{x}Ax_S
+
\transpose{x_S}Ax_S + C
\\
\transpose{b}x + c
&=
-\transpose{x_S}Ax - \transpose{x}Ax_S + \transpose{x_S}Ax_S + C
\end{align*}
ist.
Der Vektor $x_S$ dazu für alle $x$ die lineare Gleichung
\[
\transpose{b}x
=
-\transpose{x_S}Ax-\transpose{x_S}\transpose{A}x 
=
-\transpose{x_S}(A+\transpose{A}) x
\qquad\Rightarrow\qquad
(\transpose{x_S}(A+\transpose{A})+\transpose{b})x
=
0
\]
erfüllen.
Dies ist nur möglich, wenn der Klammerausdruck der Nullvektor ist, wenn also
\begin{equation}
(\transpose{A}+A)x_S = -b
\label{buch:fuvar:quadratisch:eqn:xS}
\end{equation}
ist.
Auch in diese Rechnung zeigt, dass für das quadratische Minimalproblem
nur der symmetrische Teil $S=\frac12(\transpose{A}+A)$ relevant ist.

Falls das Gleichungssystem \eqref{buch:fuvar:quadratisch:eqn:xS} eine Lösung
hat, bekommt die Funktion die Form
\begin{align*}
f(x)
&=
\transpose{(x-x_S)}A(x-x_S) + \transpose{x_S}Ax_S + c
\\
&=
\transpose{(x-x_S)}S(x-x_S) + \transpose{x_S}Sx_S + c.
\end{align*}
Im Gegensatz zum eindimensionalen Fall lässt sich daraus aber noch nicht
schliessen, dass $x_S$ das Minimum ist, da der quadratische Term Werte
verschiedenen Vorzeichens annehmen kann.

%
% Lösung des quadratischen Minimalproblems
%
\subsection{Lösung des quadratischen Minimalproblems}
Das quaratische Minimalproblem für eine symmetrische Funktion muss nicht
unbedingt eine Lösung haben.
Nur für positiv definite Matrizen kann man eine Lösung erwarten.

%
% Minimalproblem für positiv definite Matrizen
%
\subsubsection{Positiv definite Matrizen}
Die quadratische Funktion $g(x)=\transpose{x}Ax$ hat die Eigenschaft,
dass
\[
g(tx)
=
t^2 \transpose{x}Ax
=
t^2 g(x).
\]
Wenn $g(x)$ negativ ist, dann nimmt $g(tx)$ daher beliebig grosse negative
Werte an, das Minimalproblem kann keine Lösung haben.
Damit das Minimalproblem überhaupt eine Lösung hat, muss also
$g(x)>0$ sein.

\begin{definition}
Die symmetrische Matrix $A\in M_{n\times n}(\mathbb{R})$ heisst
{\em positiv definit}, wenn $\transpose{x}Ax>0$ ist für alle Vektoren
$x\in\mathbb{R}^n\setminus\{0\}$.
Sie heisst {\em positiv semidefinit}, wenn $\transpose{x}Ax\ge 0$ für
alle Vektor $x\in\mathbb{R}^n$.
\end{definition}

Ein quadratisches Minimalproblem mit einer positiv definiten Matrix
hat daher eine Lösung, die durch Lösen der Gleichung
\eqref{buch:fuvar:quadratisch:eqn:xS} als
\[
x_S
=
-
\frac12 A^{-1}b
\]
bestimmt werden kann.

%
% Lösung des quadratischen Minimalproblems mit der Cholesky-Zerlegung
%
\subsubsection{Lösung des quadratischen Minimalproblems mit der
Cholesky-Zerlegung}
In der linearen Algebra lernt man, dass eine positiv definite
symmetrische Matrix in eine Produkt $A=L\transpose{L}$ mit einer
unteren Dreiecksmatrix $L$ zerlegt werden kann, die auch regulär ist.
Damit kann man die Funktion $f(x)$ jetzt schreiben als
\begin{align*}
f(x)
&=
\transpose{x}L\transpose{L}x
+\transpose{b}x
+
c
\\
&=
\transpose{(\transpose{L}x)} \transpose{L}x
+
\transpose{b}
(\transpose{L})\mathstrut^{-1}
\transpose{L}x
+
c.
\intertext{Mit den Abkürzungen $y=\transpose{L}x$ und $d=\frac12L^{-1}b$
wird dies}
&=
\transpose{y}y
+
2
\transpose{d}y
+
c
\\
&=
\transpose{y}y
+
\transpose{d}y
+
\transpose{y}d
+
\transpose{d}d
-
\transpose{d}d
+
c
\\
&=
\transpose{(y+d)}(y+d)
-
\frac14\transpose{(L^{-1}b)}L^{-1}b
+c.
\end{align*}
Daraus kann man ablesen, dass das Minimum an der Stelle
\[
y
=
-d
\qquad\Rightarrow\qquad
\transpose{L}x = - \frac12 L^{-1}b
\qquad\Rightarrow\qquad
x
=
-\frac12(\transpose{L})^{-1}L^{-1} b
=
-\frac12 A^{-1}b
\]
ist.
Das Resultat stimmt natürlich mit dem früher gefundenen Wert
überein, aber die Inverse einer Dreiecksmatrix ist sehr viel
einfacher zu bestimmen.

%
% Nullstellen des Gradienten
%
\subsubsection{Nullstellen des Gradienten}
Natürlich kann man das Minimum auch als Nullstelle des Gradienten
finden.
Dazu muss die Ableitung von $f(x)$ berechnet werden.
Wegen
\begin{align*}
\frac{\partial f}{\partial x_j}(x)
&=
\frac{\partial}{\partial x_j}\sum_{i,k=1}^n a_{ik} x_ix_k
+
\frac{\partial}{\partial x_j}\sum_{i=1}^n b_ix_i
\\
&=
\sum_{k=1}^n a_{jk}x_k
+
\sum_{i=1}^n a_{i\!j}x_i
+
b_j
\\
&=
\sum_{k=1}^n (a_{jk}+a_{k\!j}) x_k
+
b_j
\\
&=
(2Ax)_j + b_j
\intertext{oder in Vektorschreibweise}
\nabla f(x)
&=
2Ax + b
\end{align*}
folgt für die Nullstelle des Gradienten
\[
0
=
\nabla f(x)
=
2Ax+b
\qquad\Rightarrow\qquad
x = -\frac12 A^{-1} b
\]
wie früher.

%
% Gradientabstieg
%
\subsubsection{Gradientabstieg}
Die Lösung eines quadratischen Minimalproblems erfordert die Lösung eines
linearen Gleichungssystems.
Für grosses $n$ ist dies ziemlich aufwendig.
Da der Gradient sehr einfach zu bestimmen ist, kann man auch die Methode
des Gradientabstiegs verwenden, um die Extremalstellen numerisch
anzunähern.
Sie wird im Abschnitt~\ref{buch:direkt:section:gradient} genauer
beschrieben.
Auf den ersten Blick scheint es keine Verbesserung, ein Problem, für
das ein exakter Algorithmus verfügbar ist, mit Hilfe eines iterativen
Verfahrenes nur näherungsweise zu lösen.
Es stellt sich allerdings heraus, dass es möglich ist, die Abstiegsschritte
so zu dimensionieren, dass die Lösung nicht nur approximativ ist, sondern
in genau $n$ Schritten die exakte Lösung gefunden wird.
Dies ist bekannt das das Verfahren der konjugierten Gradienten.

Der Vorteil des iterativen Verfahrens ist, dass sich eine ausreichend
genaue Lösung mit weniger Abstiegsschritten finden lässt.
Für grosses $n$ kann daher angemessen genaue Lösung mit wesentlich
weniger Aufwand gefunden werden, also mit der vollständigen Lösung des
Gleichungssystems.


