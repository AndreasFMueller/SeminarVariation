%
% nebenbedingung.tex -- Nebenbedingung in 2 Dimensionen
%
% (c) 2021 Prof Dr Andreas Müller, OST Ostschweizer Fachhochschule
%
\documentclass[tikz]{standalone}
\usepackage{amsmath}
\usepackage{times}
\usepackage{txfonts}
\usepackage{pgfplots}
\usepackage{csvsimple}
\usetikzlibrary{arrows,intersections,math,calc}
\begin{document}
\def\skala{1}
\begin{tikzpicture}[>=latex,thick,scale=\skala,
declare function={
	R(\t,\a) = (1+0.2*(cos(2*\t)-(cos(6*\t)/3)+(cos(10*\t)/5)))*(1+\a);
	X(\t,\a) = 2+(R(\t,\a)*cos(\t-3*\a));
	Y(\t,\a) = 1+(R(\t,\a)*sin(\t-3*\a));
	XX(\t,\a) = X(\t,\a);
	YY(\t,\a) = Y(\t,\a);
}]

\begin{scope}
\clip (-0.1,-0.1) rectangle (11.3,8.0);
\fill[color=gray!10] (-1,-1) rectangle (12,9);
\foreach \a in {3,6,...,60}{
	\fill[color=blue!\a,opacity=0.1]
		plot[domain=0:360,samples=72]
			({XX(\x,(60-\a)/6)},{YY(\x,(60-\a)/6)});
}
\foreach \a in {0,2,...,12}{
	\draw[color=blue] plot[domain=0:360,samples=72]
		({X(\x,\a)},{Y(\x,\a)});
	\draw[color=blue] plot[domain=0:360,samples=72]
		({XX(\x,\a)},{YY(\x,\a)});
}
%\draw[color=blue] plot[domain=0:360,samples=100]
%	({0.01*\x},{R(\x)});
\draw[color=red,line width=2pt] (0,-3) circle[radius=10];
\end{scope}

\draw[->] (-0.1,0) -- (11.6,0) coordinate[label={$x_1$}];
\draw[->] (0,-0.1) -- (0,8.3) coordinate[label={left:$x_2$}];
\node at (0,0) [below left] {$0\mathstrut$};

\end{tikzpicture}
\end{document}

