%
% polar.tex -- template for standalon tikz images
%
% (c) 2021 Prof Dr Andreas Müller, OST Ostschweizer Fachhochschule
%
\documentclass[tikz]{standalone}
\usepackage{amsmath}
\usepackage{times}
\usepackage{txfonts}
\usepackage{pgfplots}
\usepackage{csvsimple}
\usetikzlibrary{arrows,intersections,math}
\definecolor{darkred}{rgb}{0.8,0,0}
\begin{document}
\def\skala{1}
\def\dx{4}
\def\dy{4}
\input{polarpfade.tex}
\begin{tikzpicture}[>=latex,thick,scale=\skala]

\draw[color=darkred,line width=1.4pt] \pfada;
\draw[color=darkred,line width=1.4pt] \pfadb;
\draw[color=darkred,line width=1.4pt] \pfadc;
\draw[color=darkred,line width=1.4pt] \pfadd;
\draw[color=darkred,line width=1.4pt] \pfade;
\draw[color=darkred,line width=1.4pt] \pfadf;
\draw[color=darkred,line width=1.4pt] \pfadg;
\draw[color=darkred,line width=1.4pt] \pfadh;
\draw[color=darkred,line width=1.4pt] \pfadi;
\draw[color=darkred,line width=1.4pt] \pfadj;
\draw[color=darkred,line width=1.4pt] \pfadk;
\draw[color=darkred,line width=1.4pt] \pfadl;
\draw[color=darkred,line width=1.4pt] \pfadm;
\draw[color=darkred,line width=1.4pt] \pfadn;
\draw[color=darkred,line width=1.4pt] \pfado;
\draw[color=darkred,line width=1.4pt] \pfadp;
\draw[color=darkred,line width=1.4pt] \pfadq;
\draw[color=darkred,line width=1.4pt] \pfadr;

\draw[->] (0,0) -- (8.4,0) coordinate[label={$r$}];
\fill (0,0) circle[radius=0.05];
\fill (4,0) circle[radius=0.05];
\node at (4,0) [below] {$1$};
\node at (0,0) [below] {$0$};

\end{tikzpicture}
\end{document}

