%
% chapter.tex
%
% (c) 2023 Prof Dr Andreas Müller
%
\chapter{Zweite Variation
\label{buch:chapter:variation2}}
\kopflinks{Zweite Variation}
Mit Hilfe der zweiten Ableitung ist es bei einem Extremalproblem
mit endlich vielen Variablen möglich, hinreichende Bedingungen dafür
zu formulieren, dass ein Maximum oder Minimum vorliegt.
Es liegt nahe, solche Kriterien auch für Variationsprobleme
zu erwarten.
Schliesslich lässt sich auch die Variation als Funktion
$f(\varepsilon) = I(y+\varepsilon\eta)$ als Potenzreihe
\[
f(\varepsilon)
=
I(y) + \varepsilon \delta I(y) + \frac{f''(0)}{2!}\varepsilon^2 + \dots
\]
in $\varepsilon$ entwickeln.
Notwendig für ein Extremum ist das verschwinden der ersten Variation
$\delta I(y)=0$, wie das in früheren Kapiteln ausführlich verwendet
worden ist.
Falls ausserdem $f''(0)>0$ ist, kann man schliessen, dass die
Funktion $f(\varepsilon)$ ein Minimum an der Stelle $\varepsilon=0$
hat.
Dies garantiert aber noch nicht, dass das Funktional $I(y)$
ein Minimum hat, dazu müssen alle denkbaren Funktionen $\eta(x)$
untersucht werden, also eine unendliche Menge von ``Richtungen''.

Für Funktionen $\mathbb{R}^n\to\mathbb{R}: x\mapsto f(x)$ von
$n$ Variablen sind Richtungsableitungen in alle Richtungen zu
untersuchen.
Nach Abschnitt~\ref{buch:fuvar:section:hessesche} liegt ein
ein Minimum vor, wenn $\operatorname{grad}f(x_*)=0$ und ausserdem
die hessesche Matrix an der gleichen Stelle positiv definit ist.

Es stellt sich aber heraus, dass es für Variationsprobleme viel
schwieriger ist, hinreichende Bedingungen für ein Extremum zu finden.
Da es unendlich viele mögliche Richtungen gibt, kann zwar für alle
möglichen Funktion $\eta$ die zweite Ableitung $f''(0)>0$ sein, aber 
sie kann trotzdem beliebig klein werden.
Damit lässt sich nicht mehr schliessen, dass der Punkt tatsächlich
ein Minimum ist.

In diesem Abschnitt werden zunächst aus der sogenannten zweiten
Variation notwendige Bedingungen für ein Maximum oder Minimum
hergeleitet.
Nach einigen Vorbereitungen über die Eigenschaften von Nullstellen
von Lösungen von linearen Differentialgleichungen zweiter Ordnung
wird dann die Jacobi-Differentialgleichung und das daraus ableitbare
Jacobi-Kriterium für ein Maximum oder Minimum vorgestellt.

%
% 2-zweitevariation.tex
%
% (c) 2023 Prof Dr Andreas Müller
%
\section{Die zweite Ableitung
\label{buch:variation2:section:zweiteableitung}}
\kopfrechts{Zweite Ableitung}
Wir betrachten wieder ein Funktional der Form
\[
I(y)
=
\int_{x_0}^{x_1}
F(x,y(x),y'(x))
\,dx
\]
und eine Variation $y(x)+\varepsilon\eta(x)$ mit $\eta(x_0)=0$
und $\eta(x_1)=0$.
Die erste Ableitung der Funktion $\varepsilon\mapsto I(y+\varepsilon\eta)$
wurde früher als
\[
\delta I(y)
=
\int_{x_0}^{x_1}
\biggl(
\frac{\partial F}{\partial y}(x,y(x),y'(x))
-
\frac{d}{dx}
\frac{\partial F}{\partial y'}(x,y(x),y'(x))
\biggr)
\,
\eta(x)
\,dx
\]
berechnet.
Wir berechnen nun zusätzlich die zweite Ableitung.
Dazu verwenden wir die Entwicklung des Integranden
$f(\varepsilon) = F(x,y(x)+\varepsilon\eta(x),y'(x)+\varepsilon\eta'(x))$
nach Potenzen von $\varepsilon$, sie ist
\begin{align*}
f(\varepsilon)
%F(x,y(x)+\varepsilon\eta(x),y'(x)+\varepsilon\eta'(x))
&=
F(x,y(x),y'(x))
+
\biggl(
\frac{\partial F}{\partial y}(x,y(x),y'(x))\eta(x)
+
\frac{\partial F}{\partial y'}(x,y(x),y'(x))\eta'(x)
\biggr)
\varepsilon
\\
&\quad
+
\frac{1}{2!}
\biggl(
\frac{\partial^2 F}{\partial y^2}(x,y(x),y'(x))
\eta(x)^2
+
2
\frac{\partial^2 F}{\partial y\,\partial y'}(x,y(x),y'(x))
\eta(x)\eta'(x)
\\
&\qquad\qquad
+
\frac{\partial^2 F}{\partial y^{\prime 2}}(x,y(x),y'(x))
\eta'(x)^2
\biggr)
\varepsilon^2
+
\dots
\end{align*}
Für die Untersuchung eines Extremums müssen wir nur den quadratischen
Term betrachten.
Der besseren Übersichtlichkeit wegen schreiben wir ihn als
\[
G(x)
=
P(x)\eta(x)^2 + 2Q(x) \eta(x)\eta'(x) + R(x)\eta'(x)^2
\]
mit
\begin{align*}
P(x) &= \frac{\partial^2 F}{\partial y^2} (x,y(x),y'(x)),
&
Q(x) &= \frac{\partial^2 F}{\partial y\,\partial y'} (x,y(x),y'(x))
\\
\text{und}
\qquad
R(x) &= \frac{\partial^2 F}{\partial y^{\prime 2}} (x,y(x),y'(x)).
\end{align*}
Wegen
\[
2Q(x)\eta(x)\eta'(x)
=
Q(x)\cdot 2\eta(x)\eta'(x)
=
Q(x)\cdot \frac{d\eta(x)^2}{dx}
\]
kann bei der Berechnung des Integrals des mittleren Terms partielle
Integration verwendet werden:
\begin{align*}
\int_{x_0}^{x_1}
2Q(x)\eta(x)\eta'(x)
\,dx
&=
\int_{x_0}^{x_1}
Q(x) \frac{d\eta(x)^2}{dx}
\,dx
=
\biggl[Q(x)\eta(x)^2\biggr]_{x_0}^{x_1}
-
\int_{x_0}^{x_1} Q'(x) \eta(x)^2\,dx
\end{align*}
Wegen der Randbedingung $\eta(x_0)=\eta(x_1)=0$ fällt der
erste Term auf der rechten Seite weg.
Das Integral von $G(x)$ heisst die zweite Variation von $I$ an der
Stelle $y$ und kann damit zu
\begin{align}
\delta^2 I(y)
&=
\int_{x_0}^{x_1} G(x)\,dx
\notag
\\
&=
\int_{x_0}^{x_1}
(\underbrace{P(x)-Q'(x)}_{\displaystyle = S(x)}) \eta(x)^2
+
R(x)\eta'(x)^2
\,dx
\notag
\\
&=
\int_{x_0}^{x_1}
S(x)\eta(x)^2 + R(x)\eta'(x)^2
\,dx
\label{buch:variation2:zweitevariation:eqn:SRintegral}
\end{align}
vereinfacht werden.
Die Faktoren $\eta(x)^2$ und $\eta'(x)^2$ sind immer positiv,
über das Vorzeichen des Integrals entscheidet also nur das Vorzeichen
der Koeffizienten $S(x)$ und $R(x)$.
Gesucht sind jetzt Bedingungen an $S(x)$ und $R(x)$, die garantieren,
dass die zweite Variation $\delta^2 I(y)\ge 0$ ist.



%
% 2-legendre.tex
%
% (c) 2024 Prof Dr Andreas Müller
%
\section{Die Legendre-Bedingung
\label{buch:variation2:section:legendre}}
\kopfrechts{Die Legendre-Bedingung}
Der Ausdruck \eqref{buch:variation2:zweitevariation:eqn:SRintegral}
für die zweite Variation ist sicher positiv, wenn sowohl $S(x)$
als auch $R(x)$ auf dem ganzen Intervall positiv sind.
In diesem Abschnitt soll untersucht werden, inwieweit die
Umkehrung auch gilt.

\begin{satz}[Legendre]
\label{buch:variation2:legendre:satz:legendre}
Dafür, dass $\delta^2I(y)\ge 0$ für jede Funktion $\eta(x)$ gilt, ist
notwendig, dass
\[
R(x)
=
\frac{\partial^2F}{\partial y^{\prime 2}}(x,y(x),y'(x))
\ge
0
\]
im ganzen Intervall $[x_1,x_2]$ ist.
\end{satz}

\begin{definition}[Legendre-Bedingung]
Man sagt, eine Lagrange-Funktion $F(x,y,y')$ erfüllt die
{\em Legendre-Bedingung}
\index{Legendre-Bedingung}%
für eine Funktion $y(x)$, wenn 
\[
\frac{\partial^2 F}{\partial y^{\prime 2}}(x,y(x),y'(x)) \ge 0.
\]
$F$ erfüllt sogar die {\em starke Legendre-Bedingung}, wenn
\[
\frac{\partial^2 F}{\partial y^{\prime 2}}(x,y(x),y'(x)) > 0
\]
gilt.
\end{definition}

Der Satz~\ref{buch:variation2:legendre:satz:legendre} besagt also,
dass dafür, dass eine Lösung $y(x)$ der
Euler-Lagrange-Diffe\-ren\-tial\-gleichung
ein lokales Minimum des Funktionals ist, die Lagrange-Funktion
die Legendre-Bedingung für die Funktion $y(x)$ erfüllt.
Da sie auch $R(x)=0$ erlaubt, kann man nicht erwarten, dass sie auch
hinreichend ist.
Die später in Abschnitt~\ref{buch:variation2:section:jacobi} gezeigt wird,
verlangt das hinreichende Jacobi-Kriterium, dass sogar die starke
Lagrange-Bedingung erfüllt ist.

\begin{proof}
%
% legendre.tex -- Legendre Bedingung
%
% (c) 2021 Prof Dr Andreas Müller, OST Ostschweizer Fachhochschule
%
\documentclass[tikz]{standalone}
\usepackage{amsmath}
\usepackage{times}
\usepackage{txfonts}
\usepackage{pgfplots}
\usepackage{csvsimple}
\usetikzlibrary{arrows,intersections,math,calc}
\definecolor{darkred}{rgb}{0.8,0,0}
\begin{document}
\def\skala{1}
\def\a{0.3}
\def\d{1}
\def\b{5}
\def\N{500}
\def\xstern{3}
\pgfmathparse{\a*(\b+0.5)/\d}
\xdef\B{\pgfmathresult}
\begin{tikzpicture}[>=latex,thick,scale=\skala,
declare function={
	S(\x) = sin(50*\x+40)+2*cos(20*(\x-4))+0.1*\x;
	R(\x) = cos(10*\x)-3*exp(-(3-\x)*(3-\x)/4);
	eta(\x) = \a*cos((\b+0.5)*180*(\x-\xstern)/\d);
	etastrich(\x) = -\a*(\b+0.5)*sin((\b+0.5)*180*(\x-\xstern)/\d)/\d;
}]


\fill[color=gray!10] ({\xstern-\d},-10) rectangle ({\xstern+\d},3);

\begin{scope}
\draw[->] (-4.2,0) -- (8.5,0) coordinate[label={$x$}];
\fill[color=darkred!20] ({\xstern-\d},0) rectangle ({\xstern+\d},{\a*\a});
\draw[->] (0,-3) -- (0,3.3) coordinate[label={right:$y$}];
\node[color=darkred] at (\xstern,{\a*\a}) [above] {$\eta(x)^2$};
\draw[color=blue] plot[domain=-4:8,samples=100] ({\x},{S(\x)});
\draw[color=darkred]
	(-4,0)
	--
	({\xstern-\d},0)
	--
	plot[domain={\xstern-\d}:{\xstern+\d},samples=\N]
		({\x},{eta(\x)*eta(\x)})
	--
	({\xstern+\d},0)
	--
	(8,0);
\draw (-4,-0.05) -- (-4,0.05);
\draw (8,-0.05) -- (8,0.05);
\node at (-4,-0.05) [below] {$x_1\mathstrut$};
\node at (8,-0.05) [below] {$x_2\mathstrut$};
\node at ({\xstern-\d},-0.05) [below left] {$x_*-d$};
\node at ({\xstern+\d},-0.05) [below right] {$x_*+d$};
\node[color=blue] at (-2,1.5) {$S(x)$};
\draw[color=blue,line width=0.3pt,shorten <= 0.3cm]
	(-2,1.5) -- (-0.5,{S(-0.5)});
\end{scope}

\begin{scope}[yshift=-7cm]
\fill[color=blue!10]
	({\xstern-\d},0) rectangle ({\xstern+\d},{R(\xstern-\d)+0.1});
\draw[->] (-4.2,0) -- (8.5,0) coordinate[label={$x$}];
\fill [color=darkred!20] ({\xstern-\d},0) rectangle ({\xstern+\d},{\B*\B});
\draw[color=blue,line width=0.2pt]
	({\xstern-\d},{R(\xstern-\d)+0.1})
	--
	({\xstern+\d},{R(\xstern-\d)+0.1});
\node[color=blue] at (\xstern,{R(\xstern-\d)+0.1}) [above] {$-a$};
\node[color=blue] at (\xstern,{R(\xstern)}) [below] {$R(x) < -a$};
\draw[->] (0,-3) -- (0,3.3) coordinate[label={right:$y$}];
%\node[color=darkred] at (\xstern,{\B*\B})
%	[above] {$\displaystyle\frac{a(b+\frac12)\pi}{d}$};
\node[color=darkred] at (\xstern,{\B*\B}) [above] {$\eta'(x)^2$};
\draw[color=blue] plot[domain=-4:8,samples=100] ({\x},{R(\x)});
\draw[color=darkred]
	(-4,0)
	--
	({\xstern-\d},0);
\draw[color=darkred,smooth]
	plot[domain={\xstern-\d}:{\xstern+\d},samples=\N]
		({\x},{etastrich(\x)*etastrich(\x)});
\draw[color=darkred]
	({\xstern+\d},0)
	--
	(8,0);
\draw (-4,-0.05) -- (-4,0.05);
\draw (8,-0.05) -- (8,0.05);
\node at (-4,-0.05) [below] {$x_1\mathstrut$};
\node at (8,-0.05) [below] {$x_2\mathstrut$};
\node at ({\xstern-\d},-0.05) [above left] {$x_*-d$};
\node at ({\xstern+\d},-0.05) [above right] {$x_*+d$};
\node[color=blue] at (-2,1.5) {$R(x)$};
\draw[color=blue,line width=0.3pt,shorten <= 0.3cm]
	(-2,1.5) -- (-1.5,{R(-1.5)});
\end{scope}

\end{tikzpicture}
\end{document}


Wir zeigen mit Hilfe eines Widerspruchs, dass es keine Stelle $x$
geben kann, für die $R(x)<0$ ist.
Nehmen wir also an, es gäbe eine Stelle $x_*$ derart, dass
$R(x_*)<0$.
Da die Funktion $R(x)$ stetig ist, wird $R(x)$ auch noch in einer
Umgebung $(x_*-d,x_*+d)$ negativ sind.
Wir dürfen daher annehmen, dass es eine positive Zahl $a$ gibt
derart, dass $R(x)<-a$ für $x$ mit $|x-x_*|<d$.

Wir müssen jetzt zeigen, dass sich immer eine Funktion $\eta(x)$ finden
lässt, die das Integral $\delta^2 I(y)$ negativ macht, selbst wenn
der Koeffizient $S(x)$ auf dem Teilintervall
$(x_*-d,x_*+d)$ positiv ist.
Wir müssen zeigen, dass $\eta(x)$ so gewählt werden kann, dass der
Term mit $\eta(x)^2$ beliebig klein, der Term mit $\eta'(x)^2$ aber
beliebig negativ gemacht werden kann.

Wir verwenden die Funktion
\[
\eta(x)
=
\begin{cases}
\displaystyle
b\cos \frac{(c+\frac12)\pi (x-x_*)}{d}&\qquad |x-x_*|<d\\[3pt]
0&\qquad \text{sonst},
\end{cases}
\]
wobei $c\in\mathbb{N}$ eine natürliche Zahl ist
(Abbildung~\ref{buch:variation2:legendre:fig:legendre}).
Dies stellt sicher, dass $\eta(x)$ stetig ist,
wenn auch nicht differenzierbar an den Stellen $x_*\pm d$.

Zunächst ist $|\eta(x)|\le b$ für alle $x$, der erste Summand
des Integrals kann daher wegen
\begin{align*}
\biggl|
\int_{x_1}^{x_2} S(x)\eta(x)^2\,dx
\biggr|
&=
\biggl|
\int_{x_*-d}^{x_*+d} S(x)\eta(x)^2\,dx
\biggr|
\\
&\le
\int_{x_*-d}^{x_*+d} |S(x)|\cdot |\eta(x)|^2\,dx
\\
&\le
b^2
\int_{x_*-d}^{x_*+d} |S(x)|\,dx
\end{align*}
beliebig klein gemacht werden, indem $a$ klein gewählt wird.
Für die Ableitung gilt
\[
\eta'(x)
=
\begin{cases}
\displaystyle
-\frac{b(c+\frac12)\pi}{d}\sin\frac{(c+\frac12)\pi(x-x_*)}{d}&\qquad |x-x_*|<d\\[3pt]
0&\qquad\text{sonst.}
\end{cases}
\]
Der zweite Summand des Integrals wird damit zu
\begin{align*}
\int_{x_1}^{x_2} R(x) \eta'(x)^2\,dx
&=
\int_{x_*-d}^{x_*+d} R(x)\eta'(x)^2\,dx
\\
&\le 
\int_{x_*-d}^{x_*+d} (-a) \eta'(x)^2\,dx
\\
&=
-ab^2\int_{x_*-d}^{x_*+d} \frac{(c+\frac12)^2\pi^2}{d^2}
\sin^2 \frac{(c+\frac12)\pi(x-x_*)}{d}\,dx.
\intertext{Das Integral von $\sin^2 t$ über eine ganz Anzahl von
Perioden ist die halbe Intervalllänge, somit wird das Integral zu}
&=
-ab^2
\frac{(c+\frac12)^2\pi^2}{d^2}
d
=
-\frac{ab^2(c+\frac12)^2\pi^2}{d}
\end{align*}
(siehe Lemma~\ref{buch:variation2:legendre:lemma:sin} weiter unten).
Die rechte Seite kann beliebig gross gemacht werden, indem ein
geeignet grosses $c$ gewählt wird.
So wird die zweite Variation
\begin{align}
\delta^2I(y)
&=
\int_{x_1}^{x_2} S(x)\eta(x)^2 + R(x)\eta'(x)^2\,dx
\\
&=
\int_{x_1}^{x_2} S(x)\eta(x)^2\,dx
+
\int_{x_1}^{x_2} R(x)\eta'(x)^2\,dx
\notag
\\
&\le
\int_{x_*-d}^{x_*+d}|S(x)|\eta(x)\,dx
+
\int_{x_*-d}^{x_*+d} R(x) \eta(x)\,dx
\notag
\\
&\le
b^2\int_{x_*-d}^{x_*+d} |S(x)|\,dx
-
\frac{ab^2(c+\frac12)^2\pi^2}{d}
\notag
\\
&=
\frac{ab^2\pi^2}{d}
\biggl(
\frac{d}{a\pi^2}
\int_{x_*-d}^{x_*+d}|S(x)|\,dx
-
(c+{\textstyle\frac12})^2
\biggr).
\label{buch:variation2:legendre:eqn:bgl}
\end{align}
Indem man 
\[
(c+{\textstyle\frac12})^2
>
\frac{d}{a\pi^2}\int_{x_*-d}^{x_*+d} |S(x)|\,dx
\]
wählt, wird die Klammer in \eqref{buch:variation2:legendre:eqn:bgl}
negativ, so dass die zweite Variation negativ wird.
Dies widerspricht der Annahme über die zweite Variation.
Der Widerspruch zeigt, dass $R(x)\ge 0$ sein muss.
\end{proof}

\begin{lemma}
\label{buch:variation2:legendre:lemma:sin}
Für $c\in\mathbb{N}$ gilt
\[
\int_{-d}^d
\sin^2 \frac{(c+\frac12)\pi t}{d}
\,dt
=
d.
\]
\end{lemma}

\begin{proof}
%
% sin2.tex -- Oszillationen von sin^2
%
% (c) 2021 Prof Dr Andreas Müller, OST Ostschweizer Fachhochschule
%
\documentclass[tikz]{standalone}
\usepackage{amsmath}
\usepackage{times}
\usepackage{txfonts}
\usepackage{pgfplots}
\usepackage{csvsimple}
\usetikzlibrary{arrows,intersections,math}
\definecolor{darkred}{rgb}{0.8,0,0}
\begin{document}
\def\skala{1}
\def\d{4.8}
\def\a{4}
\def\b{3}
\begin{tikzpicture}[>=latex,thick,scale=\skala,
declare function={
	f(\x) = \a*sin((\b+0.5)*180*\x/\d);
	g(\x) = f(\x)*f(\x)/\a;
}]

\fill[color=gray!20] (-\d,0) rectangle (\d,\a);
\fill[color=darkred!20]
	(-\d,0)
	--
	plot[domain=-\d:\d,samples=100]
	({\x},{g(\x)})
	--
	(\d,0)
	--
	cycle;

\draw[color=blue,smooth,line width=1.2pt]
	plot[domain=-\d:\d,samples=100] ({\x},{f(\x)});
\draw[color=darkred,smooth,line width=1.2pt]
	plot[domain=-\d:\d,samples=100] ({\x},{g(\x)});

\draw[->] (-5,0) -- (7.3,0) coordinate[label={$t$}];
\draw[->] (0,-4.1) -- (0,4.5) coordinate[label={right:$y$}];

\draw (-0.05,4) -- (0.05,4);
\node at (-0.05,4) [above left] {$1$};
\draw (-0.05,-4) -- (0.05,-4);
\node at (-0.05,-4) [left] {$-1$};

\draw (-\d,-0.05) -- (-\d,0.05);
\node at (-\d,-0.05) [below] {$-d\mathstrut$};
\draw (\d,-0.05) -- (\d,0.05);
\node at (\d,-0.05) [below] {$d\mathstrut$};

\coordinate (A) at ({-3*\d/7},{-6*\a/7});
\node[color=blue] at (A) {$\displaystyle\sin\frac{(c+\frac12)\pi t}{d}$};
\draw[color=blue,line width=0.2pt,shorten <= 0.6cm]
	(A) -- ({-9*\d/14},{f(-9*\d/14)});

\coordinate (B) at (6.0,{3*\a/4});
\node[color=darkred] at (B) {$\displaystyle\sin^2\frac{(c+\frac12)\pi t}{d}$};
\draw[color=darkred,line width=0.2pt,shorten <= 0.65cm]
	(B) -- ({13*\d/14},{g(13*\d/14)});

\end{tikzpicture}
\end{document}


Wegen
\[
\sin^2 t
=
\frac12 - \frac12\cos 2t
\]
ist das Integral die Fläche unter einer um $\frac12$ nach oben
verschobenen Cosinus-Kurve mit doppelter Frequenz.
Diese Kurve halbiert das Rechteck mit den Ecken $(-d,0)$ und $(d,1)$
mit dem Flächeninhalt $2d$.
Daher ist der Wert des Integrals $2d/2=d$.
\end{proof}

\begin{beispiel}
\label{buch:variation2:legendre:bsp:ebene}
Das Variationsproblem mit der Legendre-Funktion
$L(x,y,y')=\sqrt{1+y^{\prime 2}}$ findet kürzeste Verbindungen in
der Ebene.
Zur Überprüfung der Legendre-Bedingung muss die zweite Ableitung von
$L$ nach $y'$ berechnet werden.
Die erste Ableitung ist
\begin{align}
\frac{\partial L}{\partial y'}
&=
\frac{y'}{\sqrt{1+y^{\prime 2}}}
\notag
\intertext{und die zweite mit der Quotientenregel}
\frac{\partial^2 L}{\partial y^{\prime 2}}
&=
\frac{1\cdot \sqrt{1+y^{\prime 2}}}{1+y^{\prime 2}}
-
\frac{y'\cdot y'}{(1+y^{\prime 2})^{\frac32}}
=
\frac{
(1+y^{\prime 2})  - y^{\prime 2}
}{
(1+y^{\prime 2})^{\frac32}
}
=
\frac{1}{
(1+y^{\prime 2})^{\frac32}
}
> 0.
\label{buch:variation2:legendre:bsp:ebene:R}
\end{align}
Die starke Legrendre-Bedingung ist also immer erfüllt.
\end{beispiel}

%
% 3-diffgl.tex
%
% (c) 2024 Prof Dr Andreas Müller
%
\section{Nullstellen von Lösungen linearer Differentialgleichungen
\label{buch:variation2:section:diffgl}}

%
% 4-jacobi.tex
%
% (c) 2024 Prof Dr Andreas Müller
%
\section{Das Jacobi-Kriterium
\label{buch:variation2:section:jacobi}}
\kopfrechts{Das Jacobi-Kriterium}
Die zweite Variation in der Form
\eqref{buch:variation2:zweitevariation:eqn:SRintegral}
zeigt, dass es für die Beurteilung, ob tatsächlich ein Extremum 
vorliegt, auf das Verhalten der Funktionen $S(x)$  und $R(x)$ ankommt.
In einem Minimum garantiert die Legendre-Bedingung bereits,
dass $R(x)\ge 0$ ist.
Es bleibt noch zu untersuchen, ob sich Bedinungen an die Funktion
$S(x)$ finden lassen, die garantieren können, dass ein Minimum
vorliegt.

%
% \delta^2 I(y)\ge 0 als Variationsproblem
%
\subsection{$\delta^2 I(y)\ge 0$ als Variationsproblem
\label{buch:variation2:jacobi:subsection:delta2I}}
Wir möchten herausfinden, ob die zweite Variation negativ werden kann.
Falls ja müsste sich eine Funktion $\eta(x)$ finden lassen, für die
der Wert des Integrals
\eqref{buch:variation2:zweitevariation:eqn:SRintegral}
negativ ist.
Ob es so eine Funktion gibt könnten wir dadurch entscheiden, dass wir
das Minimum des Integrals
\eqref{buch:variation2:zweitevariation:eqn:SRintegral}
suchen.
Damit ist ein Variationsproblem entstanden, auf das wir wieder die
bereits entwickelte Theorie anwenden können.

Zur Lösung des Variationsproblems scheiben wir $u(x)$ für die gesuchte
Variationsfunktion $\eta(x)$ und betrachten das Funktional
\begin{equation}
K(u)
=
\int_{x_0}^{x_1}
S(x) u(x)^2 + R(x) u'(x)^2
\,dx
\label{buch:variation2:jacobi:eqn:K}
\end{equation}
mit Randbedingungen $u(x_0)=u(x_1)=0$ und versuchen das Minimum
zu bestimmen.
Für $u(x)=0$ ist $K(0)=0$.
Sollte es aber Funktionen $u(x)$ geben, die $K(u)$ negativ machen,
dann kann die Funktion $y(x)$ von der ausgehen die Funktionen $S(x)$
und $R(x)$ konstruiert wurden, kein Minimum sein.

%
% Die Jacobi-Differentialgleichung
%
\subsubsection{Die Jacobi-Differentialgleichung}
Das Funktional
\eqref{buch:variation2:jacobi:eqn:K}
hat die Lagrange-Funk\-tion 
\[
G(x,u,u') = S(x) u^2 + R(x) u^{\prime 2}
\]
mit den Ableitungen
\begin{align*}
\frac{\partial G}{\partial u}
&=
2S(x) u
&
\frac{\partial G}{\partial u'}
&=
2R(x) u'.
\end{align*}
Daraus ergibt sich die Euler-Lagrange-Differentialgleichung
\[
2S(x) u(x) - 2\frac{d}{dx} R(x) u'(x) = 0.
\]
Der gemeinsame Faktor $2$ kann weggelassen werden.
Ausserdem ist es üblich, die Differentialgleichung mit umgekehrtem
Vorzeichen hinzuschreiben, also als
\begin{equation}
\frac{d}{dx} R(x)\frac{d}{dx} u(x) - S(x) u(x) = 0.
\label{buch:variation2:jacobi:eqn:jacobisl}
\end{equation}
Dies ist eine lineare Differentialgleichung zweiter Ordnung für die Funktion
$u(x)$.

\begin{definition}[Jacobi-Differentialgleichung]
Die Differentialgleichung
\eqref{buch:variation2:jacobi:eqn:jacobisl}
heisst die {\em Jacobi-Differentialgleichung}
für die Extermale $y(x)$.
\end{definition}

\subsection{Nullstellen von Lösungen der Jacobi-Differentialgleichung
\label{buch:variation2:jacobi:subection:nullstellen}}
Die Jacobi-Differentialgleichung ist nahe an der Form einer
Sturm-Liouville-Differential\-gleichung,
wie sie in Abschnitt~\ref{buch:variation2:section:diffgl} genauer
untersucht wurde.
Insbesondere wurden dort Aussagen über die Lage der Nullstellen von
Lösungen gefunden, die weiter unten nützlich sein werden.

%
% Die starke Legendre-Bedingung
%
\subsubsection{Die starke Legendre-Bedingung}
Durchführung der Ableitung im ersten Term der Jacobi-Differentialgleichung
führt auf
\[
R(x) u''(x) + R'(x) u'(x) - S(x) u(x) = 0,
\]
während eine Sturm-Liouville-Differentialgleichung bei $u''(x)$ den
Koeffizienten $1$ hat.
Die Form einer Sturm-Liouville-Differentialgleichung kann erreicht
werden, wenn $R(x)>0$ ist, denn in diesem Fall kann man die
Differentialgleichung durch $R(x)$ dividieren und die Gleichung
\[
u''(x) + \frac{R'(x)}{R(x)} u'(x) -\frac{S(x)}{R(x)} u(x) = 0
\]
erhalten.

\begin{definition}[Starke Legendre-Bedingung]
Die Lagrange-Funktion $F(x,y,y')$ erfüllt die starke Legendre-Bedingung
für die Extremale $y(x)$, wenn gilt
\[
\frac{\partial^2 F}{\partial y^{\prime 2}}(x,y(x),y'(x)) > 0
\]
für alle $x\in [x_0,x_1]$.
\end{definition}

Unter der Voraussetzung der starken Legendre-Bedingung ist die
Jacobi-Differential\-glei\-chung also äquivalent zu einer
Sturm-Liouville-Differentialgleichung.
In Abschnitt~\ref{buch:variation2:section:diffgl}
wurde die die Vorzeichen und Nullstellen der Lösungen einer solchen
Gleichung untersucht.
Diese Eigenschaften gelten daher auch für die Jacobi-Differentialgleichung.

%
% Berechnung von K(u) aus L(u)
%
\subsubsection{Berechnung von $K(u)$ aus $L(u)$}
Wir schreiben die linke Seite der Jacobi-Differentialgleichung auch
in der Form
\begin{equation}
L(u)
=
\frac{d}{dx}\frac{\partial G}{\partial y'}
-
\frac{\partial G}{\partial y}
=
\frac{d}{dx}(Ru') 
-
Su.
\label{buch:variation2:jacobi:eqn:L}
\end{equation}
Falls die Funktion $u(x)$, die keine Lösung der Jacobi-Differentialgleichung
zu sein braucht, die Randbedingungen $u(x_0)=u(x_1)=0$ erfüllt,
kann das Integral
\begin{align}
\int_{x_0}^{x_1} u(x) L(u(x)) \,dx
&=
\int_{x_0}^{x_1} u(x)
\biggl(
\frac{d}{dx}\bigl(R(x)u'(x)\bigr) 
-
S(x) u(x)
\biggr)
\,dx
\notag
\intertext{mit partieller Integration im ersten Term berechnet werden.
Es ergibt}
&=
\biggl[
u(x)R(x)u'(x)
\biggr]_{x_0}^{x_1}
-
\int_{x_0}^{x_1}
u'(x) \bigl( R(x)u'(x) \bigr)
+
u(x) S(x) u(x)
\,dx.
\notag
\intertext{Der erste Term verschwindet wegen der Randbedingung, der
Rest kann zu}
&=
-
\int_{x_0}^{x_1} R(x) u'(x)^2 + S(x) u(x)^2 \,dx
=
- K(u).
\label{buch:variation2:jacobi:eqn:LK}
\end{align}
vereinfacht werden.
Das Funktional $K(u)$ kann also mit der linken Seite der
Jacobi-Diffe\-ren\-tial\-gleichung ausgedrückt werden.

%
% Nullstellen im Inneren von [x_0,x_1]
%
\subsubsection{Nullstellen im Inneren von $[x_0,x_1]$}
Auf die Jacobi-Differentialgleichung sind wir gestossen, weil wir nach
Variationsfunktionen $\eta$ gesucht haben, die die zweite Variation
negativ machen könnten.
Unter der Voraussetzung der starken Legendre-Bedingung sind den
Nullstellen einer Lösung starke Einschränkungen auferlegt.
Sei also $u_0(x)$ eine Lösung der Jacobi-Differentialgleichung mit den
Anfangsbedingungen
\[
u_0(x_0) = 0
\qquad\text{und}\qquad
u_0'(x_0) = 1
\]
und sei $x_*$ die kleinste, von $x_0$ verschiedene Nullstelle von $u_0(x)$
im Inneren von $[x_0,x_1]$.
Insbesondere ist $u_0(x)>0$ im Teilintervall $(x_0,x_*)$.

Aus der Funktion $u_0(x)$ konstruieren wir jetzt eine neue Funktion
\[
u_1(x)
=
\begin{cases}
u_0(x) &\qquad \text{für $x\in [x_0,x_*]$}\\
0      &\qquad \text{sonst.}
\end{cases}
\]
Diese Funktion ist stetig, aber nicht differenzierbar an der Stelle
$x_*$.
Sie erfüllt aber die Randbedingungen $u_1(x_0)=u_1(x_1)=0$, so dass
\eqref{buch:variation2:jacobi:eqn:LK}
zur Berechnung von $K(u_1)$ verwendet werden kann.
Insbesondere gilt
\[
K(u_1)
=
-
\int_{x_0}^{x_*}
u_0
\underbrace{L(u_0)}_{\displaystyle=0}
\,dx
-
\int_{x_*}^{x_1}
0
\cdot L(0)
\,dx
=
0.
\]

Die Funktion $u_1$ ist aber an der Stelle $x_*$ nicht differenzierbar,
sie kann also nur dann ein Extremum sein, wenn sie an der Stelle
$x_*$ die weierstrasse-erdmannsche Eckenbedingung erfüllt.

XXX Nachprüfung der Eckenbedingung

\begin{satz}
Erfüllt die Extremale $y(x)$ die starke Legendre-Bedingung und hat
die Lösung $u_0(x)$ der Jacobi-Differentialgleichung mit Anfangsbedingung
$u_0(x_0)=0$ und $u_0'(x_1)=1$ eine Nullstelle im Inneren des Intervalls
$[x_1,x_2]$, dann ist $I(y)$ kein Minimum.
\end{satz}

%
%
%
\subsubsection{}

%
%
%
\subsubsection{}

Ist die Funktion $u(x)$ eine Lösung der
Differentialgleichung~\eqref{buch:variation2:jacobi:eqn:jacobisl}
mit $u(x_0)=u(x_1)=0$, dann ist auch jedes Vielfache von $u(x)$
eine Lösung mit den gleichen Randbedingungen, und die zweite Variation
des ursprünglichen Funktionals verschwindet für alle Werte von $\varepsilon$.
Falls es keine solche Funktion $u(x)$ gibt, dann ist die zweite
Variation $\ne 0$.
Die Eigenschaft, ob ein Minimum oder Maximum vorliegt, wird also
dadurch entschieden, ob es eine Lösung der Differentialgleichung
\eqref{buch:variation2:jacobi:eqn:jacobisl} mit
Randwerten $u(x_0)=u(x_1)=0$ gibt.

Ist $u(x)$ eine nichttriviale Lösung der Differentialgleichung
\eqref{buch:variation2:jacobi:eqn:jacobisl} mit $u(x_0)=0$,
dann muss $u'(x_0)\ne 0$ sein.
Durch Division durch $u'(x_0)$ erhalten wir eine Lösung, die
sogar $u'(x_0)=1$ hat.
Durch diese Anfangswerte ist aber die Lösung der Differentialgleichung
vollständig bestimmt.
Die Frage ob ein Minimum oder Maximum vorliegt, wird also
durch die Lage der Nullstellen von Lösungen der Differentialgleichung
\eqref{buch:variation2:jacobi:eqn:jacobisl} bestimmt.


%Aus der Definition des Funktionals $K(u)$ erhält man durch
%partielle Integration für Funktionen mit den Randbedingungen
%$u(x_0)=u(x_1)=0$ 
%\begin{align*}
%K(u)
%&=
%\int_{x_0}^{x_1} S(x)u(x)^2 + R(x) u'(x)^2 \,dx
%\\
%&=
%\int_{x_0}^{x_1} S(x)u(x)^2 \,dx
%+
%\int_{x_0}^{x_1} R(x) u'(x)^2 \,dx
%\\
%&=
%\int_{x_0}^{x_1} S(x)u(x)^2 \,dx
%+
%\int_{x_0}^{x_1} (R(x) u'(x)) u'(x) \,dx
%\\
%&=
%\int_{x_0}^{x_1} S(x)u(x)^2 \,dx
%+
%\biggl[R(x)u'(x)u(x)\biggr]_{x_0}^{x_1}
%-
%\int_{x_0}^{x_1} \frac{d}{dx}(R(x) u'(x)) u(x) \,dx
%\\
%&=
%\int_{x_0}^{x_1}
%\biggl(S(x) u(x) - \frac{d}{dx}\bigl(R(x)u'(x)\bigr)\biggr) u(x)
%\,dx
%\\
%&=
%-\int_{x_0}^{x_1} L(u) u(x) \,dx.
%\end{align*}


%
% Kürzeste Verbindung auf einem Rotationsellipsoid
%
\subsection{Kürzeste Verbindung auf einem Rotationsellipsoid}
Seit über 2000 Jahren ist der Menschheit bekannt, dass die Erde
ungefähr die Form einer Kugel hat.
Erathostenes gelang es als erstem, den Umfang der Erde mit
einem Fehler von weniger als $2\%$ zu bestimmen.
Die Entfernungsmessung auf einer Kugeloberfläche ist deutlich
komplizierter als auf einer Ebene.
Bereits im Altertum wurde die sphärische Trigonometrie entwickelt,
mit der sich Winkel und Seitenlägen von beliebigen Dreiecken auf
einer Kugeloberfläche berechnen lassen.
Van Brummelen \cite{buch:heavenly} gibt eine sorgfältige Einführung
in die Geschichte und die Anwendungen der sphärischen Geometrie.

\begin{beispiel}
Die kürzesten Verbindungen zweier Punkte auf einer Kugel ist ein
Grosskreis.
Liegen die Punkte weniger als $180^\circ$ Grad auseinander, gibt es
nur eine kürzeste Verbindung, das Längenfunktional nimmt ein
absolutes Minimum an.

Sind die Punkte jedoch Antipodenpunkte, dann gibt es unendliche viele
Grosskreise durch beide Punkte.
Das Längenfunktional hat kein absolutes Minimum mehr, die Drehung eines
Grosskreises um die Achse durch die beiden Punkte ist eine Variation,
die das Funktional nicht ändert.
Die beiden Punkte sind konjugierte Punkte.
\end{beispiel}

Die Erde ist aber nicht eine Kugel, sondern eher ein Rotationsellipsoid,
der Durchmesser entlang der Achse ist verschieden vom Durchmesser des 
Äquators.
Im 17. Jahrhundert gab es einen wissenschaftlichen Disput darüber,
ob der Durchmesser entlang der Achse kleiner oder grösser als der
Äquatordurchmesser ist.
Newton und Huygens hatten aufgrund der allgemeinen Mechanik eine Abplattung
der Erde vorhergesagt, die durch von Jean Richter und Philipp de La Hire
durch die erste Vermessung des Merdianbogens durch Paris bestätigt 
worden war.
Die Sache war aber immer noch nicht vollständig geklärt.
Der Astronom Jacques Cassini schloss zu Beginn des 18. Jahrhunderts,
dass der Polradius grösser als der Äquatorradius sein müsse.
König Ludwig XV.~beauftragte daher Maupertuis mit einer Expedition
nach Lappland.
Als Teil der Expedition soll der Abstand zweier Breitengrade
vermessen werden.
Wenn der Radius im hohen Norden zunehmen abnimmt, wird auch die Länge
eines Merdianbogens zu vorgegebenem Winkel kürzer.
Die Expedition von Maupertuis und eine zeitgleich stattfindende Expedtion
nach Ecuador bestätigten beide die Abplattung.

\begin{beispiel}
Unter der Annahme, dass die Erde ein Rotationsellipsoid mit Äquatorradius
$1$ und Polradius $r$ ist, soll der kürzeste Weg zwischen zwei Punkten
auf dem Äquator bestimmt werden.
\end{beispiel}





