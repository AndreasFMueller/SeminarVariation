%
% chapter.tex
%
% (c) 2023 Prof Dr Andreas Müller
%
\chapter{Zweite Variation
\label{buch:chapter:variation2}}
\kopflinks{Zweite Variation}
Mit Hilfe der zweiten Ableitung ist es bei einem Extremalproblem
mit endlich vielen Variablen möglich, hinreichende Bedingungen dafür
zu formulieren, dass ein Maximum oder Minimum vorliegt.
Es liegt nahe, solche Kriterien auch für Variationsproblem
zu erwarten.
Schliesslich lässt sich auch die Variation als Funktion
$f(\varepsilon) = I(y+\varepsilon\eta)$ als Potenzreihe
\[
f(\varepsilon)
=
I(y) + \varepsilon \delta I(y) + \frac{f''(0)}{2!}\varepsilon^2 + \dots
\]
in $\varepsilon$ entwickeln.
Notwendig für ein Extremum ist das verschwinden der ersten Variation
$\delta I(y)=0$, wie das in früheren Kapiteln ausführlich verwendet
worden ist.
Falls ausserdem $f''(0)>0$ ist, kann man schliessen, dass die
Funktion $f(\varepsilon)$ ein Minimum an der Stelle $\varepsilon=0$
hat.
Dies garantiert aber noch nicht, dass das Funktional $I(y)$
ein Minimum hat, dazu müssen alle denkbaren Funktionen $\eta(x)$
untersucht werden, also eine unendliche Menge von ``Richtungen''.

Für Funktionen $\mathbb{R}^n\to\mathbb{R}: x\mapsto f(x)$ von
$n$ Variablen sind Richtungsableitungen in alle Richtungen zu
untersuchen.
Nach Abschnitt~\ref{buch:fuvar:section:hessesche} liegt ein
ein Minimum vor, wenn $\operatorname{grad}f(x_*)=0$ und ausserdem
die hessesche Matrix an der gleichen Stelle positiv definit ist.

Es stellt sich aber heraus, dass es für Variationsprobleme viel
schwieriger ist, hinreichende Bedingungen für ein Extremum zu finden.
Da es unendlich viele mögliche Richtungen gibt, kann zwar für alle
möglichen Funktion $\eta$ die zweite Ableitung $f''(0)>0$ sein, aber 
sie kann trotzdem beliebig klein werden.
Damit lässt sich nicht mehr schliessen, dass der Punkt tatsächlich
ein Minimum ist.

In diesem Abschnitt werden zunächst aus der sogenannten zweiten
Variation notwendige Bedingungen für ein Maximum oder Minimum
hergeleitet.
Nach einigen Vorbereitungen über die Eigenschaften von Nullstellen
von Lösungen von linearen Differentialgleichungen zweiter Ordnung
wird dann die Jacobi-Differentialgleichung und das daraus ableitbare
Jacobi-Kriterium für ein Maximum oder Minimum vorgestellt.

\input{chapters/060-variation2/1-zweitevariation.tex}
\input{chapters/060-variation2/2-legendre.tex}
\input{chapters/060-variation2/3-diffgl.tex}
\input{chapters/060-variation2/4-jacobi.tex}

