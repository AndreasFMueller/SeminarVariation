%
% sin2.tex -- Oszillationen von sin^2
%
% (c) 2021 Prof Dr Andreas Müller, OST Ostschweizer Fachhochschule
%
\documentclass[tikz]{standalone}
\usepackage{amsmath}
\usepackage{times}
\usepackage{txfonts}
\usepackage{pgfplots}
\usepackage{csvsimple}
\usetikzlibrary{arrows,intersections,math}
\definecolor{darkred}{rgb}{0.8,0,0}
\begin{document}
\def\skala{1}
\def\d{4.8}
\def\a{4}
\def\b{3}
\begin{tikzpicture}[>=latex,thick,scale=\skala,
declare function={
	f(\x) = \a*sin((\b+0.5)*180*\x/\d);
	g(\x) = f(\x)*f(\x)/\a;
}]

\fill[color=gray!20] (-\d,0) rectangle (\d,\a);
\fill[color=darkred!20]
	(-\d,0)
	--
	plot[domain=-\d:\d,samples=100]
	({\x},{g(\x)})
	--
	(\d,0)
	--
	cycle;

\draw[color=blue,smooth,line width=1.2pt]
	plot[domain=-\d:\d,samples=100] ({\x},{f(\x)});
\draw[color=darkred,smooth,line width=1.2pt]
	plot[domain=-\d:\d,samples=100] ({\x},{g(\x)});

\draw[->] (-5,0) -- (7.3,0) coordinate[label={$t$}];
\draw[->] (0,-4.1) -- (0,4.5) coordinate[label={right:$y$}];

\draw (-0.05,4) -- (0.05,4);
\node at (-0.05,4) [above left] {$1$};
\draw (-0.05,-4) -- (0.05,-4);
\node at (-0.05,-4) [left] {$-1$};

\draw (-\d,-0.05) -- (-\d,0.05);
\node at (-\d,-0.05) [below] {$-d\mathstrut$};
\draw (\d,-0.05) -- (\d,0.05);
\node at (\d,-0.05) [below] {$d\mathstrut$};

\coordinate (A) at ({-3*\d/7},{-6*\a/7});
\node[color=blue] at (A) {$\displaystyle\sin\frac{(c+\frac12)\pi t}{d}$};
\draw[color=blue,line width=0.2pt,shorten <= 0.6cm]
	(A) -- ({-9*\d/14},{f(-9*\d/14)});

\coordinate (B) at (6.0,{3*\a/4});
\node[color=darkred] at (B) {$\displaystyle\sin^2\frac{(c+\frac12)\pi t}{d}$};
\draw[color=darkred,line width=0.2pt,shorten <= 0.65cm]
	(B) -- ({13*\d/14},{g(13*\d/14)});

\end{tikzpicture}
\end{document}

