%
% 3-diffgl.tex
%
% (c) 2024 Prof Dr Andreas Müller
%
\section{Nullstellen von Lösungen linearer Differentialgleichungen
\label{buch:variation2:section:diffgl}}
\kopfrechts{Nullstellen von Lösungen}
In Abschnitt~\ref{buch:variation2:section:jacobi} wird das
Jacobi-Kriterium für die Existenz eines Extremums formuliert.
Es basiert auf Eigenschaften der Lösungen der Jacobi-Differentialgleichung,
einer linearen Differentialgleichung zweiter Ordnung.
Diese Eigenschaften sind nicht spezifisch für das Variationsproblem
und werden daher in diesem Abschnitt allgemein dargestellt.

Wir betrachten die lineare Differentialgleichung zweiter Ordnung
\begin{equation}
y'' + p(x) y' + q(x) = 0.
\label{buch:variation2:diffgl:eqn:dgl}
\end{equation}
Die Koeffizienten $p(x)$ und $q(x)$ sind stetige Funktionen auf dem
abgeschlossenen Intervall $[x_0,x_1]$.
In expliziter Form kann sie auch als
\begin{equation}
y'' = - p(x) y' - q(x)
\label{buch:variation2:diffgl:eqn:explizit}
\end{equation}
geschrieben werden.
Für die numerische Rechnung und für Fragen über Existenz und Eindeutigkeit
der Lösung ist die Vektorschreibeise
\begin{equation}
\frac{d}{dx}
\begin{pmatrix} y\\ y' \end{pmatrix}
=
\underbrace{
\begin{pmatrix}
    0&    1\\
-q(x)&-p(x)
\end{pmatrix}
}_{\displaystyle =A(x)}
\begin{pmatrix} y\\ y' \end{pmatrix}
\qquad\Rightarrow\qquad
\frac{d}{dx}v = A(x) v
\label{buch:variation2:diffgl:eqn:vektordgl}
\end{equation}
der Differentialgleichung nützlicher.
Der Satz von Picard und Lindelöf besagt, dass die Differentialgleichung
\eqref{buch:variation2:diffgl:eqn:vektordgl} eine eindeutige
Lösung hat, wenn die rechte Seite eine Lipschitz-Bedingung
erfüllt.
Die Lipschitz-Bedingung verlangt im vorliegenden Fall, dass
\[
|A(x)v_1 - A(x)v_2|
\le
L |v_1-v_2|
\]
für eine geeignete Lipschitz-Konstante $L$ gilt.
Wegen
\[
|A(x)v_1-A(x)v_2|
=
|A(x)(v_1-v_2)|
\le
\|A(x)\|\cdot |v_1-v_2|
\]
ist die Lipschitz-Bedingung mit der Lipschitz-Konstanten
$L=\|A(x)\|$ immer erfüllt.
Zu vorgegebenen Anfangswerten $y_0$ und $y'_0$ gibt es also
immer eine eindeutige Lösung $y(x)$ mit $y(x_0)=y_0$ und
$y'(x_0)=y_0$.

%
% Keine mehrfachen Nullstellen
%
\subsection{Keine mehrfachen Nullstellen
\label{buch:variation2:diffgl:subsection:mehrfachenullstellen}}
Sei $y(x)$ eine nichttriviale Lösung der Differentialgleichung
\eqref{buch:variation2:diffgl:eqn:dgl} und $x_0$ eine
Nullstelle $y(x_0)=0$.
Die Stelle $x=x_0$ kann keine mehrfache Nullstelle sein, denn dann
wäre auch $y'(x_0)=0$, $y(x)$ wäre also eine Lösung der Differentialgleichung
mit Anfangsbedingungen $y(x_0)=0$ und $y'(x_0)=0$. 
Da die Nullfunktion eine Lösung mit den selben Anfangsbedingungen ist,
folgt aus dem Satz von Picard-Lindelöf, dass $y(x)=0$ sein müsste, im
Widerspruch zur Annahme, dass $y(x)$ eine nichttriviale Lösung ist.

Für jede Nullstelle $y_0$ von $y(x)$ ist also $y(x_0)\ne 0$.
Insbesondere wechselt $y(x)$ bei jeder Nullstelle das Vorzeichen.


%
% Linear unabhängige Lösungen
%
\subsection{Linear unabhängige Lösungen
\label{buch:variation2:diffgl:subsetion:linunabh}}
Seien jetzt $y_1(x)$ und $y_2(x)$ zwei Lösungen
der Differentialgleichung~\eqref{buch:variation2:diffgl:eqn:dgl}.
Falls an einer Stelle $x_0$ die Vektoren
\[
v_1(x_0)
=
\begin{pmatrix}
y_1(x_0)\\
y'_1(x_0)
\end{pmatrix}
\qquad\text{und}\qquad
v_2(x_0)
=
\begin{pmatrix}
y_2(x_0)\\
y'_2(x_0)
\end{pmatrix}
\]
linear abhängig sind, dann gibt es eine Zahl $c$ mit $cv_1(x_0)=v_2(x_0)$.
Dies bedeutet, dass die Funktion $ay_1(x)$ an der Stelle $x_0$ 
die gleichen Anfangsbedingungen erfüllt wie $y_2(x)$, nach dem
Satz von Picard-Lindelöf folgt daher, dass $y_2(x)=cy_1(x)$ für 
\index{Picard-Lindelof@Picard-Lindelöf, Satz von}%
alle $x$.
Die Vektoren $v_1(x)$ und $v_2(x)$ sind daher immer für alle
$x$ linear abhängig und die Funktionen $y_1(x)$ und $y_2(x)$ haben
die gleichen Nullstellen.

Wir betrachten jetzt zwei linear unabhängige Lösungen
$y_1(x)$ und $y_2(x)$.
Die Vektoren $v_1(x)$ und $v_2(x)$ müssen dann
für alle $x$ linear unabhängig sein, denn wären sie an einer Stelle
linear abhängig und müssten damit nach dem vorangegangenen Absatz
überall linear abhängig sein.
Insbesondere ist die Wronski-Determinante
\index{Wronski-Determinante}%
\[
W(y_1,y_2)
=
\det(v_1,v_2)
=
\left|
\begin{matrix}
 y_1(x) &  y_2(x) \\
y'_1(x) & y'_2(x)
\end{matrix}
\right|
\ne
0.
\]
Die Wronski-Determinante wechselt das Vorzeichen nicht.

Die Wronski-Determinante hat die Ableitung
\begin{align*}
\frac{d}{dx} W(y_1(x),y_2(x))
&=
\frac{d}{dx} \bigl(y_1(x)y'_2(x) - y_2(x)y'_1(x)\bigr)
\\
&=
y'_1(x)y'_2(x) + y_1(x)y''_2(x) - y'_2(x)y'_1(x) - y_2(x)y''_1(x)
\\
&=
y_1(x)y''_2(x) - y_2(x)y''_1(x).
\intertext{Durch Einsetzen der expliziten Form
\eqref{buch:variation2:diffgl:eqn:explizit} der
Differentialgleichung für $y''(x)$ wird daraus}
&=
y_1(x)(-p(x)y'_2(x)-q(x)) - y_2(x)(-p(x)y'_1(x)-q(x))
\\
&=
-p(x)
\bigl(
y_1(x)y'_2(x) - y_2(x)y'_1(x)
\bigr)
\\
&=
-p(x) W(y_1(x),y_2(x)).
\end{align*}
Die Wronski-Determinante erfüllt also die Differentialgleichung
\[
u' = -p(x) u,
\]
die sich separieren lässt, indem man sie
\[
\frac{du}{u}
=
-p(x) \,dx
\qquad\Rightarrow\qquad
\int \frac{du}{u}
=
-\int p(x)\,dx
\]
schreibt.
Ihre Lösung
\[
\log |u(x)|
=
-\int p(x)\,dx
\qquad\Rightarrow\qquad
u(x) = u_0 e^{-\int_{x_0}^s p(\xi)\,d\xi}
\]
erfüllt die Anfangsbedingungen $u(x_0)=u_0$.
Es folgt, dass die Wronski-Determinante
\[
W(y_1(x),y_2(x))
=
W(y_1(x_0),y_2(x_0))
e^{-\int_{x_0}^x p(\xi)\,d\xi}
\]
erfüllt.
Da die Exponentialfunktion keine Nullstellen hat, ergibt sich erneut
die Eigenschaft, dass die Wronski-Determinante das Vorzeichen nicht
wechselt.

%
% Alternierende Nullstellen
%
\subsection{Alternierende Nullstellen
\label{buch:variation2:diffgl:subsection:alternierendenullstellen}}
Seien wieder $y_1(x)$ und $y_2(x)$ zwei linear unabhängige Lösungen
der Differentialgleichung \eqref{buch:variation2:diffgl:eqn:dgl}.
Da die Funktionen linear unabhängig sind, kann der
Quotient $y_2(x)/y_1(x)$ nicht konstant sein.
Wir berechnen daher
\[
\frac{d}{dx}\biggl(
\frac{y_2(x)}{y_1(x)}
\biggr)
=
\frac{y_2'(x)y_1(x)-y_1'(x)y_2(x)}{y_1(x)^2}
=
-
\frac{W(y_1(x),y_2(x))}{y_1(x)^2}
=
-
\frac{\Delta_0}{y_1(x)^2}
e^{-\int_{x_0}^x p(\xi)\,d\xi}
\]
mit $\Delta_0 = W(y_1(x_0),y_2(x_0))$.
Die Ableitung des Quotienten hat daher im ganzen Definitionsbereich
keine Vorzeichenwechsel.
Der Quotient ist somit monoton wachsend oder monoton fallend im 
ganzen Definitionsbereich.

Bei den Nullstellen der Funktion $y_1(x)$ hat der Quotient
$y_2(x)/y_1(x)$ Pole, auf beiden Seiten einer solchen Nullstelle
divergiert der Quotient mit verschiedenem Vorzeichen.
Da der Quotient zwischen zwei solchen Nullstellen monoton, kann es
nur eine einzige Nullstelle geben.
Folglich hat $y_2(x)$ zwischen zwei Nullstellen von $y_1(x)$ genau
eine Nullstelle.
Dasselbe gilt mit vertauschten Rollen von $y_1(x)$ und $y_2(x)$
auch für $y_1(x)$, diese Funktion hat zwischen zwei Nullstellen
von $y_2(x)$ genau eine Nullstelle.
Die Funktionen $y_1(x)$ und $y_2(x)$ haben also alternierende
Nullstellen.



