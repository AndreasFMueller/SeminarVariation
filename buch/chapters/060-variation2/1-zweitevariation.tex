%
% 1-zweitevariation.tex
%
% (c) 2023 Prof Dr Andreas Müller
%
\section{Die zweite Ableitung
\label{buch:variation2:section:zweiteableitung}}
\kopfrechts{Zweite Ableitung}
Wir betrachten wieder ein Funktional der Form
\[
I(y)
=
\int_{x_0}^{x_1}
F(x,y(x),y'(x))
\,dx
\]
und eine Variation $y(x)+\varepsilon\eta(x)$ mit $\eta(x_0)=0$
und $\eta(x_1)=0$.
Die erste Ableitung der Funktion $\varepsilon\mapsto I(y+\varepsilon\eta)$
wurde früher als
\[
\delta I(y)
=
\int_{x_0}^{x_1}
\biggl(
\frac{\partial F}{\partial y}(x,y(x),y'(x))
-
\frac{d}{dx}
\frac{\partial F}{\partial y'}(x,y(x),y'(x))
\biggr)
\,
\eta(x)
\,dx
\]
berechnet.
Wir berechnen nun zusätzlich die zweite Ableitung.
Dazu verwenden wir die Entwicklung des Integranden
$f(\varepsilon) = F(x,y(x)+\varepsilon\eta(x),y'(x)+\varepsilon\eta'(x))$
nach Potenzen von $\varepsilon$, sie ist
\begin{align*}
f(\varepsilon)
%F(x,y(x)+\varepsilon\eta(x),y'(x)+\varepsilon\eta'(x))
&=
F(x,y(x),y'(x))
+
\biggl(
\frac{\partial F}{\partial y}(x,y(x),y'(x))\eta(x)
+
\frac{\partial F}{\partial y'}(x,y(x),y'(x))\eta'(x)
\biggr)
\varepsilon
\\
&\quad
+
\frac{1}{2!}
\biggl(
\frac{\partial^2 F}{\partial y^2}(x,y(x),y'(x))
\eta(x)^2
+
2
\frac{\partial^2 F}{\partial y\,\partial y'}(x,y(x),y'(x))
\eta(x)\eta'(x)
\\
&\qquad\qquad
+
\frac{\partial^2 F}{\partial y^{\prime 2}}(x,y(x),y'(x))
\eta'(x)^2
\biggr)
\varepsilon^2
+
\dots
\end{align*}
Für die Untersuchung eines Extremums müssen wir nur den quadratischen
Term betrachten.
Der besseren Übersichtlichkeit wegen schreiben wir ihn als
\[
G(x)
=
P(x)\eta(x)^2 + 2Q(x) \eta(x)\eta'(x) + R(x)\eta'(x)^2
\]
mit
\begin{align*}
P(x) &= \frac{\partial^2 F}{\partial y^2} (x,y(x),y'(x)),
&
Q(x) &= \frac{\partial^2 F}{\partial y\,\partial y'} (x,y(x),y'(x))
\\
\text{und}
\qquad
R(x) &= \frac{\partial^2 F}{\partial y^{\prime 2}} (x,y(x),y'(x)).
\end{align*}
Wegen
\[
2Q(x)\eta(x)\eta'(x)
=
Q(x)\cdot 2\eta(x)\eta'(x)
=
Q(x)\cdot \frac{d\eta(x)^2}{dx}
\]
kann bei der Berechnung des Integrals des mittleren Terms partielle
Integration verwendet werden:
\begin{align*}
\int_{x_0}^{x_1}
2Q(x)\eta(x)\eta'(x)
\,dx
&=
\int_{x_0}^{x_1}
Q(x) \frac{d\eta(x)^2}{dx}
\,dx
=
\biggl[Q(x)\eta(x)^2\biggr]_{x_0}^{x_1}
-
\int_{x_0}^{x_1} Q'(x) \eta(x)^2\,dx
\end{align*}
Wegen der Randbedingung $\eta(x_0)=\eta(x_1)=0$ fällt der
erste Term auf der rechten Seite weg.
Das Integral von $G(x)$ heisst die zweite Variation von $I$ an der
Stelle $y$ und kann damit zu
\begin{align}
\delta^2 I(y)
&=
\int_{x_0}^{x_1} G(x)\,dx
\notag
\\
&=
\int_{x_0}^{x_1}
(\underbrace{P(x)-Q'(x)}_{\displaystyle = S(x)}) \eta(x)^2
+
R(x)\eta'(x)^2
\,dx
\notag
\\
&=
\int_{x_0}^{x_1}
S(x)\eta(x)^2 + R(x)\eta'(x)^2
\,dx
\label{buch:variation2:zweitevariation:eqn:SRintegral}
\end{align}
vereinfacht werden.
Die Faktoren $\eta(x)^2$ und $\eta'(x)^2$ sind immer positiv,
über das Vorzeichen des Integrals entscheidet also nur das Vorzeichen
der Koeffizienten $S(x)$ und $R(x)$.
Gesucht sind jetzt Bedingungen an $S(x)$ und $R(x)$, die garantieren,
dass die zweite Variation $\delta^2 I(y)\ge 0$ ist.


