%
% 4-jacobi.tex
%
% (c) 2024 Prof Dr Andreas Müller
%
\section{Jacobi-Kriterium
\label{buch:variation2:section:jacobi}}
Die zweite Variation in der Form
\eqref{buch:variation2:zweitevariation:eqn:SRintegral}
zeigt, dass es für die Beurteilung, ob tatsächlich ein Extremum 
vorliegt, auf das Verhalten der Funktion $S(x)$ ankommt.
In einem Minimum garantiert die Legendre-Bedingung bereits,
dass $R(x)\ge 0$ ist.
Es bleibt noch zu untersuchen, ob sich Bedinungen an die Funktion
$S(x)$ finden lassen, die garantieren können, dass ein Minimum
vorliegt.
Zu diesem Zweck betrachten wir das Funktional
\begin{equation}
K(u)
=
\int_{x_0}^{x_1}
S(x) u(x)^2 + R(x) u'(x)^2
\,dx
\label{buch:variation2:jacobi:eqn:K}
\end{equation}
mit Randbedingungen $u(x_0)=u(x_1)=0$ und versuchen das Minimum
zu bestimmen.
Für $u(x)=0$ ist $K(0)=0$.
Sollte es aber Funktionen $u(x)$ geben, die $K(u)$ negativ machen,
dann kann die Funktion $y(x)$ von der ausgehen die Funktionen $S(x)$
und $R(x)$ konstruiert wurden, kein Minimum sein.

Die Euler-Lagrange-Differentialgleichung des Funktionals
\eqref{buch:variation2:jacobi:eqn:K}
mit der Lagrange-Funk\-tion 
\[
G(x,u,u') = S(x) u^2 + R(x) u^{\prime 2}
\]
können wir nach trivialen Vereinfachungen
\[
L(u)
=
\frac{d}{dx}\frac{\partial G}{\partial y'}
-
\frac{\partial G}{\partial y}
=
\frac{d}{dx}(Ru') 
-
Su
=
0
\]
schreiben,
oder auch
\begin{equation}
\frac{d}{dx} R(x)\frac{d}{dx} u(x) - S(x) u(x) = 0,
\label{buch:variation2:jacobi:eqn:jacobisl}
\end{equation}
eine lineare Differentialgleichung zweiter Ordnung für die Funktion $u(x)$.
Sie ist ausserdem von der Form einer Sturm-Liouville-Differentialgleichung,
wie sie in Abschnitt~\ref{buch:variation2:section:diffgl} genauer
untersucht wurden.

Ist die Funktion $u(x)$ eine Lösung der
Differentialgleichung~\eqref{buch:variation2:jacobi:eqn:jacobisl}
mit $u(x_0)=u(x_1)=0$, dann ist auch jedes Vielfache von $u(x)$
eine Lösung mit den gleichen Randbedingungen, und die zweite Variation
des ursprünglichen Funktionals verschwindet für alle Werte von $\varepsilon$.
Falls es keine solche Funktion $u(x)$ gibt, dann ist die zweite
Variation $\ne 0$.
Die Eigenschaft, ob ein Minimum oder Maximum vorliegt, wird also
dadurch entschieden, ob es eine Lösung der Differentialgleichung
\eqref{buch:variation2:jacobi:eqn:jacobisl} mit
Randwerten $u(x_0)=u(x_1)=1$ gibt.

Ist $u(x)$ eine nichttriviale Lösung der Differentialgleichung
\eqref{buch:variation2:jacobi:eqn:jacobisl} mit $u(x_0)=0$,
dann muss $u'(x_0)\ne 0$ sein.
Durch Division druch $u'(x_0)$ erhalten wir eine Lösung, die
sogar $u'(x_0)=1$ hat.
Durch diese Anfangswerte ist aber die Lösung der Differentialgleichung
vollständig bestimmt.
Die Frage ob ein Minimum oder Maximum vorliegt, wird also
durch die Lage der Nullstellen von Lösungen der Differentialgleichung
\eqref{buch:variation2:jacobi:eqn:jacobisl} bestimmt.

\begin{definition}[Jacobi-Differentialgleichung]
Die Differentialgleichung
\eqref{buch:variation2:jacobi:eqn:jacobisl}
heisst die {\em Jacobi-Differentialgleichung}.
\end{definition}

%Aus der Definition des Funktionals $K(u)$ erhält man durch
%partielle Integration für Funktionen mit den Randbedingungen
%$u(x_0)=u(x_1)=0$ 
%\begin{align*}
%K(u)
%&=
%\int_{x_0}^{x_1} S(x)u(x)^2 + R(x) u'(x)^2 \,dx
%\\
%&=
%\int_{x_0}^{x_1} S(x)u(x)^2 \,dx
%+
%\int_{x_0}^{x_1} R(x) u'(x)^2 \,dx
%\\
%&=
%\int_{x_0}^{x_1} S(x)u(x)^2 \,dx
%+
%\int_{x_0}^{x_1} (R(x) u'(x)) u'(x) \,dx
%\\
%&=
%\int_{x_0}^{x_1} S(x)u(x)^2 \,dx
%+
%\biggl[R(x)u'(x)u(x)\biggr]_{x_0}^{x_1}
%-
%\int_{x_0}^{x_1} \frac{d}{dx}(R(x) u'(x)) u(x) \,dx
%\\
%&=
%\int_{x_0}^{x_1}
%\biggl(S(x) u(x) - \frac{d}{dx}\bigl(R(x)u'(x)\bigr)\biggr) u(x)
%\,dx
%\\
%&=
%-\int_{x_0}^{x_1} L(u) u(x) \,dx.
%\end{align*}





