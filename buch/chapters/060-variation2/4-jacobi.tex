%
% 4-jacobi.tex
%
% (c) 2024 Prof Dr Andreas Müller
%
\section{Das Jacobi-Kriterium
\label{buch:variation2:section:jacobi}}
\kopfrechts{Das Jacobi-Kriterium}
Die zweite Variation in der Form
\eqref{buch:variation2:zweitevariation:eqn:SRintegral}
zeigt, dass es für die Beurteilung, ob tatsächlich ein Extremum 
vorliegt, auf das Verhalten der Funktion $S(x)$ ankommt.
In einem Minimum garantiert die Legendre-Bedingung bereits,
dass $R(x)\ge 0$ ist.
Es bleibt noch zu untersuchen, ob sich Bedinungen an die Funktion
$S(x)$ finden lassen, die garantieren können, dass ein Minimum
vorliegt.
Zu diesem Zweck betrachten wir das Funktional
\begin{equation}
K(u)
=
\int_{x_0}^{x_1}
S(x) u(x)^2 + R(x) u'(x)^2
\,dx
\label{buch:variation2:jacobi:eqn:K}
\end{equation}
mit Randbedingungen $u(x_0)=u(x_1)=0$ und versuchen das Minimum
zu bestimmen.
Für $u(x)=0$ ist $K(0)=0$.
Sollte es aber Funktionen $u(x)$ geben, die $K(u)$ negativ machen,
dann kann die Funktion $y(x)$ von der ausgehen die Funktionen $S(x)$
und $R(x)$ konstruiert wurden, kein Minimum sein.

Die Euler-Lagrange-Differentialgleichung des Funktionals
\eqref{buch:variation2:jacobi:eqn:K}
mit der Lagrange-Funk\-tion 
\[
G(x,u,u') = S(x) u^2 + R(x) u^{\prime 2}
\]
können wir nach trivialen Vereinfachungen
\[
L(u)
=
\frac{d}{dx}\frac{\partial G}{\partial y'}
-
\frac{\partial G}{\partial y}
=
\frac{d}{dx}(Ru') 
-
Su
=
0
\]
schreiben,
oder auch
\begin{equation}
\frac{d}{dx} R(x)\frac{d}{dx} u(x) - S(x) u(x) = 0,
\label{buch:variation2:jacobi:eqn:jacobisl}
\end{equation}
eine lineare Differentialgleichung zweiter Ordnung für die Funktion $u(x)$.
Sie ist ausserdem von der Form einer Sturm-Liouville-Differentialgleichung,
wie sie in Abschnitt~\ref{buch:variation2:section:diffgl} genauer
untersucht wurden.

Ist die Funktion $u(x)$ eine Lösung der
Differentialgleichung~\eqref{buch:variation2:jacobi:eqn:jacobisl}
mit $u(x_0)=u(x_1)=0$, dann ist auch jedes Vielfache von $u(x)$
eine Lösung mit den gleichen Randbedingungen, und die zweite Variation
des ursprünglichen Funktionals verschwindet für alle Werte von $\varepsilon$.
Falls es keine solche Funktion $u(x)$ gibt, dann ist die zweite
Variation $\ne 0$.
Die Eigenschaft, ob ein Minimum oder Maximum vorliegt, wird also
dadurch entschieden, ob es eine Lösung der Differentialgleichung
\eqref{buch:variation2:jacobi:eqn:jacobisl} mit
Randwerten $u(x_0)=u(x_1)=1$ gibt.

Ist $u(x)$ eine nichttriviale Lösung der Differentialgleichung
\eqref{buch:variation2:jacobi:eqn:jacobisl} mit $u(x_0)=0$,
dann muss $u'(x_0)\ne 0$ sein.
Durch Division durch $u'(x_0)$ erhalten wir eine Lösung, die
sogar $u'(x_0)=1$ hat.
Durch diese Anfangswerte ist aber die Lösung der Differentialgleichung
vollständig bestimmt.
Die Frage ob ein Minimum oder Maximum vorliegt, wird also
durch die Lage der Nullstellen von Lösungen der Differentialgleichung
\eqref{buch:variation2:jacobi:eqn:jacobisl} bestimmt.

\begin{definition}[Jacobi-Differentialgleichung]
Die Differentialgleichung
\eqref{buch:variation2:jacobi:eqn:jacobisl}
heisst die {\em Jacobi-Differentialgleichung}.
\end{definition}

%Aus der Definition des Funktionals $K(u)$ erhält man durch
%partielle Integration für Funktionen mit den Randbedingungen
%$u(x_0)=u(x_1)=0$ 
%\begin{align*}
%K(u)
%&=
%\int_{x_0}^{x_1} S(x)u(x)^2 + R(x) u'(x)^2 \,dx
%\\
%&=
%\int_{x_0}^{x_1} S(x)u(x)^2 \,dx
%+
%\int_{x_0}^{x_1} R(x) u'(x)^2 \,dx
%\\
%&=
%\int_{x_0}^{x_1} S(x)u(x)^2 \,dx
%+
%\int_{x_0}^{x_1} (R(x) u'(x)) u'(x) \,dx
%\\
%&=
%\int_{x_0}^{x_1} S(x)u(x)^2 \,dx
%+
%\biggl[R(x)u'(x)u(x)\biggr]_{x_0}^{x_1}
%-
%\int_{x_0}^{x_1} \frac{d}{dx}(R(x) u'(x)) u(x) \,dx
%\\
%&=
%\int_{x_0}^{x_1}
%\biggl(S(x) u(x) - \frac{d}{dx}\bigl(R(x)u'(x)\bigr)\biggr) u(x)
%\,dx
%\\
%&=
%-\int_{x_0}^{x_1} L(u) u(x) \,dx.
%\end{align*}


%
% Kürzeste Verbindung auf einem Rotationsellipsoid
%
\subsection{Kürzeste Verbindung auf einem Rotationsellipsoid}
Seit über 2000 Jahren ist der Menschheit bekannt, dass die Erde
ungefähr die Form einer Kugel hat.
Erathostenes gelang es als erstem, den Umfang der Erde mit
einem Fehler von weniger als $2\%$ zu bestimmen.
Die Entfernungsmessung auf einer Kugeloberfläche ist deutlich
komplizierter als auf einer Ebene.
Bereits im Altertum wurde die sphärische Trigonometrie entwickelt,
mit der sich Winkel und Seitenlägen von beliebigen Dreiecken auf
einer Kugeloberfläche berechnen lassen.
Van Brummelen \cite{buch:heavenly} gibt eine sorgfältige Einführung
in die Geschichte und die Anwendungen der sphärischen Geometrie.

\begin{beispiel}
Die kürzesten Verbindungen zweier Punkte auf einer Kugel ist ein
Grosskreis.
Liegen die Punkte weniger als $180^\circ$ Grad auseinander, gibt es
nur eine kürzeste Verbindung, das Längenfunktional nimmt ein
absolutes Minimum an.

Sind die Punkte jedoch Antipodenpunkte, dann gibt es unendliche viele
Grosskreise durch beide Punkte.
Das Längenfunktional hat kein absolutes Minimum mehr, die Drehung eines
Grosskreises um die Achse durch die beiden Punkte ist eine Variation,
die das Funktional nicht ändert.
Die beiden Punkte sind konjugierte Punkte.
\end{beispiel}

Die Erde ist aber nicht eine Kugel, sondern eher ein Rotationsellipsoid,
der Durchmesser entlang der Achse ist verschieden vom Durchmesser des 
Äquators.
Im 17. Jahrhundert gab es einen wissenschaftlichen Disput darüber,
ob der Durchmesser entlang der Achse kleiner oder grösser als der
Äquatordurchmesser ist.
Newton und Huygens hatten aufgrund der allgemeinen Mechanik eine Abplattung
der Erde vorhergesagt, die durch von Jean Richter und Philipp de La Hire
durch die erste Vermessung des Merdianbogens durch Paris bestätigt 
worden war.
Die Sache war aber immer noch nicht vollständig geklärt.
Der Astronom Jacques Cassini schloss zu Beginn des 18. Jahrhunderts,
dass der Polradius grösser als der Äquatorradius sein müsse.
König Ludwig XV.~beauftragte daher Maupertuis mit einer Expedition
nach Lappland.
Als Teil der Expedition soll der Abstand zweier Breitengrade
vermessen werden.
Wenn der Radius im hohen Norden zunehmen abnimmt, wird auch die Länge
eines Merdianbogens zu vorgegebenem Winkel kürzer.
Die Expedition von Maupertuis und eine zeitgleich stattfindende Expedtion
nach Ecuador bestätigten beide die Abplattung.

\begin{beispiel}
Unter der Annahme, dass die Erde ein Rotationsellipsoid mit Äquatorradius
$1$ und Polradius $r$ ist, soll der kürzeste Weg zwischen zwei Punkten
auf dem Äquator bestimmt werden.
\end{beispiel}



