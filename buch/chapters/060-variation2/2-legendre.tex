%
% 1-legrendre.tex
%
% (c) 2024 Prof Dr Andreas Müller
%
\section{Die Legendre-Bedingung
\label{buch:variation2:section:legendre}}
\kopfrechts{Die Legendre-Bedingung}
Der Ausdruck \eqref{buch:variation2:zweitevariation:eqn:SRintegral}
für die zweite Variation ist sicher positiv, wenn sowohl $S(x)$
als auch $R(x)$ auf dem ganzen Intervall positiv sind.
In diesem Abschnitt soll untersucht werden, inwieweit die
Umkehrung auch gilt.

\begin{satz}[Legendre-Bedingung]
\label{buch:variation2:legendre:satz:legendre}
Dafür, dass $\delta^2I(y)\ge 0$ für jede Funktion $\eta(x)$ gilt ist
notwendig, dass
\[
R(x)
=
\frac{\partial^2F}{\partial y\,\partial y'}(x,y(x),y'(x))
\ge
0
\]
im ganzen Intervall $[x_0,x_1]$ ist.
\end{satz}

\begin{proof}
%
% legendre.tex -- Legendre Bedingung
%
% (c) 2021 Prof Dr Andreas Müller, OST Ostschweizer Fachhochschule
%
\documentclass[tikz]{standalone}
\usepackage{amsmath}
\usepackage{times}
\usepackage{txfonts}
\usepackage{pgfplots}
\usepackage{csvsimple}
\usetikzlibrary{arrows,intersections,math,calc}
\definecolor{darkred}{rgb}{0.8,0,0}
\begin{document}
\def\skala{1}
\def\a{0.3}
\def\d{1}
\def\b{5}
\def\N{500}
\def\xstern{3}
\pgfmathparse{\a*(\b+0.5)/\d}
\xdef\B{\pgfmathresult}
\begin{tikzpicture}[>=latex,thick,scale=\skala,
declare function={
	S(\x) = sin(50*\x+40)+2*cos(20*(\x-4))+0.1*\x;
	R(\x) = cos(10*\x)-3*exp(-(3-\x)*(3-\x)/4);
	eta(\x) = \a*cos((\b+0.5)*180*(\x-\xstern)/\d);
	etastrich(\x) = -\a*(\b+0.5)*sin((\b+0.5)*180*(\x-\xstern)/\d)/\d;
}]


\fill[color=gray!10] ({\xstern-\d},-10) rectangle ({\xstern+\d},3);

\begin{scope}
\draw[->] (-4.2,0) -- (8.5,0) coordinate[label={$x$}];
\fill[color=darkred!20] ({\xstern-\d},0) rectangle ({\xstern+\d},{\a*\a});
\draw[->] (0,-3) -- (0,3.3) coordinate[label={right:$y$}];
\node[color=darkred] at (\xstern,{\a*\a}) [above] {$\eta(x)^2$};
\draw[color=blue] plot[domain=-4:8,samples=100] ({\x},{S(\x)});
\draw[color=darkred]
	(-4,0)
	--
	({\xstern-\d},0)
	--
	plot[domain={\xstern-\d}:{\xstern+\d},samples=\N]
		({\x},{eta(\x)*eta(\x)})
	--
	({\xstern+\d},0)
	--
	(8,0);
\draw (-4,-0.05) -- (-4,0.05);
\draw (8,-0.05) -- (8,0.05);
\node at (-4,-0.05) [below] {$x_1\mathstrut$};
\node at (8,-0.05) [below] {$x_2\mathstrut$};
\node at ({\xstern-\d},-0.05) [below left] {$x_*-d$};
\node at ({\xstern+\d},-0.05) [below right] {$x_*+d$};
\node[color=blue] at (-2,1.5) {$S(x)$};
\draw[color=blue,line width=0.3pt,shorten <= 0.3cm]
	(-2,1.5) -- (-0.5,{S(-0.5)});
\end{scope}

\begin{scope}[yshift=-7cm]
\fill[color=blue!10]
	({\xstern-\d},0) rectangle ({\xstern+\d},{R(\xstern-\d)+0.1});
\draw[->] (-4.2,0) -- (8.5,0) coordinate[label={$x$}];
\fill [color=darkred!20] ({\xstern-\d},0) rectangle ({\xstern+\d},{\B*\B});
\draw[color=blue,line width=0.2pt]
	({\xstern-\d},{R(\xstern-\d)+0.1})
	--
	({\xstern+\d},{R(\xstern-\d)+0.1});
\node[color=blue] at (\xstern,{R(\xstern-\d)+0.1}) [above] {$-a$};
\node[color=blue] at (\xstern,{R(\xstern)}) [below] {$R(x) < -a$};
\draw[->] (0,-3) -- (0,3.3) coordinate[label={right:$y$}];
%\node[color=darkred] at (\xstern,{\B*\B})
%	[above] {$\displaystyle\frac{a(b+\frac12)\pi}{d}$};
\node[color=darkred] at (\xstern,{\B*\B}) [above] {$\eta'(x)^2$};
\draw[color=blue] plot[domain=-4:8,samples=100] ({\x},{R(\x)});
\draw[color=darkred]
	(-4,0)
	--
	({\xstern-\d},0);
\draw[color=darkred,smooth]
	plot[domain={\xstern-\d}:{\xstern+\d},samples=\N]
		({\x},{etastrich(\x)*etastrich(\x)});
\draw[color=darkred]
	({\xstern+\d},0)
	--
	(8,0);
\draw (-4,-0.05) -- (-4,0.05);
\draw (8,-0.05) -- (8,0.05);
\node at (-4,-0.05) [below] {$x_1\mathstrut$};
\node at (8,-0.05) [below] {$x_2\mathstrut$};
\node at ({\xstern-\d},-0.05) [above left] {$x_*-d$};
\node at ({\xstern+\d},-0.05) [above right] {$x_*+d$};
\node[color=blue] at (-2,1.5) {$R(x)$};
\draw[color=blue,line width=0.3pt,shorten <= 0.3cm]
	(-2,1.5) -- (-1.5,{R(-1.5)});
\end{scope}

\end{tikzpicture}
\end{document}


Wir zeigen mit Hilfe eines Widerspruchs, dass es keine Stelle $x$
geben kann, für die $R(x)<0$ ist.
Nehmen wir also an, es gäbe eine Stelle $x_*$ derart, dass
$R(x_*)<0$.
Da die Funktion $R(x)$ stetig ist, wird $R(x)$ auch noch in einer
Umgebung $(x_*-d,x_*+d)$ negativ sind.
Wir dürfen daher annehmen, dass es eine positive Zahl $a$ gibt
derart, dass $R(x)<-a$ für $x$ mit $|x-x_*|<d$.

Wir müssen jetzt zeigen, dass sich immer eine Funktion $\eta(x)$ finden
lässt, die das Integral $\delta^2 I(y)$ negativ macht, selbst wenn
der Koeffizient $S(x)$ auf dem Teilintervall
$(x_*-d,x_*+d)$ positiv ist.
Wir müssen zeigen, dass $\eta(x)$ so gewählt werden kann, dass der
Term mit $\eta(x)^2$ beliebig klein, der Term mit $\eta'(x)^2$ aber
beliebig negativ gemacht werden kann.

Wir verwenden die Funktion
\[
\eta(x)
=
\begin{cases}
\displaystyle
b\cos \frac{(c+\frac12)\pi (x-x_*)}{d}&\qquad |x-x_*|<d\\[3pt]
0&\qquad \text{sonst},
\end{cases}
\]
wobei $c\in\mathbb{N}$ eine natürliche Zahl ist
(Abbildung~\ref{buch:variation2:legendre:fig:legendre}).
Dies stellt sicher, dass $\eta(x)$ stetig ist,
wenn auch nicht differenzierbar an den Stellen $x_*\pm d$.

Zunächst ist $|\eta(x)|\le b$ für alle $x$, der erste Summand
des Integrals kann daher wegen
\[
\biggl|
\int_{x_0}^{x_1} S(x)\eta(x)^2\,dx
\biggr|
=
\biggl|
\int_{x_*-d}^{x_*+d} S(x)\eta(x)^2\,dx
\biggr|
\le
\int_{x_*-d}^{x_*+d} |S(x)|\cdot |\eta(x)|^2\,dx
\le
b^2
\int_{x_*-d}^{x_*+d} |S(x)|\,dx
\]
beliebig klein gemacht werden, indem $a$ klein gewählt wird.
Für die Ableitung gilt
\[
\eta'(x)
=
\begin{cases}
\displaystyle
-\frac{b(c+\frac12)\pi}{d}\sin\frac{(c+\frac12)\pi(x-x_*)}{d}&\qquad |x-x_*|<d\\[3pt]
0&\qquad\text{sonst.}
\end{cases}
\]
Der zweite Summand des Integrals wird damit zu
\begin{align*}
\int_{x_0}^{x_1} R(x) \eta'(x)^2\,dx
&=
\int_{x_*-d}^{x_*+d} R(x)\eta'(x)^2\,dx
\\
&\le 
\int_{x_*-d}^{x_*+d} (-a) \eta'(x)^2\,dx
\\
&=
-ab^2\int_{x_*-d}^{x_*+d} \frac{(c+\frac12)^2\pi^2}{d^2}
\sin^2 \frac{(c+\frac12)\pi(x-x_*)}{d}\,dx.
\intertext{Das Integral von $\sin^2 t$ über eine ganz Anzahl von
Perioden ist die halbe Intervalllänge, somit wird das Integral zu}
&=
-ab^2
\frac{(c+\frac12)^2\pi^2}{d^2}
2d
=
-\frac{ab^2(c+\frac12)^2\pi^2}{d}
\end{align*}
(siehe Lemma~\ref{buch:variation2:legendre:lemma:sin} weiter unten).
Die rechte Seite kann beliebig gross gemacht werden, indem ein
geeignet grosses $c$ gewählt wird.
So wird die zweite Variation
\begin{align}
\delta^2I(y)
=
\int_{x_0}^{x_1} S(x)\eta(x)^2 + R(x)\eta'(x)^2\,dx
&=
\int_{x_0}^{x_1} S(x)\eta(x)^2\,dx
+
\int_{x_0}^{x_1} R(x)\eta'(x)^2\,dx
\notag
\\
&\le
\int_{x_*-d}^{x_*+d}|S(x)|\eta(x)\,dx
+
\int_{x_*-d}^{x_*+d} R(x) \eta(x)\,dx
\notag
\\
&\le
b^2\int_{x_*-d}^{x_*+d} |S(x)|\,dx
-
\frac{ab^2(c+\frac12)^2\pi^2}{d}
\notag
\\
&=
\frac{ab^2\pi^2}{d}
\biggl(
\frac{d}{a\pi^2}
\int_{x_*-d}^{x_*+d}|S(x)|\,dx
-
(c+{\textstyle\frac12})^2
\biggr).
\label{buch:variation2:legendre:eqn:bgl}
\end{align}
Indem man 
\[
(c+{\textstyle\frac12})^2
>
\frac{d}{a\pi^2}\int_{x_*-d}^{x_*+d} |S(x)|\,dx
\]
wählt, wird die Klammer in \eqref{buch:variation2:legendre:eqn:bgl}
negativ, so dass die zweite Variation negativ wird.
Dies widerspricht der Annahme über die zweite Variation.
Der Widerspruch zeigt, dass $R(x)\ge 0$ sein muss.
\end{proof}

\begin{lemma}
\label{buch:variation2:legendre:lemma:sin}
Für $c\in\mathbb{N}$ gilt
\[
\int_{-d}^d
\sin^2 \frac{(c+\frac12)\pi t}{d}
\,dt
=
d.
\]
\end{lemma}

\begin{proof}
%
% sin2.tex -- Oszillationen von sin^2
%
% (c) 2021 Prof Dr Andreas Müller, OST Ostschweizer Fachhochschule
%
\documentclass[tikz]{standalone}
\usepackage{amsmath}
\usepackage{times}
\usepackage{txfonts}
\usepackage{pgfplots}
\usepackage{csvsimple}
\usetikzlibrary{arrows,intersections,math}
\definecolor{darkred}{rgb}{0.8,0,0}
\begin{document}
\def\skala{1}
\def\d{4.8}
\def\a{4}
\def\b{3}
\begin{tikzpicture}[>=latex,thick,scale=\skala,
declare function={
	f(\x) = \a*sin((\b+0.5)*180*\x/\d);
	g(\x) = f(\x)*f(\x)/\a;
}]

\fill[color=gray!20] (-\d,0) rectangle (\d,\a);
\fill[color=darkred!20]
	(-\d,0)
	--
	plot[domain=-\d:\d,samples=100]
	({\x},{g(\x)})
	--
	(\d,0)
	--
	cycle;

\draw[color=blue,smooth,line width=1.2pt]
	plot[domain=-\d:\d,samples=100] ({\x},{f(\x)});
\draw[color=darkred,smooth,line width=1.2pt]
	plot[domain=-\d:\d,samples=100] ({\x},{g(\x)});

\draw[->] (-5,0) -- (7.3,0) coordinate[label={$t$}];
\draw[->] (0,-4.1) -- (0,4.5) coordinate[label={right:$y$}];

\draw (-0.05,4) -- (0.05,4);
\node at (-0.05,4) [above left] {$1$};
\draw (-0.05,-4) -- (0.05,-4);
\node at (-0.05,-4) [left] {$-1$};

\draw (-\d,-0.05) -- (-\d,0.05);
\node at (-\d,-0.05) [below] {$-d\mathstrut$};
\draw (\d,-0.05) -- (\d,0.05);
\node at (\d,-0.05) [below] {$d\mathstrut$};

\coordinate (A) at ({-3*\d/7},{-6*\a/7});
\node[color=blue] at (A) {$\displaystyle\sin\frac{(c+\frac12)\pi t}{d}$};
\draw[color=blue,line width=0.2pt,shorten <= 0.6cm]
	(A) -- ({-9*\d/14},{f(-9*\d/14)});

\coordinate (B) at (6.0,{3*\a/4});
\node[color=darkred] at (B) {$\displaystyle\sin^2\frac{(c+\frac12)\pi t}{d}$};
\draw[color=darkred,line width=0.2pt,shorten <= 0.65cm]
	(B) -- ({13*\d/14},{g(13*\d/14)});

\end{tikzpicture}
\end{document}


Wegen
\[
\sin^2 t
=
\frac12 - \frac12\cos 2t
\]
ist das Integral die Fläche unter einer um $\frac12$ nach oben
verschobenen Cosinus-Kurve mit doppelter Frequenz.
Diese Kurve halbiert das Rechteck mit den Ecken $(-d,0)$ und $(d,1)$
mit dem Flächeninhalt $2d$.
Daher ist der Wert des Integrals $2d/2=d$.
\end{proof}

