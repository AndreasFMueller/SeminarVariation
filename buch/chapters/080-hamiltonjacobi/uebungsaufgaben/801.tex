Optimales Laden eines Kondensators.
Eine Spannungsquelle soll einen Kondensator $C$ innert der Zeit $t_1$
auf eine vorgegebene Zielspannung $x_1$ aufladen.
Durch den Innenwiderstand $R$ des Kondensators entstehen Verluste.
Wie muss die Ladespannung $u(t)$ verlaufen, damit die Verluste minimal
sind.

\begin{loesung}
Wir bezeichnen die Spannung am Kondensator zur Zeit $t$ mit $x(t)$ und
die Ladespannung mit $u(t)$.
Die am Innenwiderstand abfallende Spannung ist $u(t)-x(t)$ und der 
Ladestrom ist nach dem ohmschen Gesetz
\[
i(t) = \frac{1}{R}(u(t)-x(t))
\]
Die durch den Ladestrom verursachte Spannungsänderung ist
\begin{equation}
\frac{d}{dt}x(t) = \frac{1}{C}i(t) = \frac{1}{RC}(u(t)-x(t))
\label{buch:801:systemdgl}
\end{equation}
Die im Widerstand zur Zeit $t$ dissipierte Leistung ist
\[
p(t) = \frac{1}{R}(u(t)-x(t))^2,
\]
so dass die gesamte im Widerstand verlorene Energie
\begin{equation}
E
=
\int_{t_0}^{t_1}
\frac{1}{R}(u(t)-x(t))^2\,dt
\end{equation}
ist.
\end{loesung}


