%
% chapter.tex -- Hamilton-Jacobi-Theorie
%
% (c) 2023 Prof Dr Andreas Müller
%
\chapter{Hamilton-Jacobi-Theorie
\label{buch:chapter:hamiltonjacobi}}
\kopflinks{Hamilton-Jacobi-Theorie}
Hamilton hat einen etwas anderen Zugang zur Variationsrechnung
gefunden, der aber besonders in der Physik (siehe
Kapitel~\ref{buch:chapter:mechanik}) und in der Regelungstechnik
erfolgreich ist.
Sie betrachtet die Randbedingungen nicht als fest, sondern definiert
den minimalen Wert des Funktionals in Abhängigkeit vom Endpunkt einer
Lösungskurve als das grundlegende Objekt, die Hamilton-Funktion.
Daraus lassen sich dann neue Differentialgleichungen erster Ordnung
gewinnen, für die sich auch eine Methode zur Lösung angeben lässt.
Die optimale Steuerungsaufgabe ist eine besonders spektakuläre 
Anwendung dieser Methode.

%
% 1-kanonisch.tex
%
% (c) 2024 Prof Dr Andreas Müller
%
\section{Kanonische Variablen
\label{buch:hamiltonjacobi:section:kanonisch}}
\kopfrechts{Kanonische Variablen}
In den Transversalitätsbedingungen taucht immer wieder Kombination
\[
F - y'\frac{\partial F}{\partial y'}.
\]
Dieser Abschnitt geht daher der Frage nach, ob die Ableitung von
$F$ nach $y'$ eine besondere Bedeutung hat.
Tatsächlich kann diese Ableitung als neue Koordinate verwendet
werden, mit denen sich die Euler-Lagrange-Differentialgleichung, die
von zweiter Ordnung ist, in ein besonders einfaches
Differentialgleichungssystem erster Ordnung umwandeln lässt.

%
% Transversaltität und kanonische Variable
%
\subsection{Transversalität und kanonische Variable}
Wir betrachten das Variationsproblem mit der Lagrange-Funktion
$F(x,y,y')$.
Eine Extremale durch die Punkte $(x_0,y_0)$ und $(x_1,y_1)$ ist
eine Lösung $y(x)$ der Euler-Lagrange-Differentialgleichung, die die
Randbedingungen $y(x_0)=y_0$ und $y(x_1)=y_1$ erfüllt.
Da die Euler-Lagrange-Differentialgleichung von zweiter Ordnung ist,
enthält die allgemeine Lösung zwei Konstanten, die bestimmt werden 
müssen, um die Randbedingungen zu erfüllen.
Diese Konstanten können der Anfangswert $y_0=y(x_0)$ und die Anfangssteigung
$y'(x_0)$.
Um das Randwertproblem zu lösen, muss also die Anfangssteigung gefunden
werden, so dass die zugehörige Lösung die Randbedingung $y(x_1)=y_1$ 
erfüllt.

%
% Die kanonischen Variablen
%
\subsubsection{Die kanonischen Variablen}
Die Transversalitätsbedingung gibt der partiellen Ableitung von $F$
nach $y'$ eine besondere Bedeutung, wir schreiben daher 
\begin{equation}
p = \frac{\partial F}{\partial y'}(x,y,y').
\label{buch:hamiltonjacobi:kanonisch:eqn:pablF}
\end{equation}
Dies ist eine Gleichung, die in jedem Punkt $(x,y)$ eine Beziehung 
zwischen der Steigung $y'$ und der Variablen $p$ herstellt.
Sie ist mindestens in einer Umgebung eines Punktes der $x$-$y$-Ebene
umkehrbar, wenn die Ableitung
\[
\frac{\partial^2 F}{\partial y^{\prime 2}}
\ne 0
\]
ist.
Dies ist immer dann erfüllt, wenn die starke Legendre-Bedingung
erfüllt ist, die notwendig ist, damit Lösungen der
Euler-Lagrange-Differentialgleichung tatsächlich das Funktional
extremal machen.
Wir nehmen daher im folgenden an, dass 
\eqref{buch:hamiltonjacobi:kanonisch:eqn:pablF}
umkehrbar ist und $y'$ durch $y$ und $p$ bestimmt werden kann.

%
% Die Hamilton-Funktion
%
\subsubsection{Die Hamilton-Funktion}
Der Ausdruck der Transversalitätsbedingung kann jetzt geschrieben werden
als 
\[
F(x,y,y') - y'\frac{\partial F}{\partial y'}(x,y,y')
=
F(x,y,y') - y'\cdot p.
\]
Die partielle Ableitung der rechten Seite nach $y'$ ist
\[
\frac{\partial}{\partial y'}
\biggl(
F(x,y,y') - y'\cdot p
\biggr)
=
\frac{\partial F}{\partial y'}(x,y,y') - p
=
0,
\]
sie hängt also nicht von $y'$ ab.
Somit muss sie als Funktion nur von $x$, $y$ und $p$ geschrieben werden
können.

\begin{definition}[Hamilton-Funktion]
\label{buch:hamiltonjacobi:kanonisch:def:hamilton-funktion}
Die Hamilton-Funktion zur Lagrange-Funktion $F$ ist die Funktion 
\[
H(x,y,p)
=
y'\cdot p - F(x,y,y')
\]
wobei $y'$ mit Hilfe von \eqref{buch:hamiltonjacobi:kanonisch:eqn:pablF}
durch $p$ ausgedruckt wird.
\end{definition}

\begin{beispiel}
\label{buch:hamiltonjacobi:kanonisch:bsp:laenge}
Wir betrachten das Funktional mit der Lagrange-Funktion
\[
F(x,y,y')
=
\sqrt{1+y^{\prime 2}},
\]
welches kürzeste Verbindung in der Ebene als Extremalen hat.
Die partielle Ableitung nach $y'$ ist
\begin{equation}
p
=
\frac{\partial F}{\partial y'}
=
\frac{y'}{\sqrt{1+y^{\prime 2}}}
\label{buch:hamiltonjacobi:kanonisch:bsp:pdef}
\end{equation}
Ist $\alpha$ der Neigungswinkel der Tangente an den Graphen von $y(x)$,
dann ist $y'=\tan\alpha$ und
\[
p
=
\frac{\tan\alpha}{\sqrt{1+\tan^2\alpha}}
=
\sin\alpha.
\]
Man kann also auch schreiben $p=\sin\arctan y'$ oder $y'=\tan\arcsin p$.

Die zweite Ableitung von $F$ ist
\[
\frac{\partial^2 F}{\partial y^{\prime2}}
=
\frac{1}{(1+y^{\prime 2})^{\frac32}} \ne 0,
\]
die starke Legendre-Bedingung ist also erfüllt und $y'$ kann durch $p$
ersetzt werden.
In der Tat kann die Gleichung \eqref{buch:hamiltonjacobi:kanonisch:bsp:pdef}
nach
\begin{equation}
y'=\frac{p}{\sqrt{1-p^2}}
\label{buch:hamiltonjacobi:kanonisch:bsp:y'ausp}
\end{equation}
aufgelöst werden.
Schon aus \eqref{buch:hamiltonjacobi:kanonisch:bsp:pdef} ist klar,
dass $|p|<1$ sein muss, der Nenner in
\eqref{buch:hamiltonjacobi:kanonisch:bsp:y'ausp} 
wird also nie $0$.

Die Lagrange-Funktion kann jetzt durch $p$ statt $y'$ ausgedrückt werden,
sie ist
\begin{align*}
F(x,y,y')
&=
\sqrt{1+y^{\prime 2}}
=
\sqrt{1+\frac{p^2}{1-p^2}}
=
\frac{1}{\sqrt{1-p^2}}
\intertext{oder ausgedrückt durch den Winkel $\alpha$}
&=
\frac{1}{\sqrt{1-\sin^2\alpha}}
=
\frac{1}{\cos\alpha}.
\end{align*}
Da der Winkel $|\alpha|<\frac{\pi}2$ ist, wird der Nenner nicht $0$.

Die Hamilton-Funktion ist 
\begin{align*}
H(x,y,p)
&=
y'p
-
F(x,y,y')
=
y'p
-
\sqrt{1+y^{\prime 2}},
\intertext{sie muss jetzt aber nur durch $p$ ausgedrückt werden, was
unter Verwendung von \eqref{buch:hamiltonjacobi:kanonisch:bsp:y'ausp}
auf}
&=
\frac{p^2}{\sqrt{1-p^2}}
-
\frac{1}{\sqrt{1-p^2}}
=
-
\frac{1-p^2}{\sqrt{1-p^2}}
=
-\sqrt{1-p^2}
\intertext{führt, oder mit Hilfe des Winkels $\alpha$ ausgedrückt:}
&=
-\cos\alpha.
\end{align*}
Die Hamilton-Funktion hat also eine einfache geometrische Interpretation.
\end{beispiel}

%
% Die kanonischen Differentialgleichungen
%
\subsection{Die kanonischen Differentialgleichungen}
Die Euler-Lagrange-Differentialgleichung für die Lagrange-Funktion $F$ ist 
die Differentialgleichung
\begin{equation}
\frac{\partial F}{\partial y}(x,y(x),y'(x))
-
\frac{d}{dx}
\frac{\partial F}{\partial y'}(x,y(x),y'(x))
=
0
\label{buch:hamiltonjacobi:kanonisch:eqn:kaneldgl}
\end{equation}
zweiter Ordnung für die Funktion $y(x)$.
Zu einer Lösung $y(x)$ gehört nach
\eqref{buch:hamiltonjacobi:kanonisch:eqn:pablF}
auch eine Funktion
\[
p(x)
=
\frac{\partial F}{\partial y'}(x,y(x),y'(x)).
\]
Eingesetzt in die Euler-Lagrange-Differentialgleichung
\eqref{buch:hamiltonjacobi:kanonisch:eqn:kaneldgl}
entsteht die Differentialgleichung
\[
\frac{\partial F}{\partial y}
-
\frac{d}{dx}p
=
0.
\]
Aus der Definition der Hamilton-Funktion folgt
\[
\frac{\partial H}{\partial y}
=
\frac{\partial y'p-F}{\partial y}
=
-\frac{\partial F}{\partial y}
\qquad\Rightarrow\qquad
\frac{dp}{dx}
=
-
\frac{\partial H}{\partial y}.
\]
Die Ableitung von $H$ nach $p$ ist
\[
\frac{\partial H}{\partial p}
=
y',
\]
da $H$ nicht von $y'$ abhängt.
Damit haben wir den folgenden Satz gefunden.

\begin{satz}[Kanonische Differentialgleichungen]
Jeder Lösung $y(x)$ der Euler-Lagrange-Differentialgleichung der
Lagrange-Funktion $F(x,y,y')$ entspricht eine Lösung
der {\em kanonischen Differentialgleichungen}
\begin{equation}
\begin{aligned}
y'&=\phantom{-}\frac{\partial H}{\partial p}
\\
p'&=-\frac{\partial H}{\partial y}.
\end{aligned}
\end{equation}
\end{satz}

Als Differentialgleichungen erster Ordnung sind die kanonischen
Differentialgleichungen oft leichter zu lösen als die
Euler-Lagrange-Differentialgleichung.

\begin{beispiel}
Für die Hamilton-Funktion 
$H(x,y,p) = -\sqrt{1-p^2}$
des Beispiels~\ref{buch:hamiltonjacobi:kanonisch:bsp:laenge} sind
die kanonischen Differentialgleichungen
\begin{align}
y'
&=
-\frac{\partial H}{\partial p}
=
\frac{p}{\sqrt{1-p^2}} = \tan\alpha,
\notag
\\
p'
&=
\frac{\partial H}{\partial y}=0
\notag
\end{align}
Aus der zweiten Gleichung folgt, dass $p$ konstant ist.
Damit lässt sich die erste Gleichung sofort lösen, die Lösung, die
durch den Punkt $(x_0,y_0)$ verläuft, ist
\[
y(x)
=
y_0+\frac{p}{\sqrt{1-p^2}}(x-x_0)
=
y_0+ (x-x_0) \tan\alpha.
\]
Die Lösungen sind Geraden mit Steigung $p/\sqrt{1-p^2}$
\end{beispiel}

Im Beispiel hing die Hamilton-Funktion $H(x,y,p)$ nur von $p$ ab,
was die Lösung besonders einfach gemacht hat, weil eine der kanonischen
Differentialgleichungen $0$ auf der rechten Seite hatte.
Dies gilt ganz allgemein, wie der folgende Satz zeigt.
Kommt eine der abhängigen Variablen in der Hamilton-Funktion nicht vor,
dann kann die Differentialgleichung durch eine Stammfunktion
gelöst werden, man sagt auch, es sei nur eine {\em Quadratur}
erforderlich.

\begin{satz}
Sei $y(x),p(x)$ eine Lösung der kanonischen Differentialgleichungen.
Hängt die Hamilton-Funktion nicht von $y$ ab, geschrieben als
$H(x,y,p)=H_1(x,p)$ dann ist $p(x)=p_0$ konstant und
$y(x)$ kann durch Quadratur bestimmt werden.
Hängt die Hamilton-Funktion nicht von $p$ ab, geschrieben als
$H(x,y,p)=H_2(x,y)$ dann ist $y(x)=y_0$ konstant und
$p(x)$ kann durch Quadratur bestimmt werden.
\end{satz}

\begin{proof}
Falls $H$ nicht von $y$ abhängt, verschwindet die rechte Seite der
zweiten kanonischen Differentialgleichung, es gilt daher $p'=0$,
$p$ ist konstant.
Die erste kanonische Differentialgleichung besagt dann
\[
y'(x)
=
\frac{\partial H_1}{\partial p}(x,p_0)
\qquad\Rightarrow\qquad
y(x)
=
\int_{x_0}^x
\frac{\partial H_1}{\partial y}(\xi,p_0)
\,d\xi
+ y_0.
\]
$y(x)$ ist damit durch Quadratur bestimmt.

Falls $H$ nicht von $p$ abhängt, verschwindet die rechte Seite
der ersten kanonischen Differentialgleichung, es gilt daher $y'=0$,
$y$ ist konstant.
Die zweite kanonische Differentialgleichung besagt dann
\[
p'(x)
=
-
\frac{\partial H_2}{\partial y}(x,y_0)
\qquad\Rightarrow\qquad
p(x)
=
\int_{x_0}^x
\frac{\partial H_2}{\partial y}(\xi,y_0)
\,d\xi
+
p_0
\]
$p(x)$ ist damit durch Quadratur bestimmt.
\end{proof}

%
% Transversalitätsbedingung
%
\subsection{Transversalitätsbedingung}
Der Startpunkt für die Einführung der kanonischen Variable $p$ und
der Hamilton-Funktion war ihr auftreten in der Transversalitätsbedingung.
Die Lösung eines Anfangspunkt-Endkurve-Problems erfüllt am rechten 
Intervallende die Transversalitätsbedingung
\[
\vec{r_1}\cdot \vec{f}_1=0,
\]
der Tangentialvektor $\vec{r}_1$ an die Endkurve ist orthogonal auf
dem Vektor
\begin{align*}
\vec{f}_1
&=
\begin{pmatrix}
\displaystyle
F(x_1,y(x_1),y'(x_1))-y'(x_1)\frac{\partial F}{\partial y'}(x_1,y(x_1),y'(x_1))
\\[5pt]
\displaystyle
\frac{\partial F}{\partial y'}(x_1,y(x_1),y'(x_1))
\end{pmatrix}
\\
&=
\begin{pmatrix}
F(x_1,y(x_1),y'(x_1)) - y'(x_1) p(x_1) \\
p(x_1)
\end{pmatrix}
=
\begin{pmatrix}
-H(x_1,y(x_1),p(x_1))\\
p(x_1)
\end{pmatrix}.
\end{align*}

\begin{beispiel}
Für das Beispiel~\ref{buch:hamiltonjacobi:kanonisch:bsp:laenge}
für kürzeste Verbindungen in der Ebene haben wir bereits gefunden,
dass die kanonische Variable $p$ und die Hamilton-Funktion durch 
den Neigungswinkel $\alpha$ des Graphen einer Extremalen $y(x)$
ausgedrückt werden kann.
Mit $y'(x)=\tan\alpha$ ist der Vektor
\[
\vec{f}_1
=
\begin{pmatrix}
-H\\
p
\end{pmatrix}
=
\begin{pmatrix}
\cos\alpha\\
\sin\alpha
\end{pmatrix}.
\]
Dies ist der Richtungsvektor des Einheitstangentialvektors an
den Graphen von $y(x)$.
Die Transversalitätsbedingung ist also genau dann erfüllt, wenn die
Lösung orthogonal auf Endkurve trifft.
Die kürzeste Kurve zwischen einem Punkt und einer Kurve ist eine
Gerade durch den Punkt, die die Kurve orthogonal schneidet.
\end{beispiel}

%
% Mehrere unabhängige Variable
%
\subsection{Mehrere unabhängige Variable}
Die Transformation auf die kanonische Variable $p$ ist auch dann möglich,
wenn $y$ eine Vektorwertige Funktion ist.
Wir betrachten daher in diesem Abschnitt ein Variationsproblem für
eine Funktion $y\colon \mathbb{R}\to\mathbb{R}^n$, die das Integral
\[
I(y)
=
\int_{x_0}^{x_1}
F(x,y(x),y'(x))
\,dx
\]
mit der Lagrange-Funktion
\[
F(x,y,y')
=
F(x,y_1,\dots,y_n,y_1',\dots,y'_n)
\]
extremal machen soll.
Das Problem wird durch die Euler-Lagrange-Differentialgleichung
\begin{align*}
0
&=
\frac{\partial F}{\partial y}(x,y(x),y'(x))
-
\frac{d}{dx}
\frac{\partial F}{\partial y'}(x,y(x),y'(x))
\intertext{oder in Komponenten}
0
&=
\frac{\partial F}{\partial y_k}(x,y(x),y'(x))
-
\frac{d}{dx}
\frac{\partial F}{\partial y'_k}(x,y(x),y'(x)).
\end{align*}
Im Folgenden sei $y(x)$ jeweils eine Lösung dieser Differentialgleichungen.

%
% Kanonische Variablen
%
\subsubsection{Kanonische Variablen}
Wie im Fall einer einzigen unabhängigen Variablen verwenden wir den
Vektor
\[
p
=
\frac{\partial F}{\partial y'}(x,y,y')
\qquad\text{mit Komponenten}\qquad
p_k
=
\frac{\partial F}{\partial y_k'}(x,y,y')
\]
als neue Variable an Stelle von $y'$.
Damit $y'$ bei festem $x$ und $y$ durch $p$ ausgedrückt werden kann,
damit also die Abbildung 
\[
y'\mapsto f(y')=\frac{\partial F}{\partial y'}(x,y,y')
\]
mindestens in einer Umgebung eines Punktes invertiert werden kann,
muss die Ableitung der Funktion $f\colon \mathbb{R}^n\to\mathbb{R}^n$
nach $y'$ eine invertierbare Matrix sein.
Die Jacobi-Matrix
\[
J
=
\frac{\partial f}{\partial y'}
\qquad\text{mit Matrixelementen}\qquad
\biggl(
\frac{\partial f}{\partial y'}
\biggr)_{ik}
=
\frac{\partial f_i}{\partial y_k'}
=
\frac{\partial^2 F}{\partial y_i'\,\partial y_k'}
\]
muss invertierbar sein.
Dies ist gleichbedeutend damit, dass die Funktionaldeterminante
$\det J\ne 0$ ist.
Die Bedingung ist automatisch erfüllt, wenn die starke Legendre-Bedingung
für $F$ erfüllt ist.
Diese besagt, dass die Matrix $J$ positiv (oder negativ) definit ist, womit
sie automatisch regulär ist.
Da die Legendre-Bedingung notwendig ist in Variationsproblemen, in denen
die Lösungen der Euler-Lagrange-Differentialgleichung das Funktional 
extremal machen, können wir meist davon ausgehen,
dass auch $y'=y'(x,y,p)$ geschrieben werden kann.

%
% Hamilton-Funktion
%
\subsubsection{Hamilton-Funktion}
Die Definition der Hamilton-Funktion im Falle $n=1$ kann jetzt
naheliegend wie folgt erweitert werden.
Die Ableitung
\begin{equation*}
\frac{\partial}{\partial y'}
\bigl(
p\cdot y'
-
F(x,y,y')
\bigr)
=
p
-
\frac{\partial F}{\partial y'}(x,y,y')
=
0
\end{equation*}
verschwindet nach Definition von $p$.
Der Klammerausdruck hängt also nicht von $y'$ ab und kann daher 
als Funktion $x$, $y$ und $p$ ausgedrückt werden.

\begin{definition}[Hamilton-Funktion]
Die Hamilton-Funktion zur Lagrange-Funktion $F$ ist die Funktion
\[
H(x,y,p)
=
p\cdot y' - F(x,y,y'),
\]
wobei $y'$ durch $x$, $y$ und $p$ ausgedrückt werden muss.
\end{definition}

Aus der Definition folgend unmittelbar die partiellen Ableitungen
von $H$ nach $y$ und $p$, die
\begin{equation}
\frac{\partial H}{\partial y}
=
-\frac{\partial F}{\partial y}
\qquad\text{und}\qquad
\frac{\partial H}{\partial p}
=
y'
\label{buch:hamiltonjacobi:kanonisch:eqn:Habln}
\end{equation}
sind.

%
% Kanonische Differentialgleichungen
%
\subsubsection{Kannonische Differentialgleichungen}
Die zweite Differentialgleichung in
\eqref{buch:hamiltonjacobi:kanonisch:eqn:Habln}
ist bereits eine der gesuchten kanonischen Differentialgleichungen.
Die Euler-Lagrange-Differentialgleichung
\[
\frac{\partial F}{\partial y}=
\frac{d}{dx}\frac{\partial F}{\partial y'}
\qquad\Rightarrow\qquad
\frac{d}{dx}p
=
\frac{\partial F}{\partial y}
=
-\frac{\partial H}{\partial y}
\]
liefert die zweite kanonische Differentialgleichung nach der
ersten Gleichung von \eqref{buch:hamiltonjacobi:kanonisch:eqn:Habln}.
Damit folgt der folgende Satz.

\begin{satz}[Kanonische Differentialgleichungen]
\label{buch:hamiltonjacobi:kanonisch:satz:kandgln}
Jeder Lösung $y(x)$ der Euler-Lagrange-Differentialgleichung der
Lagrange-Funktion $F(x,y,y')$ entspricht eine Lösung
der {\em kanonischen Differentialgleichungen}
\begin{equation}
\begin{aligned}
y'&=\phantom{-}\frac{\partial H}{\partial p}
\\
p'&=-\frac{\partial H}{\partial y}.
\end{aligned}
\end{equation}
\end{satz}

\begin{beispiel}
\label{buch:hamiltonjacobi:kanonisch:bsp:3dlaenge}
Wir erweitern das Beispiel~\ref{buch:hamiltonjacobi:kanonisch:bsp:laenge}
auf den dreidimensionalen Fall.
Wir bezeichnen die drei Raumkoordinaten mit $x$, $y_1$ und $y_2$ und 
verwenden die Lagrange-Funktion
\[
F(x,y,y')
=
\sqrt{1+y^{\prime 2}}
=
\sqrt{1+y_1^{\prime 2}+y_2^{\prime 2}}.
\]
Die kanonischen Variablen entstehen durch die Ableitung nach $y'$, wir
setzen also
\begin{align}
p
&=
\frac{\partial F}{\partial y'}
=
\frac{y'}{\sqrt{1+y^{\prime 2}}}.
\label{buch:hamiltonjacobi:kanonisch:bsp:py}
\end{align}
Die Jacobi-Matrix
\begin{align*}
\frac{\partial^2 F}{\partial y^{\prime 2}}
&=
\begin{pmatrix}
\displaystyle
\frac{1+y_2^2}{(1+y^{\prime 2})^{\frac32}}
&
\displaystyle
-\frac{y_1'y_2'}{(1+y^{\prime 2})^{\frac32}}
\\[5pt]
\displaystyle
-\frac{y_1'y_2'}{(1+y^{\prime 2})^{\frac32}}
&
\displaystyle
\frac{1+y_1^2}{(1+y^{\prime 2})^{\frac32}}
\end{pmatrix}
\intertext{hat die Determinante}
\det
\frac{\partial^2 F}{\partial y^{\prime 2}}
&=
\frac{1}{(1+y^{\prime 2})^3}
\bigl(
(1+y_1^{\prime 2})(1+y_2^{\prime 2}) - y_1^{\prime 2}y_2^{\prime 2}
\bigr)
=
\frac{
1
+
y_1^{\prime 2} + y_2^{\prime 2}
}{
(1+y^{\prime 2})^3
}
\\
&=
\frac{1+y^{\prime 2}}{ (1+y^{\prime 2})^3}
=
\frac{1}{\sqrt{1+y^{\prime 2}}}
>
0,
\\
\operatorname{Spur}
\frac{\partial^2 F}{\partial y^{\prime 2}}
&=
\frac{2+y_1^{\prime 2}+y_2^{\prime 2}}{(1+y^{\prime 2})^{\frac32}}
>0,
\end{align*}
somit ist die Ableitung positiv definit, die starke Legendre-Bedingung
ist erfüllt und $y'$ kann durch $x$, $y$ und $p$ ausgedrückt werden.

Die Gleichungen~\label{buch:hamiltonjacobi:kanonisch:bsp:py} können
invertiert werden, wenn der skalare Nenner durch $p$ ausgedrückt
werden kann.
Dazu multipliziert man die Gleichung mit sich selbst und erhält
\[
\begin{aligned}
p^2
&=
\frac{y^{\prime 2}}{1+y^{\prime 2}}
&&\Rightarrow&
p^2(1+y^{\prime 2})&=y^{\prime2}
\\
&&&\Rightarrow&
p^2
&=
(1-p^2)y^{\prime 2}
\\
&&&\Rightarrow&
y^{\prime 2}
&=
\frac{p^2}{1-p^2}
\\
&&&\Rightarrow&
1+y^{\prime2}
&=
\frac{1}{1-p^2}
\\
&&&\Rightarrow&
\sqrt{1+y^{\prime2}}
&=
\frac{1}{\sqrt{1-p^2}}.
\end{aligned}
\]
Damit kann man jetzt $y'$ durch $p$ als
\[
y'
=
\frac{p}{\sqrt{1-p^2}}
\]
ausdrücken.
Auch die Legendre-Funktion kann durch $p$ ausgedrückt werden,
da
\[
F(x,y,y')
=
\sqrt{1+y^{\prime 2}}
=
\frac{1}{\sqrt{1-p^2}}
\]
ist.

Die Hamilton-Funktion ist
\begin{align*}
H(x,y,p)
&=
y'\cdot p - F(x,y,y')
=
\frac{p}{\sqrt{1-p^2}}\cdot p - \frac{1}{\sqrt{1-p^2}}
=
\frac{p^2-1}{\sqrt{1-p^2}}
=
-\sqrt{1-p^2},
\end{align*}
insbesondere hängt sie nicht von $y$ ab.
Die Ableitung nach $p$ ist
\[
\frac{\partial H}{\partial p}
=
\frac{p}{\sqrt{1-p^2}}.
\]
Damit werden die kanonischen Differentialgleichungen
\begin{align*}
y' &= \frac{p}{\sqrt{1-p^2}} \\
p' &= 0.
\end{align*}
Somit ist $p$ konstant und $y'$ kann durch Quadratur als
\[
y(x)
=
y_0 + \frac{p}{\sqrt{1-p^2}}(x-x_0)
\]
bestimmt werden.
\end{beispiel}

%
% Transversalitätsbedingung
%
\subsubsection{Transversalitätsbedingung}
Die Verallgemeinerung der Transversalitätsbedingungen auf mehrere
Funktionen verlangt, dass jeder Tangentialvektor an die Randfläche
senkrecht steht auf dem Vektor
\[
\vec{f}_1
=
\begin{pmatrix}
\displaystyle
F(x_1,y(x_1),y'(x_1))
-
y'(x_1)\cdot \frac{\partial F}{\partial y'}(x_1,y(x_1),y'(x_1))
\\
\displaystyle
\frac{\partial F}{\partial y_1'}(x_1,y(x_1),y'(x_1))
\\[9pt]
\displaystyle
\frac{\partial F}{\partial y_2'}(x_1,y(x_1),y'(x_1))
\end{pmatrix}
=
\begin{pmatrix}
-H(x_1,y(x_1),p(x_1)\\
p_1(x_1)\\
p_2(x_1)
\end{pmatrix}.
\]

\begin{beispiel}
In Beispiel~\ref{buch:hamiltonjacobi:kanonisch:bsp:3dlaenge} wurde
die Hamilton-Funktion für das Variationsproblem für kürzeste Verbindungen
im Raum bestimmt.
Der Vektor $\vec{f}$ 
\begin{align*}
\vec{f}_1
&=
\begin{pmatrix}
-H(x_1,y(x_1),p(x_1)\\
p_1(x_1)\\
p_2(x_2)
\end{pmatrix}
=
\begin{pmatrix}
\sqrt{1-p^2}\\
p_1\\
p_2
\end{pmatrix}
\end{align*}
ist ein Vektor mit Länge
\[
|\vec{f}_1|
=
(1-p^2) + p_1^2+p_2^2
=
1,
\]
Der Vektor $\vec{f}_1$ ist also ein Tangentialeinheitsvektor an die
durch $y(x)$ beschriebene Kurve.
Die Transversalitätsbedingung sagt daher, dass die Kurve kürzester
Länge von einem Punkt zu einer Fläche senkrecht auf die Fläche trifft.
\end{beispiel}



%
% 2-jacobi.tex
%
% (c) 2024 Prof Dr Andreas Müller
%
\section{Jacobi-Theorie
\label{buch:hamiltonjacobi:section:jacobi}}
\kopfrechts{Jacobi-Theorie}
%
% kreis.tex -- template for standalon tikz images
%
% (c) 2021 Prof Dr Andreas Müller, OST Ostschweizer Fachhochschule
%
\documentclass[tikz]{standalone}
\usepackage{amsmath}
\usepackage{times}
\usepackage{txfonts}
\usepackage{pgfplots}
\usepackage{csvsimple}
\usetikzlibrary{arrows,intersections,math}
\definecolor{darkred}{rgb}{0.8,0,0}
\begin{document}
\def\skala{1}
\def\dx{4.5}
\def\dy{4.5}
\input{kreispfad.tex}
\begin{tikzpicture}[>=latex,thick,scale=\skala]

\draw[color=blue!40] \pfada;
\draw[color=blue!40] \pfadb;
\draw[color=blue!40] \pfadc;
\draw[color=blue!40] \pfadd;
\draw[color=blue!40] \pfade;
\draw[color=blue!40] \pfadf;
\draw[color=blue!40] \pfadg;
\draw[color=blue!40] \pfadh;
\draw[color=blue!40] \pfadi;
\draw[color=blue!40] \pfadj;
\draw[color=blue!40] \pfadk;
\draw[color=blue!40] \pfadl;
\draw[color=blue!40] \pfadm;
\draw[color=blue!40] \pfadn;
\draw[color=blue!40] \pfado;
\draw[color=blue!40] \pfadp;
\draw[color=blue!40] \pfadq;

\faechera
\faecherb
\faecherc
\faecherd
\faechere
\faecherf
\faecherg
\faecherh
\faecheri
\faecherj
\faecherk
\faecherl
\faecherm
\faechern
\faechero
\faecherp
\faecherq

\draw[color=darkred!40,line width=1.2pt] \kreisfaechera;
\draw[color=darkred!40,line width=1.2pt] \kreisfaecherb;
\draw[color=darkred!40,line width=1.2pt] \kreisfaecherc;
\draw[color=darkred!40,line width=1.2pt] \kreisfaecherd;
\draw[color=darkred!40,line width=1.2pt] \kreisfaechere;
\draw[color=darkred!40,line width=1.2pt] \kreisfaecherf;
\draw[color=darkred!40,line width=1.2pt] \kreisfaecherg;
\draw[color=darkred!40,line width=1.2pt] \kreisfaecherh;
\draw[color=darkred!40,line width=1.2pt] \kreisfaecheri;
\draw[color=darkred!40,line width=1.2pt] \kreisfaecherj;
\draw[color=darkred!40,line width=1.2pt] \kreisfaecherk;
\draw[color=darkred!40,line width=1.2pt] \kreisfaecherl;
\draw[color=darkred!40,line width=1.2pt] \kreisfaecherm;
\draw[color=darkred!40,line width=1.2pt] \kreisfaechern;
\draw[color=darkred!40,line width=1.2pt] \kreisfaechero;
\draw[color=darkred!40,line width=1.2pt] \kreisfaecherp;
\draw[color=darkred!40,line width=1.2pt] \kreisfaecherq;

\draw[color=darkred,line width=1.4pt] \kreiseins;
\draw[color=darkred,line width=1.4pt] \kreiszwei;

\draw[->,color=black] (-0.1,0) -- ({2.65*\dx},0) coordinate[label={$x$}];
\draw[->,color=black] (0,{-0.8*\dy}) -- (0,{1.40*\dy}) coordinate[label={right:$y$}];

\end{tikzpicture}
\end{document}


In diesem Abschnitt entwickeln wir eine geometrische Beschreibung des
Variationsproblems, die besondere Bedeutung der Hamilton-Funktion hervorhebt.
Wir gehen dazu aus von einem Funktional
\begin{equation}
J(y)
=
\int_{x_0}^{x_1}
F(x,y(x),y'(x))
\,dx
\label{buch:hamiltonjacobi:jacobi:eqn:funktionalJ}
\end{equation}
mit einer Lagrange-Funktion $F(x,y,y')$.
Als Beispiel wird jeweils die Lichtausbreitung in einem inhomogenen
Medium mit Brechungsindex $n(y) = 1+\nu y$ dienen, welches die
Lagrange-Funktion
\begin{equation}
F(x,y,y')
=
n(y) \sqrt{1+y^{\prime 2}}
\label{buch:hamiltonjacobi:jacobi:eqn:beispielF}
\end{equation}
verwendet.
Im Übrigen verwenden wir wieder die Bezeichnungen des vorangegangenen
Abschnitts.

%
% Quasi-Länge und Quasikreise
%
\subsection{Quasilänge und Quasikreise
\label{buch:hamiltonjacobi:jacobi:subsection:quasi}}
Der Wert des Funktionals $J(y)$ von
\eqref{buch:hamiltonjacobi:jacobi:eqn:funktionalJ}
kann als ein alternatives Mass für die Entfernung zwischen
zwei Punkten verwendet werden.
In Kapitel~\ref{chapter:geodaeten} wird das übliche Riemannsche
Funktional zur Berechnung der Länge einer Kurve vorgestellt, mit dem
man sowohl die Geodäten als die kürzesten Verbindungen wie auch
den geodätischen Abstand als den Wert des Funktionals definiert.
Die geometrische Sprechweise für das Funktional 
\eqref{buch:hamiltonjacobi:jacobi:eqn:funktionalJ}
verwendet werden.

%
% Quasilänge und $J$-Abstand
%
\subsubsection{Quasilänge und $J$-Abstand}
Sei $y(x)$ eine Funktion, deren Graph die Punkte $P_0=(x_0,y_0)$
und $P_1=(x_1,y_1)$ verbindet.
Dann nennen wir den Wert des Funktionals $J(y)$ den {\em $J$-Länge}
der verbindenden Kurve.
Wenn es unter allen Kurven, die die Punkte $P_0$  und $P_1$ verbinden,
auch eine Extremale $y(x)$ gibt, dann bezeichnen wir den extremalen
Wert des Funktionals $J(y)$ den $J$-Abstand der beiden Punkte.
Wir bezeichnen den $J$-Abstand auch mit $J(P_0,P_1)$.

%
% Quasikreise
%
\subsubsection{Quasikreise}
Wir halten den Punkt $P_0=(x_0,y_0)$ weiterhin fest und betrachten die
Menge
\[
K_J(P_0,\varrho)
=
\{
P_1=(x_1,y_1)\in\mathbb{R}^2
\mid
J(P_0,P_1) = \varrho
\}
\]
aller Punkte $P_1=(x_1,y_1)$, für die der $J$-Abstand zwischen
zu $P_0$ einen vorgegebenen Wert $\varrho$ annimmt.
$K_J(P_0,\varrho)$ heisst auch der {\em Quasikreis} oder {\em $J$-Kreis}
mit Zentrum $P_0$ und $J$-Radius oder Quasiradius $\varrho$.

\begin{beispiel}
In Abbildung~\ref{buch:hamiltonjacobi:hamiltonjacobi:kreis} ist der
$J$-Kreis um den Punkt $P_0=(0,0)$ mit Radius $1$ für das Funktional
mit der Lagrange-Funktion~\eqref{buch:hamiltonjacobi:jacobi:eqn:beispielF}
dargestellt.
Der hellgrüne Hintergrund symbolisiert die optische Dichte des Mediums,
in dem sich die Lichstrahlen ausbreiten.
Die Lichtgeschwindigkeit ist also im unteren Teil des Bildes grösser,
die Lichstrahlen krümmen sich daher nach oben.

In der gleichen Zeit können die Lichstrahlen im unteren Teil auch
eine grössere Strecke überwinden.
Die beiden eingezeichneten Quasikreise mit Radius 1 bzw.~2 sind 
daher vor allem dann nach rechts verzerrt, wenn die dort
endenden Extremalen, weit im negativen Bereich der Ebene verlaufen.
Die Extremalen mit grosser Steigung erreichen schnell Bereich höherer
optischer Dichte, wo Lichstrahlen mehr Zeit für die gleiche Distanz
brauchen.
Die Quasikreise nähern sich daher für grösser werdendes $y$ an.
\end{beispiel}

%
% Transversalitätsbedingung
%
\subsubsection{Transversalitätsbedingung}
Zu jedem Punkt $P_1\in K_J(P_0,\varrho)$ gibt es eine Lösung des
Anfangspunkt-Endpunkt-Problems mit dem Funktional
\eqref{buch:hamiltonjacobi:jacobi:eqn:funktionalJ}.
Jede solche Lösung ist aber auch eine Lösung des
Anfangspunkt-Endkurve-Problems mit Anfangspunkt $P_0$ und
Endkurve $K_J(P_0,\varrho)$.
Insbesondere erfüllt jede solche Extremale die Transversalitätsbedingung,
die man am einfachsten mit der Hamilton-Funktion schreiben kann.

\begin{beispiel}
\label{buch:hamiltonjacobi:jacobi:bsp:beispielFH}
Für die Lagrange-Funktion~\eqref{buch:hamiltonjacobi:jacobi:eqn:beispielF}
ist die konjugierte Koordinate
\[
p
=
\frac{\partial F}{\partial y}
=
n'(y)\sqrt{1+y^{\prime 2}}
=
\nu \sqrt{1+y^{\prime 2}}.
\]
Aufgelöst nach $y'$ ist dies
\begin{align*}
\frac{p}{\nu}
&=
\sqrt{1+y^{\prime 2}}
\\
y'
&=
\sqrt{ \frac{p^2}{\nu^2} -1 }
\end{align*}
Damit kann man jetzt die Hamilton-Funktion
\[
H(x,y,p)
=
y'p - F(x,y,y')
=
p\sqrt{\frac{p^2}{\nu^2}-1}
-
n(y) \frac{p}{\nu}
\]
durch $y$ und $p$ ausdrücken.
Damit sind die Koeffizienten für die Transfersalitätsbedingung
\[
\vec{f}_1
=
\begin{pmatrix} 
-H\\
p
\end{pmatrix}
\]
gefunden.
Der Vektor $\vec{f}_1$ steht auf der Quasikugel senkrecht.
\end{beispiel}

%
% Quasikugeln
%
\subsubsection{Quasikugeln}
Die Idee der Quasilänge und der Quasikugeln ist nicht auf eine abhängige
Variable beschränkt.
Für die Lagrange-Funktion 
$F\colon \mathbb{R}\times\mathbb{R}^n\times\mathbb{R}^n\to\mathbb{R}$
können wir das Funktional
\[
J(y) = \int_{x_0}^{x_1} F(x,y(x),y'(x))\,dx
\]
verwenden, um den $J$-Abstand oder Quasiabstand zwischen zwei Punkten
definieren.
Die $(n+1)$-dimensionale Quasikugel um den Punkt $P_0$ mit Radius $\varrho$
besteht dann aus den Punkten, die den $J$-Abstand $\varrho$ von $P_0$ haben.
Die Extremalen zwischen $P_0$ und einem Punkt $P_1$ auf der Quasikugel
erfüllen im Punkt $P_1$ die Transversalitätsbedingung, der Vektor
\[
\vec{f}_1
=
\begin{pmatrix}
-H(x,y,p)\\
p_1\\
\vdots\\
p_n
\end{pmatrix}
\]
steht auf der Quasikugel senkrecht.

%
% Die Fundamentalfunktion $S$
%
\subsection{Die Fundamentalfunktion $S$}
Wir betrachten jetzt ein Problem mit einer gegeben Anfangskurve,
die wir mit $\gamma_0$ bezeichnen.

%
% Konstruktion der Funktion $S$
%
\subsubsection{Konstruktion der Funktion $S$}
Zu jedem Punkt $P_1=(x_1,y_1)$ suchen wir eine Lösung des
Anfangskurve-Endpunkt-Problems mit der Anfangskurve $\gamma_0$
und dem Endpunkt $P_1$.
Die Lösung ist eine Funktion $y(x)$, die einen Punkt $P_0=(x_0,y_0)$ auf der
Kurve $\gamma_0$ mit dem Punkt $P_1$ verbindet.
Wir definieren den Wert der {\em Fundamentalfunktion} $S$ auf dem
\index{Fundamentalfunktion}%
Punkt $P_1$ als
\begin{equation}
S(x_1,y_1)
:=
J(y).
\label{buch:hamiltonjacobi:jacobi:eqn:Sdef}
\end{equation}
Die Funktion $S$ hat die Eigenschaft, dass $S(P)=0$ ist für alle
Punkte $P$ auf der Kurve $\gamma_0$.
Die Extremale $y(x)$ erfüllt im Punkt $P_0$ die Transversalitätsbedingung.

%
% Das huygenssche Prinzip
%
\subsubsection{Das huygenssche Prinzip}
\index{Prinzip von Huygens}%
\index{huygensches Prinzip}%
Die Menge 
\[
L(\varrho)
=
\{
P_1=(x_1,y_1)\in\mathbb{R}^2
\mid
S(P_1)=\varrho
\}
\]
besteht aus den Punkten, für die es eine Extremale der Quasilänge
$\varrho$ von der Anfangskurve $\gamma_0$ gibt.
Eine solche Extremale erfüllt die Transversalitätsbedingung für die
Anfangskurve im Anfangspunkt und die Transversalitätsbedingung
für die Kurve $L(\varrho)$ für den Endpunkt $P_1$.

Diese Konstruktion erinnert an das huygensche Prinzip der Wellenausbreitung.
Die Kurve $S(\varrho)$ ist die Wellenfront zur Zeit $t=\varrho$ einer Welle,
die sich zur Zeit $t=0$ von der Anfangskurve gelöst hat.
Sie entsteht als Einhüllende aller Quasikreise vom Radius $\varrho$ 
um Punkte der Anfangskurve $\gamma_0$.

\begin{beispiel}
In Abbildung~\ref{buch:hamiltonjacobi:hamiltonjacobi:kreis} sind
ausgehend von Punkten $P$ des Kreises $K_J((0,0),1)$ die Kreise
$K_J(P,1)$ hellrot dargestellt.
Sie berühren alle den roten Kreis $K_J((0,0),2)$.
\end{beispiel}

%
% Der Gradient von S
%
\subsubsection{Der Gradient von $S$}
Aus der Transversalitätsbedingung, die die Extremalen bei $P_1$ erfüllen
müssen folgt, dass die Normale auf die Kurve $L(\varrho)$ parallel
zum Vektor $\vec{f}_1$ sein muss.
Die Normale ist auch parallel zum Gradienten von $S$.
In diesem Abschnitt zeigen wir, dass der Gradient sogar mit dem 
Vektor $\vec{f}_1$ übereinstimmt.

Wir untersuchen, wie sich die der Wert $S(P)$ der Funktion $S$ ändert,
wenn man den Punkt $P$ verschiebt.
Sei als eine Variation der Extremalen $y(x)$ gegeben, die den Punkt $P_0$
mit $P$ verbindet.
Wir schreiben die Variation als Vektor
\[
\begin{pmatrix}
\delta x\\
\delta y
\end{pmatrix}
\]
Die allgemeine Variationsformel, die wir in 
Abschnitt~\ref{buch:hamiltonjacobi:section:kanonisch}
in der kanoninschen Variable und der Hamilton-Funktion ausgedrückt
haben, liefert für diese Variation
\[
\delta J
=
\Bigl[ -H\delta x + p \delta y \Bigr]_{x_0}^{x_1}
+
\int_{x_0}^{x_1}
\biggl(
\frac{\partial F}{\partial y}(x,y(x),y'(x))
-
\frac{d}{dx}
\frac{\partial F}{\partial y'}(x,y(x),y'(x))
\biggr)
\delta y
\,dx.
\]
Da $y(x)$ eine Extremale ist, trägt das Integral nichts zur
Variation bei.
Die Änderung von $S$ ist nach Definition die Variation $\delta J$,
wir können daher
\[
\delta S
=
-H\delta x(x_1) + p\delta y(x_1)
\]
schreiben.
Die Koeffizienten von $\delta x(x_1)$ und $\delta y(x_1)$ sind die
partiellen Ableitungen von $S$ nach den Variablen $x$ und $y$, also
gilt
\[
\operatorname{grad} S
=
\begin{pmatrix}
-H\\
p
\end{pmatrix}.
\]

%
% Hamilton-Jacobi-Differentialgleichung
%
\subsection{Hamilton-Jacobi-Differentialgleichung
\label{buch:hamiltonjacobi:jacobi:subsection:HJ-DGL}}
Die Fundamtenalfunktion $S(x,y)$ hat die partiellen
Ableitungen
\[
\frac{\partial S}{\partial x}
=
-H(x,y,p)
\qquad\text{und}\qquad
\frac{\partial S}{\partial y}
=
p.
\]
Durch Eliminieren von $p$ entsteht die partielle Differentialgleichung
\begin{equation}
\frac{\partial S}{\partial x}
+
H\biggl(x,y,\frac{\partial S}{\partial y}\biggr).
\label{buch:hamiltonjacobi:jacobi:eqn:hamilton-jacobi-dgl}
\end{equation}
Dies ist im Allgemeinen eine nichtlineare Differentialgleichung
erster Ordnung.

\begin{beispiel}
Wir betrachten das Funktional mit der Lagrange-Funktion 
$L=\sqrt{1+y^{\prime 2}}$.
Die zugehörige Hamilton-Funktion wurde bereits in
Beispiel~\ref{buch:hamiltonjacobi:section:kanonisch}
bestimmt, sie war
\[
H(x,y,p)
=
-\sqrt{1-p^2}.
\]
Die Hamilton-Jacobi-Differentialgleichung ist daher
\begin{align}
\frac{\partial S}{\partial x}
+
H\biggl(x,y,\frac{\partial S}{\partial y}\biggr)
&=
0
\notag
\\
\frac{\partial S}{\partial x}
&=
\sqrt{1-\biggl(\frac{\partial S}{\partial y}\biggr)^2}.
\notag
\intertext{Durch Quadrieren können alle Ableitungen von $S$ auf die
linke Seite gebracht werden, es bleibt die Differentialgleichung}
\biggr(\frac{\partial S}{\partial x}\biggr)^2
+
\biggr(\frac{\partial S}{\partial y}\biggr)^2
&=
1.
\label{buch:hamiltonjacobi:jacobi:eqn:eikonal}
\end{align}
Diese Gleichung heisst auch {\em Eikonalgleichung}.
\index{Eikonalgleichung}%
\end{beispiel}

\begin{beispiel}
In Beispiel~\ref{buch:hamiltonjacobi:jacobi:bsp:beispielFH} wurde
die Hamilton-Funktion für das Funktional mit der Lagrange-Funktion
\eqref{buch:hamiltonjacobi:jacobi:eqn:beispielF} bereits berechnet.
Um die Hamilton-Jacobi-Differentialgleichung zu erhalten, müssen
wir in
\[
\frac{\partial S}{\partial x}
=
\frac{1}{\nu}p\sqrt{p^2-\nu^2} -\frac{n(y)}{\nu} p
\]
$p$ durch $\partial S/\partial y$ ersetzen. 
%Bevor wir dies tun, quadrieren wir die Gleichung und formen die
%rechte Seite um:
%\begin{align*}
%\biggl(
%\frac{\partial S}{\partial x}
%\biggr)^2
%=
%\frac{1}{\nu^2} p^2 (p^2-\nu^2)
%-2\frac{n(y)}{\nu^2}p^2\sqrt{p^2-\nu^2}
%+
%\frac{n(y)^2}{\nu^2}p^2
%\end{align*}
%Wir können daher die Hamilton-Jacobi-Differentialgleichung aufstellen,
%indem wir $p$ durch $\partial S/\partial y$ ersetzen:
und erhalten
\begin{align*}
\frac{\partial S}{\partial x}
&=
\frac{1}{\nu}
\frac{\partial S}{\partial y}
\sqrt{
\bigg(\frac{\partial S}{\partial y}\biggr)^2
-
\nu^2
}
-
\frac{n(y)}{\nu}
\frac{\partial S}{\partial y}
\end{align*}
als Hamilton-Jacobi-Differentialgleichung.
\end{beispiel}

Um die Lösung der Hamilton-Jacobi-Differentialgleichung eindeutig
festzulegen, müssen noch Randbedingungen spezifiziert werden.
Die ausgehend von der Anfangskurve $\gamma_0$ konstruiert Funktion
$S(x,y)$ erfüllt zusätzlich die homogene Dirichlet Randbedingung
$S(P)=0$ für $P\in\gamma_0$.
Umgekehrt ist eine
beliebige Lösung der Hamilton-Jacobi-Diffe\-ren\-tial\-gleichung
auch die Fundamentalfunktion für die Anfangskurve ist, die aus den
Punkten $P$ mit $S(P)=0$ besteht.

%
% Lösungen der kanonischen Differentialgleichungen
%
\subsection{Lösungen der kanonischen Differentialgleichungen}
Die Extremalen sind Lösungen der kanonischen Differentialgleichungen
\begin{equation}
\begin{aligned}
\frac{dy}{dx}&=\phantom{-}\frac{\partial H}{\partial p}
\\
\frac{dp}{dx}&=-\frac{\partial H}{\partial y}.
\end{aligned}
\label{buch:hamiltonjacobi:jacobi:eqn:kanonisch}
\end{equation}
Im Punkt $(x_0,y_0)$ sei $\vec{r}_0$ der Tangentialvektor an die
Anfangskurve.
Wählt man $p_0$ so, dass die Transversalitätsbedingung
\[
\vec{r}_0
\cdot
\begin{pmatrix}
-H(x_0,y_0,p_0)\\
p_0
\end{pmatrix}
=
0
\]
erfüllt ist, dann ist die Lösung von
\eqref{buch:hamiltonjacobi:jacobi:eqn:kanonisch}
mit Anfangsbedingung $y(x_0)=y_0$ und $p(x_0)=p_0$ eine Extremale.
Der Gradient der Fundamentalfunktion ist durch $-H$ und $p$ gegeben,
ihr Wert entlang der Lösungskurve wird daher durch das Integral
\begin{align*}
S(x_1,y_1)
&=
\int_{x_0}^{x_1}
-H(x,y(x),p(x))
+
p(x) y'(x)
\,dx
\\
&=
\int_{x_0}^{x_1}
-H(x,y(x),p(x))
+
p(x)\frac{\partial H}{\partial p}
\,dx
\end{align*}
gegeben.
Aus einer Lösung des Differentialgleichungssystems
\eqref{buch:hamiltonjacobi:jacobi:eqn:kanonisch}
lässt sich also immer auch die Fundamentalfunktion berechnen.

%
% Der Satz von Jacobi
%
\subsubsection{Der Satz von Jacobi}
Wir untersuchen jetzt die umgekehrt Schlussrichtung: Kann man aus
der Fundamentalfunktion $S$ eine Lösung der kanonischen Gleichungen 
ableiten.
Sei also $S$ eine Lösung der Differentialgleichung
\begin{equation}
\frac{\partial S}{\partial x}
=
H\biggl(x,y,\frac{\partial S}{\partial y}\biggr).
\label{buch:hamiltonjacobi:jacobi:hjdgl1}
\end{equation}
In der Konstruktion der Fundamentalfunktion aus den kanonischen Gleichungen
musste erst ein zusätzlicher Anfangswert $p_0$ gefunden werden,
der von der Steigung der Anfangskurve abhing.
Die Lösung $S$ der
Differentialgleichung~\eqref{buch:hamiltonjacobi:jacobi:hjdgl1}
soll daher zusätzlich von einem Parameter $a$ abhängen, soll also
die Form
\(
S(x,y,a)
\)
haben.
Eine solche Lösung heisst ein {\em vollständiges Integral} von
\label{buch:hamiltonjacobi:jacobi:hjdgl1}.

Die Gleichung
\begin{align}
\frac{\partial S}{\partial a}(x,y,a)
&=
b
\label{buch:hamiltonjacobi:jacobi:eqn:Sa1}
\intertext{definiert implizit eine Funktion $y(x)$, aus der mit Gleichung}
\frac{\partial S}{\partial y}(x,y(x),a)
= 
p(x)
\label{buch:hamiltonjacobi:jacobi:eqn:Sa2}
\end{align}
bestimmt werden kann.
Die implizite Definition \eqref{buch:hamiltonjacobi:jacobi:eqn:Sa1}
definiert die Funktion $y(x)$ nur dann, wenn die Ableitung der linken
Seite nach $y$ nicht verschwindet. 
Wir müssen also zusätzlich von der Funktion $S$ die Bedingung
\begin{equation}
\frac{\partial^2 S}{\partial y\,\partial a}\ne 0
\label{buch:hamiltonjacobi:jacobi:invertierungsbedingung}
\end{equation}
verlangen.

Wir prüfen nach, dass die Funktion $y(x)$ und $p(x)$ eine Lösung der
kanonischen Differentialgleichungen
\eqref{buch:hamiltonjacobi:jacobi:hjdgl1}
sind.
Da die Gleichung
\label{buch:hamiltonjacobi:jacobi:eqn:Sa1}
die Funktion implizit definiert, leiten wir die Gleichung
\eqref{buch:hamiltonjacobi:jacobi:eqn:Sa1}
nach $x$ und die Hamilton-Jacobi-Differentialgleichung
\eqref{buch:hamiltonjacobi:jacobi:hjdgl1}
nach $a$ ab und erhalten
\begin{align*}
\frac{\partial^2 S}{\partial x\,\partial a}
+
\frac{\partial^2 S}{\partial y\,\partial a}
\frac{dy}{dx}
&=0
\\
\frac{\partial^2 S}{\partial a\,\partial x}
+
\frac{\partial H}{\partial p}\frac{\partial^2 S}{\partial a\,\partial y}
&=0.
\intertext{Die Differenz dieser Differentialgleichungen ist}
\frac{\partial^2 S}{\partial y\,\partial a}
\biggl(\frac{dy}{dx}-\frac{\partial H}{\partial p}\biggr)
&=
0.
\end{align*}
Wegen der Bedingung
\eqref{buch:hamiltonjacobi:jacobi:invertierungsbedingung} können wir
schliessen, dass die Klammer verschwindet, dass also die erste
der kanonischen Differentialgleichungen erfüllt ist.

Ähnlich erhält man durch Ableitung von
\eqref{buch:hamiltonjacobi:jacobi:eqn:Sa2} nach $x$ und
der Hamilton-Jacobi-Differen\-tial\-gleichung nach $y$ die Gleichungen
\begin{align*}
\frac{dp}{dx}
&=
\frac{\partial^2 S}{\partial x\,\partial y}
+
\frac{\partial^2 S}{\partial y^2} \frac{dy}{dx}
\\
0
&=
\frac{\partial^2 S}{\partial y\,\partial x}
+
\frac{\partial H}{\partial y}
+
\frac{\partial H}{\partial p}\frac{\partial^2 S}{\partial y^2}
\intertext{mit der Differenz}
\frac{dp}{dx}
&=
\frac{\partial^2 S}{\partial y^2}
\biggl(
\frac{dy}{dx}-\frac{\partial H}{\partial p}
\biggr)
-
\frac{\partial H}{\partial y}.
\intertext{Wir wissen bereits, dass die Klammer verschwindet, es bleibt}
\frac{dy}{dx}
&=
-\frac{\partial H}{\partial y},
\end{align*}
die zweite der kanonischen Differentialgleichungen ist also auch erfüllt.
Damit haben wir den einfachsten Fall des folgenden Satzes von Jacobi
bewiesen.

\begin{satz}[Jacobi]
Ist die Funktion $S(x,y,a)$ ein vollständiges Integral der
Hamilton-Jacobi-Differentialgleichung und ist
\[
\frac{\partial^2 S}{\partial a\,\partial y}\ne 0,
\]
dann definieren die Gleichungen
\begin{align*}
\frac{\partial S}{\partial a}(x,y(x),a) &= b \\
\frac{\partial S}{\partial y}(x,y(x),a) &= p(x)
\end{align*}
die allgemeine Lösung der kanonischen Differentialgleichungen, die von
den zwei Konstanten $a$ und $b$ abhängt.
\end{satz}

%
% Der Satz on Jacobi für das $n$-dimensionale Problem
%
\subsubsection{Der Satz von Jacobi für das $n$-dimensionale Problem}
Die Funktion $S(x,y,a)$ mit $a\in\mathbb{R}^n$ heisst ein
vollständiges Integral für die Hamilton-Jacobi-Differentialgleichung
\begin{equation}
\frac{\partial S}{\partial x}
+
H\biggl(x,y
\frac{\partial S}{\partial y}
\biggr),
\label{buch:hamiltonjacobi:jacobi:eqn:hjn}
\end{equation}
wenn $(x,y)\mapsto S(x,y,a)$ ein Lösung 
\eqref{buch:hamiltonjacobi:jacobi:eqn:hjn} ist.
Sei ausserdem die Matrix mit den Einträgen
\begin{equation}
\frac{\partial^2 S}{\partial y_i\,\partial a_k}
\label{buch:hamiltonjacobi:jacobi:eqn:regulaer}
\end{equation}
regulär.
Dann gilt der folgende Satz.

\begin{satz}[Jacobi]
Ist $S$ ein vollständiges Integral der Hamilton-Jacobi-Differentialgleichung
\eqref{buch:hamiltonjacobi:jacobi:eqn:hjn}, welches die Regularitätsbedingung
\eqref{buch:hamiltonjacobi:jacobi:eqn:regulaer} erfüllt,
dann definieren die Gleichungen
\begin{align}
\frac{\partial S}{\partial a}(x,y(x),a) &= b
\label{buch:hamiltonjacobi:jacobi:eqn:implizit1}
\\
\frac{\partial S}{\partial y}(x,y(x),a) &= p(x)
\label{buch:hamiltonjacobi:jacobi:eqn:implizit2}
\end{align}
für jedes $b\in\mathbb{R}^n$ zwei Funktionen $y(x)$ und $p(x)$, die
Lösungen der kanonischen Differentialgleichungen
\begin{align*}
\frac{dy}{dx} &= \phantom{-}\frac{\partial H}{\partial p}
\\
\frac{dp}{dx} &= -\frac{\partial H}{\partial y}
\end{align*}
sind.
\end{satz}

\begin{proof}
Der Beweis folgt genau der Rechnung für den eindimensionalen Fall.
Die Ableitung der Gleichung
\eqref{buch:hamiltonjacobi:jacobi:eqn:implizit1}
nach $y$ und Ableitung der Hamilton-Jacobi-Differentialgleichung
nach $a$ ergeben die Gleichungen
\begin{align*}
\frac{\partial^2 S}{\partial x\,\partial a_k}
+
\sum_{i=1}^n
\frac{\partial^2 S}{\partial y_i \partial a_k}
\frac{dy_i}{dx}
&= 0
\\
\frac{\partial^2 S}{\partial a_k\,\partial x}
+
\sum_{i=1}^n
\frac{\partial H}{\partial p_i}
\frac{\partial^2 S}{\partial a_k\partial y_i}
&=0
\intertext{mit der Differenz}
\frac{\partial^2 S}{\partial y_i\,\partial a_k}
\biggl(
\frac{dy_i}{dx}-\frac{\partial H}{\partial p_i}
\biggr)
&=0.
\end{align*}
Dank der Bedingung
\eqref{buch:hamiltonjacobi:jacobi:eqn:regulaer}
muss die Klammer verschwinden, die ersten kanonischen Differentialgleichungen
sind damit erfüllt.

Die Ableitung von
\eqref{buch:hamiltonjacobi:jacobi:eqn:implizit2}
nach $x$
und der Hamilton-Jacobi-Differentialgleichung nach $y$
erhalten wir
\begin{align*}
\frac{dp_k}{dx}
&=
\frac{\partial^2 S}{\partial x\,\partial y_k}
+
\sum_{i=1}^n
\frac{\partial^2 S}{\partial y_i\,\partial y_k}\frac{dy_i}{dx}
\\
0
&=
\frac{\partial^2 S}{\partial y_k\,\partial x}
+
\sum_{i=1}^n
\frac{\partial H}{\partial p_i}
\frac{\partial^2 S}{\partial y_k\,\partial y_i}
+
\frac{\partial H}{\partial y_k}
\intertext{mit der Differenz}
\frac{dp_k}{dx}
&=
\sum_{i=1}^n
\frac{\partial^2 S}{\partial y_i\,\partial y_k}
\biggl(
\underbrace{
\frac{dy_i}{dx}
-
\frac{\partial H}{\partial p_i}
}_{\displaystyle=0}
\biggr)
-
\frac{\partial H}{\partial y_k}.
\end{align*}
Es bleibt die zweite kanonische Differentialgleichung.
\end{proof}

%
% 3-oc.tex
%
% (c) 2024 Prof Dr Andreas Müller
%
\section{Optimale Steuerung
\label{buch:hamiltonjacobi:section:oc}}
\kopfrechts{Optimale Steuerung}
Die vorangegangenen Kapitel haben verschiedene Beispiele gezeigt,
in denen Funktionen gesucht waren, für die ein gewisses Integral
einen extremalen Wert annimmt.
In der Praxis tauchen aber auch Probleme auf, in denen die
gesuchte Funktion nicht direkt für die Berechnung des zu minimierenden
Integrals verwendet wird.
Stattdessen fliesst die Funktion in eine Differentialgleichung
ein, deren Lösung dann erst das Funktional ergibt.
Das optimale Steuerungsproblem ist ein Problem dieser Art: gesucht
wird eine Steuerfunktion derart, dass das gesteuerte System, welches
mit einer gewöhnlichen Differentialgleichung beschrieben ist, ein
Kostenfunktional minimieren soll.

In diesem Abschnitt wird die Lösung des optimalen Steuerungsproblem
mit Hilfe des Minimum-Prinzips von Pontryagin skizziert.
In Abschnitt~\ref{buch:hamiltonjacobi:oc:subsection:problem}
wird das Problem exakt formuliert.
Danach wird die Vorgehensweise Anhang eines Variationsproblems
mit Nebenbedingungen motiviert und die zugehörige Hamilton-Funktion
hergeleitet.
In Abschnitt~\ref{buch:hamiltonjacobi:oc:subsection:minimum}
wird dann das Minimum-Prinzip von Pontryagin hergeleitet, mit dem
sich viele dieser Probleme lösen lassen.

%
% Das optimale Steuerungsproblem
%
\subsection{Das optimale Steuerungsproblem
\label{buch:hamiltonjacobi:oc:subsection:problem}}
Ein {\em Steuerungsproblem} für ein System mit Differentialgleichung
der Form
\[
\dot{x}
=
f(t, x, u),
\]
wobei $f\colon\mathbb{R}^n\times\mathbb{R}^m\to\mathbb{R}^n$
eine differenzierbare Funktion ist.
Der Vektor $u$ beeinflusst die Zeitentwicklung der Lösung $x(t)$, man
kann ihn als die Steuersignale betrachten, mit denen man die Lösung
im Rahmen der physikalischen Grenzen des Systems beeinflussen kann.
Durch geeignete Wahl der Funktion $u(t)$ kann man erreichen, dass 
die Lösung mit gewissen Anfangsbedingungen $x(t_0)$ in einem
gewählten Endpunkt $x(t_1)$ endet.

Da man die Zeit ebenfalls als eine Steuervariable auffassen kann, 
kann man die Systemgleichung zu
\begin{equation*}
\dot{x}
=
f(x,u)
\end{equation*}
vereinfachen.
Man darf also annehmen, dass $f$ nicht explizit von der Zeit
abhängt.

%
% Das Kostenfunktional
%
\subsubsection{Das Kostenfunktional}
Wie in früher untersuchten Variationsproblemen können der Funktion $x(t)$
Kosten in Form eines Integrals der Form
\begin{equation}
\int_{t_0}^{t_1}
L(x(t),\dot{x}(t))
\,dt
\label{buch:hamiltonjacobi:oc:eqn:jxdotx}
\end{equation}
zugeordnet werden.
Man kann jetzt fragen, für welche Funktion $u(t)$ die sich daraus
ergebende Bahnkurve $x(t)$ das Funktional
\eqref{buch:hamiltonjacobi:oc:eqn:jx}
minimiert.

In den meisten Fällen umfasst der Zustandsvektor $x(t)$ für die
Systemdifferentialgleichung auch die Geschwindigkeit, so dass
es nicht länger sinnvoll ist, dass die Funktion $L$ auch von
$\dot{x}$ abhängig ist.
Das Funktional wird daher
\begin{equation}
\int_{t_0}^{t_1}
L(x(t))
\,dt
\label{buch:hamiltonjacobi:oc:eqn:jx}
\end{equation}

Dies ist aber noch nicht ganz allgemein genug, denn die Steuerung selbst
ist oft ebenfalls mit Kosten verbunden.
Man denke etwa an die Steuerung einer Rakete, wo jeder Steuerimpuls 
Raketentreibstoff verbraucht, von dem man aus Gewichtsgründen möglichst
wenig mitnehmen möchte.
Die Lagrange-Funktion sollte daher auch noch von $u$ abhängen, die
Kostenfunktion wird daher
\begin{equation}
\int_{t_0}^{t_1}
L(x(t),u(t))
\,dt.
\end{equation}

Doch auch dies ist noch nicht die endgültige Fassung.
In vielen Problemen spielt es eine wichtige Rolle, wann die
Operation abgeschlossen ist.
Zum Beispiel können im späteren Verlauf zusätzliche Kosten entstehen,
wenn die Steuerungsaufgabe erst später abgeschlossen wird.
Erreicht ein Raumschiff die Umlaufbahn einer Raumstation verspätet,
wird zusätzlicher Treibstoff für Orbitalmanöver benötigt, um die Station
zu erreichen.
Wir fügen dem zu minimierenden Funktional daher noch einen Randterm
hinzu und erhalten
\begin{equation}
J(u)
=
\int_{t_0}^{t_1}
L(x(t),u(t))
\,dt
+
K(x(t_1))
\label{buch:hamiltonjacobi:oc:eqn:jk}
\end{equation}
als endgültiges Funktional, welches zu minimieren ist.

%
% Die optimale Steuerungsaufgabe
%
\subsubsection{Die optimale Steuerungsaufgabe}
Damit lässt sich jetzt die Aufgabe formulieren, die in diesem
Abschnitt gelöst werden soll.

\begin{aufgabe}[optimale Steuerung]
\label{buch:hamiltonjacobi:oc:aufgabe}
Für das System mit der Differentialgleichung
\begin{equation}
\dot{x}
=
f(x,u), \qquad f\colon \mathbb{R}^n\times\mathbb{R}^m\to\mathbb{R}^n
\label{buch:hamiltonjacobi:oc:eqn:dgl}
\end{equation}
soll eine Funktion $u\colon [t_0,t_1]\to\mathbb{R}^m$ gefunden werden,
für die 
\begin{equation}
J(u)
=
\int_{t_0}^{t_1}
L(x(t),u(t))
\,dt
+
K(x(t_1))
\label{buch:hamiltonjacobi:oc:eqn:kosten}
\end{equation}
minimiert, wobei $x(t)$ eine Lösung der Differentialgleichung
\eqref{buch:hamiltonjacobi:oc:eqn:dgl} ist, die die Anfangsbedinung
$x(t_0)=x_0$ erfüllt.
\end{aufgabe}

Die Aufgabe verlangt also, eine Steuerfunktion $u(t)$ so zu bestimmen,
dass das System der Bahn $x(t)$ mit minimalen Kosten vom Startpunkt $x_0$
zum Zielpunkt $x_1$ folgt.

%
% Das optimale Steuerungsproblem als Variationsproblem mit Nebenbedingungen
%
\subsubsection{Das optimale Steuerungsproblem als Variationsproblem
mit Nebenbedingungen}
Die Formulierung von Aufgabe~\ref{buch:hamiltonjacobi:oc:aufgabe}
betrachtet die Funktionen $x(t)$ als von $u(t)$ abhängig.
Man muss also erst die
Differentialgleichung~\eqref{buch:hamiltonjacobi:oc:eqn:dgl}
lösen, bevor man das
Kostenfunktional~\eqref{buch:hamiltonjacobi:oc:eqn:kosten}
berechnen kann.
Diese Formulierung macht es unmöglich, die Aufgabe als 
Variationsproblem zu sehen.

Die Aufgabenstellung kann aber auch als Variationsproblem mit
einer Nebenbedingung gesehen werden.
Dazu werden sowohl die Variablen $x$ wie auch die Steuerinputs $u$
als unabhängige Variablen $q=(x,u)$ betrachtet.
Zu bestimmen ist jetzt eine Funktion $q(t)=(x(t),u(t))$, die
das Integral
\[
\int_{t_0}^{t_1}
L(x(t), u(t))
\,dt
\]
minimiert.
Jetzt sieht die Aufgabenstellung wie ein Variationsproblem aus,
allerdings wird damit noch nicht berücksichtigt, dass $x(t)$ eine
Lösung der Differentialgleichung~\eqref{buch:hamiltonjacobi:oc:eqn:dgl}
sein muss.

Der durch die Differentialgleichung~\eqref{buch:hamiltonjacobi:oc:eqn:dgl}
gegebene Zusammenhang von $x(t)$ und $u(t)$ kann als Nebenbedingung 
in der Form
\begin{equation}
G(x,\dot{x}, u)
=
f(x,u) - \dot{x} = 0
\label{buch:hamiltonjacobi:oc:eqn:dglneben}
\end{equation}
betrachtet werden.
Im Gegensatz zu den Problemen, die in
Abschnitt~\ref{buch:nebenbedingungen:lagrangemult:subsection:nebenbedingungen}
untersucht wurden, hat die
Nebenbedingung~\eqref{buch:hamiltonjacobi:oc:eqn:dglneben}
nicht die der Form eines Integrals.
Es reicht also nicht, eine zusätzliche Vektorvariable $\lambda\in\mathbb{R}^n$
einzuführen und damit ein Funktional mit einer Lagrange-Funktion der Form
\[
L(x,u) - \lambda \cdot G(x,\dot{x},u)
\]
zu minimieren, womit
Abschnitt~\ref{buch:nebenbedingungen:lagrangemult:subsection:nebenbedingungen}
erfolgreich gewesen war.

Als Erweiterung bietet sich an, die Variable $\lambda$ durch eine
Funktion $p(t)$ zu ersetzen.
So entsteht ein neues Funktional
\[
F
\colon
\mathbb{R}^n\times\mathbb{R}^m\times\mathbb{R}^n
\to\mathbb{R}
:
(x,u,p)
\mapsto
F(x,u,p)
=
L(x,u) + p\cdot (f(x,u) - \dot{x}),
\]
für welches jetzt untersucht werden soll, ob sich damit das Problem lösen
lässt.

%
% Die Euler-Lagrange-Differentialgleichung für F
%
\subsubsection{Die Euler-Lagrange-Differentialgleichung für $F$}
Gesucht ist jetzt eine Funktion $q(t)=(x(t),u(t),p(t))$ derart,
dass das Funktional mit der Lagrange-Funktion
\[
F(q,\dot{q})
=
L(x,u) +p\cdot f(x,u) - p\cdot \dot{x}
\]
minimiert.
Die Euler-Lagrange-Differentialgleichung von $F$ ist
\begin{equation}
\frac{\partial F}{\partial q}(q(t),\dot{q}(t))
=
\frac{d}{dt}
\frac{\partial F}{\partial \dot{q}}(q(t),\dot{q}(t)).
\label{buch:hamiltonjacobi:oc:eqn:Feulerlagrange}
\end{equation}
Die partiellen Ableitungen nach $q$ und $\dot{q}$ lassen sich aufteilen
in die Komponenten $x$, $u$ und $p$ von $q$.
Wir berechnen nur die Komponenten für $p$, sie ist
\begin{align}
\frac{\partial F}{\partial p}
&=
f(x,u)
-
\dot{x}
&
\frac{\partial F}{\partial \dot{p}}
&=
0
&&\Rightarrow&
f(x,u)-\dot{x}&=0
\label{buch:hamiltonjacobi:oc:eqn:Fsystemdgl}
\end{align}
Die letzte Gleichung ist die Nebenbedingung, eine Lösung der
Euler-Lagrange-Differential\-gleichung erfüllt also automatisch auch die
Systemdifferentialgleichung~\eqref{buch:hamiltonjacobi:oc:eqn:dgl}.

%
% Die Lösung der optimalen Steuerungsaufgaben
%
\subsubsection{Die Lösung der optimalen Steuerungsaufgaben}
Die Euler-Lagrange-Differentialgleichung zusammen mit geeigneten
Anfangsbedingungen für die gesuchten Funktionen können die
optimale Steuerungsaufgaben lösen, wie der folgende Satz zeigt.

\begin{satz}[Optimale Steuerung, Lagrange-Form]
\label{buch:hamiltonjacobi:oc:satz:optimal-lagrange}
Eine Extremale des Funktionals
\begin{equation}
I(q)
=
\int_{t_0}^{t_1}
F(q(t),\dot{q}(t))
\,dt
+
K(x(t))
\end{equation}
mit der Lagrange-Funktion
\begin{equation}
F(q,\dot{q}) = L(x,u) + p\cdot(f(x,u) - \dot{x})
\label{buch:hamiltonjacobi:oc:optF}
\end{equation}
und den Randbedingungen
\[
x(t_0)=x_0
\qquad\text{und}\qquad
p(t_1)=\frac{\partial K}{\partial x}(x(t_1))
\]
ist eine Lösung der optimale
Steuerungsaufgabe~\ref{buch:hamiltonjacobi:oc:aufgabe}.
\end{satz}

\begin{proof}
Da nach
\eqref{buch:hamiltonjacobi:oc:eqn:Fsystemdgl}
eine Lösung der Euler-Lagrange-Differentialgleichung von $F$ immer
auch die Systemdifferentialgleichung löst, ist 
\begin{equation*}
F(q_*(t),\dot{q}_*(t))
=
L(x_*(t),u_*(t))
+
p_*(t)\cdot (
\underbrace{f(x_*(t),u_*(t))-\dot{x}_*(t)}_{\displaystyle=0})
=
L(x_*(t),u_*(t)).
\end{equation*}
Insbesondere ist auch
\[
\int_{t_0}^{t_1}
F(q_*(t),\dot{q}_*(t))
\,dt
=
\int_{t_0}^{t_1}
L(x_*(t),u_*(t))
\,dt,
\]
Der Wert des Integrals im Funktion ändert also nicht.

Am Intervallende bei $t_1$ sind die Werte von $x_*(t)$ nicht
vorgegeben.
Der Satz~\ref{buch:nebenbedingungen:transversal:satz:randterme}
erklärt, wie der Randterm $K(x(t))$ sich in Randbedingungen
\[
\frac{\partial F}{\partial \dot{q}}(q(t),\dot{q}(t))
+
\frac{\partial K}{\partial q}(q(t_1))
=
0
\]
für $q(t_1)$ umrechnen lässt.
Indem man die Ableitungen nach den Komponenten von $q$ separat betrachtet,
findet man
\begin{align*}
0
&=
\frac{\partial F}{\partial\dot{x}}(q(t_1),\dot{q}(t_1))
+
\frac{\partial K}{\partial x}(q(t_1))
=
-p_*(t_1)
+
\frac{\partial K}{\partial x}(x_*(t_1))
&&\Rightarrow&p_*(t_1)
&=
\frac{\partial K}{\partial x}(x_*(t_1))
\intertext{Dies ist die gesuchte Randbedingung für $p_*(t)$.
Die verbleibenden Komponenten sind}
0
&=
\frac{\partial F}{\partial\dot{u}}(q(t_1),\dot{q}(t_1))
+
\frac{\partial K}{\partial u}(q(t_1))
\\
0
&=
\frac{\partial F}{\partial\dot{p}}(q(t_1),\dot{q}(t_1))
+
\frac{\partial K}{\partial p}(q(t_1)).
\end{align*}
Da die Funktionen auf der rechten Seite nicht von den Variablen
abhängen, nach denen sie abgeleitet werden, sind diese Bedingungen
automatisch erfüllt.
\end{proof}

Die Funktion $F(q,\dot{q})$ hängt nicht von der unabhängigen
Variablen $t$ ab, somit ist die Beltrami-Identität
(Satz~\ref{buch:variation:eulerlagrange:satz:beltrami})
anwendbar.
Sie sagt, dass die Grösse
\begin{align}
F(q,\dot{q})
-
\dot{q}\cdot\frac{\partial F}{\partial\dot{q}}(q,\dot{q})
&=
F(q,\dot{q})
-
\dot{x}\cdot
\underbrace{
\frac{\partial F}{\partial\dot{x}}
}_{\displaystyle=-p}
\mathstrut-
\dot{u}\cdot
\underbrace{
\frac{\partial F}{\partial\dot{u}}
}_{\displaystyle=0}
\mathstrut-
\dot{p}\cdot
\underbrace{
\frac{\partial F}{\partial\dot{p}}
}_{\displaystyle=0}
\notag
\\
&=
L(x,u)+p\cdot f(x,u)-p\cdot\dot{x}
-
\dot{x}\cdot
(-p)
\notag
\\
&=
L(x,u) + p\cdot f(x,u)
\label{buch:hamiltonjacobi:oc:hamilton}
\end{align}
konstant ist.
Diese Funktion spielt daher in der nachfolgenden Konstruktion zur
Lösung der optimalen Steuerungsaufgabe eine grosse Rolle.

%
% Die Hamilton-Differentialgleichungen
%
\subsection{Die Hamilton-Differentialgleichungen
\label{buch:hamiltonjacobi:oc:subsection:hamilton}}
Für eine Lagrange-Funktion $L(x,u)$ haben wir gefunden, dass der
Ausdruck~\eqref{buch:hamiltonjacobi:oc:hamilton} eine besondere
Bedeutung hat.
Wir geben ihm daher einen Namen.

\begin{definition}[Steuerungs-Hamiltonfunktion]
Die Funktion
\[
H
\colon
\mathbb{R}^n\times\mathbb{R}^n\times\mathbb{R}^m
\to
\mathbb{R}
:
(x,p,u)
\mapsto
H(x,p,u)
=
p\cdot f(x,u) + L(x,u)
\]
heisst die {\em Steuerungs-Hamilton-Funktion}
der optimalen Steuerungsaufgabe~\ref{buch:hamiltonjacobi:oc:aufgabe}.
\end{definition}

Der Name rechtfertigt sich dadurch, dass sich aus $H$ analoge
Differentialgleichungen für die Funktionen $x(t)$ und $p(t)$ bilden
lassen.
Damit wird es möglich, auch die Funktion $p(t)$ zu berechnen.
Wenn sich ausserdem eine Bedingung für $u(t)$ finden lässt, kann
die Aufgabe~\ref{buch:hamiltonjacobi:oc:aufgabe}
vollständig gelöst werden.
Eine solche Bedingung wird später im Minimum-Prinzip gefunden.
Als erstes sollen jetzt die Differentialgleichungen konstruiert
werden, die sich aus $H$ ableiten lassen.

%
% Partielle Ableitungen
%
\subsubsection{Partielle Ableitungen}
Zu diesem Zweck sei $q_*(t)=(x_*(t),u_*(t),p_*(t))$ eine Lösung der
Euler-Lagrange-Differen\-tialgleichung
\eqref{buch:hamiltonjacobi:oc:eqn:Feulerlagrange}
für die Lagrange-Funktion $F(q,\dot{q})$.
Als Vorbereitung für die Berechnung der partiellen Ableitungen von $H$
schreiben wir die Euler-Lagrange-Differen\-tialglei\-chungen explizit
für die einzelnen Komponenten von $q$ hin:
\begin{align}
0
&=
\frac{\partial F}{\partial x_k}(x_*(t),u_*(t),p_*(t))
-
\frac{d}{dt}
\frac{\partial F}{\partial \dot{x}_k}(x_*(t),u_*(t),p_*(t))
\notag
\\
&=
\frac{\partial L}{\partial x_k}(x_*(t),u_*(t))
+
p_*(t)\cdot\frac{\partial f}{\partial x_k}(x_*(t),u_*(t))
+
\dot{p}_{*k}
\notag
\\
\Rightarrow\qquad
\dot{p}_*(t)
&=
-\frac{\partial L}{\partial x}(x_*(t),u_*(t))
-
p_*(t)\cdot\frac{\partial f}{\partial x}(x_*(t),u_*(t))
\label{buch:hamiltonjacobi:oc:eqn:Fxabl}
\intertext{Die Variablen $\dot{u}$ und $\dot{p}$ kommen in $F$ nicht
vor, der Zeitableitungsterm kommt daher in den zugehörigen
Euler-Lagrange-Gleichungen nicht vor:}
0
&=
\frac{\partial F}{\partial u_k}(x_*(t),u_*(t),p_*(t))
-
\frac{d}{dt}
\frac{\partial F}{\partial \dot{u}_k}(x_*(t),u_*(t),p_*(t))
\notag
\\
&=
\frac{\partial L}{\partial u_k}(x_*(t),u_*(t))
+
p_*(t)\cdot\frac{\partial f}{\partial u_k}(x_*(t),u_*(t))
\label{buch:hamiltonjacobi:oc:eqn:Fuabl}
\\
0
&=
\frac{\partial F}{\partial p_k}(x_*(t),u_*(t),p_*(t))
-
\frac{d}{dt}
\frac{\partial F}{\partial \dot{p}_k}(x_*(t),u_*(t),p_*(t))
\notag
\\
&=
f_k(x_*(t),u_*(t))-\dot{x}_{*k}(t)
\notag
\\
\Rightarrow\qquad
\dot{x}_*(t)
&=
f(x_*(t),u_*(t)).
\label{buch:hamiltonjacobi:oc:eqn:Fpabl}
\end{align}

Nach diesen Vorbereitungen können jetzt auch die partiellen Ableitungen
von $H$ berechnet werden.
Unter Verwendung von 
\eqref{buch:hamiltonjacobi:oc:eqn:Fxabl},
\eqref{buch:hamiltonjacobi:oc:eqn:Fuabl}
und
\eqref{buch:hamiltonjacobi:oc:eqn:Fpabl}
Folgen die Gleichungen
\begin{equation}
\renewcommand{\arraycolsep}{2pt}
\renewcommand{\arraystretch}{2.0}
\begin{array}{rclcl}
\displaystyle
\frac{\partial H}{\partial x_k}(x_*(t),u_*(t),p_*(t))
&=&
\displaystyle
\frac{\partial L}{\partial x}(x_*(t),u_*(t))
+
p_*(t)\cdot\frac{\partial f}{\partial x_k}(x_*(t),u_*(t))
&=&-\dot{p}_{*k}(t)
\\
\displaystyle
\frac{\partial H}{\partial u_k}(x_*(t),u_*(t),p_*(t))
&=&
\displaystyle
p_*(t)\cdot
\frac{\partial f}{\partial u_k}(x_*(t),u_*(t))
+
\frac{\partial L}{\partial u_k}(x_*(t),u_*(t))
&=&0
\\
\displaystyle
\frac{\partial H}{\partial p_k}(x_*(t),u_*(t),p_*(t))
&=&
f_k(x_*(t),u_*(t))
&=&
\dot{x}_{*k}(t)
\end{array}
\label{buch:hamiltonjacobi:oc:Habl}
\end{equation}

%
% Randbedingungen
%
\subsubsection{Randbedingungen}
Die Anfangsbedingung $x_*(t_0)=x_0$ für $x_*(t)$ ist bereits durch
die Aufgabenstellung gegeben.
Für die Funktion $p_*(t)$ wird eine weitere Randbedingung benötigt,
damit die Differentialgleichungen~\eqref{buch:hamiltonjacobi:oc:Habl}
Beide Funktionen $x_*(t)$ und $p_*(t)$ bestimmen können.
Der Satz~\ref{buch:hamiltonjacobi:oc:satz:optimal-lagrange}
definiert auch die Randbedingung
\[
p_*(t_1)
=
\frac{\partial K}{\partial x}(x_*(t_1)).
\]

%
% Differentialgleichung
%
\subsubsection{Differentialgleichung}
Die Differentialgleichungen~\eqref{buch:hamiltonjacobi:oc:Habl} zusammen
mit den Randbedingungen des vorangegangenen Abschnitts ergeben jetzt
den folgenden Satz.

\begin{satz}[Hamilton-Gleichungen]
\label{buch:hamiltonjacobi:oc:satz:hamilton-gleichungen}
Eine Lösung $(x_*(t),u_*(t),p_*(t))$ der optimalen Steuerungsaufgabe
erfüllt die {\em Hamilton-Differentialgleichungen}
\begin{align*}
\dot{x}_*(t)
&=
\phantom{-}
\frac{\partial H}{\partial p}(x_*(t),u_*(t),p_*(t)),
&
x_*(t_0)
&=
x_0,
\\
\dot{p}_*(t)
&=
-
\frac{\partial H}{\partial x}(x_*(t),u_*(t),p_*(t)),
&
p_*(t_1)
&=
\frac{\partial K}{\partial x}(x_*(t_1)).
\end{align*}
Ausserdem verschwindet die Ableitung nach $u$:
\begin{equation}
\frac{\partial H}{\partial u}(x_*(t),u_*(t),p_*(t))=0.
\label{buch:hamiltonjacobi:oc:eqn:ablHu0}
\end{equation}
\end{satz}

Der Satz bestimmt $x_*(t)$ und $p_*(t)$ sobald $u_*(t)$ gegeben ist,
er macht aber ausser der Bedingung, dass die
Ableitung~\eqref{buch:hamiltonjacobi:oc:eqn:ablHu0} verschwindet,
keine Aussage darüber, wie $u_*(t)$ gefunden werden kann.
Ein mögliche Wahl für $u_*(t)$ könnte sein, dass $u_*(t)$ jeweils
als das Extremum der Funktion $u\mapsto H(x_*(t),u,p_*(t))$
gewählt wird.
Dass dies auch die optimale Wahl sein kann, zeigt das Minimum-Prinzip
von Pontryagin, welches im
Abschnitt~\ref{buch:hamiltonjacobi:oc:subsection:minimum}
besprochen werden soll.

%
% Das Minimum-Prinzip
%
\subsection{Das Minimum-Prinzip
\label{buch:hamiltonjacobi:oc:subsection:minimum}}
In Satz~\ref{buch:hamiltonjacobi:oc:satz:hamilton-gleichungen}
und in der daran anschliessenden Diskussion wird angedeutet, dass
wegen der Bedingung~\eqref{buch:hamiltonjacobi:oc:eqn:ablHu0}
die Funktion $u_*(t)$ ein Extremum der Funktion $H$ zu den gegebenen
Werten $x_*(t)$ und $p_*(t)$ sei.
Eine solche Bedingung vervollständigt die Lösung des optimalen
Steuerungsproblems.

\begin{satz}[Pontryagin]
Seien die Funktionen $f(x,u)$ und $L(x,u)$ steteig differenzierbar in 
$x$ und $u$ und sei $K(x)$ stetig differenzierbar in $x$.
Sei die Funktion $u_*\colon[t_0,t_1]\to\mathbb{R}$ eine Lösung des
optimalen Steuerungsproblems Aufgabe~\ref{buch:hamiltonjacobi:oc:aufgabe}
und sei $x_*(t)$ die resultierende optimale Bahnkurve, dann gibt es
eine eindeutig bestimmte Funktion $p_*\colon[t_0,t_1]\to\mathbb{R}$,
So dass $x_*(t),u_*(t),p_*(t)$ die Hamilton-Differentialgleichungen
von Satz~\ref{buch:hamiltonjacobi:oc:satz:hamilton-gleichungen}
erfüllen.
Die Funktion $u_*(t)$ erfüllt
\begin{equation}
u_*(t)
=
\operatorname{argmin}_{u\in\mathbb{R}} H(x_*(t),u,p_*(t)).
\label{buch:hamiltonjacobi:oc:eqn:argminH}
\end{equation}
für alle Stellen $t\in[t_0,t_1]$, an denen $u_*(t)$ stetig ist.
\end{satz}

Dass die Bedingung \eqref{buch:hamiltonjacobi:oc:eqn:argminH} nur
an Stellen gelten muss, an denen $u_*(t)$ stetig ist, lässt zu,
dass die Funktion $u_*$ nicht stetig ist.
Dies wird vom folgenden Beispiel illustriert.

%
% oc.tex -- einfache OC Beispiele
%
% (c) 2021 Prof Dr Andreas Müller, OST Ostschweizer Fachhochschule
%
\documentclass[tikz]{standalone}
\usepackage{amsmath}
\usepackage{times}
\usepackage{txfonts}
\usepackage{pgfplots}
\usepackage{csvsimple}
\usetikzlibrary{arrows,intersections,math}
\definecolor{darkred}{rgb}{0.8,0,0}
\definecolor{darkgreen}{rgb}{0,0.6,0}
\begin{document}
\def\skala{1}
\begin{tikzpicture}[>=latex,thick,scale=\skala]

\clip (-3.8,3.0) rectangle (8.8,-9.8);

\begin{scope}[xshift=-3.2cm]

\begin{scope}
\draw[color=darkred,line width=1.4pt] (0,2.5) -- (5,0);
\node[color=darkred] at (2.5,1.25) [above right] {$p_*(t)$};
\draw[->] (0,-0.1) -- (0,2.8) coordinate[label={right:$p$}];
\draw[->] (-0.1,0) -- (5.4,0) coordinate[label={$t$}];
\draw (5,-0.05) -- (5,0.05);
\draw (-0.05,2.5) -- (0.05,2.5);
\node at (-0.05,2.5) [left] {$1$};
\node at (-0.05,0.0) [left] {$0$};
\draw[line width=0.3pt] (5,2.6) -- +(0,-11.9);
\end{scope}

\begin{scope}[yshift=-3.3cm]
\draw[color=darkred,line width=1.4pt] (0,-2.5) -- (5,-2.5);
\node[color=darkred] at (2.5,-2.5) [above] {$u_*(t)$};
\draw[->] (0,-2.6) -- (0,2.8) coordinate[label={right:$u$}];
\draw[->] (-0.1,0) -- (5.4,0) coordinate[label={$t$}];
\draw (5,-0.05) -- (5,0.05);
\draw (-0.05,2.5) -- (0.05,2.5);
\node at (-0.05,2.5) [left] {$1$};
\draw (-0.05,-2.5) -- (0.05,-2.5);
\node at (-0.05,-2.5) [left] {$-1$};
\node at (-0.05,0.0) [left] {$0$};
\end{scope}

\begin{scope}[yshift=-8.2cm]
\draw[color=darkred,line width=1.4pt] (0,1.5) -- (5,-1);
\node[color=darkred] at (2.5,0.25) [above right] {$x_*(t)$};
\draw[->] (0,-1.1) -- (0,1.9) coordinate[label={right:$x$}];
\draw[->] (-0.1,0) -- (5.4,0) coordinate[label={$t$}];
\draw (5,-0.05) -- (5,0.05);
\node at (-0.05,0.0) [left] {$0$};
\draw (-0.05,1.5) -- (0.05,1.5);
\node at (-0.05,1.5) [left] {$x_0$};
\end{scope}

\node at (2.5,-9.6) {(a)};

\end{scope}

\begin{scope}[xshift=3.2cm]
\fill[color=blue!10,opacity=0.5] (0,2.6) rectangle (2.5,-9.3);
\fill[color=darkgreen!10,opacity=0.5] (2.5,2.6) rectangle (5,-9.3);

\begin{scope}[yshift=1.25cm]
\draw[color=darkred,line width=1.4pt] (0,1.25) -- (5,-1.25);
\node[color=darkred] at (2.5,0) [above right] {$p_*(t)$};
\draw[->] (0,-1.35) -- (0,1.55) coordinate[label={right:$p$}];
\draw[->] (-0.1,0) -- (5.4,0) coordinate[label={$t$}];
\draw (5,-0.05) -- (5,0.05);
\draw (-0.05,1.25) -- (0.05,1.25);
\node at (-0.05,1.25) [left] {$\frac12$};
\draw (-0.05,-1.25) -- (0.05,-1.25);
\node at (-0.05,-1.25) [left] {$-\frac12$};
\node at (-0.05,0.0) [left] {$0$};
\draw[line width=0.3pt] (5,1.35) -- +(0,-11.9);
\end{scope}

\begin{scope}[yshift=-3.3cm]
\draw[color=darkred,line width=1.4pt] (0,-2.5) -- (2.5,-2.5);
\draw[color=darkred,line width=0.3pt] (2.5,-2.5) -- (2.5,2.5);
\draw[color=darkred,line width=1.4pt] (2.5,2.5) -- (5,2.5);
\node[color=darkred] at (1.25,-2.5) [above] {$u_*(t)$};
\node[color=darkred] at (3.75,2.5) [below] {$u_*(t)$};
\draw[->] (0,-2.6) -- (0,2.8) coordinate[label={right:$u$}];
\draw[->] (-0.1,0) -- (5.4,0) coordinate[label={$t$}];
\draw (5,-0.05) -- (5,0.05);
\draw (-0.05,2.5) -- (0.05,2.5);
\node at (-0.05,2.5) [left] {$1$};
\draw (-0.05,-2.5) -- (0.05,-2.5);
\node at (-0.05,-2.5) [left] {$-1$};
\node at (-0.05,0.0) [left] {$0$};
\end{scope}

\begin{scope}[yshift=-8.2cm]
\draw[color=darkred,line width=1.4pt] (0,1.5) -- (2.5,0.25) -- (5,1.5);
\node[color=darkred] at (2.5,0.35) [above] {$x_*(t)$};
\draw[->] (0,-1.1) -- (0,1.9) coordinate[label={right:$x$}];
\draw[->] (-0.1,0) -- (5.4,0) coordinate[label={$t$}];
\draw (5,-0.05) -- (5,0.05);
\node at (-0.05,0.0) [left] {$0$};
\draw (-0.05,1.5) -- (0.05,1.5);
\node at (-0.05,1.5) [left] {$x_0$};
\end{scope}

\node at (2.5,-9.6) {(b)};

\end{scope}

\end{tikzpicture}
\end{document}



\begin{beispiel}
\label{buch:hamiltonjacobi:oc:bsp:simple}
Gesucht ist die Lösung des eindimensionalen optimalen Steuerproblems
auf dem Intervall zwischen $t_0=0$ und $t_1=1$ mit
\begin{align*}
f(x,u) &= u\in U=[-1,1],
&
x(0) &= x_0\in\mathbb{R}
&
&\text{und}&
J(u)
=
\int_0^1 x(t)\,dt.
\end{align*}
Die Lagrange-Funktion ist $L(x,u)=x$ und die Funktion $K(x)$ ist die
Nullfunktion.

Für die Lösung des Problems berechnen wird zunächst die
Steuerungs-Hamilton-Funktion
\[
H(x,u,p)
=
L(u,x) + pf(x,u)
=
x + pu
\]
und ihre Ableitungen
\begin{equation}
\left.
\begin{aligned}
\frac{\partial H}{\partial x}
&=
1,
&&&
\frac{\partial H}{\partial\dot{x}}
&=
0,
\\
\frac{\partial H}{\partial u}
&=
p,
&&&
\frac{\partial H}{\partial\dot{u}}
&=
0,
\\
\frac{\partial H}{\partial p}
&=
u,
&&&
\frac{\partial H}{\partial\dot{p}}
&=
0.
\end{aligned}
\qquad
\right\}
\label{buch:hamiltonjacobi:oc:bsp:ableitungen}
\end{equation}
Daraus folgen die Hamilton-Differentialgleichungen
\begin{align*}
&\text{für $x(t)$:}&
\dot{x}(t)&=u(t)
&&\text{mit Anfangsbedingung}&
x(0)&=x_0
\qquad\text{und}
\\
&\text{für $p(t)$:}&
\dot{p}(t)&=-1
&&\text{mit Randbedingung}&
p(1)&=\frac{\partial K}{\partial x}(1) = 0.
\end{align*}
Aus der Gleichung für $p$ folgt $p_*(t)=1-t$
(Abbildung~\ref{buch:hamiltonjacobi:oc:fig:oc} (a)).
Das Minimumprinzip sagt jetzt, dass für $u_*(t)$ das Argument $u$
gewählt werden muss, welches
\[
H(x_*(t),u,p_*(t))
=
x_*(t)+p_*(t)u
=
x_*(t)+(1-t)u
\]
minimiert.
Da $1-t<0$ ist für alle $t<1$, wird ganz unabhängig von $x_*(t)$
der kleinstmögliche Wert bei $u_*(t)=-1$ angenommen.
Mit diesem $u_*(t)$ kann man jetzt auch die Lösung $x_*(t)=-t$
berechnen.

Diese Lösung ist auch plausibel, denn weil kein Randterm vorhanden ist,
kommt es nur darauf an, dass $x_*(t)$ möglichst schnell abnimmt, um
das Integral
\[
\int_0^1 x(t)\,dt
=
\int_0^1 -t\,dt
=
-\biggl[\frac12t^2\biggr]_0^1
=
-\frac12
\]
zu minimieren.
\end{beispiel}

\begin{beispiel}
\label{buch:hamiltonjacobi:oc:bsp:switch}
Lösen Sie das optimale Steuerungsproblem für das System auf dem Intervall
$[t_0,t_1]=[0,1]$ mit der Systemgleichung
\[
\dot{x} = f(x,u)=u\in U=[-1,1],\quad x(0) = x_0
\]
und dem Funktional
\[
J(u)
=
\int_0^1 x(t)\,dt -\frac12 x(1)
\]
mit Lagrange-Funktion und Randterm
\begin{align*}
L(x,u)
&=
x
&&\text{und}&
K(x)&=-\frac12x.
\end{align*}
Die Steuerungs-Hamilton-Funktion ist
\[
H(x,u,p)
=
L(x,u) + pf(x,u)
=
x + pu
\]
mit den Ableitungen wie in~\eqref{buch:hamiltonjacobi:oc:bsp:ableitungen}.
Die Hamilton-Differentialgleichungen sind jetzt
\begin{align*}
&\text{für $x(t)$:}&
\dot{x}(t)&=u(t)
&&\text{mit der Anfangsbedingung:}&
x(0)&=x_0\qquad\text{und}
\\
&\text{für $p(t)$:}&
\dot{p}(t)&=-1
&&\text{mit der Randbedingung:}&
p(1)&=\frac{\partial K}{\partial x}(1) = -\frac12.
\end{align*}
Aus der Gleichung für $p$ folgt diesmal $p_*(t)=\frac12-t$.
Zur Bestimmung von $u_*(t)$ müssen wir die Hamilton-Funktion
\[
H(x_*(t),u,p_*(t))
=
x_*(t)+p_*(t)u
=
x_*(t)+({\textstyle\frac12}-t)u
\]
betrachten.
Welcher Werte von $u$ den kleinsten Wert von $H$ ergibt, hängt
vom Vorzeichen von $p_*(t)=\frac12-t$.
Für positives Vorzeichen von $p_*(t)$ muss $u=-1$ gewählt werden,
sonst $u=1$.
Also ist
\[
u_*(t)
=
\left\{
\begin{aligned}
&\phantom{-}1&&\text{falls $p_*(t)<0$}&&\Leftrightarrow&t&>\textstyle\frac12
\\
&{}-1&&\text{falls $p_*(t)>0$}&&\Leftrightarrow&t&<\textstyle\frac12.
\end{aligned}
\right.
\]
Daraus kann man jetzt auch $x_*(t)$ berechnen:
\[
x_*(t)
=
\int_0^t u_*(\tau)\,d\tau
=
\begin{cases}
x_0+{\displaystyle\int_0^t 1\,d\tau} = x_0+t&\qquad\text{für $t\le\frac12$}
\\[8pt]
x_*(\frac12)-{\displaystyle\int_{\frac12}^t 1\,d\tau}
=
x_0+\frac12-(t-\frac12)
=
x_0+1-t&\qquad\text{für $t>\frac12$.}
\end{cases}
\]
Die Lösungsfunktionen sind in
Abbildung~\ref{buch:hamiltonjacobi:oc:fig:oc} (b)
dargestellt.
Der Wert des Funktionals kann jetzt ebenfalls berechnet werden, er ist
\begin{align*}
J(u_*)
&=
\int_0^1 x_*(t)\,dt + K(x_*(1))
\\
&=
\int_0^{\frac12} x_0+t \,dt
+
\int_{\frac12}^1 x_0+1-t\,dt
-
\frac12x_0
\\
&=
\biggl[x_0t+\frac12t^2\biggr]_0^{\frac12}
+
\biggl[x_0t+t-\frac12t^2\biggr]_{\frac12}^1 - \frac12x_0
\\
&=
\frac12x_0 + \frac18 + x_0+1-\frac12 - \frac12x_0-\frac12 +\frac18 - \frac12x_0
\\
&=\frac12x_0+\frac14.
\qedhere
\end{align*}
\end{beispiel}

\begin{proof}[Beweis des Minimum-Prinzips]

XXX Todo

\end{proof}

%
% Quadratische optimale Steuerungsprobleme
%
\subsection{Quadratische optimale Steuerungsprobleme
\label{buch:hamiltonjacobi:oc:subsection:quadratisch}}
Ein spezielle Fall, in dem die Reechnungen vollständig durchführbar sind,
ist das {\em linear quaratische optimale Steuerungsproblem}.
\index{linear quadratisches optimales Steuerungsproblem}%

\begin{aufgabe}
\label{buch:hamiltonjacobi:oc:aufgabe:lqc}
Das System mit der linearen Systemgleichung
\begin{equation}
\dot{x}
=
A(t)x
+
B(t)u
\end{equation}
und der Anfangsbedinung $x(t_0)=x_0$ soll so gesteuert werden,
dass das Funktional
\begin{equation}
J(u)
=
\frac12 x(t_1)^t S_1 x(t_1)
+
\frac12
\int_{t_0}^{t_1}
x(t)^t Q(t) x(t)
+
u(t)^t R(t) u(t)
\,dt
\end{equation}
minimal wird.
\end{aufgabe}
Wir lösen das Problem hier nur für den Fall, dass die Matrizen $A$, $B$,
$Q$ und $R$ nicht von der Zeit abhängen.
In diesem Fall ist die Lagrange-Funktion
\[
L(x,u)
=
\frac12 x^t Qx + \frac12 u^t Ru
\]
dieses Funktionals ist von der Zeit unabhängig,
ebenso die Systemgleichung $f(x,u)=Ax+bu$ wir haben also tatsächlich ein
optimales Steuerungsproblem der Art, wie wir es studiert haben.

Zur Lösung dieses Problems muss erst die Steuerungs-Hamilton-Funktion
\[
H(x,u,p)
=
L(x,u) + p\cdot f(u,x)
=
\frac12 x^tQx + \frac12 u^tRu + p^tAx + p^tBu
\]
bestimmt werden.
Die kanonischen Differentialgleichungen sind jetzt
\begin{align*}
\dot{x}
&=
\frac{\partial H}{\partial p}
=
Ax
+
Bu
\\
\dot{p}
&=
-\frac{\partial H}{\partial x}
=
-Qx - p^tA.
\end{align*}
Die zweite Gleichung ist natürlich keine Überraschung, dies ist die
Systemgleichung.

Der Randterm $K(x)$ ist ebenfalls quadratisch, es ist
\[
K(x) = \frac12 x^t S_1 x,
\]
mit der Ableitung
\begin{equation*}
\frac{\partial K}{\partial x}
=
S_1x,
\end{equation*}
wie sie später als Randbedingung für $p_*(t)$ gebraucht wird.

Da  Minimumprinzip besagt, dass $u$ jeweils so gehält werden muss,
dass $H(x,u,p)$ bei gegebenenm $x$ und $p$ das Minimum annimmt.
Da $H$ quadratisch ist in $u$, muss also nach Weglassen der konstanten
Terme in $H$ für $u$ das quadratische Minimalproblem
\[
f(u)
=
\frac12 u^t R u + p^t Bu
\to
\min
\]
gelöst werden.
Dies ist möglich, wenn $R$ positiv definit ist.
Die Ableitung nach $u$ ist
\[
0
=
\frac{\partial f}{\partial u}
=
Ru + B^tp
\qquad\Rightarrow\qquad
u
=
-
R^{-1}B^t p.
\]
Damit kann $u$ aus den Differentialgleichungen eliminiert werden, 
es bleiben die Gleichungen
\begin{equation}
\left.
\begin{aligned}
\dot{x}
&=
Ax-BR^{-1}B^t p
\\
\dot{p}
&=
-Qx -p^tA
\end{aligned}
\quad
\right\}
\qquad\Rightarrow\qquad
\frac{d}{dt}
\begin{pmatrix}
x\\
p
\end{pmatrix}
=
\begin{pmatrix}
A & -BR^{-1}B^t \\
-Q & A^t
\end{pmatrix}
\begin{pmatrix}
x\\
p
\end{pmatrix}
\end{equation}
Die Anfangsbedingung für $x$ ist $x(t_0)=x_0$, die Randbedingung bei
$t_1$ für $p$ ist mit
$p(t_1)=S_1x(t_1)$.
Da eine solche Beziehung für jeden denkbaren Endzeitpunkt $t$
gelten muss, wird jetzt eine Matrix $S(t)$ gesucht derart,
dass $p(t)=S(t)x(t)$ eine Lösung ist.
Durch Einsetzen in die Differentialgleichung folgt
\begin{align}
\dot{x} &= (A - BR^{-1}B^t S)x
\label{buch:hamiltonjacobi:oc:lqc:xdot}
\\
\dot{S}x+S\dot{x} &= -(Q+A^tS)x
\label{buch:hamiltonjacobi:oc:lqc:Sdot}
\end{align}
Durch Einsetzen von $\dot{x}$ aus der ersten Differentialgleichung
in die zweite entsteht die Gleichung
\[
\dot{S}x
=
(
-Q-A^tS  - SA+BR^{-1}B S
)
x.
\]
Da der Vektor $x$ durch Veränderung der Anfangsbedingung $x_0$ beliebige
Werte annehmen kann, muss die Matrix $S(t)$ die Matrixdifferentialgleichung
\begin{equation}
\dot{S}
=
-SA -A^tS+SBR^{-1}B^tS-Q
\label{buch:hamiltonjacobi:oc:eqn:riccati}
\end{equation}
mit der Randbedingung $S(t_1)=S_1$ erfüllen.
Die Differentialgleichung
\eqref{buch:hamiltonjacobi:oc:eqn:riccati}
heisst die {\em Riccati-Differentialgleichung}.
\index{Riccati-Differentialgleichung}%

Sobald $S(t)$ bestimmt ist, kann aus
\eqref{buch:hamiltonjacobi:oc:lqc:xdot}
durch durch Lösen der Differentialgleichung 
\begin{align*}
\dot{x} = (A-BR^{-1}B^tS(t)) x
\end{align*}
auch $x(t)$ bestimmt werden.
Daraus ergeben sich schliesslich
\[
p(t)
=
S(t)x(t)
\qquad
\text{und schliesslich}
\qquad
u(t)
=
-R^{-1}B^tp(t)=-R^{-1}B^tS(t)x(t).
\]
Damit ist das optimale
Steuerungsproblem~\ref{buch:hamiltonjacobi:oc:aufgabe:lqc}
vollständig gelöst.

\begin{beispiel}
Wir betrachten das eindimensionale optimale Steuerungsproblem mit
der Systemgleichung
\[
\dot x
=
ax + bu,
\qquad
x(0)=x_0
\]
und dem Funktional
\[
J(u)
=
\frac12 s_1x(t_1)^2
+
\frac12
\int_{t_0}^{t_1}
qx(t)^2 + ru(t)^2
\,dt.
\]
Nach der oben erarbeiteten Lösung muss $r>0$ sein und es muss
die Riccati-Differentialgleichung
\[
\dot{s}
=
-q-as -sa + \frac{b^2}rs
=
-q-2as+\frac{b^2}rs^2
\qquad\text{mit Randbedingung}\qquad
s(t_1)=s_1
\]
gelöst werden.

\end{beispiel}


%\uebungsabschnitt
%\aufgabetoplevel{chapters/080-hamiltonjacobi/uebungsaufgaben}
%\begin{uebungsaufgaben}
%\uebungsaufgabe{801}
%\end{uebungsaufgaben}
%\enduebungsabschnitt

