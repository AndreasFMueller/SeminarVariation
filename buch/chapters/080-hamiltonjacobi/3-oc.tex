%
% 3-oc.tex
%
% (c) 2024 Prof Dr Andreas Müller
%
\section{Optimale Steuerung
\label{buch:hamiltonjacobi:section:oc}}
\kopfrechts{Optimale Steuerung}
Die vorangegangenen Kapitel haben verschiedene Beispiele gezeigt,
in denen Funktionen gesucht waren, für die ein gewisses Integral
einen extremalen Wert annimmt.
In der Praxis tauchen aber auch Probleme auf, in denen die
gesuchte Funktion nicht direkt für die Berechnung des zu minimierenden
Integrals verwendet wird.
Stattdessen fliesst die Funktion in eine Differentialgleichung
ein, deren Lösung dann erst das Funktional ergibt.
Das optimale Steuerungsproblem ist ein Problem dieser Art: gesucht
wird eine Steuerfunktion derart, dass das gesteuerte System, welches
mit einer gewöhnlichen Differentialgleichung beschrieben ist, ein
Kostenfunktional minimieren soll.

In diesem Abschnitt wird die Lösung des optimalen Steuerungsproblem
mit Hilfe des Minimum-Prinzips von Pontryagin skizziert.
In Abschnitt~\ref{buch:hamiltonjacobi:oc:subsection:problem}
wird das Problem exakt formuliert.
Danach wird die Vorgehensweise Anhang eines Variationsproblems
mit Nebenbedingungen motiviert und die zugehörige Hamilton-Funktion
hergeleitet.
In Abschnitt~\ref{buch:hamiltonjacobi:oc:subsection:minimum}
wird dann das Minimum-Prinzip von Pontryagin hergeleitet, mit dem
sich viele dieser Probleme lösen lassen.

%
% Das optimale Steuerungsproblem
%
\subsection{Das optimale Steuerungsproblem
\label{buch:hamiltonjacobi:oc:subsection:problem}}
Ein {\em Steuerungsproblem} für ein System mit Differentialgleichung
der Form
\[
\dot{x}
=
f(t, x, u),
\]
wobei $f\colon\mathbb{R}^n\times\mathbb{R}^m\to\mathbb{R}^n$
eine differenzierbare Funktion ist.
Der Vektor $u$ beeinflusst die Zeitentwicklung der Lösung $x(t)$, man
kann ihn als die Steuersignale betrachten, mit denen man die Lösung
im Rahmen der physikalischen Grenzen des Systems beeinflussen kann.
Durch geeignete Wahl der Funktion $u(t)$ kann man erreichen, dass 
die Lösung mit gewissen Anfangsbedingungen $x(t_0)$ in einem
gewählten Endpunkt $x(t_1)$ endet.

Da man die Zeit ebenfalls als eine Steuervariable auffassen kann, 
kann man die Systemgleichung zu
\begin{equation*}
\dot{x}
=
f(x,u)
\end{equation*}
vereinfachen.
Man darf also annehmen, dass $f$ nicht explizit von der Zeit
abhängt.

%
% Das Kostenfunktional
%
\subsubsection{Das Kostenfunktional}
Wie in früher untersuchten Variationsproblemen können der Funktion $x(t)$
Kosten in Form eines Integrals der Form
\begin{equation}
\int_{t_0}^{t_1}
L(x(t),\dot{x}(t))
\,dt
\label{buch:hamiltonjacobi:oc:eqn:jxdotx}
\end{equation}
zugeordnet werden.
Man kann jetzt fragen, für welche Funktion $u(t)$ die sich daraus
ergebende Bahnkurve $x(t)$ das Funktional
\eqref{buch:hamiltonjacobi:oc:eqn:jx}
minimiert.

In den meisten Fällen umfasst der Zustandsvektor $x(t)$ für die
Systemdifferentialgleichung auch die Geschwindigkeit, so dass
es nicht länger sinnvoll ist, dass die Funktion $L$ auch von
$\dot{x}$ abhängig ist.
Das Funktional wird daher
\begin{equation}
\int_{t_0}^{t_1}
L(x(t))
\,dt
\label{buch:hamiltonjacobi:oc:eqn:jx}
\end{equation}

Dies ist aber noch nicht ganz allgemein genug, denn die Steuerung selbst
ist oft ebenfalls mit Kosten verbunden.
Man denke etwa an die Steuerung einer Rakete, wo jeder Steuerimpuls 
Raketentreibstoff verbraucht, von dem man aus Gewichtsgründen möglichst
wenig mitnehmen möchte.
Die Lagrange-Funktion sollte daher auch noch von $u$ abhängen, die
Kostenfunktion wird daher
\begin{equation}
\int_{t_0}^{t_1}
L(x(t),u(t))
\,dt.
\end{equation}

Doch auch dies ist noch nicht die endgültige Fassung.
In vielen Problemen spielt es eine wichtige Rolle, wann die
Operation abgeschlossen ist.
Zum Beispiel können im späteren Verlauf zusätzliche Kosten entstehen,
wenn die Steuerungsaufgabe erst später abgeschlossen wird.
Erreicht ein Raumschiff die Umlaufbahn einer Raumstation verspätet,
wird zusätzlicher Treibstoff für Orbitalmanöver benötigt, um die Station
zu erreichen.
Wir fügen dem zu minimierenden Funktional daher noch einen Randterm
hinzu und erhalten
\begin{equation}
J(u)
=
\int_{t_0}^{t_1}
L(x(t),u(t))
\,dt
+
K(x(t_1))
\label{buch:hamiltonjacobi:oc:eqn:jk}
\end{equation}
als endgültiges Funktional, welches zu minimieren ist.

%
% Die optimale Steuerungsaufgabe
%
\subsubsection{Die optimale Steuerungsaufgabe}
Damit lässt sich jetzt die Aufgabe formulieren, die in diesem
Abschnitt gelöst werden soll.

\begin{aufgabe}[optimale Steuerung]
\label{buch:hamiltonjacobi:oc:aufgabe}
Für das System mit der Differentialgleichung
\begin{equation}
\dot{x}
=
f(x,u), \qquad f\colon \mathbb{R}^n\times\mathbb{R}^m\to\mathbb{R}^n
\label{buch:hamiltonjacobi:oc:eqn:dgl}
\end{equation}
soll eine Funktion $u\colon [t_0,t_1]\to\mathbb{R}^m$ gefunden werden,
für die 
\begin{equation}
J(u)
=
\int_{t_0}^{t_1}
L(x(t),u(t))
\,dt
+
K(x(t_1))
\label{buch:hamiltonjacobi:oc:eqn:kosten}
\end{equation}
minimiert, wobei $x(t)$ eine Lösung der Differentialgleichung
\eqref{buch:hamiltonjacobi:oc:eqn:dgl} ist, die die Anfangsbedinung
$x(t_0)=x_0$ erfüllt.
\end{aufgabe}

Die Aufgabe verlangt also, eine Steuerfunktion $u(t)$ so zu bestimmen,
dass das System der Bahn $x(t)$ mit minimalen Kosten vom Startpunkt $x_0$
zum Zielpunkt $x_1$ folgt.

%
% Das optimale Steuerungsproblem als Variationsproblem mit Nebenbedingungen
%
\subsubsection{Das optimale Steuerungsproblem als Variationsproblem
mit Nebenbedingungen}
Die Formulierung von Aufgabe~\ref{buch:hamiltonjacobi:oc:aufgabe}
betrachtet die Funktionen $x(t)$ als von $u(t)$ abhängig.
Man muss also erst die
Differentialgleichung~\eqref{buch:hamiltonjacobi:oc:eqn:dgl}
lösen, bevor man das
Kostenfunktional~\eqref{buch:hamiltonjacobi:oc:eqn:kosten}
berechnen kann.
Diese Formulierung macht es unmöglich, die Aufgabe als 
Variationsproblem zu sehen.

Die Aufgabenstellung kann aber auch als Variationsproblem mit
einer Nebenbedingung gesehen werden.
Dazu werden sowohl die Variablen $x$ wie auch die Steuerinputs $u$
als unabhängige Variablen $q=(x,u)$ betrachtet.
Zu bestimmen ist jetzt eine Funktion $q(t)=(x(t),u(t))$, die
das Integral
\[
\int_{t_0}^{t_1}
L(x(t), u(t))
\,dt
\]
minimiert.
Jetzt sieht die Aufgabenstellung wie ein Variationsproblem aus,
allerdings wird damit noch nicht berücksichtigt, dass $x(t)$ eine
Lösung der Differentialgleichung~\eqref{buch:hamiltonjacobi:oc:eqn:dgl}
sein muss.

Der durch die Differentialgleichung~\eqref{buch:hamiltonjacobi:oc:eqn:dgl}
gegebene Zusammenhang von $x(t)$ und $u(t)$ kann als Nebenbedingung 
in der Form
\begin{equation}
G(x,\dot{x}, u)
=
f(x,u) - \dot{x} = 0
\label{buch:hamiltonjacobi:oc:eqn:dglneben}
\end{equation}
betrachtet werden.
Im Gegensatz zu den Problemen, die in
Abschnitt~\ref{buch:nebenbedingungen:lagrangemult:subsection:nebenbedingungen}
untersucht wurden, hat die
Nebenbedingung~\eqref{buch:hamiltonjacobi:oc:eqn:dglneben}
nicht die der Form eines Integrals.
Es reicht also nicht, eine zusätzliche Vektorvariable $\lambda\in\mathbb{R}^n$
einzuführen und damit ein Funktional mit einer Lagrange-Funktion der Form
\[
L(x,u) - \lambda \cdot G(x,\dot{x},u)
\]
zu minimieren, womit
Abschnitt~\ref{buch:nebenbedingungen:lagrangemult:subsection:nebenbedingungen}
erfolgreich gewesen war.

Als Erweiterung bietet sich an, die Variable $\lambda$ durch eine
Funktion $p(t)$ zu ersetzen.
So entsteht ein neues Funktional
\[
F
\colon
\mathbb{R}^n\times\mathbb{R}^m\times\mathbb{R}^n
\to\mathbb{R}
:
(x,u,p)
\mapsto
F(x,u,p)
=
L(x,u) + p\cdot (f(x,u) - \dot{x}),
\]
für welches jetzt untersucht werden soll, ob sich damit das Problem lösen
lässt.

%
% Die Euler-Lagrange-Differentialgleichung für F
%
\subsubsection{Die Euler-Lagrange-Differentialgleichung für $F$}
Gesucht ist jetzt eine Funktion $q(t)=(x(t),u(t),p(t))$ derart,
dass das Funktional mit der Lagrange-Funktion
\[
F(q,\dot{q})
=
L(x,u) +p\cdot f(x,u) - p\cdot \dot{x}
\]
minimiert.
Die Euler-Lagrange-Differentialgleichung von $F$ ist
\begin{equation}
\frac{\partial F}{\partial q}(q(t),\dot{q}(t))
=
\frac{d}{dt}
\frac{\partial F}{\partial \dot{q}}(q(t),\dot{q}(t)).
\label{buch:hamiltonjacobi:oc:eqn:Feulerlagrange}
\end{equation}
Die partiellen Ableitungen nach $q$ und $\dot{q}$ lassen sich aufteilen
in die Komponenten $x$, $u$ und $p$ von $q$.
Wir berechnen nur die Komponenten für $p$, sie ist
\begin{align}
\frac{\partial F}{\partial p}
&=
f(x,u)
-
\dot{x}
&
\frac{\partial F}{\partial \dot{p}}
&=
0
&&\Rightarrow&
f(x,u)-\dot{x}&=0
\label{buch:hamiltonjacobi:oc:eqn:Fsystemdgl}
\end{align}
Die letzte Gleichung ist die Nebenbedingung, eine Lösung der
Euler-Lagrange-Differential\-gleichung erfüllt also automatisch auch die
Systemdifferentialgleichung~\eqref{buch:hamiltonjacobi:oc:eqn:dgl}.

%
% Die Lösung der optimalen Steuerungsaufgaben
%
\subsubsection{Die Lösung der optimalen Steuerungsaufgaben}
Die Euler-Lagrange-Differentialgleichung zusammen mit geeigneten
Anfangsbedingungen für die gesuchten Funktionen können die
optimale Steuerungsaufgaben lösen, wie der folgende Satz zeigt.

\begin{satz}[Optimale Steuerung, Lagrange-Form]
\label{buch:hamiltonjacobi:oc:satz:optimal-lagrange}
Eine Extremale des Funktionals
\begin{equation}
I(q)
=
\int_{t_0}^{t_1}
F(q(t),\dot{q}(t))
\,dt
+
K(x(t))
\end{equation}
mit der Lagrange-Funktion
\begin{equation}
F(q,\dot{q}) = L(x,u) + p\cdot(f(x,u) - \dot{x})
\label{buch:hamiltonjacobi:oc:optF}
\end{equation}
und den Randbedingungen
\[
x(t_0)=x_0
\qquad\text{und}\qquad
p(t_1)=\frac{\partial K}{\partial x}(x(t_1))
\]
ist eine Lösung der optimale
Steuerungsaufgabe~\ref{buch:hamiltonjacobi:oc:aufgabe}.
\end{satz}

\begin{proof}
Da nach
\eqref{buch:hamiltonjacobi:oc:eqn:Fsystemdgl}
eine Lösung der Euler-Lagrange-Differentialgleichung von $F$ immer
auch die Systemdifferentialgleichung löst, ist 
\begin{equation*}
F(q_*(t),\dot{q}_*(t))
=
L(x_*(t),u_*(t))
+
p_*(t)\cdot (
\underbrace{f(x_*(t),u_*(t))-\dot{x}_*(t)}_{\displaystyle=0})
=
L(x_*(t),u_*(t)).
\end{equation*}
Insbesondere ist auch
\[
\int_{t_0}^{t_1}
F(q_*(t),\dot{q}_*(t))
\,dt
=
\int_{t_0}^{t_1}
L(x_*(t),u_*(t))
\,dt,
\]
Der Wert des Integrals im Funktion ändert also nicht.

Am Intervallende bei $t_1$ sind die Werte von $x_*(t)$ nicht
vorgegeben.
Der Satz~\ref{buch:nebenbedingungen:transversal:satz:randterme}
erklärt, wie der Randterm $K(x(t))$ sich in Randbedingungen
\[
\frac{\partial F}{\partial \dot{q}}(q(t),\dot{q}(t))
+
\frac{\partial K}{\partial q}(q(t_1))
=
0
\]
für $q(t_1)$ umrechnen lässt.
Indem man die Ableitungen nach den Komponenten von $q$ separat betrachtet,
findet man
\begin{align*}
0
&=
\frac{\partial F}{\partial\dot{x}}(q(t_1),\dot{q}(t_1))
+
\frac{\partial K}{\partial x}(q(t_1))
=
-p_*(t_1)
+
\frac{\partial K}{\partial x}(x_*(t_1))
&&\Rightarrow&p_*(t_1)
&=
\frac{\partial K}{\partial x}(x_*(t_1))
\intertext{Dies ist die gesuchte Randbedingung für $p_*(t)$.
Die verbleibenden Komponenten sind}
0
&=
\frac{\partial F}{\partial\dot{u}}(q(t_1),\dot{q}(t_1))
+
\frac{\partial K}{\partial u}(q(t_1))
\\
0
&=
\frac{\partial F}{\partial\dot{p}}(q(t_1),\dot{q}(t_1))
+
\frac{\partial K}{\partial p}(q(t_1)).
\end{align*}
Da die Funktionen auf der rechten Seite nicht von den Variablen
abhängen, nach denen sie abgeleitet werden, sind diese Bedingungen
automatisch erfüllt.
\end{proof}

Die Funktion $F(q,\dot{q})$ hängt nicht von der unabhängigen
Variablen $t$ ab, somit ist die Beltrami-Identität
(Satz~\ref{buch:variation:eulerlagrange:satz:beltrami})
anwendbar.
Sie sagt, dass die Grösse
\begin{align}
F(q,\dot{q})
-
\dot{q}\cdot\frac{\partial F}{\partial\dot{q}}(q,\dot{q})
&=
F(q,\dot{q})
-
\dot{x}\cdot
\underbrace{
\frac{\partial F}{\partial\dot{x}}
}_{\displaystyle=-p}
\mathstrut-
\dot{u}\cdot
\underbrace{
\frac{\partial F}{\partial\dot{u}}
}_{\displaystyle=0}
\mathstrut-
\dot{p}\cdot
\underbrace{
\frac{\partial F}{\partial\dot{p}}
}_{\displaystyle=0}
\notag
\\
&=
L(x,u)+p\cdot f(x,u)-p\cdot\dot{x}
-
\dot{x}\cdot
(-p)
\notag
\\
&=
L(x,u) + p\cdot f(x,u)
\label{buch:hamiltonjacobi:oc:hamilton}
\end{align}
konstant ist.
Diese Funktion spielt daher in der nachfolgenden Konstruktion zur
Lösung der optimalen Steuerungsaufgabe eine grosse Rolle.

%
% Die Hamilton-Differentialgleichungen
%
\subsection{Die Hamilton-Differentialgleichungen
\label{buch:hamiltonjacobi:oc:subsection:hamilton}}
Für eine Lagrange-Funktion $L(x,u)$ haben wir gefunden, dass der
Ausdruck~\eqref{buch:hamiltonjacobi:oc:hamilton} eine besondere
Bedeutung hat.
Wir geben ihm daher einen Namen.

\begin{definition}[Steuerungs-Hamiltonfunktion]
Die Funktion
\[
H
\colon
\mathbb{R}^n\times\mathbb{R}^n\times\mathbb{R}^m
\to
\mathbb{R}
:
(x,p,u)
\mapsto
H(x,p,u)
=
p\cdot f(x,u) + L(x,u)
\]
heisst die {\em Steuerungs-Hamilton-Funktion}
der optimalen Steuerungsaufgabe~\ref{buch:hamiltonjacobi:oc:aufgabe}.
\end{definition}

Der Name rechtfertigt sich dadurch, dass sich aus $H$ analoge
Differentialgleichungen für die Funktionen $x(t)$ und $p(t)$ bilden
lassen.
Damit wird es möglich, auch die Funktion $p(t)$ zu berechnen.
Wenn sich ausserdem eine Bedingung für $u(t)$ finden lässt, kann
die Aufgabe~\ref{buch:hamiltonjacobi:oc:aufgabe}
vollständig gelöst werden.
Eine solche Bedingung wird später im Minimum-Prinzip gefunden.
Als erstes sollen jetzt die Differentialgleichungen konstruiert
werden, die sich aus $H$ ableiten lassen.

%
% Partielle Ableitungen
%
\subsubsection{Partielle Ableitungen}
Zu diesem Zweck sei $q_*(t)=(x_*(t),u_*(t),p_*(t))$ eine Lösung der
Euler-Lagrange-Differen\-tialgleichung
\eqref{buch:hamiltonjacobi:oc:eqn:Feulerlagrange}
für die Lagrange-Funktion $F(q,\dot{q})$.
Als Vorbereitung für die Berechnung der partiellen Ableitungen von $H$
schreiben wir die Euler-Lagrange-Differen\-tialglei\-chungen explizit
für die einzelnen Komponenten von $q$ hin:
\begin{align}
0
&=
\frac{\partial F}{\partial x_k}(x_*(t),u_*(t),p_*(t))
-
\frac{d}{dt}
\frac{\partial F}{\partial \dot{x}_k}(x_*(t),u_*(t),p_*(t))
\notag
\\
&=
\frac{\partial L}{\partial x_k}(x_*(t),u_*(t))
+
p_*(t)\cdot\frac{\partial f}{\partial x_k}(x_*(t),u_*(t))
+
\dot{p}_{*k}
\notag
\\
\Rightarrow\qquad
\dot{p}_*(t)
&=
-\frac{\partial L}{\partial x}(x_*(t),u_*(t))
-
p_*(t)\cdot\frac{\partial f}{\partial x}(x_*(t),u_*(t))
\label{buch:hamiltonjacobi:oc:eqn:Fxabl}
\intertext{Die Variablen $\dot{u}$ und $\dot{p}$ kommen in $F$ nicht
vor, der Zeitableitungsterm kommt daher in den zugehörigen
Euler-Lagrange-Gleichungen nicht vor:}
0
&=
\frac{\partial F}{\partial u_k}(x_*(t),u_*(t),p_*(t))
-
\frac{d}{dt}
\frac{\partial F}{\partial \dot{u}_k}(x_*(t),u_*(t),p_*(t))
\notag
\\
&=
\frac{\partial L}{\partial u_k}(x_*(t),u_*(t))
+
p_*(t)\cdot\frac{\partial f}{\partial u_k}(x_*(t),u_*(t))
\label{buch:hamiltonjacobi:oc:eqn:Fuabl}
\\
0
&=
\frac{\partial F}{\partial p_k}(x_*(t),u_*(t),p_*(t))
-
\frac{d}{dt}
\frac{\partial F}{\partial \dot{p}_k}(x_*(t),u_*(t),p_*(t))
\notag
\\
&=
f_k(x_*(t),u_*(t))-\dot{x}_{*k}(t)
\notag
\\
\Rightarrow\qquad
\dot{x}_*(t)
&=
f(x_*(t),u_*(t)).
\label{buch:hamiltonjacobi:oc:eqn:Fpabl}
\end{align}

Nach diesen Vorbereitungen können jetzt auch die partiellen Ableitungen
von $H$ berechnet werden.
Unter Verwendung von 
\eqref{buch:hamiltonjacobi:oc:eqn:Fxabl},
\eqref{buch:hamiltonjacobi:oc:eqn:Fuabl}
und
\eqref{buch:hamiltonjacobi:oc:eqn:Fpabl}
Folgen die Gleichungen
\begin{equation}
\renewcommand{\arraycolsep}{2pt}
\renewcommand{\arraystretch}{2.0}
\begin{array}{rclcl}
\displaystyle
\frac{\partial H}{\partial x_k}(x_*(t),u_*(t),p_*(t))
&=&
\displaystyle
\frac{\partial L}{\partial x}(x_*(t),u_*(t))
+
p_*(t)\cdot\frac{\partial f}{\partial x_k}(x_*(t),u_*(t))
&=&-\dot{p}_{*k}(t)
\\
\displaystyle
\frac{\partial H}{\partial u_k}(x_*(t),u_*(t),p_*(t))
&=&
\displaystyle
p_*(t)\cdot
\frac{\partial f}{\partial u_k}(x_*(t),u_*(t))
+
\frac{\partial L}{\partial u_k}(x_*(t),u_*(t))
&=&0
\\
\displaystyle
\frac{\partial H}{\partial p_k}(x_*(t),u_*(t),p_*(t))
&=&
f_k(x_*(t),u_*(t))
&=&
\dot{x}_{*k}(t)
\end{array}
\label{buch:hamiltonjacobi:oc:Habl}
\end{equation}

%
% Randbedingungen
%
\subsubsection{Randbedingungen}
Die Anfangsbedingung $x_*(t_0)=x_0$ für $x_*(t)$ ist bereits durch
die Aufgabenstellung gegeben.
Für die Funktion $p_*(t)$ wird eine weitere Randbedingung benötigt,
damit die Differentialgleichungen~\eqref{buch:hamiltonjacobi:oc:Habl}
Beide Funktionen $x_*(t)$ und $p_*(t)$ bestimmen können.
Der Satz~\ref{buch:hamiltonjacobi:oc:satz:optimal-lagrange}
definiert auch die Randbedingung
\[
p_*(t_1)
=
\frac{\partial K}{\partial x}(x_*(t_1)).
\]

%
% Differentialgleichung
%
\subsubsection{Differentialgleichung}
Die Differentialgleichungen~\eqref{buch:hamiltonjacobi:oc:Habl} zusammen
mit den Randbedingungen des vorangegangenen Abschnitts ergeben jetzt
den folgenden Satz.

\begin{satz}[Hamilton-Gleichungen]
\label{buch:hamiltonjacobi:oc:satz:hamilton-gleichungen}
Eine Lösung $(x_*(t),u_*(t),p_*(t))$ der optimalen Steuerungsaufgabe
erfüllt die {\em Hamilton-Differentialgleichungen}
\begin{align*}
\dot{x}_*(t)
&=
\phantom{-}
\frac{\partial H}{\partial p}(x_*(t),u_*(t),p_*(t)),
&
x_*(t_0)
&=
x_0,
\\
\dot{p}_*(t)
&=
-
\frac{\partial H}{\partial x}(x_*(t),u_*(t),p_*(t)),
&
p_*(t_1)
&=
\frac{\partial K}{\partial x}(x_*(t_1)).
\end{align*}
Ausserdem verschwindet die Ableitung nach $u$:
\begin{equation}
\frac{\partial H}{\partial u}(x_*(t),u_*(t),p_*(t))=0.
\label{buch:hamiltonjacobi:oc:eqn:ablHu0}
\end{equation}
\end{satz}

Der Satz bestimmt $x_*(t)$ und $p_*(t)$ sobald $u_*(t)$ gegeben ist,
er macht aber ausser der Bedingung, dass die
Ableitung~\eqref{buch:hamiltonjacobi:oc:eqn:ablHu0} verschwindet,
keine Aussage darüber, wie $u_*(t)$ gefunden werden kann.
Ein mögliche Wahl für $u_*(t)$ könnte sein, dass $u_*(t)$ jeweils
als das Extremum der Funktion $u\mapsto H(x_*(t),u,p_*(t))$
gewählt wird.
Dass dies auch die optimale Wahl sein kann, zeigt das Minimum-Prinzip
von Pontryagin, welches im
Abschnitt~\ref{buch:hamiltonjacobi:oc:subsection:minimum}
besprochen werden soll.

%
% Das Minimum-Prinzip
%
\subsection{Das Minimum-Prinzip
\label{buch:hamiltonjacobi:oc:subsection:minimum}}
In Satz~\ref{buch:hamiltonjacobi:oc:satz:hamilton-gleichungen}
und in der daran anschliessenden Diskussion wird angedeutet, dass
wegen der Bedingung~\eqref{buch:hamiltonjacobi:oc:eqn:ablHu0}
die Funktion $u_*(t)$ ein Extremum der Funktion $H$ zu den gegebenen
Werten $x_*(t)$ und $p_*(t)$ sei.
Eine solche Bedingung vervollständigt die Lösung des optimalen
Steuerungsproblems.

\begin{satz}[Pontryagin]
Seien die Funktionen $f(x,u)$ und $L(x,u)$ stetig differenzierbar in 
$x$ und $u$ und sei $K(x)$ stetig differenzierbar in $x$.
Sei die Funktion $u_*\colon[t_0,t_1]\to\mathbb{R}$ eine Lösung des
optimalen Steuerungsproblems Aufgabe~\ref{buch:hamiltonjacobi:oc:aufgabe}
und sei $x_*(t)$ die resultierende optimale Bahnkurve, dann gibt es
eine eindeutig bestimmte Funktion $p_*\colon[t_0,t_1]\to\mathbb{R}$,
So dass $x_*(t),u_*(t),p_*(t)$ die Hamilton-Differentialgleichungen
von Satz~\ref{buch:hamiltonjacobi:oc:satz:hamilton-gleichungen}
erfüllen.
Die Funktion $u_*(t)$ erfüllt
\begin{equation}
u_*(t)
=
\operatorname{argmin}_{u\in\mathbb{R}} H(x_*(t),u,p_*(t)).
\label{buch:hamiltonjacobi:oc:eqn:argminH}
\end{equation}
für alle Stellen $t\in[t_0,t_1]$, an denen $u_*(t)$ stetig ist.
\end{satz}

Dass die Bedingung \eqref{buch:hamiltonjacobi:oc:eqn:argminH} nur
an Stellen gelten muss, an denen $u_*(t)$ stetig ist, lässt zu,
dass die Funktion $u_*$ nicht stetig ist.
Dies wird von den folgenden Beispielen illustriert.

%
% oc.tex -- einfache OC Beispiele
%
% (c) 2021 Prof Dr Andreas Müller, OST Ostschweizer Fachhochschule
%
\documentclass[tikz]{standalone}
\usepackage{amsmath}
\usepackage{times}
\usepackage{txfonts}
\usepackage{pgfplots}
\usepackage{csvsimple}
\usetikzlibrary{arrows,intersections,math}
\definecolor{darkred}{rgb}{0.8,0,0}
\definecolor{darkgreen}{rgb}{0,0.6,0}
\begin{document}
\def\skala{1}
\begin{tikzpicture}[>=latex,thick,scale=\skala]

\clip (-3.8,3.0) rectangle (8.8,-9.8);

\begin{scope}[xshift=-3.2cm]

\begin{scope}
\draw[color=darkred,line width=1.4pt] (0,2.5) -- (5,0);
\node[color=darkred] at (2.5,1.25) [above right] {$p_*(t)$};
\draw[->] (0,-0.1) -- (0,2.8) coordinate[label={right:$p$}];
\draw[->] (-0.1,0) -- (5.4,0) coordinate[label={$t$}];
\draw (5,-0.05) -- (5,0.05);
\draw (-0.05,2.5) -- (0.05,2.5);
\node at (-0.05,2.5) [left] {$1$};
\node at (-0.05,0.0) [left] {$0$};
\draw[line width=0.3pt] (5,2.6) -- +(0,-11.9);
\end{scope}

\begin{scope}[yshift=-3.3cm]
\draw[color=darkred,line width=1.4pt] (0,-2.5) -- (5,-2.5);
\node[color=darkred] at (2.5,-2.5) [above] {$u_*(t)$};
\draw[->] (0,-2.6) -- (0,2.8) coordinate[label={right:$u$}];
\draw[->] (-0.1,0) -- (5.4,0) coordinate[label={$t$}];
\draw (5,-0.05) -- (5,0.05);
\draw (-0.05,2.5) -- (0.05,2.5);
\node at (-0.05,2.5) [left] {$1$};
\draw (-0.05,-2.5) -- (0.05,-2.5);
\node at (-0.05,-2.5) [left] {$-1$};
\node at (-0.05,0.0) [left] {$0$};
\end{scope}

\begin{scope}[yshift=-8.2cm]
\draw[color=darkred,line width=1.4pt] (0,1.5) -- (5,-1);
\node[color=darkred] at (2.5,0.25) [above right] {$x_*(t)$};
\draw[->] (0,-1.1) -- (0,1.9) coordinate[label={right:$x$}];
\draw[->] (-0.1,0) -- (5.4,0) coordinate[label={$t$}];
\draw (5,-0.05) -- (5,0.05);
\node at (-0.05,0.0) [left] {$0$};
\draw (-0.05,1.5) -- (0.05,1.5);
\node at (-0.05,1.5) [left] {$x_0$};
\end{scope}

\node at (2.5,-9.6) {(a)};

\end{scope}

\begin{scope}[xshift=3.2cm]
\fill[color=blue!10,opacity=0.5] (0,2.6) rectangle (2.5,-9.3);
\fill[color=darkgreen!10,opacity=0.5] (2.5,2.6) rectangle (5,-9.3);

\begin{scope}[yshift=1.25cm]
\draw[color=darkred,line width=1.4pt] (0,1.25) -- (5,-1.25);
\node[color=darkred] at (2.5,0) [above right] {$p_*(t)$};
\draw[->] (0,-1.35) -- (0,1.55) coordinate[label={right:$p$}];
\draw[->] (-0.1,0) -- (5.4,0) coordinate[label={$t$}];
\draw (5,-0.05) -- (5,0.05);
\draw (-0.05,1.25) -- (0.05,1.25);
\node at (-0.05,1.25) [left] {$\frac12$};
\draw (-0.05,-1.25) -- (0.05,-1.25);
\node at (-0.05,-1.25) [left] {$-\frac12$};
\node at (-0.05,0.0) [left] {$0$};
\draw[line width=0.3pt] (5,1.35) -- +(0,-11.9);
\end{scope}

\begin{scope}[yshift=-3.3cm]
\draw[color=darkred,line width=1.4pt] (0,-2.5) -- (2.5,-2.5);
\draw[color=darkred,line width=0.3pt] (2.5,-2.5) -- (2.5,2.5);
\draw[color=darkred,line width=1.4pt] (2.5,2.5) -- (5,2.5);
\node[color=darkred] at (1.25,-2.5) [above] {$u_*(t)$};
\node[color=darkred] at (3.75,2.5) [below] {$u_*(t)$};
\draw[->] (0,-2.6) -- (0,2.8) coordinate[label={right:$u$}];
\draw[->] (-0.1,0) -- (5.4,0) coordinate[label={$t$}];
\draw (5,-0.05) -- (5,0.05);
\draw (-0.05,2.5) -- (0.05,2.5);
\node at (-0.05,2.5) [left] {$1$};
\draw (-0.05,-2.5) -- (0.05,-2.5);
\node at (-0.05,-2.5) [left] {$-1$};
\node at (-0.05,0.0) [left] {$0$};
\end{scope}

\begin{scope}[yshift=-8.2cm]
\draw[color=darkred,line width=1.4pt] (0,1.5) -- (2.5,0.25) -- (5,1.5);
\node[color=darkred] at (2.5,0.35) [above] {$x_*(t)$};
\draw[->] (0,-1.1) -- (0,1.9) coordinate[label={right:$x$}];
\draw[->] (-0.1,0) -- (5.4,0) coordinate[label={$t$}];
\draw (5,-0.05) -- (5,0.05);
\node at (-0.05,0.0) [left] {$0$};
\draw (-0.05,1.5) -- (0.05,1.5);
\node at (-0.05,1.5) [left] {$x_0$};
\end{scope}

\node at (2.5,-9.6) {(b)};

\end{scope}

\end{tikzpicture}
\end{document}



\begin{beispiel}
\label{buch:hamiltonjacobi:oc:bsp:simple}
Gesucht ist die Lösung des eindimensionalen optimalen Steuerproblems
auf dem Intervall zwischen $t_0=0$ und $t_1=1$ mit
\begin{align*}
f(x,u) &= u\in U=[-1,1],
&
x(0) &= x_0\in\mathbb{R}
&
&\text{und}&
J(u)
=
\int_0^1 x(t)\,dt.
\end{align*}
Die Lagrange-Funktion ist $L(x,u)=x$ und die Funktion $K(x)$ ist die
Nullfunktion.

Für die Lösung des Problems berechnen wird zunächst die
Steuerungs-Hamilton-Funktion
\[
H(x,u,p)
=
L(u,x) + pf(x,u)
=
x + pu
\]
und ihre Ableitungen
\begin{equation}
\left.
\begin{aligned}
\frac{\partial H}{\partial x}
&=
1,
&&&
\frac{\partial H}{\partial\dot{x}}
&=
0,
\\
\frac{\partial H}{\partial u}
&=
p,
&&&
\frac{\partial H}{\partial\dot{u}}
&=
0,
\\
\frac{\partial H}{\partial p}
&=
u,
&&&
\frac{\partial H}{\partial\dot{p}}
&=
0.
\end{aligned}
\qquad
\right\}
\label{buch:hamiltonjacobi:oc:bsp:ableitungen}
\end{equation}
Daraus folgen die Hamilton-Differentialgleichungen
\begin{align*}
&\text{für $x(t)$:}&
\dot{x}(t)&=u(t)
&&\text{mit Anfangsbedingung}&
x(0)&=x_0
\qquad\text{und}
\\
&\text{für $p(t)$:}&
\dot{p}(t)&=-1
&&\text{mit Randbedingung}&
p(1)&=\frac{\partial K}{\partial x}(1) = 0.
\end{align*}
Aus der Gleichung für $p$ folgt $p_*(t)=1-t$
(Abbildung~\ref{buch:hamiltonjacobi:oc:fig:oc} (a)).
Das Minimumprinzip sagt jetzt, dass für $u_*(t)$ das Argument $u$
gewählt werden muss, welches
\[
H(x_*(t),u,p_*(t))
=
x_*(t)+p_*(t)u
=
x_*(t)+(1-t)u
\]
minimiert.
Da $1-t<0$ ist für alle $t<1$, wird ganz unabhängig von $x_*(t)$
der kleinstmögliche Wert bei $u_*(t)=-1$ angenommen.
Mit diesem $u_*(t)$ kann man jetzt auch die Lösung $x_*(t)=-t$
berechnen.

Diese Lösung ist auch plausibel, denn weil kein Randterm vorhanden ist,
kommt es nur darauf an, dass $x_*(t)$ möglichst schnell abnimmt, um
das Integral
\[
\int_0^1 x(t)\,dt
=
\int_0^1 -t\,dt
=
-\biggl[\frac12t^2\biggr]_0^1
=
-\frac12
\]
zu minimieren.
\end{beispiel}

\begin{beispiel}
\label{buch:hamiltonjacobi:oc:bsp:switch}
Lösen Sie das optimale Steuerungsproblem für das System auf dem Intervall
$[t_0,t_1]=[0,1]$ mit der Systemgleichung
\[
\dot{x} = f(x,u)=u\in U=[-1,1],\quad x(0) = x_0
\]
und dem Funktional
\[
J(u)
=
\int_0^1 x(t)\,dt -\frac12 x(1)
\]
mit Lagrange-Funktion und Randterm
\begin{align*}
L(x,u)
&=
x
&&\text{und}&
K(x)&=-\frac12x.
\end{align*}
Die Steuerungs-Hamilton-Funktion ist
\[
H(x,u,p)
=
L(x,u) + pf(x,u)
=
x + pu
\]
mit den Ableitungen wie in~\eqref{buch:hamiltonjacobi:oc:bsp:ableitungen}.
Die Hamilton-Differentialgleichungen sind jetzt
\begin{align*}
&\text{für $x(t)$:}&
\dot{x}(t)&=u(t)
&&\text{mit der Anfangsbedingung:}&
x(0)&=x_0\qquad\text{und}
\\
&\text{für $p(t)$:}&
\dot{p}(t)&=-1
&&\text{mit der Randbedingung:}&
p(1)&=\frac{\partial K}{\partial x}(1) = -\frac12.
\end{align*}
Aus der Gleichung für $p$ folgt diesmal $p_*(t)=\frac12-t$.
Zur Bestimmung von $u_*(t)$ müssen wir die Hamilton-Funktion
\[
H(x_*(t),u,p_*(t))
=
x_*(t)+p_*(t)u
=
x_*(t)+({\textstyle\frac12}-t)u
\]
betrachten.
Welcher Werte von $u$ den kleinsten Wert von $H$ ergibt, hängt
vom Vorzeichen von $p_*(t)=\frac12-t$.
Für positives Vorzeichen von $p_*(t)$ muss $u=-1$ gewählt werden,
sonst $u=1$.
Also ist
\[
u_*(t)
=
\left\{
\begin{aligned}
&\phantom{-}1&&\text{falls $p_*(t)<0$}&&\Leftrightarrow&t&>\textstyle\frac12
\\
&{}-1&&\text{falls $p_*(t)>0$}&&\Leftrightarrow&t&<\textstyle\frac12.
\end{aligned}
\right.
\]
Daraus kann man jetzt auch $x_*(t)$ berechnen:
\[
x_*(t)
=
\int_0^t u_*(\tau)\,d\tau
=
\begin{cases}
x_0+{\displaystyle\int_0^t 1\,d\tau} = x_0+t&\qquad\text{für $t\le\frac12$}
\\[8pt]
x_*(\frac12)-{\displaystyle\int_{\frac12}^t 1\,d\tau}
=
x_0+\frac12-(t-\frac12)
=
x_0+1-t&\qquad\text{für $t>\frac12$.}
\end{cases}
\]
Die Lösungsfunktionen sind in
Abbildung~\ref{buch:hamiltonjacobi:oc:fig:oc} (b)
dargestellt.
Der Wert des Funktionals kann jetzt ebenfalls berechnet werden, er ist
\begin{align*}
J(u_*)
&=
\int_0^1 x_*(t)\,dt + K(x_*(1))
\\
&=
\int_0^{\frac12} x_0+t \,dt
+
\int_{\frac12}^1 x_0+1-t\,dt
-
\frac12x_0
\\
&=
\biggl[x_0t+\frac12t^2\biggr]_0^{\frac12}
+
\biggl[x_0t+t-\frac12t^2\biggr]_{\frac12}^1 - \frac12x_0
\\
&=
\frac12x_0 + \frac18 + x_0+1-\frac12 - \frac12x_0-\frac12 +\frac18 - \frac12x_0
\\
&=\frac12x_0+\frac14.
\qedhere
\end{align*}
\end{beispiel}

\begin{proof}[Beweis des Minimum-Prinzips]
Der Beweis erfolgt durch Widerspruch.
Wir nehmen dazu an, dass sich der Wert der Hamilton-Funktion an einer
Stelle $\bar{t}$ durch Wahl eines von $u_*(\bar{t})$ abweichenden Wertes
verkleinern lässt.
Die Änderung eines einzelnen Funktionswertes beeinflusst aber $x_*(t)$ nicht.
Nur wenn $u_*(t)$ in einem kleinen Intervall in einer Weise abgeändert
werden kann, dass $H(x_*(t),u_*(t),p_*(t))$ verringert wird, ergibt sich
auch eine merkliche Veränderung von $x_*(t)$ und des Kostenfunktionals $J(u)$.
Dazu muss angenommen werden dürfen, dass $u_*(t)$ an der Stelle $\bar{t}$
stetig ist.
Wir nehmen daher an, dass sogar in einem kleinen Intervall 
$[\bar{t},\bar{t}+\varepsilon]$ noch gilt
\[
H(x_*(t),\bar{u},p_*(t)) < H(x_*(t), u_*(t), p_*(t)).
\]
Dann ist für genügend kleines $\varepsilon$
\begin{equation}
H(x_*(t),\bar{u},p_*(t)) - H(x_*(t),u_*(t),p_*(t)) < c < 0
\label{buch:hamiltonjacobi:oc:eqn:Hunterschied}
\end{equation}
mit einer negativen Konstanten $c\in\mathbb{R}$.

Wir ändern $u_*(t)$ um die Funktion
\[
\delta u(t)
=
\begin{cases}
\bar{u}&\qquad t\in [\bar{t},\bar{t}+\varepsilon] \\
0      &\qquad\text{sonst}.
\end{cases}
\]
Mit den veränderten Steuerinputs $\bar{u}(t) = u_*(t) + \delta u(t)$ ändert
sich jetzt auch der Systemzustand $x_*(t)$.
Wir schreiben die Änderung als Funktion $\delta x(t)$, die neue
Zustandsfunktion ist somit $\bar{x}(t) = x_*(t)+\delta x(t)$.
Sie erfüllt die Systemgleichungen, es gilt also
\[
\frac{d}{dt}\bar{x}(t)
=
\dot{x}_*(t) + (\delta x)^{\dot{}}(t)
=
f(\bar{x}(t),\bar{u}(t))
=
f(x_*(t)+\delta x(t), u_*(t)+\delta u(t)).
\]
Die zeitliche Ableitung von $\delta x(t)=\bar{x}(t)-x_*(t)$ schreiben wir als
$(\delta x)^{\dot{}}(t)$, da $\delta\dot{x}(t)$ auch als die Differenz
$\dot{\bar{x}}(t)-\dot{x}_*(t)$ missverstanden werden könnte.
Nach allgemeinen Sätzen über die stetig Abhängigkeit der Lösung einer
Differentialgleichung von $u(t)$ folgt, dass sich
$\delta x(t)=O(\varepsilon)$ ist für $t>\bar{t}$.

Wir werden zeigen, dass durch diese Änderungen das Kostenfunktional
$J(\bar{u})$ reduziert wird im Widerspruch zur Voraussetzung, dass 
mit $u_*(t)$ bereits das Minimum erreicht wird.

Die Änderung des Kostenfunktionals ist
\begin{align*}
\Delta
&=
J(\bar{u}) - J(u_*)
\\
&=
K(x_*(t_1)+\delta x(t_1)) - K(x_*(t_1))
+
\int_{t_0}^{t_1} 
L(x_*(t) + \delta x(t), u_*(t)+\delta u(t))
-
L(x_*(t), u_*(t))
\,dt.
\intertext{Da $K$ differenzierbar ist, können die ersten beiden Terme 
bis auf ein $o(\varepsilon)$ durch die Ableitung approximiert werden.
Den Integranden drücken wir mittels $L=-p\cdot f+H$ durch die
Hamilton-Funktion aus und erhalten}
&=
\frac{\partial K}{\partial x}(x_*(t_1))\cdot \delta x(t_1) + o(\varepsilon)
\\
&\quad
-
\int_{t_0}^{t_1}
p_*(t)
\cdot
(f(\bar{x}(t),\bar{u}(t))-f(x_*(t),u_*(t))
\,dt
\\
&\quad
+
\int_{t_0}^{t_1}
H(\bar{x}(t),\bar{u}(t),p_*(t)) - H(x_*(t),u_*(t),p_*(t))
\,dt.
\end{align*}
Die Funktionen $\bar{x}(t)$ und $x_*(t)$ erfüllen die
Systemgleichung für die Steuerinputs $\bar{u}(t)$ und $u_*(t)$, so dass
die Differenz wegen der Linearität der Ableitung die Ableitung
\[
f(\bar{x}(t),\bar{u}(t))-f(x_*(t),u_*(t)
=
\dot{\bar{x}}(t) - \dot{x}_*(t)
=
\frac{d}{dt} (\bar{x}(t)-x_*(t))
=
(\delta x)^{\dot{}}(t)
\]
der Differenz ist.
Durch Addieren und Subtrahieren von $H(x_*(t), \bar{u}(t),p_*(t))$ kann
das zweite Integral in einfacher zu handhabende Integrale aufgeteilt
werden, wir erhalten
\begin{align*}
\Delta
&=
p_*(t_1)\cdot \delta x(t_1) + o(\varepsilon)
-
\int_{t_0}^{t_1}
p_*(t)\frac{d}{dt}(\delta x)(t)\,dt
\\
&\quad
+
\int_{t_0}^{t_1}
H(x_*(t)+\delta x(t), u_*(t) + \delta u(t), p_*(t))
-
H(x_*(t), u_*(t) + \delta u(t), p_*(t))
\,dt
\\
&\quad
+
\int_{t_0}^{t_1}
H(x_*(t), u_*(t) + \delta u(t), p_*(t))
-
H(x_*(t), u_*(t), p_*(t))
\,dt
\intertext{Die Differenz im zweiten Integral kann durch die Ableitung nach $x$
approximiert und mit dem ersten Integral zusammengefasst werden:}
&=
p_*(t_1)\cdot \delta x(t_1) + o(\varepsilon)
\\
&\quad
+
\int_{t_0}^{t_1}
-p_*(t)\frac{d}{dt}(\delta x)(t)\,dt
+
\frac{\partial H}{\partial x}(x_*(t),u_*(t)+\delta u(t),p_*(t))
\cdot 
\delta x(t)
+
o(\varepsilon)
\,dt
\\
&\quad
+
\int_{t_0}^{t_1}
H(x_*(t), u_*(t) + \delta u(t), p_*(t))
-
H(x_*(t), u_*(t), p_*(t))
\,dt.
\intertext{Die Funktion $\bar{u}(t)$ ist nur im Intervall
$[\bar{t},\bar{t}+\varepsilon]$  von $u_*(t)$ verschieden, daher ist auch
die Ableitung von $H$ im ersten Integral nur in diesem Intervall
von $-\dot{p}_*(t)$ verschieden.
Ausserdem ist auch $\delta x(t)$ nur von der Grössenordnung
$o(\varepsilon)$, so dass wird das erste Integral ersetzen können:}
&=
p_*(t_1)\cdot \delta x(t_1) + o(\varepsilon)
+
\int_{t_0}^{t_1}
-p_*(t)
\cdot
\frac{d}{dt}\delta x(t)\,dt
-
\frac{d}{dt}p_*(t)
\cdot 
\delta x(t)
+
o(\varepsilon)
\,dt
\\
&\quad
+
\int_{t_0}^{t_1}
H(x_*(t), u_*(t) + \delta u(t), p_*(t))
-
H(x_*(t), u_*(t), p_*(t))
\,dt.
\intertext{Die ersten beiden Terme im ersten Integral sind die
Ableitung von $-p_*(t)\cdot \delta x(t)$, mit dieser Stammfunktion
kann das Integral ausgwertet werden.
Der Integrand im zweiten Integral ist nur im Intervall
$[\bar{t},\bar{t}+\varepsilon]$ von $0$ verschieden und ist nach
\eqref{buch:hamiltonjacobi:oc:eqn:Hunterschied}
kleiner als $c$.
Es folgt}
&<
p_*(t_1)\cdot \delta x(t_1) + o(\varepsilon)
-
\Bigl[
p_*(t)\cdot \delta x(t)
\Bigr]_{t_0}^{t_1}
+
c\varepsilon
\\
&=
\underbrace{
p_*(t_1)\cdot \delta x(t_1)
-
p_*(t_1)\cdot \delta x(t_1)
}_{\displaystyle = 0}
+
p_*(t_0)\cdot \delta x(t_0)
+ o(\varepsilon)
+ c\varepsilon.
\intertext{Zur Zeit $t_0$ ist $\delta x(t_0) = 0$ und damit}
\Delta
&<
o(\varepsilon) + c\varepsilon
=
\varepsilon \biggl(c + \frac{o(\varepsilon)}{\varepsilon}\biggr).
\end{align*}
Der zweite Term in der Klammer konvergiert gegen $0$, für genügend
kleines $\varepsilon$ ist die Klammer daher negativ und damit folgt,
dass $\Delta <0$ ist.
Damit ist der Widerspruch gezeigt, die Annahme wiederlegt und damit
der Satz bewiesen.
\end{proof}

%
% Linear-Quadratische optimale Steuerungsprobleme
%
\subsection{Linear-Quadratische optimale Steuerungsprobleme
\label{buch:hamiltonjacobi:oc:subsection:quadratisch}}
Ein spezielle Fall, in dem die Rechnungen vollständig durchführbar sind,
ist das {\em linear-quaratische optimale Steuerungsproblem}.
\index{linear-quadratisches optimales Steuerungsproblem}%

\begin{aufgabe}
\label{buch:hamiltonjacobi:oc:aufgabe:lqc}
Das System mit der linearen Systemgleichung
\begin{equation}
\dot{x}
=
A(t)x
+
B(t)u
\end{equation}
und der Anfangsbedinung $x(t_0)=x_0$ soll so gesteuert werden,
dass das quadratische Funktional
\begin{equation}
J(u)
=
\frac12 x(t_1)^t S_1 x(t_1)
+
\frac12
\int_{t_0}^{t_1}
x(t)^t Q(t) x(t)
+
u(t)^t R(t) u(t)
\,dt
\end{equation}
minimal wird.
\end{aufgabe}
Wir lösen das Problem hier nur für den Fall, dass die Matrizen $A$, $B$,
$Q$ und $R$ nicht von der Zeit abhängen.
In diesem Fall ist die Lagrange-Funktion
\[
L(x,u)
=
\frac12 x^t Qx + \frac12 u^t Ru
\]
dieses Funktionals ist von der Zeit unabhängig,
ebenso die Systemgleichung $f(x,u)=Ax+bu$ wir haben also tatsächlich ein
optimales Steuerungsproblem der Art, wie wir es studiert haben.

Zur Lösung dieses Problems muss erst die Steuerungs-Hamilton-Funktion
\[
H(x,u,p)
=
L(x,u) + p\cdot f(u,x)
=
\frac12 x^tQx + \frac12 u^tRu + p^tAx + p^tBu
\]
bestimmt werden.
Die kanonischen Differentialgleichungen sind jetzt
\begin{align*}
\dot{x}
&=
\frac{\partial H}{\partial p}
=
Ax
+
Bu
\\
\dot{p}
&=
-\frac{\partial H}{\partial x}
=
-Qx - p^tA.
\end{align*}
Die erste Gleichung ist natürlich keine Überraschung, dies ist die
Systemgleichung.

Der Randterm $K(x)$ ist ebenfalls quadratisch, es ist
\[
K(x) = \frac12 x^t S_1 x,
\]
mit der Ableitung
\begin{equation*}
\frac{\partial K}{\partial x}
=
S_1x,
\end{equation*}
wie sie später als Randbedingung
\[
p_*(t_1)
=
\frac{\partial K}{\partial x}(x_*(t_1))
=
S_1x_*(t_1)
\]
gebraucht wird.

%
% Optimale Steuerung für das linear-quadratische Problem
%
\subsubsection{Optimale Steuerung für das linear-quadratische Problem}
Da  Minimumprinzip besagt, dass $u$ jeweils so gewählt werden muss,
dass $H(x,u,p)$ bei gegebenem $x$ und $p$ das Minimum annimmt.
Da $H$ quadratisch ist in $u$, muss also nach Weglassen der konstanten
Terme in $H$ für $u$ das quadratische Minimalproblem
\[
f(u)
=
\frac12 u^t R u + p^t Bu
\to
\min
\]
gelöst werden.
Dies ist möglich, wenn $R$ positiv definit ist.
Die Ableitung nach $u$ ist
\[
0
=
\frac{\partial f}{\partial u}
=
Ru + B^tp
\qquad\Rightarrow\qquad
u
=
-
R^{-1}B^t p.
\]
Damit kann $u$ aus den Differentialgleichungen eliminiert werden, 
es bleiben die Gleichungen
\begin{equation}
\left.
\begin{aligned}
\dot{x}
&=
Ax-BR^{-1}B^t p
\\
\dot{p}
&=
-Qx -p^tA
\end{aligned}
\quad
\right\}
\qquad\Rightarrow\qquad
\frac{d}{dt}
\begin{pmatrix}
x\\
p
\end{pmatrix}
=
\begin{pmatrix}
 A & -BR^{-1}B^t \\
-Q & -A^t
\end{pmatrix}
\begin{pmatrix}
x\\
p
\end{pmatrix}.
\label{buch:hamiltonjacobi:oc:eqn:blockkanonisch}
\end{equation}

\begin{definition}[Hamilton-Matrix]
Die Matrix
\[
\mathscr{H}
=
\begin{pmatrix}
 A & -BR^{-1}B^t \\
-Q & -A^t
\end{pmatrix}
\]
heisst die {\em Hamilton-Matrix} für das linear-quadratische
Problem~\ref{buch:hamiltonjacobi:oc:aufgabe:lqc}.
\end{definition}

%
% Die Steuerungskosten
%
\subsubsection{Die Steuerungskosten}
Sei $x_*(t)$, $p_*(t)$ und $u_*(t)$ eine Lösung des Steuerungsproblems.
Wir möchten den optimalen Wert des Kostenfunktionals $J(u_*)$ berechnen.
Dazu berechnen wir die Ableitung
\begin{align*}
\frac12
\frac{d}{dt}(p_*(t)^t x_*(t))
&=
\dot{p}_*(t)^t x_*(t) + p_*(t)^t \dot{x}_*(t)
\intertext{mit Hilfe der kanonischen Differentialgleichungen}
&=
\frac12
(-Qx_*(t)-A^tp_*(t))^t x_*(t)
+
\frac12
p_*(t)^t (Ax_*(t)-BR^{-1}B^tp_*(t))
\\
&=
-
\frac12
x_*(t)^tQx_*(t)
-
\frac12
p_*(t)^t BR^{-1}B^t p_*(t)
\\
&=
-
\frac12
x_*(t)^tQx_*(t)
-
\frac12
\underbrace{p_*(t)^t BR^{-1}}_{\displaystyle=u_*(t)}
R
\underbrace{R^{-1}B^t p_*(t)}_{\displaystyle=u_*(t)^t}
\\
&=
-
\frac12
x_*(t)^tQx_*(t)
-
\frac12
u_*(t)^t Ru_*(t).
\end{align*}
Durch Integrieren wird die rechte Seite zum Integral im Funktional
$J(u_*)$:
\begin{align*}
\biggl[
\frac12 p_*(t)^t x_*(t)
\biggr]_{t_0}^{t_1}
&=
-\int_{t_0}^{t_1} \frac12 x_*(t)^t Qx_*(t) + \frac12 u_*(t)^t Ru_*(t)\,dt
\end{align*}
oder
\begin{align*}
\frac12 p_*(t_1)^tx_*(t_1)
+
\int_{t_0}^{t_1} \frac12 x_*(t)^t Qx_*(t) + \frac12 u_*(t)^t Ru_*(t)\,dt
&=
\frac12 p_*(t_0)^t x_*(t_0).
\intertext{Die Randbedingung $p_*(t_1)=S_1x_*(t_1)$ kann ersetzt werden
und ergibt}
\frac12
(S_1x_*(t_1))^t
x_*(t_1)
+
\int_{t_0}^{t_1} \frac12 x_*(t)^t Qx_*(t) + \frac12 u_*(t)^t Ru_*(t)\,dt
&=
J(u_*)
=
\frac12 p_*(t_0)^t x_*(t_0).
\end{align*}

\begin{satz}
\label{buch:hamiltonjacobi:oc:satz:optimale-kosten}
Das Kostenfunktional hat den optimalen Wert
\[
J(u_*)
=
\frac12 p_*(t_0)^t x_*(t_0).
\]
\end{satz}


%
% Randbedingungen
%
\subsubsection{Randbedingungen}
Die Anfangsbedingung für $x$ ist $x(t_0)=x_0$, die Randbedingung bei
$t_1$ für $p$ ist mit $p(t_1)=S_1x(t_1)$.
Es ist nicht klar, dass eine solche Lösung tatsächlich existiert.
Da die Matrizen $A$, $B$, $R$ und $Q$ konstant sind, kann die
Matrixdifferentialgleichung~\eqref{buch:hamiltonjacobi:oc:eqn:blockkanonisch}
mit Hilfe der Matrixexponentialfunktion gelöst werden.
Tatsächlich ist
\begin{equation}
\begin{pmatrix}
x(t)\\
p(t)
\end{pmatrix}
=
e^{\mathscr{H}t}
\begin{pmatrix}
x_0\\
p_(0)
\end{pmatrix}.
\label{buch:hamiltonjacobi:oc:eqn:kansol}
\end{equation}
Am Ende des Zeitintervalls muss die Randbedingung
\[
p(t_1) = S_1x(t_1)
\qquad\Leftrightarrow\qquad
Sx(t_1)
-
Ip(t_1)
=
0
\]
erfüllt sein.
Dies kann in Matrixform als
\begin{equation}
\begin{pmatrix}
S & -I
\end{pmatrix}
\begin{pmatrix}
x(t_1)\\
p(t_1)
\end{pmatrix}
=
0
\qquad\Rightarrow\qquad
\begin{pmatrix}
S & -I
\end{pmatrix}
e^{\mathscr{H}t_1}
\begin{pmatrix}
x_0\\
p(0)
\end{pmatrix}
=
0
\label{buch:hamiltonjacobi:oc:p0cond}
\end{equation}
geschrieben werden.
Dies ist ein lineares Gleichungssystem zur Bestimmung von $p(0)$.
Um es in die Standardform zu bringen, schreiben wir die $2n\times 2n$-Matrix
$e^{\mathscr{H}t_1}$ als Blockmatrix mit $n\times n$-Blöcken in der Form
\[
e^{\mathscr{H}t}
=
\Phi(t)
=
\begin{pmatrix}
\Phi_{11}(t) &
\Phi_{12}(t) \\
\Phi_{21}(t) &
\Phi_{22}(t) 
\end{pmatrix}.
\]
Einsetzen in \eqref{buch:hamiltonjacobi:oc:p0cond} ergibt
\[
\begin{pmatrix}
S & -I
\end{pmatrix}
e^{\mathscr{H}t_1}
=
\begin{pmatrix}
S\Phi_{11}(t_1)
-\Phi_{21}(t_1)
&
S\Phi_{12}(t_1)
-\Phi_{22}(t_1)
\end{pmatrix}.
\]
Die Gleichung für $p(0)$ wird damit
\[
(S\Phi_{11}(t_1)
-\Phi_{21}(t)_1)x_0
+
(S\Phi_{12}(t_1)
-\Phi_{22}(t_1))p(0)
=
0
\]
oder
\[
(S\Phi_{12}(t_1)
-\Phi_{22}(t_1))
p(0)
=
-
(S\Phi_{11}(t_1)
-\Phi_{21}(t_1))
x_0
\]
Sie hat sicher eine Lösung, wenn der Faktor $S\Phi_{12}(t_1)-\Phi_{22}(t_1)$
regulär ist, in diesem Fall ist
\[
p(0)
=
-
(S\Phi_{12}(t_1)
-\Phi_{22}(t_1))^{-1}
(S\Phi_{11}(t_1)
-\Phi_{21}(t_1))
x_0.
\]
Das Problem ist also leicht lösbar, wenn die Matrix
\begin{equation}
M
=
(S\Phi_{12}(t_1)
-\Phi_{22}(t_1))^{-1}
(S\Phi_{11}(t_1)
-\Phi_{21}(t_1))
\label{buch:hamiltonjacobi:oc:eqn:Mmatrix}
\end{equation}
wohldefiniert ist.
Es stellt sich heraus, dass dies unter den gegebenen Voraussetzungen
immer der Fall ist.

\begin{satz}
Falls $S_1$ und $Q$ positiv semidefinit sind und $R$ positiv definit,
dann ist die Matrix $M$ von
\eqref{buch:hamiltonjacobi:oc:eqn:Mmatrix}
wohldefiniert.
\end{satz}

\begin{proof}
XXX TODO
\end{proof}

Die Berechnung des Kostenfunktionals in
Satz~\ref{buch:hamiltonjacobi:oc:satz:optimale-kosten}
hat die Anfangsbedingung $p_*(0)$ benötigt, die jetzt mit der
Matrix $M$ berechnen kann.
Der Wert kann jetzt mit
\[
J(u_*)
=
\frac12 p_*(t_0)^t x_0
=
\frac12 (Mx_0)^t x_0
=
\frac12 x_0^t Mx_0
\]
als quadratische Funktion der Anfangsbedingung $x_0$ ausgedrückt werden.


%Da eine solche Beziehung für jeden denkbaren Endzeitpunkt $t$
%gelten muss, wird jetzt eine Matrix $S(t)$ gesucht derart,
%dass $p(t)=S(t)x(t)$ eine Lösung ist.
%Durch Einsetzen in die Differentialgleichung folgt
%\begin{align}
%\dot{x} &= (A - BR^{-1}B^t S)x
%\label{buch:hamiltonjacobi:oc:lqc:xdot}
%\\
%\dot{S}x+S\dot{x} &= -(Q+A^tS)x.
%\label{buch:hamiltonjacobi:oc:lqc:Sdot}
%\end{align}
%Durch Einsetzen von $\dot{x}$ aus der ersten Differentialgleichung
%in die zweite entsteht die Gleichung
%\[
%\dot{S}x
%=
%(
%-Q-A^tS  - SA+BR^{-1}B S
%)
%x.
%\]
%Da der Vektor $x$ durch Veränderung der Anfangsbedingung $x_0$ beliebige
%Werte annehmen kann, muss die Matrix $S(t)$ die Matrixdifferentialgleichung
%\begin{equation}
%\dot{S}
%=
%-SA -A^tS+SBR^{-1}B^tS-Q
%\label{buch:hamiltonjacobi:oc:eqn:riccati}
%\end{equation}
%mit der Randbedingung $S(t_1)=S_1$ erfüllen.
%Die Differentialgleichung
%\eqref{buch:hamiltonjacobi:oc:eqn:riccati}
%heisst die {\em Riccati-Differentialgleichung}.
%\index{Riccati-Differentialgleichung}%
%Sie kann im Allgemeinen nicht in geschlossener Form gelöst werden.
%Für konstante Matrizen ist eine Darstellung der Lösung mit Hilfe
%der Matrixexponentialfunktion möglich.
%
%Sobald $S(t)$ bestimmt ist, kann aus
%\eqref{buch:hamiltonjacobi:oc:lqc:xdot}
%durch durch Lösen der Differentialgleichung 
%\begin{align*}
%\dot{x} = (A-BR^{-1}B^tS(t)) x
%\end{align*}
%auch $x(t)$ bestimmt werden.
%Daraus ergeben sich schliesslich
%\[
%p(t)
%=
%S(t)x(t)
%\qquad
%\text{und schliesslich}
%\qquad
%u(t)
%=
%-R^{-1}B^tp(t)=-R^{-1}B^tS(t)x(t).
%\]
%Damit ist das optimale
%Steuerungsproblem~\ref{buch:hamiltonjacobi:oc:aufgabe:lqc}
%vollständig gelöst.

\begin{beispiel}
\label{buch:hamiltonjacobi:oc:bsp:eindim}
%
% eindim.tex -- optimale Regelung eines eindimensionalen Systems
%
% (c) 2021 Prof Dr Andreas Müller, OST Ostschweizer Fachhochschule
%
\documentclass[tikz]{standalone}
\usepackage{amsmath}
\usepackage{times}
\usepackage{txfonts}
\usepackage{pgfplots}
\usepackage{csvsimple}
\definecolor{darkred}{rgb}{0.8,0,0}
\definecolor{darkgreen}{rgb}{0,0.6,0}
\usetikzlibrary{arrows,intersections,math}
\begin{document}
\def\skala{1}
\input{paths.tex}
\begin{tikzpicture}[>=latex,thick,scale=\skala]

\def\dt{11}
\def\dx{4}
\def\dP{0.4}
\def\du{0.4}

\def\timeaxis{
	\draw[->] (-0.1,0) -- (11.5,0) coordinate[label={$t$}];
	\foreach \t in {0.1,0.2,...,1.0}{
		\draw ({\t*\dt},-0.05) -- ({\t*\dt},0.05);
	}
	\node at ({\dt},-0.05) [below] {$1\mathstrut$};
}

\begin{scope}
\draw[color=darkred,line width=1.2pt] \xpatheins;
\draw[color=blue,line width=1.2pt] \xpathzwei;
\draw[color=darkgreen,line width=1.2pt] \xpathdrei;
\draw[color=orange,line width=1.2pt] \xpathvier;
\draw[->] (0,-0.1) -- (0,4.4) coordinate[label={right:$x$}];
\timeaxis
\foreach \x in {0.1,0.2,...,1.0}{
	\draw (-0.05,{\x*\dx}) -- (0.05,{\x*\dx});
}
\node at (-0.05,{1.0*\dx}) [left] {$1.0\mathstrut$};
\node at (-0.05,{0.5*\dx}) [left] {$0.5\mathstrut$};
\node at (-0.05,{0.0*\dx}) [left] {$0.0\mathstrut$};
\end{scope}

\begin{scope}[yshift=-5.0cm]
\draw[color=darkred,line width=1.2pt] \ppatheins;
\draw[color=blue,line width=1.2pt] \ppathzwei;
\draw[color=darkgreen,line width=1.2pt] \ppathdrei;
\draw[color=orange,line width=1.2pt] \ppathvier;
\draw[->] (0,-0.1) -- (0,4.4) coordinate[label={right:$p$}];
\timeaxis
\foreach \p in {1,2,...,10}{
	\draw (-0.05,{\p*\dP}) -- (0.05,{\p*\dP});
}
\node at (-0.05,{0*\dP}) [left] {$0\mathstrut$};
\node at (-0.05,{5*\dP}) [left] {$5\mathstrut$};
\node at (-0.05,{10*\dP}) [left] {$10\mathstrut$};
\end{scope}

\begin{scope}[yshift=-6cm]
\draw[color=darkred,line width=1.2pt] \upatheins;
\draw[color=blue,line width=1.2pt] \upathzwei;
\draw[color=darkgreen,line width=1.2pt] \upathdrei;
\draw[color=orange,line width=1.2pt] \upathvier;
\draw[->] (0,-4.1) -- (0,0.4) coordinate[label={right:$u$}];
\timeaxis
\foreach \p in {1,2,...,10}{
	\draw (-0.05,{-\p*\dP}) -- (0.05,{-\p*\dP});
}
\node at (-0.05,{0*\dP}) [left] {$0\mathstrut$};
\node at (-0.05,{-5*\dP}) [left] {$-5\mathstrut$};
\node at (-0.05,{-10*\dP}) [left] {$-10\mathstrut$};
\end{scope}

\begin{scope}[yshift=-10.2cm]

\begin{scope}[xshift=2cm]
\color{darkred}
\parametereins
\end{scope}

\begin{scope}[xshift=4.5cm]
\color{blue}
\parameterzwei
\end{scope}

\begin{scope}[xshift=7cm]
\color{darkgreen}
\parameterdrei
\end{scope}

\begin{scope}[xshift=9.5cm]
\color{orange}
\parametervier
\end{scope}
\end{scope}

\end{tikzpicture}
\end{document}


Wir betrachten das eindimensionale optimale Steuerungsproblem mit
der Systemgleichung
\[
\dot x
=
ax + bu,
\qquad
x(0)=x_0
\]
und dem Funktional
\[
J(u)
=
\frac12 s_1x(t_1)^2
+
\frac12
\int_{t_0}^{t_1}
qx(t)^2 + ru(t)^2
\,dt.
\]
Die Hamilton-Matrix ist
\[
\mathscr{H}
=
\begin{pmatrix}
 a& -b^2/r \\
-q& -a
\end{pmatrix}
\]
mit dem charakteristischen Polynom
\[
\det (\mathscr{H}-\lambda I)
=
-(a-\lambda)(a+\lambda)-qb^2/r
=
-(a^2+b^2q/r) + \lambda^2 
\]
mit den Nullstellen
\[
\lambda_\pm
=
\pm \sqrt{a^2 +b^2q/r}.
\]
%Nach der oben erarbeiteten Lösung muss $r>0$ sein und es muss
%die Riccati-Differential\-gleichung
%\[
%\dot{s}
%=
%-q-as -sa + \frac{b^2}rs
%=
%-q-2as+\frac{b^2}rs^2
%\qquad\text{mit Randbedingung}\qquad
%s(t_1)=s_1
%\]
%gelöst werden.

Abbildung~\ref{buch:hamiltonjacobi:oc:bsp:eindim} zeigt, wie die Parameter
$q$, $r$ und $s_1$ die Konvergenzgeschwindigkeit der Steuerung beeinflussen.
\end{beispiel}
