%
% 2-jacobi.tex
%
% (c) 2024 Prof Dr Andreas Müller
%
\section{Jacobi-Theorie
\label{buch:hamiltonjacobi:section:jacobi}}
\kopfrechts{Jacobi-Theorie}
%
% kreis.tex -- template for standalon tikz images
%
% (c) 2021 Prof Dr Andreas Müller, OST Ostschweizer Fachhochschule
%
\documentclass[tikz]{standalone}
\usepackage{amsmath}
\usepackage{times}
\usepackage{txfonts}
\usepackage{pgfplots}
\usepackage{csvsimple}
\usetikzlibrary{arrows,intersections,math}
\definecolor{darkred}{rgb}{0.8,0,0}
\begin{document}
\def\skala{1}
\def\dx{4.5}
\def\dy{4.5}
\input{kreispfad.tex}
\begin{tikzpicture}[>=latex,thick,scale=\skala]

\draw[color=blue!40] \pfada;
\draw[color=blue!40] \pfadb;
\draw[color=blue!40] \pfadc;
\draw[color=blue!40] \pfadd;
\draw[color=blue!40] \pfade;
\draw[color=blue!40] \pfadf;
\draw[color=blue!40] \pfadg;
\draw[color=blue!40] \pfadh;
\draw[color=blue!40] \pfadi;
\draw[color=blue!40] \pfadj;
\draw[color=blue!40] \pfadk;
\draw[color=blue!40] \pfadl;
\draw[color=blue!40] \pfadm;
\draw[color=blue!40] \pfadn;
\draw[color=blue!40] \pfado;
\draw[color=blue!40] \pfadp;
\draw[color=blue!40] \pfadq;

\faechera
\faecherb
\faecherc
\faecherd
\faechere
\faecherf
\faecherg
\faecherh
\faecheri
\faecherj
\faecherk
\faecherl
\faecherm
\faechern
\faechero
\faecherp
\faecherq

\draw[color=darkred!40,line width=1.2pt] \kreisfaechera;
\draw[color=darkred!40,line width=1.2pt] \kreisfaecherb;
\draw[color=darkred!40,line width=1.2pt] \kreisfaecherc;
\draw[color=darkred!40,line width=1.2pt] \kreisfaecherd;
\draw[color=darkred!40,line width=1.2pt] \kreisfaechere;
\draw[color=darkred!40,line width=1.2pt] \kreisfaecherf;
\draw[color=darkred!40,line width=1.2pt] \kreisfaecherg;
\draw[color=darkred!40,line width=1.2pt] \kreisfaecherh;
\draw[color=darkred!40,line width=1.2pt] \kreisfaecheri;
\draw[color=darkred!40,line width=1.2pt] \kreisfaecherj;
\draw[color=darkred!40,line width=1.2pt] \kreisfaecherk;
\draw[color=darkred!40,line width=1.2pt] \kreisfaecherl;
\draw[color=darkred!40,line width=1.2pt] \kreisfaecherm;
\draw[color=darkred!40,line width=1.2pt] \kreisfaechern;
\draw[color=darkred!40,line width=1.2pt] \kreisfaechero;
\draw[color=darkred!40,line width=1.2pt] \kreisfaecherp;
\draw[color=darkred!40,line width=1.2pt] \kreisfaecherq;

\draw[color=darkred,line width=1.4pt] \kreiseins;
\draw[color=darkred,line width=1.4pt] \kreiszwei;

\draw[->,color=black] (-0.1,0) -- ({2.65*\dx},0) coordinate[label={$x$}];
\draw[->,color=black] (0,{-0.8*\dy}) -- (0,{1.40*\dy}) coordinate[label={right:$y$}];

\end{tikzpicture}
\end{document}


In diesem Abschnitt entwickeln wir eine geometrische Beschreibung des
Variationsproblems, die besondere Bedeutung der Hamilton-Funktion hervorhebt.
Wir gehen dazu aus von einem Funktional
\begin{equation}
J(y)
=
\int_{x_0}^{x_1}
F(x,y(x),y'(x))
\,dx
\label{buch:hamiltonjacobi:jacobi:eqn:funktionalJ}
\end{equation}
mit einer Lagrange-Funktion $F(x,y,y')$.
Als Beispiel wird jeweils die Lichtausbreitung in einem inhomogenen
Medium mit Brechungsindex $n(y) = 1+\nu y$ dienen, welches die
Lagrange-Funktion
\begin{equation}
F(x,y,y')
=
n(y) \sqrt{1+y^{\prime 2}}
\label{buch:hamiltonjacobi:jacobi:eqn:beispielF}
\end{equation}
verwendet.
Im Übrigen verwenden wir wieder die Bezeichnungen des vorangegangenen
Abschnitts.

%
% Quasi-Länge und Quasikreise
%
\subsection{Quasilänge und Quasikreise
\label{buch:hamiltonjacobi:jacobi:subsection:quasi}}
Der Wert des Funktionals $J(y)$ von
\eqref{buch:hamiltonjacobi:jacobi:eqn:funktionalJ}
kann als ein alternatives Mass für die Entfernung zwischen
zwei Punkten verwendet werden.
In Kapitel~\ref{chapter:geodaeten} wird das übliche Riemannsche
Funktional zur Berechnung der Länge einer Kurve vorgestellt, mit dem
man sowohl die Geodäten als die kürzesten Verbindungen wie auch
den geodätischen Abstand als den Wert des Funktionals definiert.
Die geometrische Sprechweise für das Funktional 
\eqref{buch:hamiltonjacobi:jacobi:eqn:funktionalJ}
verwendet werden.

%
% Quasilänge und $J$-Abstand
%
\subsubsection{Quasilänge und $J$-Abstand}
Sei $y(x)$ eine Funktion, deren Graph die Punkte $P_0=(x_0,y_0)$
und $P_1=(x_1,y_1)$ verbindet.
Dann nennen wir den Wert des Funktionals $J(y)$ den {\em $J$-Länge}
der verbindenden Kurve.
Wenn es unter allen Kurven, die die Punkte $P_0$  und $P_1$ verbinden,
auch eine Extremale $y(x)$ gibt, dann bezeichnen wir den extremalen
Wert des Funktionals $J(y)$ den $J$-Abstand der beiden Punkte.
Wir bezeichnen den $J$-Abstand auch mit $J(P_0,P_1)$.

%
% Quasikreise
%
\subsubsection{Quasikreise}
Wir halten den Punkt $P_0=(x_0,y_0)$ weiterhin fest und betrachten die
Menge
\[
K_J(P_0,\varrho)
=
\{
P_1=(x_1,y_1)\in\mathbb{R}^2
\mid
J(P_0,P_1) = \varrho
\}
\]
aller Punkte $P_1=(x_1,y_1)$, für die der $J$-Abstand zwischen
zu $P_0$ einen vorgegebenen Wert $\varrho$ annimmt.
$K_J(P_0,\varrho)$ heisst auch der {\em Quasikreis} oder {\em $J$-Kreis}
mit Zentrum $P_0$ und $J$-Radius oder Quasiradius $\varrho$.

\begin{beispiel}
In Abbildung~\ref{buch:hamiltonjacobi:hamiltonjacobi:kreis} ist der
$J$-Kreis um den Punkt $P_0=(0,0)$ mit Radius $1$ für das Funktional
mit der Lagrange-Funktion~\eqref{buch:hamiltonjacobi:jacobi:eqn:beispielF}
dargestellt.
Der hellgrüne Hintergrund symbolisiert die optische Dichte des Mediums,
in dem sich die Lichstrahlen ausbreiten.
Die Lichtgeschwindigkeit ist also im unteren Teil des Bildes grösser,
die Lichstrahlen krümmen sich daher nach oben.

In der gleichen Zeit können die Lichstrahlen im unteren Teil auch
eine grössere Strecke überwinden.
Die beiden eingezeichneten Quasikreise mit Radius 1 bzw.~2 sind 
daher vor allem dann nach rechts verzerrt, wenn die dort
endenden Extremalen, weit im negativen Bereich der Ebene verlaufen.
Die Extremalen mit grosser Steigung erreichen schnell Bereich höherer
optischer Dichte, wo Lichstrahlen mehr Zeit für die gleiche Distanz
brauchen.
Die Quasikreise nähern sich daher für grösser werdendes $y$ an.
\end{beispiel}

%
% Transversalitätsbedingung
%
\subsubsection{Transversalitätsbedingung}
Zu jedem Punkt $P_1\in K_J(P_0,\varrho)$ gibt es eine Lösung des
Anfangspunkt-Endpunkt-Problems mit dem Funktional
\eqref{buch:hamiltonjacobi:jacobi:eqn:funktionalJ}.
Jede solche Lösung ist aber auch eine Lösung des
Anfangspunkt-Endkurve-Problems mit Anfangspunkt $P_0$ und
Endkurve $K_J(P_0,\varrho)$.
Insbesondere erfüllt jede solche Extremale die Transversalitätsbedingung,
die man am einfachsten mit der Hamilton-Funktion schreiben kann.

\begin{beispiel}
\label{buch:hamiltonjacobi:jacobi:bsp:beispielFH}
Für die Lagrange-Funktion~\eqref{buch:hamiltonjacobi:jacobi:eqn:beispielF}
ist die konjugierte Koordinate
\[
p
=
\frac{\partial F}{\partial y}
=
n'(y)\sqrt{1+y^{\prime 2}}
=
\nu \sqrt{1+y^{\prime 2}}.
\]
Aufgelöst nach $y'$ ist dies
\begin{align*}
\frac{p}{\nu}
&=
\sqrt{1+y^{\prime 2}}
\\
y'
&=
\sqrt{ \frac{p^2}{\nu^2} -1 }
\end{align*}
Damit kann man jetzt die Hamilton-Funktion
\[
H(x,y,p)
=
y'p - F(x,y,y')
=
p\sqrt{\frac{p^2}{\nu^2}-1}
-
n(y) \frac{p}{\nu}
\]
durch $y$ und $p$ ausdrücken.
Damit sind die Koeffizienten für die Transfersalitätsbedingung
\[
\vec{f}_1
=
\begin{pmatrix} 
-H\\
p
\end{pmatrix}
\]
gefunden.
Der Vektor $\vec{f}_1$ steht auf der Quasikugel senkrecht.
\end{beispiel}

%
% Quasikugeln
%
\subsubsection{Quasikugeln}
Die Idee der Quasilänge und der Quasikugeln ist nicht auf eine abhängige
Variable beschränkt.
Für die Lagrange-Funktion 
$F\colon \mathbb{R}\times\mathbb{R}^n\times\mathbb{R}^n\to\mathbb{R}$
können wir das Funktional
\[
J(y) = \int_{x_0}^{x_1} F(x,y(x),y'(x))\,dx
\]
verwenden, um den $J$-Abstand oder Quasiabstand zwischen zwei Punkten
definieren.
Die $(n+1)$-dimensionale Quasikugel um den Punkt $P_0$ mit Radius $\varrho$
besteht dann aus den Punkten, die den $J$-Abstand $\varrho$ von $P_0$ haben.
Die Extremalen zwischen $P_0$ und einem Punkt $P_1$ auf der Quasikugel
erfüllen im Punkt $P_1$ die Transversalitätsbedingung, der Vektor
\[
\vec{f}_1
=
\begin{pmatrix}
-H(x,y,p)\\
p_1\\
\vdots\\
p_n
\end{pmatrix}
\]
steht auf der Quasikugel senkrecht.

%
% Die Fundamentalfunktion $S$
%
\subsection{Die Fundamentalfunktion $S$}
Wir betrachten jetzt ein Problem mit einer gegeben Anfangskurve,
die wir mit $\gamma_0$ bezeichnen.

%
% Konstruktion der Funktion $S$
%
\subsubsection{Konstruktion der Funktion $S$}
Zu jedem Punkt $P_1=(x_1,y_1)$ suchen wir eine Lösung des
Anfangskurve-Endpunkt-Problems mit der Anfangskurve $\gamma_0$
und dem Endpunkt $P_1$.
Die Lösung ist eine Funktion $y(x)$, die einen Punkt $P_0=(x_0,y_0)$ auf der
Kurve $\gamma_0$ mit dem Punkt $P_1$ verbindet.
Wir definieren den Wert der {\em Fundamentalfunktion} $S$ auf dem
\index{Fundamentalfunktion}%
Punkt $P_1$ als
\begin{equation}
S(x_1,y_1)
:=
J(y).
\label{buch:hamiltonjacobi:jacobi:eqn:Sdef}
\end{equation}
Die Funktion $S$ hat die Eigenschaft, dass $S(P)=0$ ist für alle
Punkte $P$ auf der Kurve $\gamma_0$.
Die Extremale $y(x)$ erfüllt im Punkt $P_0$ die Transversalitätsbedingung.

%
% Das huygenssche Prinzip
%
\subsubsection{Das huygenssche Prinzip}
\index{Prinzip von Huygens}%
\index{huygensches Prinzip}%
Die Menge 
\[
L(\varrho)
=
\{
P_1=(x_1,y_1)\in\mathbb{R}^2
\mid
S(P_1)=\varrho
\}
\]
besteht aus den Punkten, für die es eine Extremale der Quasilänge
$\varrho$ von der Anfangskurve $\gamma_0$ gibt.
Eine solche Extremale erfüllt die Transversalitätsbedingung für die
Anfangskurve im Anfangspunkt und die Transversalitätsbedingung
für die Kurve $L(\varrho)$ für den Endpunkt $P_1$.

Diese Konstruktion erinnert an das huygensche Prinzip der Wellenausbreitung.
Die Kurve $S(\varrho)$ ist die Wellenfront zur Zeit $t=\varrho$ einer Welle,
die sich zur Zeit $t=0$ von der Anfangskurve gelöst hat.
Sie entsteht als Einhüllende aller Quasikreise vom Radius $\varrho$ 
um Punkte der Anfangskurve $\gamma_0$.

\begin{beispiel}
In Abbildung~\ref{buch:hamiltonjacobi:hamiltonjacobi:kreis} sind
ausgehend von Punkten $P$ des Kreises $K_J((0,0),1)$ die Kreise
$K_J(P,1)$ hellrot dargestellt.
Sie berühren alle den roten Kreis $K_J((0,0),2)$.
\end{beispiel}

%
% Der Gradient von S
%
\subsubsection{Der Gradient von $S$}
Aus der Transversalitätsbedingung, die die Extremalen bei $P_1$ erfüllen
müssen folgt, dass die Normale auf die Kurve $L(\varrho)$ parallel
zum Vektor $\vec{f}_1$ sein muss.
Die Normale ist auch parallel zum Gradienten von $S$.
In diesem Abschnitt zeigen wir, dass der Gradient sogar mit dem 
Vektor $\vec{f}_1$ übereinstimmt.

Wir untersuchen, wie sich die der Wert $S(P)$ der Funktion $S$ ändert,
wenn man den Punkt $P$ verschiebt.
Sei als eine Variation der Extremalen $y(x)$ gegeben, die den Punkt $P_0$
mit $P$ verbindet.
Wir schreiben die Variation als Vektor
\[
\begin{pmatrix}
\delta x\\
\delta y
\end{pmatrix}
\]
Die allgemeine Variationsformel, die wir in 
Abschnitt~\ref{buch:hamiltonjacobi:section:kanonisch}
in der kanoninschen Variable und der Hamilton-Funktion ausgedrückt
haben, liefert für diese Variation
\[
\delta J
=
\Bigl[ -H\delta x + p \delta y \Bigr]_{x_0}^{x_1}
+
\int_{x_0}^{x_1}
\biggl(
\frac{\partial F}{\partial y}(x,y(x),y'(x))
-
\frac{d}{dx}
\frac{\partial F}{\partial y'}(x,y(x),y'(x))
\biggr)
\delta y
\,dx.
\]
Da $y(x)$ eine Extremale ist, trägt das Integral nichts zur
Variation bei.
Die Änderung von $S$ ist nach Definition die Variation $\delta J$,
wir können daher
\[
\delta S
=
-H\delta x(x_1) + p\delta y(x_1)
\]
schreiben.
Die Koeffizienten von $\delta x(x_1)$ und $\delta y(x_1)$ sind die
partiellen Ableitungen von $S$ nach den Variablen $x$ und $y$, also
gilt
\[
\operatorname{grad} S
=
\begin{pmatrix}
-H\\
p
\end{pmatrix}.
\]

%
% Hamilton-Jacobi-Differentialgleichung
%
\subsection{Hamilton-Jacobi-Differentialgleichung
\label{buch:hamiltonjacobi:jacobi:subsection:HJ-DGL}}
Die Fundamtenalfunktion $S(x,y)$ hat die partiellen
Ableitungen
\[
\frac{\partial S}{\partial x}
=
-H(x,y,p)
\qquad\text{und}\qquad
\frac{\partial S}{\partial y}
=
p.
\]
Durch Eliminieren von $p$ entsteht die partielle Differentialgleichung
\begin{equation}
\frac{\partial S}{\partial x}
+
H\biggl(x,y,\frac{\partial S}{\partial y}\biggr).
\label{buch:hamiltonjacobi:jacobi:eqn:hamilton-jacobi-dgl}
\end{equation}
Dies ist im Allgemeinen eine nichtlineare Differentialgleichung
erster Ordnung.

\begin{beispiel}
\label{buch:hamiltonjacobi:jacobi:bsp:eikonal}
Wir betrachten das Funktional mit der Lagrange-Funktion 
$L=\sqrt{1+y^{\prime 2}}$.
Die zugehörige Hamilton-Funktion wurde bereits in
Beispiel~\ref{buch:hamiltonjacobi:section:kanonisch}
bestimmt, sie war
\[
H(x,y,p)
=
-\sqrt{1-p^2}.
\]
Die Hamilton-Jacobi-Differentialgleichung ist daher
\begin{align}
\frac{\partial S}{\partial x}
+
H\biggl(x,y,\frac{\partial S}{\partial y}\biggr)
&=
0
\notag
\\
\frac{\partial S}{\partial x}
&=
\sqrt{1-\biggl(\frac{\partial S}{\partial y}\biggr)^2}.
\notag
\intertext{Durch Quadrieren können alle Ableitungen von $S$ auf die
linke Seite gebracht werden, es bleibt die Differentialgleichung}
\biggr(\frac{\partial S}{\partial x}\biggr)^2
+
\biggr(\frac{\partial S}{\partial y}\biggr)^2
&=
1.
\label{buch:hamiltonjacobi:jacobi:eqn:eikonal}
\end{align}
Diese Gleichung heisst auch {\em Eikonalgleichung}.
\index{Eikonalgleichung}%
\end{beispiel}

\begin{beispiel}
In Beispiel~\ref{buch:hamiltonjacobi:jacobi:bsp:beispielFH} wurde
die Hamilton-Funktion für das Funktional mit der Lagrange-Funktion
\eqref{buch:hamiltonjacobi:jacobi:eqn:beispielF} bereits berechnet.
Um die Hamilton-Jacobi-Differentialgleichung zu erhalten, müssen
wir in
\[
\frac{\partial S}{\partial x}
=
\frac{1}{\nu}p\sqrt{p^2-\nu^2} -\frac{n(y)}{\nu} p
\]
$p$ durch $\partial S/\partial y$ ersetzen. 
%Bevor wir dies tun, quadrieren wir die Gleichung und formen die
%rechte Seite um:
%\begin{align*}
%\biggl(
%\frac{\partial S}{\partial x}
%\biggr)^2
%=
%\frac{1}{\nu^2} p^2 (p^2-\nu^2)
%-2\frac{n(y)}{\nu^2}p^2\sqrt{p^2-\nu^2}
%+
%\frac{n(y)^2}{\nu^2}p^2
%\end{align*}
%Wir können daher die Hamilton-Jacobi-Differentialgleichung aufstellen,
%indem wir $p$ durch $\partial S/\partial y$ ersetzen:
und erhalten
\begin{align*}
\frac{\partial S}{\partial x}
&=
\frac{1}{\nu}
\frac{\partial S}{\partial y}
\sqrt{
\bigg(\frac{\partial S}{\partial y}\biggr)^2
-
\nu^2
}
-
\frac{n(y)}{\nu}
\frac{\partial S}{\partial y}
\end{align*}
als Hamilton-Jacobi-Differentialgleichung.
\end{beispiel}

Um die Lösung der Hamilton-Jacobi-Differentialgleichung eindeutig
festzulegen, müssen noch Randbedingungen spezifiziert werden.
Die ausgehend von der Anfangskurve $\gamma_0$ konstruiert Funktion
$S(x,y)$ erfüllt zusätzlich die homogene Dirichlet Randbedingung
$S(P)=0$ für $P\in\gamma_0$.
Umgekehrt ist eine
beliebige Lösung der Hamilton-Jacobi-Diffe\-ren\-tial\-gleichung
auch die Fundamentalfunktion für die Anfangskurve ist, die aus den
Punkten $P$ mit $S(P)=0$ besteht.

%
% Lösungen der kanonischen Differentialgleichungen
%
\subsection{Lösungen der kanonischen Differentialgleichungen}
Die Extremalen sind Lösungen der kanonischen Differentialgleichungen
\begin{equation}
\begin{aligned}
\frac{dy}{dx}&=\phantom{-}\frac{\partial H}{\partial p}
\\
\frac{dp}{dx}&=-\frac{\partial H}{\partial y}.
\end{aligned}
\label{buch:hamiltonjacobi:jacobi:eqn:kanonisch}
\end{equation}
Im Punkt $(x_0,y_0)$ sei $\vec{r}_0$ der Tangentialvektor an die
Anfangskurve.
Wählt man $p_0$ so, dass die Transversalitätsbedingung
\[
\vec{r}_0
\cdot
\begin{pmatrix}
-H(x_0,y_0,p_0)\\
p_0
\end{pmatrix}
=
0
\]
erfüllt ist, dann ist die Lösung von
\eqref{buch:hamiltonjacobi:jacobi:eqn:kanonisch}
mit Anfangsbedingung $y(x_0)=y_0$ und $p(x_0)=p_0$ eine Extremale.
Der Gradient der Fundamentalfunktion ist durch $-H$ und $p$ gegeben,
ihr Wert entlang der Lösungskurve wird daher durch das Integral
\begin{align*}
S(x_1,y_1)
&=
\int_{x_0}^{x_1}
-H(x,y(x),p(x))
+
p(x) y'(x)
\,dx
\\
&=
\int_{x_0}^{x_1}
-H(x,y(x),p(x))
+
p(x)\frac{\partial H}{\partial p}
\,dx
\end{align*}
gegeben.
Aus einer Lösung des Differentialgleichungssystems
\eqref{buch:hamiltonjacobi:jacobi:eqn:kanonisch}
lässt sich also immer auch die Fundamentalfunktion berechnen.

%
% Der Satz von Jacobi
%
\subsubsection{Der Satz von Jacobi}
Wir untersuchen jetzt die umgekehrt Schlussrichtung: Kann man aus
der Fundamentalfunktion $S$ eine Lösung der kanonischen Gleichungen 
ableiten.
Sei also $S$ eine Lösung der Differentialgleichung
\begin{equation}
\frac{\partial S}{\partial x}
=
H\biggl(x,y,\frac{\partial S}{\partial y}\biggr).
\label{buch:hamiltonjacobi:jacobi:hjdgl1}
\end{equation}
In der Konstruktion der Fundamentalfunktion aus den kanonischen Gleichungen
musste erst ein zusätzlicher Anfangswert $p_0$ gefunden werden,
der von der Steigung der Anfangskurve abhing.
Die Lösung $S$ der
Differentialgleichung~\eqref{buch:hamiltonjacobi:jacobi:hjdgl1}
soll daher zusätzlich von einem Parameter $a$ abhängen, soll also
die Form
\(
S(x,y,a)
\)
haben.
Eine solche Lösung heisst ein {\em vollständiges Integral} von
\label{buch:hamiltonjacobi:jacobi:hjdgl1}.

Die Gleichung
\begin{align}
\frac{\partial S}{\partial a}(x,y,a)
&=
b
\label{buch:hamiltonjacobi:jacobi:eqn:Sa1}
\intertext{definiert implizit eine Funktion $y(x)$, aus der mit Gleichung}
\frac{\partial S}{\partial y}(x,y(x),a)
= 
p(x)
\label{buch:hamiltonjacobi:jacobi:eqn:Sa2}
\end{align}
bestimmt werden kann.
Die implizite Definition \eqref{buch:hamiltonjacobi:jacobi:eqn:Sa1}
definiert die Funktion $y(x)$ nur dann, wenn die Ableitung der linken
Seite nach $y$ nicht verschwindet. 
Wir müssen also zusätzlich von der Funktion $S$ die Bedingung
\begin{equation}
\frac{\partial^2 S}{\partial y\,\partial a}\ne 0
\label{buch:hamiltonjacobi:jacobi:invertierungsbedingung}
\end{equation}
verlangen.

Wir prüfen nach, dass die Funktion $y(x)$ und $p(x)$ eine Lösung der
kanonischen Differentialgleichungen
\eqref{buch:hamiltonjacobi:jacobi:hjdgl1}
sind.
Da die Gleichung
\label{buch:hamiltonjacobi:jacobi:eqn:Sa1}
die Funktion implizit definiert, leiten wir die Gleichung
\eqref{buch:hamiltonjacobi:jacobi:eqn:Sa1}
nach $x$ und die Hamilton-Jacobi-Differentialgleichung
\eqref{buch:hamiltonjacobi:jacobi:hjdgl1}
nach $a$ ab und erhalten
\begin{align*}
\frac{\partial^2 S}{\partial x\,\partial a}
+
\frac{\partial^2 S}{\partial y\,\partial a}
\frac{dy}{dx}
&=0
\\
\frac{\partial^2 S}{\partial a\,\partial x}
+
\frac{\partial H}{\partial p}\frac{\partial^2 S}{\partial a\,\partial y}
&=0.
\intertext{Die Differenz dieser Differentialgleichungen ist}
\frac{\partial^2 S}{\partial y\,\partial a}
\biggl(\frac{dy}{dx}-\frac{\partial H}{\partial p}\biggr)
&=
0.
\end{align*}
Wegen der Bedingung
\eqref{buch:hamiltonjacobi:jacobi:invertierungsbedingung} können wir
schliessen, dass die Klammer verschwindet, dass also die erste
der kanonischen Differentialgleichungen erfüllt ist.

Ähnlich erhält man durch Ableitung von
\eqref{buch:hamiltonjacobi:jacobi:eqn:Sa2} nach $x$ und
der Hamilton-Jacobi-Differen\-tial\-gleichung nach $y$ die Gleichungen
\begin{align*}
\frac{dp}{dx}
&=
\frac{\partial^2 S}{\partial x\,\partial y}
+
\frac{\partial^2 S}{\partial y^2} \frac{dy}{dx}
\\
0
&=
\frac{\partial^2 S}{\partial y\,\partial x}
+
\frac{\partial H}{\partial y}
+
\frac{\partial H}{\partial p}\frac{\partial^2 S}{\partial y^2}
\intertext{mit der Differenz}
\frac{dp}{dx}
&=
\frac{\partial^2 S}{\partial y^2}
\biggl(
\frac{dy}{dx}-\frac{\partial H}{\partial p}
\biggr)
-
\frac{\partial H}{\partial y}.
\intertext{Wir wissen bereits, dass die Klammer verschwindet, es bleibt}
\frac{dy}{dx}
&=
-\frac{\partial H}{\partial y},
\end{align*}
die zweite der kanonischen Differentialgleichungen ist also auch erfüllt.
Damit haben wir den einfachsten Fall des folgenden Satzes von Jacobi
bewiesen.

\begin{satz}[Jacobi]
Ist die Funktion $S(x,y,a)$ ein vollständiges Integral der
Hamilton-Jacobi-Differentialgleichung und ist
\begin{equation}
\frac{\partial^2 S}{\partial a\,\partial y}\ne 0,
\label{buch:hamiltonjacobi:jacobi:satz:jacobi:eqn:regularitaet}
\end{equation}
dann definieren die Gleichungen
\begin{equation}
\begin{aligned}
\frac{\partial S}{\partial a}(x,y(x),a) &= b \\
\frac{\partial S}{\partial y}(x,y(x),a) &= p(x)
\end{aligned}
\label{buch:hamiltonjacobi:jacobi:satz:jacobi:eqn:equations}
\end{equation}
die allgemeine Lösung der kanonischen Differentialgleichungen, die von
den zwei Konstanten $a$ und $b$ abhängt.
\end{satz}

\begin{beispiel}
In Beispiel~\ref{buch:hamiltonjacobi:jacobi:bsp:eikonal} wurde als
Hamilton-Jacobi-Differentialgleichung für die Lagrange-Funktion
$L(x,y,y')=\sqrt{1+y^{\prime 2}}$ für den kürzesten Abstand in der
Ebene bereits die Eikonal-Gleichung
\[
\biggl(\frac{\partial S}{\partial x}\biggr)^2
+
\biggl(\frac{\partial S}{\partial y}\biggr)^2
=
1
\]
gefunden.
Für die Lösung versuchen wir eine Separationsansatz der Form
\[
S(x,y) = X(x) + Y(y),
\]
setzen ihn in die Eikonal-Gleichung ein und erhalten
\[
X'(x)^2 + Y'(y)^2 = 1
\qquad\Rightarrow\qquad
X'(x)^2 = 1 - Y'(y)^2.
\]
Da die linke Seite nur von $x$, die rechte aber nur von $y$ abhängt,
müssen beide Seiten konstant sein.
Wenn aber $X'(x)$ und $Y'(x)$ konstant sind, dann sind $X(x)$ und $Y(y)$
lineare Funktionen.
Somit muss
\[
S(x,y) = k_xx + k_yy + c
\]
sein.
Der konstante Term $c$ ist nicht relevant, aber auch $k_x$ und $k_y$ sind
nicht beliebig.
Einsetzen in die Eikonal-Gleichung ergibt
\[
\frac{\partial S}{\partial x} = k_x
\,\;;
\frac{\partial S}{\partial y} = k_y
\quad\Rightarrow\quad
\biggl(\frac{\partial S}{\partial x}\biggr)^2
+
\biggl(\frac{\partial S}{\partial y}\biggr)^2
=
k_x^2 + k_y^2 = 1
\quad\Rightarrow\quad
k_x = \sqrt{1-k_y^2}.
\]
Wir können somit $k_y$ als Parameter $a$ wählen und
\[
S(x,y,a)
=
\sqrt{1-a^2}x+ay
\]
schreiben.

Wir müssen die Bedingung
\eqref{buch:hamiltonjacobi:jacobi:satz:jacobi:eqn:regularitaet}
überprüfen:
\[
\frac{\partial S}{\partial y}
=
a,
\quad
\frac{\partial^2 S}{\partial a\,\partial y}
=
1
>
0,
\]
die Regularitätsbedingung ist also erfüllt und $S(x,y,a)$ ist ein
vollständiges Integral der Hamilton-Jacobi-Differentialgleichung.
Der Parameter $a$ muss im Intervall $[-1,1]$ liegen.

Nach dem Satz von Jacobi können jetzt Lösungen durch die Gleichungen
\eqref{buch:hamiltonjacobi:jacobi:satz:jacobi:eqn:equations}
definiert werden.
Die Ableitung von $S$ nach $a$ ist
\begin{align*}
\frac{\partial S}{\partial a}
&=
-
\frac{a}{\sqrt{1-a^2}}
x
+
y
=
b
\\
\frac{\partial S}{\partial y}
&=
a
=
p
\end{align*}
Die erste Gleichung kann nach $y$ aufgelöst werden und ergibt
\[
y
=
\frac{a}{\sqrt{1-a^2}} x + b.
\]
Dies ist eine Gerade mit Steigung 
\[
\frac{a}{\sqrt{1-a^2}}
=
\tan\arcsin a
\]
und Achsabschnitt $b$.
Der Parameter $a$ hat also die Bedeutung des Sinus des Winkels, den die
Gerade mit der $x$-Achse einschliesst.

Die kanonischen Differentialgleichungen sind
\begin{align*}
\frac{dp}{dx}
&=
0
=
\frac{\partial H}{\partial y}
\\
\frac{dy}{dx}
&=
\frac{a}{\sqrt{1-a^2}}
=
\frac{p}{\sqrt{1-p^2}}
=
-\frac{\partial H}{\partial p}(x,y(x),a),
\end{align*}
sie sind also tatsächlich erfüllt.
\end{beispiel}

%
% Der Satz on Jacobi für das $n$-dimensionale Problem
%
\subsubsection{Der Satz von Jacobi für das $n$-dimensionale Problem}
Die Funktion $S(x,y,a)$ mit $a\in\mathbb{R}^n$ heisst ein
vollständiges Integral für die Hamilton-Jacobi-Differentialgleichung
\begin{equation}
\frac{\partial S}{\partial x}
+
H\biggl(x,y
\frac{\partial S}{\partial y}
\biggr),
\label{buch:hamiltonjacobi:jacobi:eqn:hjn}
\end{equation}
wenn $(x,y)\mapsto S(x,y,a)$ ein Lösung 
\eqref{buch:hamiltonjacobi:jacobi:eqn:hjn} ist.
Sei ausserdem die Matrix mit den Einträgen
\begin{equation}
\frac{\partial^2 S}{\partial y_i\,\partial a_k}
\label{buch:hamiltonjacobi:jacobi:eqn:regulaer}
\end{equation}
regulär.
Dann gilt der folgende Satz.

\begin{satz}[Jacobi]
Ist $S$ ein vollständiges Integral der Hamilton-Jacobi-Differentialgleichung
\eqref{buch:hamiltonjacobi:jacobi:eqn:hjn}, welches die Regularitätsbedingung
\eqref{buch:hamiltonjacobi:jacobi:eqn:regulaer} erfüllt,
dann definieren die Gleichungen
\begin{align}
\frac{\partial S}{\partial a}(x,y(x),a) &= b
\label{buch:hamiltonjacobi:jacobi:eqn:implizit1}
\\
\frac{\partial S}{\partial y}(x,y(x),a) &= p(x)
\label{buch:hamiltonjacobi:jacobi:eqn:implizit2}
\end{align}
für jedes $b\in\mathbb{R}^n$ zwei Funktionen $y(x)$ und $p(x)$, die
Lösungen der kanonischen Differentialgleichungen
\begin{align*}
\frac{dy}{dx} &= \phantom{-}\frac{\partial H}{\partial p}
\\
\frac{dp}{dx} &= -\frac{\partial H}{\partial y}
\end{align*}
sind.
\end{satz}

\begin{proof}
Der Beweis folgt genau der Rechnung für den eindimensionalen Fall.
Die Ableitung der Gleichung
\eqref{buch:hamiltonjacobi:jacobi:eqn:implizit1}
nach $y$ und Ableitung der Hamilton-Jacobi-Differentialgleichung
nach $a$ ergeben die Gleichungen
\begin{align*}
\frac{\partial^2 S}{\partial x\,\partial a_k}
+
\sum_{i=1}^n
\frac{\partial^2 S}{\partial y_i \partial a_k}
\frac{dy_i}{dx}
&= 0
\\
\frac{\partial^2 S}{\partial a_k\,\partial x}
+
\sum_{i=1}^n
\frac{\partial H}{\partial p_i}
\frac{\partial^2 S}{\partial a_k\partial y_i}
&=0
\intertext{mit der Differenz}
\frac{\partial^2 S}{\partial y_i\,\partial a_k}
\biggl(
\frac{dy_i}{dx}-\frac{\partial H}{\partial p_i}
\biggr)
&=0.
\end{align*}
Dank der Bedingung
\eqref{buch:hamiltonjacobi:jacobi:eqn:regulaer}
muss die Klammer verschwinden, die ersten kanonischen Differentialgleichungen
sind damit erfüllt.

Die Ableitung von
\eqref{buch:hamiltonjacobi:jacobi:eqn:implizit2}
nach $x$
und der Hamilton-Jacobi-Differentialgleichung nach $y$
erhalten wir
\begin{align*}
\frac{dp_k}{dx}
&=
\frac{\partial^2 S}{\partial x\,\partial y_k}
+
\sum_{i=1}^n
\frac{\partial^2 S}{\partial y_i\,\partial y_k}\frac{dy_i}{dx}
\\
0
&=
\frac{\partial^2 S}{\partial y_k\,\partial x}
+
\sum_{i=1}^n
\frac{\partial H}{\partial p_i}
\frac{\partial^2 S}{\partial y_k\,\partial y_i}
+
\frac{\partial H}{\partial y_k}
\intertext{mit der Differenz}
\frac{dp_k}{dx}
&=
\sum_{i=1}^n
\frac{\partial^2 S}{\partial y_i\,\partial y_k}
\biggl(
\underbrace{
\frac{dy_i}{dx}
-
\frac{\partial H}{\partial p_i}
}_{\displaystyle=0}
\biggr)
-
\frac{\partial H}{\partial y_k}.
\end{align*}
Es bleibt die zweite kanonische Differentialgleichung.
\end{proof}
