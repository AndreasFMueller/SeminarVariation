%
% oc.tex -- einfache OC Beispiele
%
% (c) 2021 Prof Dr Andreas Müller, OST Ostschweizer Fachhochschule
%
\documentclass[tikz]{standalone}
\usepackage{amsmath}
\usepackage{times}
\usepackage{txfonts}
\usepackage{pgfplots}
\usepackage{csvsimple}
\usetikzlibrary{arrows,intersections,math}
\definecolor{darkred}{rgb}{0.8,0,0}
\definecolor{darkgreen}{rgb}{0,0.6,0}
\begin{document}
\def\skala{1}
\begin{tikzpicture}[>=latex,thick,scale=\skala]

\clip (-3.8,3.0) rectangle (8.8,-9.8);

\begin{scope}[xshift=-3.2cm]

\begin{scope}
\draw[color=darkred,line width=1.4pt] (0,2.5) -- (5,0);
\node[color=darkred] at (2.5,1.25) [above right] {$p_*(t)$};
\draw[->] (0,-0.1) -- (0,2.8) coordinate[label={right:$p$}];
\draw[->] (-0.1,0) -- (5.4,0) coordinate[label={$t$}];
\draw (5,-0.05) -- (5,0.05);
\draw (-0.05,2.5) -- (0.05,2.5);
\node at (-0.05,2.5) [left] {$1$};
\node at (-0.05,0.0) [left] {$0$};
\draw[line width=0.3pt] (5,2.6) -- +(0,-11.9);
\end{scope}

\begin{scope}[yshift=-3.3cm]
\draw[color=darkred,line width=1.4pt] (0,-2.5) -- (5,-2.5);
\node[color=darkred] at (2.5,-2.5) [above] {$u_*(t)$};
\draw[->] (0,-2.6) -- (0,2.8) coordinate[label={right:$u$}];
\draw[->] (-0.1,0) -- (5.4,0) coordinate[label={$t$}];
\draw (5,-0.05) -- (5,0.05);
\draw (-0.05,2.5) -- (0.05,2.5);
\node at (-0.05,2.5) [left] {$1$};
\draw (-0.05,-2.5) -- (0.05,-2.5);
\node at (-0.05,-2.5) [left] {$-1$};
\node at (-0.05,0.0) [left] {$0$};
\end{scope}

\begin{scope}[yshift=-8.2cm]
\draw[color=darkred,line width=1.4pt] (0,1.5) -- (5,-1);
\node[color=darkred] at (2.5,0.25) [above right] {$x_*(t)$};
\draw[->] (0,-1.1) -- (0,1.9) coordinate[label={right:$x$}];
\draw[->] (-0.1,0) -- (5.4,0) coordinate[label={$t$}];
\draw (5,-0.05) -- (5,0.05);
\node at (-0.05,0.0) [left] {$0$};
\draw (-0.05,1.5) -- (0.05,1.5);
\node at (-0.05,1.5) [left] {$x_0$};
\end{scope}

\node at (2.5,-9.6) {(a)};

\end{scope}

\begin{scope}[xshift=3.2cm]
\fill[color=blue!10,opacity=0.5] (0,2.6) rectangle (2.5,-9.3);
\fill[color=darkgreen!10,opacity=0.5] (2.5,2.6) rectangle (5,-9.3);

\begin{scope}[yshift=1.25cm]
\draw[color=darkred,line width=1.4pt] (0,1.25) -- (5,-1.25);
\node[color=darkred] at (2.5,0) [above right] {$p_*(t)$};
\draw[->] (0,-1.35) -- (0,1.55) coordinate[label={right:$p$}];
\draw[->] (-0.1,0) -- (5.4,0) coordinate[label={$t$}];
\draw (5,-0.05) -- (5,0.05);
\draw (-0.05,1.25) -- (0.05,1.25);
\node at (-0.05,1.25) [left] {$\frac12$};
\draw (-0.05,-1.25) -- (0.05,-1.25);
\node at (-0.05,-1.25) [left] {$-\frac12$};
\node at (-0.05,0.0) [left] {$0$};
\draw[line width=0.3pt] (5,1.35) -- +(0,-11.9);
\end{scope}

\begin{scope}[yshift=-3.3cm]
\draw[color=darkred,line width=1.4pt] (0,-2.5) -- (2.5,-2.5);
\draw[color=darkred,line width=0.3pt] (2.5,-2.5) -- (2.5,2.5);
\draw[color=darkred,line width=1.4pt] (2.5,2.5) -- (5,2.5);
\node[color=darkred] at (1.25,-2.5) [above] {$u_*(t)$};
\node[color=darkred] at (3.75,2.5) [below] {$u_*(t)$};
\draw[->] (0,-2.6) -- (0,2.8) coordinate[label={right:$u$}];
\draw[->] (-0.1,0) -- (5.4,0) coordinate[label={$t$}];
\draw (5,-0.05) -- (5,0.05);
\draw (-0.05,2.5) -- (0.05,2.5);
\node at (-0.05,2.5) [left] {$1$};
\draw (-0.05,-2.5) -- (0.05,-2.5);
\node at (-0.05,-2.5) [left] {$-1$};
\node at (-0.05,0.0) [left] {$0$};
\end{scope}

\begin{scope}[yshift=-8.2cm]
\draw[color=darkred,line width=1.4pt] (0,1.5) -- (2.5,0.25) -- (5,1.5);
\node[color=darkred] at (2.5,0.35) [above] {$x_*(t)$};
\draw[->] (0,-1.1) -- (0,1.9) coordinate[label={right:$x$}];
\draw[->] (-0.1,0) -- (5.4,0) coordinate[label={$t$}];
\draw (5,-0.05) -- (5,0.05);
\node at (-0.05,0.0) [left] {$0$};
\draw (-0.05,1.5) -- (0.05,1.5);
\node at (-0.05,1.5) [left] {$x_0$};
\end{scope}

\node at (2.5,-9.6) {(b)};

\end{scope}

\end{tikzpicture}
\end{document}

