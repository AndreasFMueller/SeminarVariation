%
% eindim.tex -- optimale Regelung eines eindimensionalen Systems
%
% (c) 2021 Prof Dr Andreas Müller, OST Ostschweizer Fachhochschule
%
\documentclass[tikz]{standalone}
\usepackage{amsmath}
\usepackage{times}
\usepackage{txfonts}
\usepackage{pgfplots}
\usepackage{csvsimple}
\definecolor{darkred}{rgb}{0.8,0,0}
\definecolor{darkgreen}{rgb}{0,0.6,0}
\usetikzlibrary{arrows,intersections,math}
\begin{document}
\def\skala{1}
\input{paths.tex}
\begin{tikzpicture}[>=latex,thick,scale=\skala]

\def\dt{11}
\def\dx{4}
\def\dP{0.4}
\def\du{0.4}

\def\timeaxis{
	\draw[->] (-0.1,0) -- (11.5,0) coordinate[label={$t$}];
	\foreach \t in {0.1,0.2,...,1.0}{
		\draw ({\t*\dt},-0.05) -- ({\t*\dt},0.05);
	}
	\node at ({\dt},-0.05) [below] {$1\mathstrut$};
}

\begin{scope}
\draw[color=darkred,line width=1.2pt] \xpatheins;
\draw[color=blue,line width=1.2pt] \xpathzwei;
\draw[color=darkgreen,line width=1.2pt] \xpathdrei;
\draw[color=orange,line width=1.2pt] \xpathvier;
\draw[->] (0,-0.1) -- (0,4.4) coordinate[label={right:$x$}];
\timeaxis
\foreach \x in {0.1,0.2,...,1.0}{
	\draw (-0.05,{\x*\dx}) -- (0.05,{\x*\dx});
}
\node at (-0.05,{1.0*\dx}) [left] {$1.0\mathstrut$};
\node at (-0.05,{0.5*\dx}) [left] {$0.5\mathstrut$};
\node at (-0.05,{0.0*\dx}) [left] {$0.0\mathstrut$};
\end{scope}

\begin{scope}[yshift=-5.0cm]
\draw[color=darkred,line width=1.2pt] \ppatheins;
\draw[color=blue,line width=1.2pt] \ppathzwei;
\draw[color=darkgreen,line width=1.2pt] \ppathdrei;
\draw[color=orange,line width=1.2pt] \ppathvier;
\draw[->] (0,-0.1) -- (0,4.4) coordinate[label={right:$p$}];
\timeaxis
\foreach \p in {1,2,...,10}{
	\draw (-0.05,{\p*\dP}) -- (0.05,{\p*\dP});
}
\node at (-0.05,{0*\dP}) [left] {$0\mathstrut$};
\node at (-0.05,{5*\dP}) [left] {$5\mathstrut$};
\node at (-0.05,{10*\dP}) [left] {$10\mathstrut$};
\end{scope}

\begin{scope}[yshift=-6cm]
\draw[color=darkred,line width=1.2pt] \upatheins;
\draw[color=blue,line width=1.2pt] \upathzwei;
\draw[color=darkgreen,line width=1.2pt] \upathdrei;
\draw[color=orange,line width=1.2pt] \upathvier;
\draw[->] (0,-4.1) -- (0,0.4) coordinate[label={right:$u$}];
\timeaxis
\foreach \p in {1,2,...,10}{
	\draw (-0.05,{-\p*\dP}) -- (0.05,{-\p*\dP});
}
\node at (-0.05,{0*\dP}) [left] {$0\mathstrut$};
\node at (-0.05,{-5*\dP}) [left] {$-5\mathstrut$};
\node at (-0.05,{-10*\dP}) [left] {$-10\mathstrut$};
\end{scope}

\begin{scope}[yshift=-10.2cm]

\begin{scope}[xshift=2cm]
\color{darkred}
\parametereins
\end{scope}

\begin{scope}[xshift=4.5cm]
\color{blue}
\parameterzwei
\end{scope}

\begin{scope}[xshift=7cm]
\color{darkgreen}
\parameterdrei
\end{scope}

\begin{scope}[xshift=9.5cm]
\color{orange}
\parametervier
\end{scope}
\end{scope}

\end{tikzpicture}
\end{document}

