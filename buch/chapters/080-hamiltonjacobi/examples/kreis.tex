%
% kreis.tex -- template for standalon tikz images
%
% (c) 2021 Prof Dr Andreas Müller, OST Ostschweizer Fachhochschule
%
\documentclass[tikz]{standalone}
\usepackage{amsmath}
\usepackage{times}
\usepackage{txfonts}
\usepackage{pgfplots}
\usepackage{csvsimple}
\usetikzlibrary{arrows,intersections,math}
\definecolor{darkred}{rgb}{0.8,0,0}
\definecolor{darkgreen}{rgb}{0,0.6,0}
\begin{document}
\def\skala{1}
\def\dx{4.5}
\def\dy{4.5}
\input{kreispfad.tex}
\begin{tikzpicture}[>=latex,thick,scale=\skala]

\clip (-0.4,-3.60) rectangle (12.2,6.5);

\begin{scope}
	\pgfmathparse{2.12}
	\xdef\h{\pgfmathresult}
	\clip (0,{-0.8*\dy}) rectangle ({2.6*\dx},{1.4*\dy});
	\foreach \k in {5,...,24}{
		\fill[color=darkgreen!\k]
			(0,{(-0.8+\h*(\k-5)/20)*\dy})
			rectangle
			({2.65*\dx},{(-0.8+\h*(\k-4)/20)*\dy+0.1});
	}
\end{scope}

\faechera
\faecherb
\faecherc
\faecherd
\faechere
\faecherf
\faecherg
\faecherh
\faecheri
\faecherj
\faecherk
\faecherl
\faecherm
\faechern
\faechero
\faecherp
\faecherq

\draw[color=blue!40] \pfada;
\draw[color=blue!40] \pfadb;
\draw[color=blue!40] \pfadc;
\draw[color=blue!40] \pfadd;
\draw[color=blue!40] \pfade;
\draw[color=blue!40] \pfadf;
\draw[color=blue!40] \pfadg;
\draw[color=blue!40] \pfadh;
\draw[color=blue!40] \pfadi;
\draw[color=blue!40] \pfadj;
\draw[color=blue!40] \pfadk;
\draw[color=blue!40] \pfadl;
\draw[color=blue!40] \pfadm;
\draw[color=blue!40] \pfadn;
\draw[color=blue!40] \pfado;
\draw[color=blue!40] \pfadp;
\draw[color=blue!40] \pfadq;

\draw[color=darkred!40,line width=1.2pt] \kreisfaechera;
\draw[color=darkred!40,line width=1.2pt] \kreisfaecherb;
\draw[color=darkred!40,line width=1.2pt] \kreisfaecherc;
\draw[color=darkred!40,line width=1.2pt] \kreisfaecherd;
\draw[color=darkred!40,line width=1.2pt] \kreisfaechere;
\draw[color=darkred!40,line width=1.2pt] \kreisfaecherf;
\draw[color=darkred!40,line width=1.2pt] \kreisfaecherg;
\draw[color=darkred!40,line width=1.2pt] \kreisfaecherh;
\draw[color=darkred!40,line width=1.2pt] \kreisfaecheri;
\draw[color=darkred!40,line width=1.2pt] \kreisfaecherj;
\draw[color=darkred!40,line width=1.2pt] \kreisfaecherk;
\draw[color=darkred!40,line width=1.2pt] \kreisfaecherl;
\draw[color=darkred!40,line width=1.2pt] \kreisfaecherm;
\draw[color=darkred!40,line width=1.2pt] \kreisfaechern;
\draw[color=darkred!40,line width=1.2pt] \kreisfaechero;
\draw[color=darkred!40,line width=1.2pt] \kreisfaecherp;
\draw[color=darkred!40,line width=1.2pt] \kreisfaecherq;

\draw[color=darkred,line width=1.4pt] \kreiseins;
\draw[color=darkred,line width=1.4pt] \kreiszwei;

\draw[->,color=black] (-0.05,0) -- ({2.65*\dx},0) coordinate[label={$x$}];
\draw[->,color=black] (0,{-0.8*\dy}) -- (0,{1.40*\dy}) coordinate[label={right:$y$}];

\draw ({1*\dx},{-0.05}) -- ({1*\dx},{0.05});
\node at ({1*\dx},{-0.05}) [below] {$1\mathstrut$};

\draw ({2*\dx},{-0.05}) -- ({2*\dx},{0.05});
\node at ({2*\dx},{-0.05}) [below] {$2\mathstrut$};

\draw (-0.05,{1*\dy}) -- (0.05,{1*\dy});
\node at (-0.05,{1*\dy}) [left] {$1\mathstrut$};

\node at (-0.05,0) [left] {$0\mathstrut$};

\node[color=darkred] at ({1.03*\dx},{-0.6*\dy}) {$J=1$};
\node[color=darkred] at ({1.75*\dx},{0.7*\dy}) {$J=2$};
\node[color=darkgreen] at ({2.5*\dx},{1.25*\dy}) [left] {$n(y)\mathstrut$};

\end{tikzpicture}
\end{document}

