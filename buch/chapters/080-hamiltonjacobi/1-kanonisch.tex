%
% 1-kanonisch.tex
%
% (c) 2024 Prof Dr Andreas Müller
%
\section{Kanonische Variablen
\label{buch:hamiltonjacobi:section:kanonisch}}
\kopfrechts{Kanonische Variablen}
In den Transversalitätsbedingungen taucht immer wieder die Kombination
\[
F - y'\frac{\partial F}{\partial y'}
\]
auf.
Dieser Abschnitt geht daher der Frage nach, ob die Ableitung von
$F$ nach $y'$ eine besondere Bedeutung hat.
Tatsächlich kann diese Ableitung als neue Koordinate verwendet
werden, mit denen sich die Euler-Lagrange-Differentialgleichung, die
von zweiter Ordnung ist, in ein besonders einfaches
Differentialgleichungssystem erster Ordnung umwandeln lässt.

%
% Transversaltität und kanonische Variable
%
\subsection{Transversalität und kanonische Variable}
Wir betrachten das Variationsproblem mit der Lagrange-Funktion
$F(x,y,y')$.
Eine Extremale durch die Punkte $(x_1,y_1)$ und $(x_2,y_2)$ ist
eine Lösung $y(x)$ der Euler-Lagrange-Differentialgleichung, die die
Randbedingungen $y(x_1)=y_1$ und $y(x_2)=y_2$ erfüllt.
Da die Euler-Lagrange-Differentialgleichung von zweiter Ordnung ist,
enthält die allgemeine Lösung zwei Konstanten, die bestimmt werden 
müssen, um die Randbedingungen zu erfüllen.
Diese Konstanten können z.~B.~der Anfangswert $y_1=y(x_1)$ und die
Anfangssteigung $y'(x_1)$ sein.
Um das Randwertproblem zu lösen, muss also die Anfangssteigung gefunden
werden, so dass die zugehörige Lösung die Randbedingung $y(x_2)=y_2$ 
erfüllt.

%
% Die kanonischen Variablen
%
\subsubsection{Die kanonischen Variablen}
Die Transversalitätsbedingung gibt der partiellen Ableitung von $F$
nach $y'$ eine besondere Bedeutung, wir schreiben daher 
\begin{equation}
p = \frac{\partial F}{\partial y'}(x,y,y').
\label{buch:hamiltonjacobi:kanonisch:eqn:pablF}
\end{equation}
Dies ist eine Gleichung, die in jedem Punkt $(x,y)$ eine Beziehung 
zwischen der Steigung $y'$ und der Variablen $p$ herstellt.
Sie ist mindestens in einer Umgebung eines Punktes der $x$-$y$-Ebene
umkehrbar, wenn die Ableitung
\[
\frac{\partial p(y')}{\partial y'}
=
\frac{\partial^2 F}{\partial y^{\prime 2}}
\ne 0
\]
ist.
Dies ist immer dann erfüllt, wenn die starke Legendre-Bedingung
erfüllt ist, die notwendig ist, damit Lösungen der
Euler-Lagrange-Differentialgleichung tatsächlich das Funktional
extremal machen.
Wir nehmen daher im folgenden an, dass 
\eqref{buch:hamiltonjacobi:kanonisch:eqn:pablF}
umkehrbar ist und $y'$ durch $y$ und $p$ bestimmt werden kann.

%
% Die Hamilton-Funktion
%
\subsubsection{Die Hamilton-Funktion}
Der Ausdruck der Transversalitätsbedingung kann jetzt
als 
\[
F(x,y,y') - y'\frac{\partial F}{\partial y'}(x,y,y')
=
F(x,y,y') - y'\cdot p
\]
geschrieben werden.
Die partielle Ableitung der rechten Seite nach $y'$ ist
\[
\frac{\partial}{\partial y'}
\biggl(
F(x,y,y') - y'\cdot p
\biggr)
=
\frac{\partial F}{\partial y'}(x,y,y') - p
=
0,
\]
sie hängt also nicht von $y'$ ab.
Somit muss sie als Funktion nur von $x$, $y$ und $p$ geschrieben werden
können.

\begin{definition}[Hamilton-Funktion]
\label{buch:hamiltonjacobi:kanonisch:def:hamilton-funktion}
Die Hamilton-Funktion zur Lagrange-Funktion $F$ ist die Funktion 
\[
H(x,y,p)
=
y'\cdot p - F(x,y,y')
\]
wobei $y'$ mit Hilfe von \eqref{buch:hamiltonjacobi:kanonisch:eqn:pablF}
durch $p$ ausgedrückt wird.
\end{definition}

\begin{beispiel}
\label{buch:hamiltonjacobi:kanonisch:bsp:laenge}
Wir betrachten das Funktional mit der Lagrange-Funktion
\[
F(x,y,y')
=
\sqrt{1+y^{\prime 2}},
\]
welches kürzeste Verbindungen in der Ebene als Extremalen hat.
Die partielle Ableitung nach $y'$ ist
\begin{equation}
p
=
\frac{\partial F}{\partial y'}
=
\frac{y'}{\sqrt{1+y^{\prime 2}}}
\label{buch:hamiltonjacobi:kanonisch:bsp:pdef}
\end{equation}
Ist $\alpha$ der Neigungswinkel der Tangente an den Graphen von $y(x)$,
dann ist $y'=\tan\alpha$ und
\[
p
=
\frac{\tan\alpha}{\sqrt{1+\tan^2\alpha}}
=
\sin\alpha.
\]
Man kann also auch $p=\sin\arctan y'$ oder $y'=\tan\arcsin p$ schreiben.

Die zweite Ableitung von $F$ ist
\[
\frac{\partial^2 F}{\partial y^{\prime2}}
=
\frac{1}{(1+y^{\prime 2})^{\frac32}} \ne 0,
\]
die starke Legendre-Bedingung ist also erfüllt und $y'$ kann durch $p$
ersetzt werden.
In der Tat kann die Gleichung \eqref{buch:hamiltonjacobi:kanonisch:bsp:pdef}
nach
\begin{equation}
y'=\frac{p}{\sqrt{1-p^2}}
\label{buch:hamiltonjacobi:kanonisch:bsp:y'ausp}
\end{equation}
aufgelöst werden.
Schon aus \eqref{buch:hamiltonjacobi:kanonisch:bsp:pdef} ist klar,
dass $|p|<1$ sein muss, der Nenner in
\eqref{buch:hamiltonjacobi:kanonisch:bsp:y'ausp} 
wird also nie $0$.

Die Lagrange-Funktion kann jetzt durch $p$ statt $y'$ ausgedrückt werden,
sie ist
\begin{align*}
F(x,y,y')
&=
\sqrt{1+y^{\prime 2}}
=
\sqrt{1+\frac{p^2}{1-p^2}}
=
\frac{1}{\sqrt{1-p^2}}
\intertext{oder ausgedrückt durch den Winkel $\alpha$}
&=
\frac{1}{\sqrt{1-\sin^2\alpha}}
=
\frac{1}{\cos\alpha}.
\end{align*}
Da der Winkel $|\alpha|<\frac{\pi}2$ ist, wird der Nenner nicht $0$.

Die Hamilton-Funktion ist 
\begin{align*}
H(x,y,p)
&=
y'p
-
F(x,y,y')
=
y'p
-
\sqrt{1+y^{\prime 2}},
\intertext{sie muss jetzt aber nur durch $p$ ausgedrückt werden, was
unter Verwendung von \eqref{buch:hamiltonjacobi:kanonisch:bsp:y'ausp}
auf}
&=
\frac{p^2}{\sqrt{1-p^2}}
-
\frac{1}{\sqrt{1-p^2}}
=
-
\frac{1-p^2}{\sqrt{1-p^2}}
=
-\sqrt{1-p^2}
\intertext{führt, oder mit Hilfe des Winkels $\alpha$ ausgedrückt:}
&=
-\cos\alpha.
\end{align*}
Die Hamilton-Funktion hat also eine einfache geometrische Interpretation.
\end{beispiel}

%
% Die kanonischen Differentialgleichungen
%
\subsection{Die kanonischen Differentialgleichungen}
Die Euler-Lagrange-Differentialgleichung für die Lagrange-Funktion $F$ ist 
die Differentialgleichung
\begin{equation}
\frac{\partial F}{\partial y}\bigl(x,y(x),y'(x)\bigr)
-
\frac{d}{dx}
\frac{\partial F}{\partial y'}\bigl(x,y(x),y'(x)\bigr)
=
0
\label{buch:hamiltonjacobi:kanonisch:eqn:kaneldgl}
\end{equation}
zweiter Ordnung für die Funktion $y(x)$.
Zu einer Lösung $y(x)$ gehört nach
\eqref{buch:hamiltonjacobi:kanonisch:eqn:pablF}
auch eine Funktion
\[
p(x)
=
\frac{\partial F}{\partial y'}\bigl(x,y(x),y'(x)\bigr).
\]
Eingesetzt in die Euler-Lagrange-Differentialgleichung
\eqref{buch:hamiltonjacobi:kanonisch:eqn:kaneldgl}
entsteht die Differentialgleichung
\[
\frac{\partial F}{\partial y}
-
\frac{d}{dx}p
=
0.
\]
Aus der Definition der Hamilton-Funktion folgt
\[
\frac{\partial H}{\partial y}
=
\frac{\partial(y'p-F)}{\partial y}
=
-\frac{\partial F}{\partial y}
\qquad\Rightarrow\qquad
\frac{dp}{dx}
=
-
\frac{\partial H}{\partial y}.
\]
Die Ableitung von $H$ nach $p$ ist
\[
\frac{\partial H}{\partial p}
=
y',
\]
da $H$ nicht von $y'$ abhängt.
Damit haben wir den folgenden Satz gefunden.

\begin{satz}[Kanonische Differentialgleichungen]
Jeder Lösung $y(x)$ der Euler-Lagrange-Differentialgleichung der
Lagrange-Funktion $F(x,y,y')$ entspricht eine Lösung
der {\em kanonischen Differentialgleichungen}
\begin{equation}
\begin{aligned}
y'&=\phantom{-}\frac{\partial H}{\partial p}
\\
p'&=-\frac{\partial H}{\partial y}.
\end{aligned}
\end{equation}
\end{satz}

Als Differentialgleichungen erster Ordnung sind die kanonischen
Differentialgleichungen oft leichter zu lösen als die
Euler-Lagrange-Differentialgleichung.

\begin{beispiel}
Für die Hamilton-Funktion 
$H(x,y,p) = -\sqrt{1-p^2}$
des Beispiels~\ref{buch:hamiltonjacobi:kanonisch:bsp:laenge} sind
die kanonischen Differentialgleichungen
\begin{align}
y'
&=
-\frac{\partial H}{\partial p}
=
\frac{p}{\sqrt{1-p^2}} = \tan\alpha,
\notag
\\
p'
&=
\frac{\partial H}{\partial y}=0
\notag
\end{align}
Aus der zweiten Gleichung folgt, dass $p$ konstant ist.
Damit lässt sich die erste Gleichung sofort lösen, die Lösung, die
durch den Punkt $(x_0,y_0)$ verläuft, ist
\[
y(x)
=
y_0+\frac{p}{\sqrt{1-p^2}}(x-x_0)
=
y_0+ (x-x_0) \tan\alpha.
\]
Die Lösungen sind Geraden mit Steigung $p/\sqrt{1-p^2}$
\end{beispiel}

Im Beispiel hing die Hamilton-Funktion $H(x,y,p)$ nur von $p$ ab,
was die Lösung besonders einfach gemacht hat, weil eine der kanonischen
Differentialgleichungen $0$ auf der rechten Seite hatte.
Dies gilt ganz allgemein, wie der folgende Satz zeigt.
Kommt eine der abhängigen Variablen in der Hamilton-Funktion nicht vor,
dann kann die Differentialgleichung durch eine Stammfunktion
gelöst werden, man sagt auch, es sei nur eine {\em Quadratur}
\index{Quadratur}%
erforderlich.

\begin{satz}
Sei $y(x),p(x)$ eine Lösung der kanonischen Differentialgleichungen.
Hängt die Hamilton-Funktion nicht von $y$ ab, geschrieben als
$H(x,y,p)=H_1(x,p)$ dann ist $p(x)=p_0$ konstant und
$y(x)$ kann durch Quadratur bestimmt werden.
Hängt die Hamilton-Funktion nicht von $p$ ab, geschrieben als
$H(x,y,p)=H_2(x,y)$ dann ist $y(x)=y_0$ konstant und
$p(x)$ kann durch Quadratur bestimmt werden.
\end{satz}

\begin{proof}
Falls $H$ nicht von $y$ abhängt, verschwindet die rechte Seite der
zweiten kanonischen Differentialgleichung, es gilt daher $p'=0$,
$p$ ist konstant.
Die erste kanonische Differentialgleichung besagt dann
\[
y'(x)
=
\frac{\partial H_1}{\partial p}(x,p_0)
\qquad\Rightarrow\qquad
y(x)
=
\int_{x_1}^x
\frac{\partial H_1}{\partial y}(\xi,p_0)
\,d\xi
+ y_0.
\]
$y(x)$ ist damit durch Quadratur bestimmt.

Falls $H$ nicht von $p$ abhängt, verschwindet die rechte Seite
der ersten kanonischen Differentialgleichung, es gilt daher $y'=0$,
$y$ ist konstant.
Die zweite kanonische Differentialgleichung besagt dann
\[
p'(x)
=
-
\frac{\partial H_2}{\partial y}(x,y_0)
\qquad\Rightarrow\qquad
p(x)
=
\int_{x_1}^x
\frac{\partial H_2}{\partial y}(\xi,y_0)
\,d\xi
+
p_0.
\]
$p(x)$ ist damit durch Quadratur bestimmt.
\end{proof}

%
% Transversalitätsbedingung
%
\subsection{Transversalitätsbedingung}
Der Startpunkt für die Einführung der kanonischen Variable $p$ und
der Hamilton-Funktion war ihr Auftreten in der Transversalitätsbedingung.
Die Lösung eines Anfangspunkt-End\-kur\-ve-Problems erfüllt am rechten 
Intervallende die Transversalitätsbedingung
\[
\vec{r_2}\cdot \vec{f}_2=0,
\]
der Tangentialvektor $\vec{r}_1$ an die Endkurve ist orthogonal auf
dem Vektor
\begin{align*}
\vec{f}_2
&=
\begin{pmatrix}
\displaystyle
F(x_2,y(x_2),y'(x_2))-y'(x_2)\frac{\partial F}{\partial y'}(x_2,y(x_2),y'(x_2))
\\[5pt]
\displaystyle
\frac{\partial F}{\partial y'}(x_2,y(x_2),y'(x_2))
\end{pmatrix}
\\
&=
\begin{pmatrix}
F(x_2,y(x_2),y'(x_2)) - y'(x_2) p(x_2) \\
p(x_2)
\end{pmatrix}
=
\begin{pmatrix}
-H(x_2,y(x_2),p(x_2))\\
p(x_2)
\end{pmatrix}.
\end{align*}

\begin{beispiel}
Für das Beispiel~\ref{buch:hamiltonjacobi:kanonisch:bsp:laenge}
für kürzeste Verbindungen in der Ebene haben wir bereits gefunden,
dass die kanonische Variable $p$ und die Hamilton-Funktion durch 
den Neigungswinkel $\alpha$ des Graphen einer Extremalen $y(x)$
ausgedrückt werden kann.
Mit $y'(x)=\tan\alpha$ ist der Vektor
\[
\vec{f}
=
\begin{pmatrix}
-H\\
p
\end{pmatrix}
=
\begin{pmatrix}
\cos\alpha\\
\sin\alpha
\end{pmatrix}.
\]
Dies ist der Richtungsvektor des Einheitstangentialvektors an
den Graphen von $y(x)$.
Die Transversalitätsbedingung ist also genau dann erfüllt, wenn die
Lösung orthogonal auf die Endkurve trifft.
Die kürzeste Kurve zwischen einem Punkt und einer Kurve ist eine
Gerade durch den Punkt, die die Kurve orthogonal schneidet.
\end{beispiel}

%
% Mehrere unabhängige Variable
%
\subsection{Mehrere unabhängige Variable}
Die Transformation auf die kanonische Variable $p$ ist auch dann möglich,
wenn $y$ eine vektorwertige Funktion ist.
Wir betrachten daher in diesem Abschnitt ein Variationsproblem für
eine Funktion $y\colon \mathbb{R}\to\mathbb{R}^n$, die das Integral
\[
I(y)
=
\int_{x_0}^{x_1}
F\bigl(x,y(x),y'(x)\bigr)
\,dx
\]
mit der Lagrange-Funktion
\[
F(x,y,y')
=
F(x,y_1,\dots,y_n,y_1',\dots,y'_n)
\]
extremal machen soll.
Das Problem wird durch die Euler-Lagrange-Differentialgleichung
\begin{align*}
0
&=
\frac{\partial F}{\partial y}\bigl(x,y(x),y'(x)\bigr)
-
\frac{d}{dx}
\frac{\partial F}{\partial y'}\bigl(x,y(x),y'(x)\bigr)
\intertext{oder in Komponenten}
0
&=
\frac{\partial F}{\partial y_k}\bigl(x,y(x),y'(x)\bigr)
-
\frac{d}{dx}
\frac{\partial F}{\partial y'_k}\bigl(x,y(x),y'(x)\bigr)
\end{align*}
gelöst.
Im Folgenden sei $y(x)$ jeweils eine Lösung dieser Differentialgleichungen.

%
% Kanonische Variablen
%
\subsubsection{Kanonische Variablen}
Wie im Fall einer einzigen unabhängigen Variablen verwenden wir den
Vektor
\[
p
=
\frac{\partial F}{\partial y'}(x,y,y')
\qquad\text{mit Komponenten}\qquad
p_k
=
\frac{\partial F}{\partial y_k'}(x,y,y')
\]
als neue Variable an Stelle von $y'$.
Damit $y'$ bei festem $x$ und $y$ durch $p$ ausgedrückt werden kann,
damit also die Abbildung 
\[
y'\mapsto f(y')=\frac{\partial F}{\partial y'}(x,y,y')
\]
mindestens in einer Umgebung eines Punktes invertiert werden kann,
muss die Ableitung der Funktion $f\colon \mathbb{R}^n\to\mathbb{R}^n$
nach $y'$ eine invertierbare Matrix sein.
Die Jacobi-Matrix
\[
J
=
\frac{\partial f}{\partial y'}
\qquad\text{mit Matrixelementen}\qquad
\biggl(
\frac{\partial f}{\partial y'}
\biggr)_{ik}
=
\frac{\partial f_i}{\partial y_k'}
=
\frac{\partial^2 F}{\partial y_i'\,\partial y_k'}
\]
muss invertierbar sein.
Dies ist gleichbedeutend damit, dass die Funktionaldeterminante
$\det J\ne 0$ ist.
Die Bedingung ist automatisch erfüllt, wenn die starke Legendre-Bedingung
für $F$ erfüllt ist.
Diese besagt, dass die Matrix $J$ positiv (oder negativ) definit ist, womit
sie automatisch regulär ist.
Da die Legendre-Bedingung notwendig ist, damit die Lösung der
Euler-Lagrange-Differentialgleichung eines Variationsproblems das
Funktional extremal macht, können wir meist davon ausgehen,
dass auch $y'=y'(x,y,p)$ geschrieben werden kann.

%
% Hamilton-Funktion
%
\subsubsection{Hamilton-Funktion}
Die Definition der Hamilton-Funktion im Falle $n=1$ kann jetzt
naheliegend wie folgt erweitert werden.
Die Ableitung
\begin{equation*}
\frac{\partial}{\partial y'}
\bigl(
p\cdot y'
-
F(x,y,y')
\bigr)
=
p
-
\frac{\partial F}{\partial y'}(x,y,y')
=
0
\end{equation*}
verschwindet nach Definition von $p$.
Der Klammerausdruck hängt also nicht von $y'$ ab und kann daher 
als Funktion $x$, $y$ und $p$ ausgedrückt werden.

\begin{definition}[Hamilton-Funktion]
Die Hamilton-Funktion zur Lagrange-Funktion $F$ ist die Funktion
\[
H(x,y,p)
=
p\cdot y' - F(x,y,y'),
\]
wobei $y'$ durch $x$, $y$ und $p$ ausgedrückt werden muss.
\end{definition}

Aus der Definition folgen unmittelbar die partiellen Ableitungen
von $H$ nach $y$ und $p$, die
\begin{equation}
\frac{\partial H}{\partial y}
=
-\frac{\partial F}{\partial y}
\qquad\text{und}\qquad
\frac{\partial H}{\partial p}
=
y'
\label{buch:hamiltonjacobi:kanonisch:eqn:Habln}
\end{equation}
sind.

%
% Kanonische Differentialgleichungen
%
\subsubsection{Kannonische Differentialgleichungen}
Die zweite Differentialgleichung in
\eqref{buch:hamiltonjacobi:kanonisch:eqn:Habln}
ist bereits eine der gesuchten kanonischen Differentialgleichungen.
Die Euler-Lagrange-Differentialgleichung
\[
\frac{\partial F}{\partial y}=
\frac{d}{dx}\frac{\partial F}{\partial y'}
\qquad\Rightarrow\qquad
\frac{d}{dx}p
=
\frac{\partial F}{\partial y}
=
-\frac{\partial H}{\partial y}
\]
ergibt die zweite kanonische Differentialgleichung nach der
ersten Gleichung von \eqref{buch:hamiltonjacobi:kanonisch:eqn:Habln}.
Damit folgt der folgende Satz.

\begin{satz}[Kanonische Differentialgleichungen]
\label{buch:hamiltonjacobi:kanonisch:satz:kandgln}
Jeder Lösung $y(x)$ der Euler-Lagrange-Differentialgleichung der
Lagrange-Funktion $F(x,y,y')$ entspricht eine Lösung
der {\em kanonischen Differentialgleichungen}
\index{kanonische Differentialgleichungen}%
\begin{equation}
\begin{aligned}
y'&=\phantom{-}\frac{\partial H}{\partial p}
\\
p'&=-\frac{\partial H}{\partial y}.
\end{aligned}
\end{equation}
\end{satz}

\begin{beispiel}
\label{buch:hamiltonjacobi:kanonisch:bsp:3dlaenge}
Wir erweitern das Beispiel~\ref{buch:hamiltonjacobi:kanonisch:bsp:laenge}
auf den dreidimensionalen Fall.
Wir bezeichnen die drei Raumkoordinaten mit $x$, $y_1$ und $y_2$ und 
verwenden die Lagrange-Funktion
\[
F(x,y,y')
=
\sqrt{1+y^{\prime 2}}
=
\sqrt{1+y_1^{\prime 2}+y_2^{\prime 2}}.
\]
Die kanonischen Variablen entstehen durch die Ableitung nach $y'$, wir
setzen also
\begin{align}
p
&=
\frac{\partial F}{\partial y'}
=
\frac{y'}{\sqrt{1+y^{\prime 2}}}.
\label{buch:hamiltonjacobi:kanonisch:bsp:py}
\end{align}
Die Jacobi-Matrix
\begin{align*}
\frac{\partial^2 F}{\partial y^{\prime 2}}
&=
\begin{pmatrix}
\displaystyle
\frac{1+y_2^2}{(1+y^{\prime 2})^{\frac32}}
&
\displaystyle
-\frac{y_1'y_2'}{(1+y^{\prime 2})^{\frac32}}
\\[5pt]
\displaystyle
-\frac{y_1'y_2'}{(1+y^{\prime 2})^{\frac32}}
&
\displaystyle
\frac{1+y_1^2}{(1+y^{\prime 2})^{\frac32}}
\end{pmatrix}
\intertext{hat die Determinante}
\det
\frac{\partial^2 F}{\partial y^{\prime 2}}
&=
\frac{1}{(1+y^{\prime 2})^3}
\bigl(
(1+y_1^{\prime 2})(1+y_2^{\prime 2}) - y_1^{\prime 2}y_2^{\prime 2}
\bigr)
=
\frac{
1
+
y_1^{\prime 2} + y_2^{\prime 2}
}{
(1+y^{\prime 2})^3
}
\\
&=
\frac{1+y^{\prime 2}}{ (1+y^{\prime 2})^3}
=
\frac{1}{\sqrt{1+y^{\prime 2}}}
>
0,
\\
\operatorname{Spur}
\frac{\partial^2 F}{\partial y^{\prime 2}}
&=
\frac{2+y_1^{\prime 2}+y_2^{\prime 2}}{(1+y^{\prime 2})^{\frac32}}
>0,
\end{align*}
somit ist die Ableitung positiv definit, die starke Legendre-Bedingung
ist erfüllt und $y'$ kann durch $x$, $y$ und $p$ ausgedrückt werden.

Die Gleichungen~\label{buch:hamiltonjacobi:kanonisch:bsp:py} können
invertiert werden, wenn der skalare Nenner durch $p$ ausgedrückt
werden kann.
Dazu multipliziert man die Gleichung mit sich selbst und erhält
\[
\begin{aligned}
p^2
&=
\frac{y^{\prime 2}}{1+y^{\prime 2}}
&&\Rightarrow&
p^2(1+y^{\prime 2})&=y^{\prime2}
\\
&&&\Rightarrow&
p^2
&=
(1-p^2)y^{\prime 2}
\\
&&&\Rightarrow&
y^{\prime 2}
&=
\frac{p^2}{1-p^2}
\\
&&&\Rightarrow&
1+y^{\prime2}
&=
\frac{1}{1-p^2}
\\
&&&\Rightarrow&
\sqrt{1+y^{\prime2}}
&=
\frac{1}{\sqrt{1-p^2}}.
\end{aligned}
\]
Damit kann man jetzt $y'$ durch $p$ als
\[
y'
=
\frac{p}{\sqrt{1-p^2}}
\]
ausdrücken.
Auch die Legendre-Funktion kann durch $p$ ausgedrückt werden,
da
\[
F(x,y,y')
=
\sqrt{1+y^{\prime 2}}
=
\frac{1}{\sqrt{1-p^2}}
\]
ist.

Die Hamilton-Funktion ist
\begin{align*}
H(x,y,p)
&=
y'\cdot p - F(x,y,y')
=
\frac{p}{\sqrt{1-p^2}}\cdot p - \frac{1}{\sqrt{1-p^2}}
=
\frac{p^2-1}{\sqrt{1-p^2}}
=
-\sqrt{1-p^2},
\end{align*}
insbesondere hängt sie nicht von $y$ ab.
Die Ableitung nach $p$ ist tatsächlich
\begin{equation}
\frac{\partial H}{\partial p}
=
\frac{p}{\sqrt{1-p^2}}.
\label{buch:hamiltonjacobi:kanonisch:bsp:dHdp}
\end{equation}
Damit werden die kanonischen Differentialgleichungen
\begin{align*}
y' &= \frac{p}{\sqrt{1-p^2}} \\
p' &= 0.
\end{align*}
Somit ist $p$ konstant und $y'$ kann durch Quadratur als
\[
y(x)
=
y_0 + \frac{p}{\sqrt{1-p^2}}(x-x_0)
\]
bestimmt werden.
\end{beispiel}

%
% Transversalitätsbedingung
%
\subsubsection{Transversalitätsbedingung}
Die Verallgemeinerung der Transversalitätsbedingungen auf mehrere
Funktionen verlangt, dass jeder Tangentialvektor an die Randfläche
senkrecht steht auf dem Vektor
\[
\vec{f}(x)
=
\begin{pmatrix}
\displaystyle
F\bigl(x,y(x),y'(x)\bigr)
-
y'(x)\cdot \frac{\partial F}{\partial y'}\bigl(x,y(x),y'(x)\bigr)
\\
\displaystyle
\frac{\partial F}{\partial y_1'}\bigl(x,y(x),y'(x)\bigr)
\\[9pt]
\displaystyle
\frac{\partial F}{\partial y_2'}\bigl(x,y(x),y'(x)\bigr)
\end{pmatrix}
=
\begin{pmatrix}
-H\bigl(x,y(x),p(x)\bigr)\\
p_1(x)\\
p_2(x)
\end{pmatrix}
\]
steht, wobei $(x,y(x))$ auf der Randfläche liegen muss.

\begin{beispiel}
In Beispiel~\ref{buch:hamiltonjacobi:kanonisch:bsp:3dlaenge} wurde
die Hamilton-Funktion für das Variationsproblem für kürzeste Verbindungen
im Raum bestimmt.
Der Vektor $\vec{f}$ 
\begin{align*}
\vec{f}(x)
&=
\begin{pmatrix}
-H\bigl(x,y(x),p(x)\bigr)\\
p_1(x_1)\\
p_2(x_2)
\end{pmatrix}
=
\begin{pmatrix}
\sqrt{1-p^2}\\
p_1\\
p_2
\end{pmatrix}
\end{align*}
ist ein Vektor mit Länge
\[
|\vec{f}(x)|
=
(1-p^2) + p_1^2+p_2^2
=
1,
\]
Der Vektor $\vec{f}$ ist also ein Tangentialeinheitsvektor an die
durch $y(x)$ beschriebene Kurve.
Die Transversalitätsbedingung sagt daher, dass die Kurve kürzester
Länge von einem Punkt zu einer Fläche senkrecht auf die Fläche trifft.
\end{beispiel}

%
% H als Erhaltungsgrösse
%
\subsection{$H$ als Erhaltungsgrösse}
Eine besondere Situation tritt ein, wenn die Lagrange-Funktion $F(x,y,y')$
nicht von $x$ abhängt.
Dieser Fall soll im Folgenden untersucht werden.
In diesem Abschnitt wird daher die
Lagrange-Funktion vereinfachend als $F(y,y')$ geschrieben.

%
% Unabhängigkeit der Hamilton-Funktion von x
%
\subsubsection{Unabhängigkeit der Hamilton-Funktion von $x$}
Da $F$ nicht explizit von $x$ abhängt, hängt auch die
Hamilton-Funktion
\[
H(x,y,p)
=
y'\cdot p - F(y,y')
\]
nicht von $x$ ab.
Wir schreiben im Folgenden auch die Hamilton-Funktion ohne $x$-Argument,
also als $H(y,p)$.

Die kanonischen Differentialgleichungen
\[
y' = \frac{\partial H}{\partial p}(y,p)
\quad\text{und}\quad
p' = -\frac{\partial H}{\partial y}(y,p)
\]
hängt ebenfalls nicht von $x$ ab.
Ein solches Differentialgleichungssystem heisst {\em autonom}.
\index{autonom}%

\begin{definition}[Autonomes Differentialgleichungssystem]
Ein Differentiagleichungssystem ist eine Funktion 
\[
f\colon \mathbb{R}\times \mathbb{R}^n\to\mathbb{R}^n
:
(x,y)\mapsto f(x,y).
\]
Eine Lösung des Differentialgleichungssystems mit der Anfangsbedingung
$y_0$ an der Stelle $x_0$ ist eine Funktion
$y\colon\mathbb{R}\to\mathbb{R}^n$ derart, dass 
\[
\frac{dy}{dx}(x)
=
f(x,y(x))
\qquad\text{und}\qquad
y(x_0)=y_0.
\]
Das Differentialgleichungssystem heisst {\em autonom}, wenn $f$ nicht von
\index{autonom}%
$x$ abhängt, wenn also
\[
\frac{\partial f}{\partial x} = 0
\]
ist.
\end{definition}

%
% Erhaltung von H
%
\subsubsection{Erhaltung von $H$}
Sei jetzt $y(x)$ und $p(x)$ eine Lösung der kanonischen
Differentialgleichungen.
Die Funktion $x\mapsto H(y(x),p(x))$ hat die Ableitung
\begin{align*}
\frac{d}{dx}H\bigl(y(x),p(x)\bigr)
&=
\frac{\partial H}{\partial y}\bigl(y(x),p(x)\bigr)\cdot y'(x)
+
\frac{\partial H}{\partial p}\bigl(y(x),p(x)\bigr)\cdot p'(x)
\\
&=
-p'(x)\cdot y'(x)
+
y'(x)\cdot p'(x)
=
0.
\end{align*}
Somit bleibt $H$ entlang einer Lösungskurve konstant.
Lösungskurven sind in den Flächen $H(y,p)=\operatorname{const}$
enthalten.

%
% Ableitungen beliebiger Funktionen von $x$, $y$ und $y'$
%
\subsubsection{Ableitungen beliebiger Funktionen von $x$, $y$ und $y'$}
Sei $G(x,y,p)$ eine beliebige Funktion von $x$, $y$ und $p$ und seien
$y(x)$ und $p(x)$ eine Lösung der kanonischen Differentialgleichungen.
Mit Hilfe der kanonischen Differentialgleichungen kann man die Ableitung
nach $x$ mit der Kettenregel
\begin{align*}
\frac{d}{dx}G\bigl(x,y(x),p(x)\bigr)
&=
\frac{\partial G}{\partial x}\bigl(x,y(x),p(x)\bigr)
\\
&\qquad
+
\frac{\partial G}{\partial y}\bigl(x,y(x),p(x)\bigr)
\cdot
y'(x)
+
\frac{\partial G}{\partial p}\bigl(x,y(x),p(x)\bigr)
\cdot
p'(x)
\intertext{berechnen und mit Hilfe der kanonischen Differenialgleichungen
in die Form}
&=
\frac{\partial G}{\partial x}\bigl(x,y(x),p(x)\bigr)
\\
&\qquad
+
\frac{\partial G}{\partial y}\bigl(x,y(x),p(x)\bigr)
\cdot
\frac{\partial H}{\partial p}\bigl(x,y(x),p(x)\bigr)
\\
&\qquad
-
\frac{\partial G}{\partial p}\bigl(x,y(x),p(x)\bigr)
\cdot
\frac{\partial H}{\partial y}\bigl(x,y(x),p(x)\bigr)
\end{align*}
bringen.
Die letzten zwei Terme auf der rechten Seite verdienen einen eigenen
Namen.

\begin{definition}[Poisson-Klammer] Die {\em Poisson-Klammer} zweier
\index{Poisson-Klammer}%
Funktionen $A(x,y,p)$ und $B(x,y,p)$ von $x$, $y$ und $p$ ist die
Funktion
\begin{align*}
\{ A,B\} (x,y,p)
&=
\frac{\partial A}{\partial y}(x,y,p)
\frac{\partial B}{\partial p}(x,y,p)
-
\frac{\partial B}{\partial y}(x,y,p)
\frac{\partial A}{\partial p}(x,y,p)
\\
&=
\sum_{k=1}^n \biggl(
\frac{\partial A}{\partial y_k}(x,y,p)
\frac{\partial B}{\partial p_k}(x,y,p)
-
\frac{\partial B}{\partial y_k}(x,y,p)
\frac{\partial A}{\partial p_k}(x,y,p)
\biggr)
\end{align*}
in Komponenten.
\end{definition}

Die Poisson-Klammer ist antisymmetrisch, es gilt $\{A,B\} = -\{B,A\}$,
wie man unmittelbar mit der Definition verifizieren kann.
Setzt man $B=A$, filgt $\{A,A\}=-\{A,A\}$, woraus $\{A,A\}=0$ folgt.
Die Poisson-Klammer erfüllt aber auch die die Jacobi-Identität.

\begin{satz}[Jacobi-Identität]
Für drei Funktionen $A(x,y,p)$, $B(x,y,p)$ und $C(x,y,p)$ gilt
\begin{equation}
\{A,\{B,C\}\}
+
\{B,\{C,A\}\}
+
\{C,\{A,B\}\}
=
0.
\label{buch:hamiltonjacobi:kanonisch:eqn:jacobiidentitaet}
\end{equation}
Die Funktionen von $x$, $y$ und $p$ bilden eine Lie-Algebra mit
der Poisson-Klammer als Lie-Klammer.
\end{satz}

\begin{proof}
Die linke Seite von
\eqref{buch:hamiltonjacobi:kanonisch:eqn:jacobiidentitaet}
wird im Folgende $L(A,B,C)$ geschrieben.
Wir berechnen die iterierte Poisson-Klammer
\begin{align*}
\{A,\{B,C\}\}
&=
\sum_{k=1}^n
\biggl(
\frac{\partial A}{\partial y_k} 
\frac{\partial \{B,C\}}{\partial p_k}
-
\frac{\partial A}{\partial p_k} 
\frac{\partial \{B,C\}}{\partial y_k}
\biggr)
\\
&=
\sum_{k=1}^n\sum_{i=1}^n
\biggl(
\frac{\partial A}{\partial y_k}
\frac{\partial}{\partial p_k}\biggl(
\frac{\partial B}{\partial y_i}
\frac{\partial C}{\partial p_i}
-
\frac{\partial B}{\partial p_i}
\frac{\partial C}{\partial y_i}
\biggr)
-
\frac{\partial A}{\partial p_k}
\frac{\partial}{\partial y_k}\biggl(
\frac{\partial B}{\partial y_i}
\frac{\partial C}{\partial p_i}
-
\frac{\partial B}{\partial p_i}
\frac{\partial C}{\partial y_i}
\biggr)
\biggr)
\\
&=
\sum_{k,i=1}^n
\bigg(
\frac{\partial A}{\partial y_k}
\frac{\partial^2 B}{\partial p_k\,\partial y_i}
\frac{\partial C}{\partial p_i}
+
\frac{\partial A}{\partial y_k}
\frac{\partial B}{\partial y_i}
\frac{\partial^2 C}{\partial p_k\,\partial p_i}
-
\frac{\partial A}{\partial y_k}
\frac{\partial^2 B}{\partial p_k\,\partial p_i}
\frac{\partial C}{\partial y_i}
-
\frac{\partial A}{\partial y_k}
\frac{\partial B}{\partial p_i}
\frac{\partial^2 C}{\partial p_k\,\partial y_i}
\\
&\qquad
-
\frac{\partial A}{\partial p_k}
\frac{\partial^2 B}{\partial y_k\,\partial y_i}
\frac{\partial C}{\partial p_i}
-
\frac{\partial A}{\partial p_k}
\frac{\partial B}{\partial y_i}
\frac{\partial^2 C}{\partial y_k\,\partial p_i}
+
\frac{\partial A}{\partial p_k}
\frac{\partial^2 B}{\partial y_k\,\partial p_i}
\frac{\partial C}{\partial y_i}
+
\frac{\partial A}{\partial p_k}
\frac{\partial B}{\partial p_i}
\frac{\partial^2 C}{\partial y_k\,\partial y_i}
\bigg).
\end{align*}
Um darin eine besser Übersicht zu bekommen, schreiben wir die ersten
Ableitungen als Spaltenvektoren
\[
(A_y)_i = \frac{\partial A}{\partial y_i}
\qquad\text{und}\qquad
(A_p)_i = \frac{\partial A}{\partial p_i}.
\]
und die zweiten Ableitungen als die symmetrischen Matrizen
\[
(B_{yy})_{ki} = \frac{\partial^2 B}{\partial y_k\,\partial y_i}
,
\qquad
(B_{yp})_{ki} = \frac{\partial^2 B}{\partial y_k\,\partial p_i}
\qquad\text{und}\qquad
(B_{pp})_{ki} = \frac{\partial^2 B}{\partial p_k\,\partial p_i}.
\]
Mit diesen Notationen
kann die iterierte Poisson-Klammer mit Hilfe des Matrizenproduktes
geschrieben werden als
\begin{equation}
\begin{aligned}
\{A,\{B,C\}\}
&=
A_y^tB_{py}C_p
+
A_y^tC_{pp}B_y
-
A_y^tB_{pp}C_y
-
A_y^tC_{py}B_p
\\
&\qquad
-
A_p^tB_{yy}C_p
-
A_p^tC_{yp}B_y
+
A_p^tB_{yp}C_y
+
A_p^tC_{yy}B_p.
\end{aligned}
\label{buch:hamiltonjacobi:kanonisch:klammerschema}
\end{equation}
Die Vektorfaktoren können vertauscht werden, wenn gleichzeitig die
Indizes des Matrixfaktors vertauscht werden:
\begin{equation}
X_a^t Y_{bc} Z_d
=
Z_d^t Y_{cb} X_a
\qquad
\text{für $a,b,c,d\in \{y,p\}$.}
\label{buch:hamiltonjacobi:kanonisch:eqn:xyz}
\end{equation}
Durch zyklische Vertauschung von $A$, $B$ und $C$ in
\eqref{buch:hamiltonjacobi:kanonisch:klammerschema} und
Addition entsteht die linke Seite
\begin{align*}
L(A,B,C)
&=
\mathstrut\phantom{\mathstrut-\mathstrut}
A_y^tB_{py}C_p
+
A_y^tC_{pp}B_y
-
A_y^tB_{pp}C_y
-
A_y^tC_{py}B_p
\\
&\phantom{\mathstrut=\mathstrut}
-
A_p^tB_{yy}C_p
-
A_p^tC_{yp}B_y
+
A_p^tB_{yp}C_y
+
A_p^tC_{yy}B_p
\\
&\phantom{\mathstrut=\mathstrut}
+
B_y^tC_{py}A_p
+
B_y^tA_{pp}C_y
-
B_y^tC_{pp}A_y
-
B_y^tA_{py}C_p
\\
&\phantom{\mathstrut=\mathstrut}
-
B_p^tC_{yy}A_p
-
B_p^tA_{yp}C_y
+
B_p^tC_{yp}A_y
+
B_p^tA_{yy}C_p
\\
&\phantom{\mathstrut=\mathstrut}
+
C_y^tA_{py}B_p
+
C_y^tB_{pp}A_y
-
C_y^tA_{pp}B_y
-
C_y^tB_{py}A_p
\\
&\phantom{\mathstrut=\mathstrut}
-
C_p^tA_{yy}B_p
-
C_p^tB_{yp}A_y
+
C_p^tA_{yp}B_y
+
C_p^tB_{yy}A_p.
\intertext{Durch Umsortieren der Terme kann man erreichen, dass auf jeder
Zeile nur  zweite Ableitungen der gleichen Funktion stehen:}
&= 
\mathstrut
-A_p^t B_{yy} C_p
+C_p^t B_{yy} A_p
-A_y^t B_{pp} C_y
+C_y^t B_{pp} A_y
\\
&\phantom{\mathstrut=\mathstrut}
+A_y^t B_{py} C_p
-C_y^t B_{yp} A_y
+A_p^t B_{yp} C_y
-C_y^t B_{py} A_p
\\
&\phantom{\mathstrut=\mathstrut}
+A_p^t C_{yy} B_p
-B_p^t C_{yy} A_p
+A_y^t C_{pp} B_y
-B_y^t C_{pp} A_y
\\
&\phantom{\mathstrut=\mathstrut}
-A_y^t C_{py} B_p
+B_p^t C_{yp} A_y
-A_p^t C_{yp} B_y
+B_y^t C_{py} A_p
\\
&\phantom{\mathstrut=\mathstrut}
+B_p^t A_{yy} C_p
-C_p^t A_{yy} B_p
+B_y^t A_{pp} C_y
-C_y^t A_{pp} B_y
\\
&\phantom{\mathstrut=\mathstrut}
-B_y^t A_{py} C_p
+C_p^t A_{yp} B_y
-B_p^t A_{yp} C_y
+C_y^t A_{py} B_p.
\intertext{Aufeinanderfolgende Terme lassen sich durch Anwendung von
\eqref{buch:hamiltonjacobi:kanonisch:eqn:xyz}
ineinander umformen und heben sich damit weg.
Es bleibt
}
L(A,B,C)&=0
\end{align*}
übrig, damit ist die Jacobi-Identität bewiesen.
\end{proof}

Die Ableitung von $G$ kann mit der Poisson-Klammer als
\begin{equation}
\frac{d}{dx}G(x,y(x),p(x))
=
\frac{\partial G}{\partial x}(x,y(x),p(x))
+
\{G,H\}(x,y(x),p(x))
\label{buch:hamiltonjacobi:kanonisch:eqn:Gabl}
\end{equation}
geschrieben werden, sie gilt auch, wenn $H$ von $x$ abhängt.
Wendet man diese Formel auf $H$ an, folgt
\[
\frac{d}{dx}
H(x,y(x),p(x))
=
\frac{\partial H}{\partial x}(x,y(x),p(x))
+
\{H,H\}(x,y(x),p(x))
=
\frac{\partial H}{\partial x}(x,y(x),p(x)).
\]
Die Hamilton-Funktion ist also genau dann konstant entlang
einer Lösung der kanonnischen Gleichungen, wenn $H$ nicht von $x$
abhängt.

\begin{beispiel}
Für die Lagrange-Funktion $F(x,y,y')=\sqrt{1+y^{\prime 2}}$ wurde
wir die Hamilton-Funktion $H(x,y,p)=-\sqrt{1-p^2}$ und die Lösung
der kanonischen Differentialgleichungen bereits bestimmt.
Die Ableitung einer Funktion $A(x,y,p)$ entlang einer Lösungskurve
der kanonischen Differentialgleichungen ist $p(x)=p_0$ konstant
und damit die Ableitung
\eqref{buch:hamiltonjacobi:kanonisch:eqn:Gabl}
\begin{align*}
\frac{d}{dx}A(x,y(x),p(x))
&=
\frac{\partial A}{\partial x}(x,y(x),p_0)
+
\{A,H\}(x,y(x),p_0)
\\
&=
\frac{\partial A}{\partial x}(x,y(x),p_0)
+
\frac{\partial A}{\partial y}(x,y(x),p_0)
\frac{\partial H}{\partial p}(x,y(x),p_0),
\intertext{da die Ableitung von $H$ nach $y$ verschwindet.
Mit \eqref{buch:hamiltonjacobi:kanonisch:bsp:dHdp} kann die Ableitung
der Hamilton-Funktion nach $p$ ausgedrückt werden, die Ableitung von $A$ wird
dann
}
&=
\frac{\partial A}{\partial x}(x,y(x),p_0)
+
\frac{\partial A}{\partial y}(x,y(x),p_0)
\frac{p_0}{\sqrt{1-p_0^2}},
\intertext{die man auch als Skalarprodukt}
&=
\begin{pmatrix}
\displaystyle
\frac{\partial A}{\partial x}\\[7pt]
\displaystyle
\frac{\partial A}{\partial y_1}\\[7pt]
\displaystyle
\frac{\partial A}{\partial y_2}
\end{pmatrix}
\cdot
\frac{1}{\sqrt{1-p_0^2}}
\begin{pmatrix}
\sqrt{1-p_0^2}\\
p_{10}\\
p_{20}
\end{pmatrix}
\end{align*}
schreiben kann.
\end{beispiel}
