%
% 32-eulerlagrangedgl.tex
%
% (c) 2023 Prof Dr Andreas Müller
%

%
% Euler-Lagrange_Differentialgleichung
%
\subsection{Euler-Lagrange-Differentialgleichung
\label{buch:variation:eulerlagrange:subsection:dgl}}
%
% variation0.tex -- Variation mit festen Enden
%
% (c) 2021 Prof Dr Andreas Müller, OST Ostschweizer Fachhochschule
%
\documentclass[tikz]{standalone}
\usepackage{amsmath}
\usepackage{times}
\usepackage{txfonts}
\usepackage{pgfplots}
\usepackage{csvsimple}
\usetikzlibrary{arrows,intersections,math}
\definecolor{darkred}{rgb}{0.8,0,0}
\begin{document}
\def\skala{1}
\def\xzero{1}
\def\xone{10}
\def\yzero{1}
\def\yone{5}
\begin{tikzpicture}[>=latex,thick,scale=\skala,
declare function={
	x(\t)   = (1-\t)*\xzero+\t*\xone;
	t(\x)   = (\x-\xzero)/(\xone-\xzero);
	eta(\t) = \t*(1-\t)*(2*cos(180*\t)+4*\t*sin(720*\t)+6);
	y(\t)   = (1-\t)*\yzero+\t*\yone+4*\t*(1-\t);
}]

\draw (\xzero,-0.05) -- (\xzero,0.05);
\draw (-0.05,\yzero) -- (0.05,\yzero);
\draw[line width=0.2pt] (\xzero,0) -- (\xzero,\yzero);
\draw[line width=0.2pt] (0,\yzero) -- (\xzero,\yzero);
\node at (\xzero,-0.05) [below] {$x_1\mathstrut$};
\node at (-0.05,\yzero) [left] {$y_1\mathstrut$};

\draw (\xone,-0.05) -- (\xone,0.05);
\draw (-0.05,\yone) -- (0.05,\yone);
\draw[line width=0.2pt] (\xone,0) -- (\xone,\yone);
\draw[line width=0.2pt] (0,\yone) -- (\xone,\yone);
\node at (\xone,-0.05) [below] {$x_2\mathstrut$};
\node at (-0.05,\yone) [left] {$y_2\mathstrut$};

\fill[color=darkred] (\xzero,\yzero) circle[radius=0.08];
\fill[color=darkred] (\xone,\yone) circle[radius=0.08];
\node at (\xzero,\yzero) [above left] {$P_1$};
\node at (\xone,\yone) [right] {$P_2$};

\foreach \e in {-1.6,-1.2,...,1.7}{
	\draw[color=gray,smooth] plot[domain=0:1] ({x(\x)},{y(\x)+\e*eta(\x)});
}
\node[color=gray] at (3.6,5.65) {$y(x)+\varepsilon\eta(x)$};

\draw[color=darkred,smooth,line width=1.4pt] plot[domain=0:1] ({x(\x)},{y(\x)});
\node[color=darkred] at (6.8,4.8) {$y(x)$};

\draw[->] (-0.1,0) -- (11.6,0) coordinate[label={$x$}];
\draw[->] (0,-0.1) -- (0,8) coordinate[label={right:$y$}];

\end{tikzpicture}
\end{document}


Das Maximum oder Minimum einer Funktionen mehrere Variablen wurde
gefunden, indem die Richtungsableitung berechnet und $=0$ gesetzt
wurde.
Um die Funktion zu bestimmen, die ein Funktional $I(y)$ zu einem
Maximum oder Minimum macht, versuchen wir, die Idee der Richtungsableitung
für ein Funktional nachzuahmen.
Wir nehmen daher an, dass $y(x)$ eine Funktion ist, die das Funktional
$I(y)$ zu einem Minimum macht.
Für die Richtungsableitung addieren wir ein Vielfaches einer
Funktion $\eta(x)$, die Summe $y(x)+\varepsilon\eta(x)$ entspricht
dann einer Geraden mit Richtung $\eta(x)$ im Funktionenraum
(Abbildung~\ref{buch:variation:fig:variation0}).
Die Funktionen $y(x)+\varepsilon\eta(x)$ sind aber nur dann Kandidaten
für eine Lösung des Problems, wenn immer noch
\begin{align*}
y(x_0) + \varepsilon \eta(x_0) &= y_0
&&\text{und}&
y(x_1) + \varepsilon \eta(x_1) &= y_1
\end{align*}
gilt.
Dies ist nur möglich, wenn $\eta(x_0)=\eta(x_1)=0$ ist.

%
% Die Richtungsableitung des Funktionals
%
\subsubsection{Die Richtungsableitung des Funktionals}
Wir berechnen jetzt die Ableitung der Funktion
$\varepsilon\mapsto I(y+\varepsilon\eta )$ an der Stelle $\varepsilon=0$.
Da die Intervallgrenzen nicht von $\varepsilon$ abhängen, können wir
die Ableitung unter das Integral nehmen:
\begin{align*}
\frac{d}{d\varepsilon}
I(y+\varepsilon\eta)
&=
\int_{x_0}^{x_1}
\frac{d}{d\varepsilon}
F(x,y(x)+\varepsilon\eta(x),y(x)+\varepsilon\eta'(x))
\,dx.
\intertext{Da $F$ differenzierbar ist, kann die Ableitung mit der
Kettenregel berechnet werden, sie ist}
&=
\int_{x_0}^{x_1}
\frac{\partial F}{\partial y}
(x,y(x)+\varepsilon\eta(x),y(x)+\varepsilon\eta'(x))
\eta(x)
\\
&\qquad
+
\frac{\partial F}{\partial y'}
(x,y(x)+\varepsilon\eta(x),y(x)+\varepsilon\eta'(x))
\eta'(x)
\,dx.
\intertext{Uns interessiert aber nur der Wert an der Stelle $\varepsilon=0$,
er ist}
\frac{d}{d\varepsilon}
I(y+\varepsilon\eta)
\bigg|_{\varepsilon=0}
&=
\int_{x_0}^{x_1}
\frac{\partial F}{\partial y}
(x,y(x),y'(x))
\,
\eta(x)
+
\frac{\partial F}{\partial y'}
(x,y(x),y'(x))
\,
\eta'(x)
\,dx
=0.
\end{align*}
Das Integral hängt von den verschiedenen Faktoren $\eta(x)$ und
von $\eta'(x)$ in den beiden Termen unter dem Integral ab.
Wir integrieren den zweiten Term partiell 
\begin{align*}
\int_{x_0}^{x_1}
\frac{\partial F}{\partial y'}(x,y(x),y'(x))\,\eta'(x)\,dx
&=
\biggl[
\frac{\partial F}{\partial y'}(x,y(x),y'(x))\,\eta(x)
\biggr]_{x_0}^{x_1}
\\
&\qquad
-
\int_{x_0}^{x_1}
\frac{d}{dx}
\frac{\partial F}{\partial y'}(x,y(x),y'(x))\,\eta(x)\,dx.
\end{align*}
Da $\eta(x_0)=\eta(x_1)=0$ verschwindet der erste Term
auf der rechten Seite, es bleibt
\begin{equation}
\frac{d}{d\varepsilon}
I(y+\varepsilon\eta)
\bigg|_{\varepsilon=0}
=
\int_{x_0}^{x_1}
\biggl(
\frac{\partial F}{\partial y}
(x,y(x),y'(x))
-
\frac{d}{dx}
\frac{\partial F}{\partial y'}
(x,y(x),y'(x))
\biggr)
\eta(x)
\,dx.
\label{buch:variation:eulerlagrange:eqn:ableitung}
\end{equation}
Dies kann auch als Skalarprodukt
\[
\biggl\langle 
\frac{\partial F}{\partial y}
(x,y(x),y'(x))
-
\frac{d}{dx}
\frac{\partial F}{\partial y'}
(x,y(x),y'(x))
,
\eta(x)
\biggr\rangle
=
0
\]
geschrieben werden.

%
% Die Differentialgleichung
%
\subsubsection{Die Differentialgleichung}
Da
\eqref{buch:variation:eulerlagrange:eqn:ableitung}
für jede differenzierbare Funktion $\eta$ mit Randwerten
$\eta(x_0)=\eta(x_1)$ verschwinden muss, folgt nach dem
Fundamentallemma~\ref{buch:variation:fundamentallemma:satz:fundamentallemma},
der folgende Satz. 

\begin{satz}[Euler-Lagrange]
\label{buch:variation:eulerlagrange:satz:eulerlagrange}
Wenn die mindestens zweimal stetig differenzierbare Funktion $y(x)$
unter allen solchen Funktionen mit $y(x_0)=y_0$ und $y(x_1)=y_1$
das Funktional
\[
I(y)
=
\int_{x_0}^{x_1}
F(x,y(x),y'(x))\,dx
\]
zu einem Maximum oder Minimum macht, dann ist $y(x)$ eine Lösung der
gewöhnlichen Differentialgleichung
\begin{equation}
\frac{d}{dx}
\frac{\partial F}{\partial y'}(x,y(x),y'(x))
-
\frac{\partial F}{\partial y}(x,y(x),y'(x))
=
0.
\label{buch:variation:eulerlagrange:eqn:eulerlagrange}
\end{equation}
Sie heisst die {\em Euler-Lagrange-Differentialgleichung}.
\end{satz}

Eine Lösung des Variationsproblems kann also als Lösung der
Euler-Lagrange-Dif\-fe\-ren\-tial\-glei\-chung mit den Randwerten
$y(x_0)=x_0$ und $y(x_1)=y_1$ gefunden werden.
Die Bedingung ist notwendig, aber nicht hinreichend.
Wie bei der Bestimmung eines Extremums bei Funktionen endlich
vieler Variablen garantiert das Verschwinden der Richtungsableitung
nicht, dass auch tatsächlich ein Extremum vorliegt.
Man sagt daher auch, dass eine Lösung $y(x)$ der
Euler-Lagrange-Differentialgleichung das Funktional $I(y)$
stationär macht.

Eine weitere Einschränkung ist, dass die Herleitung der
Euler-Lagrange-Differential\-gleichung vorausgesetzt hat,
dass die Lösungsfunktion $y(x)$ mindestens zweimal 
stetig differenzierbar ist.
Es gibt aber durchaus Variationsprobleme, deren Lösungen
nicht differenzierbar sind, dazu mehr im Kapitel~\ref{buch:chapter:nichtdiff}.

\begin{beispiel}
\label{buch:variation:eulerlagrange:beispiel:gerade}
Wir lösen das Variationsproblem von Beispiel
\ref{buch:variation:eulerlagrange:beispiel:gerade}
mit der Lagrange-Funk\-tion
\eqref{buch:variation:eulerlagrange:eqn:geradeL}.
Da die Lagrange-Funktion nicht von $y$ abhängt, bleibt von der 
Euler-Lagrange-Gleichung nur
\[
\frac{d}{dx}
\frac{\partial L}{\partial y'}(x,y(x),y'(x))
=
0
\]
übrig.
Berechnung der Ableitung liefert
\begin{equation}
\frac{\partial}{\partial y'}
\sqrt{1+y^{\prime 2}}
=
\frac{y'}{\sqrt{1+y^{\prime 2}}}.
\label{buch:variation:eulerlagrange:eqn:ableitungFyp}
\end{equation}
Die Ableitung nach $x$ ergibt
\begin{align*}
\frac{d}{dx}
\frac{\partial}{\partial y'}
\sqrt{1+y^{\prime 2}}
&=
\frac{d}{dx}
\frac{y'}{\sqrt{1+y^{\prime 2}}}
\\
&=
\frac{
y''\sqrt{1+y^{\prime 2}}-y'\cdot \frac{y'y''}{\sqrt{1+y^{\prime 2}}}
}{
1+y^{\prime 2}
}
\\
&=
y''
\frac{
1+y^{\prime 2}-y^{\prime 2}
}{
(1+y^{\prime 2})^{\frac32}
}.
\intertext{Die Euler-Lagrange-Differentialgleichung ist daher}
0
&=
\frac{y''}{(1+y^{\prime 2})^{\frac32}} .
\end{align*}
Der Nenner auf der rechten Seite ist immer $\ge 1$, die Gleichung kann
also nur erfüllt sein, wenn $y''=0$ ist.
Die Funktion $y(x)$ muss also eine lineare Funktion $y=ax+b$ sein.
Die Randbedingung wird erfüllt für die Geradengleichung
\[
y(x)
=
\frac{y_1-y_0}{x_1-x_0}(x-x_0) + y_0.
\]
Kürzeste Verbindungen in der Ebene sind daher Geraden.
\end{beispiel}

%
% Die Beltrami-Identität
%
\subsubsection{Die Beltrami-Identität}
\index{Beltrami-Identität}%
Falls die Lagrange-Funktion nicht explizit von $x$ abhängt, lässt sich
die Euler-Lagrange-Differentialgleichung in die folgende besonders
nützliche Form bringen.

\begin{satz}[Beltrami-Identität]
\label{buch:variation:eulerlagrange:satz:beltrami}
Ist $L(y,y')$ eine Lagrange-Funktion, die nicht von $x$ abhängt, und
$y(x)$ eine Lösung der Euler-Lagrange-Differentialgleichung, dann ist
\begin{equation}
H(y(x),y'(x)
=
L(y(x),y'(x))
-
y'(x)\cdot \frac{\partial L}{\partial y'}(y(x),y'(x))
=
\text{const}.
\label{buch:variation:eulerlagrange:eqn:beltrami}
\end{equation}
\end{satz}

\begin{proof}
Wir müssen die Ableitungen nach $x$ der Grösse $H(y(x),y'(x))$ 
berechnen:
\begin{align*}
\frac{d}{dx}H(y(x),y'(x))
&=
\frac{d}{dx}
\biggl(
L(y(x),y'(x))
-
y'(x)\cdot \frac{\partial L}{\partial y'}(y(x),y'(x))
\biggr)
\\
&=
\frac{d}{dx}L(y(x),y'(x))
-
\frac{d}{dx}\biggl(
y'(x)\cdot \frac{\partial L}{\partial y'}(y(x),y'(x))
\biggr)
\\
&=
\biggl(
y'(x)\cdot\frac{\partial L}{\partial y}(y(x),y'(x))
+
y''(x)\cdot\frac{\partial L}{\partial y'}(y(x),y'(x))
\biggr)
\\
&\qquad
-
\biggl(
y''(x)\cdot \frac{\partial L}{\partial y'}(y(x),y'(x))
+
y'(x)
\cdot
\frac{d}{dt}\frac{\partial L}{\partial y'}(y(x),y'(x))
\biggr)
\intertext{Darin heben sich die Terme mit $y''(x)$ weg und in den
verbleibenden Termen kann der gemeinsame Faktor $y'(x)$  ausgeklammert
werden, was auf}
&=
y'(x)\cdot\biggl(
\frac{\partial L}{\partial y}(y(x),y'(x))
-
\frac{d}{dx}
\frac{\partial L}{\partial y'}(y(x),y'(x))
\biggr)
\intertext{führt.
Da $y(x)$ eine Lösung der Euler-Lagrange-Differentialgleichung
ist, verschwindet die Klammer auf der rechten Seite und es folgt}
\frac{d}{dx}H(y(x),y'(x))
&=
0.
\end{align*}
Somit ist $H(y(x),y'(x))$ konstant.
\end{proof}

Die Beltrami-Identität ist oft nützlich, weil Sie die
Euler-Lagrange-Differentialgleichung als Differentialgleichung zweiter
Ordnung auf eine Differentialgleichung erster Ordnung reduziert.
Mit der Reduktion der Ordnung geht eine aus den Randbedingungen zu
bestimmende Konstante einher.



