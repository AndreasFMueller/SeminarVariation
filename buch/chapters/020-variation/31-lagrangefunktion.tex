%
% 31-lagrangefunktion.tex
%
% (c) 2023 Prof Dr Andreas Müller
%

%
% Die Lagrange-Funktion
%
\subsection{Die Lagrange-Funktion
\label{buch:variation:eulerlagrange:subsection:lagrange-funktion}}
Wir betrachten Variationsprobleme der folgenden Art.
Gesucht ist eine auf dem Intervall $[x_0,x_1]$ definirte
Funktion $y(x)$, die das Integral
\begin{equation}
I(y)
=
\int_{x_0}^{x_1}
F(x, y(x), y'(x))
\,dx
\label{buch:variation:eulerlagrange:eqn:funktional}
\end{equation}
maximiert oder minimiert.
Der Ausdruck~\eqref{buch:variation:eulerlagrange:eqn:funktional}
wird ein Funktional genannt.
Die Funktion
\[
F
\colon
\mathbb{R}\times
\mathbb{R}\times
\mathbb{R}
\to
\mathbb{R}
\]
von drei Variablen heisst die {\em Lagrange-Funktion}
des Funktionals \eqref{buch:variation:eulerlagrange:eqn:funktional}.

\begin{beispiel}
Die Lagrange-Funktion des Brachistochronenproblems ist
\[
F(x,y,y')
=
\sqrt{ \frac{1+y^{\prime 2}}{y} }.
\]
Die Funktion hängt nicht von $x$ ab, was bedeutet, dass eine
Verschiebung in $x$-Richtung die Form der Lösungsfunktion des
Variationsproblems nicht ändert.
\end{beispiel}

\begin{beispiel}
\label{buch:variation:eulerlagrange:beispiel:gerade}
Wir formulieren die Aufgabe, die kürzeste Verbindung der Punkte
$(x_0,y_0)$ und $(x_1,y_1)$ in einer Ebene zu finden, als Variationsproblem.
Die Länge einer Kurve $y(x)$ ist das Integral
\[
l(y)
=
\int_{x_0}^{x_1}
\sqrt{1+y'(x)^2}\,dx.
\]
Daraus lesen wir ab, dass die Lagrange-Funktion dieses Variationsproblems
\begin{equation}
F(x,y,y') = \sqrt{1+y^{\prime 2}}
\label{buch:variation:eulerlagrange:eqn:geradeL}
\end{equation}
ist.
Die Funktion hängt weder von $x$ noch von $y$ ab.
Dies ist auch zu erwarten, denn die Länge einer Kurve hängt nicht davon
ob, wo in der Ebene sie platziert ist.
Eine Verschiebung in $x$-Richtung würde das $x$-Argument ändern,
eine Verschiebung in $y$-Richtung die $y$-Werte.
Wäre $F$ von $x$ oder $y$ abhängig, könnte auch die Länge der Kurve
davon abhängen.
\end{beispiel}

