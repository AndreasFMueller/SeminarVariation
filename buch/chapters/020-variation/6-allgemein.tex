%
% 6-allgemein.tex
%
% (c) 2024 Prof Dr Andreas Müller
%
\section{Die allgemeine Theorie der ersten Variation
\label{buch:variation:section:allgemein}}
\kopfrechts{Die allgemeine Theorie der ersten Variation}
In Abschnitt~\ref{buch:variation:eulerlagrange:subsection:freierb}
wurde gezeigt, wie sich in einem Variationsproblem mit vorgegebenem
Randwert nur an einem Ende des Intervalls aus der ersten Variation
automatisch eine zusätzliche Randbedingung am anderen Ende des
Intervalls ergibt.
Die Endpunkte $x_0$ und $x_1$ des Intervalls waren aber immer noch 
fest.
In diesem Abschnitt soll gezeigt werden, wie sich diese Einschränkung
aufheben lässt.

%
% Allgemeine Variation einer Funktion
%
\subsection{Allgemeine Variation einer Funktion $y(x)$
\label{buch:variation:allgemein:subsection:vary}}
In Abschnitt~\ref{buch:variation:section:eulerlagrange} wurde die
erste Variation mit Hilfe von Funktionen der Form
\begin{equation}
x
\mapsto
y(x)+\varepsilon\eta(x)
\label{buch:variation:allgemein:eqn:ansatz}
\end{equation}
konstruiert.
Mit diesem Ansatz ist es nicht möglich, die Endpunkte $x_0$ und $x_1$
des Intervalls zu varieren.
Es war zwar möglich, die $y$-Werte zu varieren, was auf die zusätzliche
Randbedingung von
Satz~\ref{buch:variation:eulerlagrange:satz:zusaetzlicherb}
geführt hat.
Wir möchten aber sogar die Intervallgrenzen varieren können, dieser
Fall wird von den bisherigen Entwicklungen nicht abgedeckt.

%
% Verallgemeinerung des Variationsansatzes
%
\subsubsection{Verallgemeinerung des Variationsansatzes}
Statt des Ansatzes~\eqref{buch:variation:allgemein:eqn:ansatz}
verwenden wir jetzt eine Parametrisierung
\begin{equation}
y
\colon
[x_0(\varepsilon),x_1(\varepsilon)]
\to
\mathbb{R}
:
x\mapsto y(x,\varepsilon)
\label{buch:variation:allgemein:eqn:ansatz2}
\end{equation}
für $\varepsilon$-Werte in einer Umgebung von $0$.
Damit
\eqref{buch:variation:allgemein:eqn:ansatz2}
eine Variation der einer Funktion $y(x)$ beschreibt, soll
$y(x) = y(x,0)$ sein.
Die Funktionen $x_0(\varepsilon)$ und $x_1(\varepsilon)$ beschreiben
die Endpunkte des Definitionsintervalls der partiellen Funktion
$x\mapsto y(x,\varepsilon)$.

Der ursprüngliche Ansatz~\eqref{buch:variation:allgemein:eqn:ansatz}
ist in \eqref{buch:variation:allgemein:eqn:ansatz2} enthalten, wenn
man $x_0(\varepsilon)=x_0$, $x_1(\varepsilon)=x_1$ und
\[
y(x,\varepsilon) = y(x) + \varepsilon \eta(x)
\]
setzt.
In der Herleitung der Euler-Lagrange-Differentialgleichung wurde
entscheidend verwendet, dass man diese Funktion nach $\varepsilon$
ableiten kann.
Dass dies möglich ist, ist nach den bisherigen Forderungen an
$y(x,\varepsilon)$ nicht garantiert.
Es muss also zusätzlich Differenzierbarkeit nach $\varepsilon$
gefordert werden.

\begin{definition}[Allgemeine Variation einer Funktion]
\label{buch:variation:allgemein:def:variation}
Eine stetig nach beiden Variablen differenzierbare Funktion $y(x,\varepsilon)$
ist eine {\em allgemeine Variation} der Funktion $y(x)$, wenn es stetig
differenzierbare Funktionen
$x_0(\varepsilon)$ und $x_1(\varepsilon)$ gibt derart, dass
die partielle Funktion $x\mapsto y(x,\varepsilon)$ auf dem
Intervall $[x_0(\varepsilon),x_1(\varepsilon)]$ stetig differenzierbar
ist und für $\varepsilon=0$ mit $y(x)$ übereinstimmt: $y(x)=y(x,0)$.
Der Einfachheit halber wird die partielle Ableitung von $y(x,\varepsilon)$
nach $x$ auch
\[
y'(x,\varepsilon) = \frac{\partial y}{\partial x}(x,\varepsilon)
\]
geschrieben.
Für die Werte an den Endpunkten schreiben wir abkürzend
\begin{align*}
x_i &= x_i(0),
&
y_i &= y(x_i(0),0) = y(x_i)
&&\text{und}&
y_i'
=
y'(x_i(0),0)
=
\frac{\partial y(x(\varepsilon),\varepsilon)}{\partial\varepsilon}
\bigg|_{\varepsilon=0}
\end{align*}
für $i=0,1$.
\end{definition}

%
% Kurven, auf denen sich die Endpunkte bewegen
%
\subsubsection{Kurven, auf denen sich die Endpunkte bewegen}
Der Ansatz~\eqref{buch:variation:allgemein:eqn:ansatz2} beschreibt
Kurven, die die beiden Punkte
\begin{align*}
P_0(\varepsilon)
&=
(x_0(\varepsilon),y(x_0(\varepsilon),\varepsilon))
&&\text{und}&
P_1
&=
(x_1(\varepsilon),y(x_1(\varepsilon),\varepsilon))
\end{align*}
miteinander verbinden.
Die Endpunkte der Kurve bewegen sich also auf den Kurven
\begin{align*}
\varepsilon
&\mapsto
P_0(\varepsilon)
=
(x_0(\varepsilon),y(x_0(\varepsilon),\varepsilon)
&&\text{und}&
\varepsilon
&\mapsto
P_1(\varepsilon)
=
(x_1(\varepsilon),y(x_1(\varepsilon),\varepsilon).
\end{align*}
Da alle beteiligten Funktionen als differenzierbar vorausgesetzt wurden,
lassen sich auch die Tangenten dieser Kurven bestimmen.
Der Richtungsvektor der Tangente im Punkt $x_i=x_i(0)$ ist
\begin{align}
\vec{r}_i(\varepsilon)
=
\frac{d}{d\varepsilon}
\begin{pmatrix}
x_i(\varepsilon)\\
y(x_i(\varepsilon),\varepsilon)
\end{pmatrix}
\bigg|_{\varepsilon=0}
\label{buch:variation:allgemein:eqn:tangential}
\end{align}
für $i=0,1$.
Die Kurve $\varepsilon\mapsto P_i(\varepsilon)$ lässt sich daher für
genügend kleine $\varepsilon$ beliebig genau durch
\[
x\mapsto
\begin{pmatrix}
x_0(0)\\
y(x,0) 
\end{pmatrix}
+
\varepsilon
\frac{d}{d\varepsilon}
\begin{pmatrix}
x_i(\varepsilon)\\
y(x_i(\varepsilon),\varepsilon)
\end{pmatrix}
\bigg|_{\varepsilon=0}
\]
approximieren.

%
% Erste Variation
%
\subsection{Erste Variation einer einzelnen Funktion
\label{buch:variation:allgemein:subsection:var1}}
Wir betrachten jetzt wieder das Funktional
\[
I(y)
=
\int_{x_0}^{x_1} F(x,y(x),y'(x)) \,dx
\]
für die Lagrange-Funktion $f(x,y,y')$.
Eine allgemeine Variation der Funktion $y(x)$ im Sinne der
Definition~\ref{buch:variation:allgemein:def:variation} macht aus
$I(y)$ eine Funktion 
\[
I(y,\varepsilon)
=
\int_{x_0(\varepsilon)}^{x_1(\varepsilon)}
F(x,y(x,\varepsilon),y'(x,\varepsilon))
\,dx
\]
von $\varepsilon$.
Davon ist wieder die Ableitung nach $\varepsilon$ an der Stelle
$\varepsilon=0$ zu ermitteln.
Sie ist
\begin{align*}
\frac{\partial}{\partial\varepsilon}I(y,\varepsilon)
&=
F(x_1(\varepsilon),y(x_1(\varepsilon),\varepsilon),
y'(x_1(\varepsilon),\varepsilon))
\frac{dx_1(\varepsilon)}{d\varepsilon}
-
F(x_0(\varepsilon),y(x_0(\varepsilon),\varepsilon),
y'(x_0(\varepsilon),\varepsilon))
\frac{dx_0(\varepsilon)}{d\varepsilon}
\\
&\qquad
+
\int_{x_0(\varepsilon)}^{x_1(\varepsilon)}
\frac{\partial F}{\partial y}(x, y(x,\varepsilon), y'(x,\varepsilon))
\frac{\partial y}{\partial \varepsilon}(x,\varepsilon)
\,dx
\\
&\qquad
+
\int_{x_0(\varepsilon)}^{x_1(\varepsilon)}
\frac{\partial F}{\partial y'}(x, y(x,\varepsilon), y'(x,\varepsilon))
\frac{\partial^2 y}{\partial x\partial \varepsilon}(x,\varepsilon)
\,dx
\end{align*}
An der Stelle $\varepsilon=0$ vereinfacht sich die Variation zu
\begin{align}
\delta I(y)
&=
F(x_1,y(x_1),y'(x_1))
\frac{dx_0(0)}{d\varepsilon}
-
F(x_0,y(x_0),y'(x_0))
\frac{dx_1(0)}{d\varepsilon}
\notag
\\
&\qquad
+
\int_{x_0}^{x_1}
\frac{\partial F}{\partial y}(x,y(x),y'(x))
\frac{\partial y}{\partial \varepsilon}(x,0)
\,dx
\notag
\\
&\qquad
+
\int_{x_0}^{x_1}
\frac{\partial F}{\partial y'}(x,y(x),y'(x))
\frac{\partial^2 y}{\partial x\partial \varepsilon}(x,0)
\,dx.
\label{buch:variation:allgemein:eqn:var1}
\end{align}
Die Ableitung $y'$ im zweiten Integral kann mit partieller Integration 
vereinfacht werden:
\begin{align*}
\int_{x_0}^{x_1}
\frac{\partial F}{\partial y'}(x,y(x),y'(x))
\frac{\partial y'}{\partial \varepsilon}(x,0)
\,dx
&=
\biggl[
\frac{\partial F}{\partial y'}(x,y(x),y'(x))
\frac{\partial y}{\partial\varepsilon}(x,0)
\biggr]_{x_0}^{x_1}
\\
&\qquad
-
\int_{x_0}^{x_1}
\frac{d}{dx}
\frac{\partial F}{\partial y'}(x,y(x),y'(x))
\frac{\partial y}{\partial\varepsilon}(x,0)
\,dx.
\end{align*}
Eingesetzt in 
\eqref{buch:variation:allgemein:eqn:var1}
ist die allgemeine Variation des Funktionals $I(y)$ daher
\begin{equation}
\begin{aligned}
\delta I(y)
&=
F(x_1,y(x_1),y'(x_1))
\frac{dx_0(0)}{d\varepsilon}
-
F(x_0,y(x_0),y'(x_0))
\frac{dx_1(0)}{d\varepsilon}
\\
&\qquad
+
\biggl[
\frac{\partial F}{\partial y'}(x,y(x),y'(x))
\frac{\partial y}{\partial\varepsilon}(x,0)
\biggr]_{x_0}^{x_1}
\\
&\qquad
+
\int_{x_0}^{x_1}
\biggl(
\frac{\partial F}{\partial y}(x,y(x),y'(x))
-
\frac{d}{dx}\frac{\partial F}{\partial y'}(x,y(x),y'(x))
\biggr)
\frac{\partial y}{\partial \varepsilon}(x,0)
\,dx.
\end{aligned}
\label{buch:variation:allgemein:eqn:variation}
\end{equation}

Sei jetzt $y(x)$ eine Funktion, die das Funktional $I(y)$ stationär macht,
für jede Wahl einer Variation von $y(x)$ ist also $\delta I(y)=0$.
Dies gilt insbesondere auch für alle Variationen, die Endpunkte der
Kurve unverändert lassen.
Für eine solche Variation verschwinden die ersten drei Terme, es bleibt
nur das Integral.
Wie früher folgt daher, dass der Ausdruck in der Klammer im Integral
verschwinden muss.
Eine Lösung des Variationsproblems muss also die
Euler-Lagrange-Differentialgleichung erfüllen.

Die verbleibenden Terme in \eqref{buch:variation:allgemein:eqn:variation}
drücken aus, wie sich das Funktional $I(y)$ ändert, wenn sich die
Endpunkte der Kurve verschieben.
Die ersten zwei Terme behandeln offensichtlich den Fall, dass die 
Intervalgrenzen verschoben werden.
Der dritte Term behandelt die Änderung, die ausschliesslich von der
Änderung des zweiten Parameters von $y(x,\varepsilon)$ ausgeht.
Der $y$-Wert kann sich auch dadurch ändern, dass $x$ variert wird.
\begin{align*}
\frac{dy(x_i(\varepsilon),\varepsilon)}{d\varepsilon}
&=
\frac{\partial y(x_i(\varepsilon),\varepsilon)}{\partial x}
\frac{dx_i(\varepsilon)}{d\varepsilon}
+
\frac{\partial y(x_i(\varepsilon), \varepsilon)}{\partial \varepsilon}
\intertext{oder an der Stelle $\varepsilon=0$}
\frac{dy(x_i(0),0)}{d\varepsilon}
&=
\frac{\partial y(x_i(0),0)}{\partial x}
\frac{dx_i(0)}{d\varepsilon}
+
\frac{\partial y(x_i(0), 0)}{\partial \varepsilon}
\intertext{Aufgelöst nach dem letzten Term ist dies}
\frac{\partial y(x_i(0), 0)}{\partial \varepsilon}
&=
\frac{dy(x_i(0),0)}{d\varepsilon}
-
\frac{\partial y(x_i(0),0)}{\partial x}
\frac{dx_i(0)}{d\varepsilon}
\end{align*}
Dies können wir in die Variation
\eqref{buch:variation:allgemein:eqn:variation}
einsetzen und erhalten
\begin{equation}
\begin{aligned}
\delta I(y)
&=
\biggl(
F(x_1,y(x_1),y'(x_1))
-
\frac{\partial y(x_1(0),0)}{\partial x}
\frac{\partial F}{\partial y'}(x_1,y(x_1),y'(x_1))
\biggr)
\frac{dx_1(0)}{d\varepsilon}
\\
&\qquad
+
\frac{\partial F}{\partial y'}(x_1,y(x_1),y'(x_1))
\frac{dy(x_1(0),0)}{d\varepsilon}
\\
&\qquad
-
\biggl(
F(x_0,y(x_0),y'(x_0))
-
\frac{\partial y(x_0(0),0)}{\partial x}
\frac{\partial F}{\partial y'}(x_0,y(x_0),y'(x_0))
\biggr)
\frac{dx_0(0)}{d\varepsilon}
\\
&\qquad
-
\frac{\partial F}{\partial y'}(x_0,y(x_0),y'(x_0))
\frac{dy(x_0(0),0)}{d\varepsilon}
\\
&\qquad
+
\int_{x_0}^{x_1}
\biggl(
\frac{\partial F}{\partial y}(x,y(x),y'(x))
-
\frac{d}{dx}\frac{\partial F}{\partial y'}(x,y(x),y'(x))
\biggr)
\frac{\partial y}{\partial \varepsilon}(x,0)
\,dx.
\end{aligned}
\label{buch:variation:allgemein:eqn:variation}
\end{equation}
Von den ersten vier Terme lassen sich jeweils zwei als ein Skalarprodukt
des Vektors
\[
\vec{f}_i
=
\begin{pmatrix}
F(x_i,y(x_i),y'(x_i))
-
\frac{\partial F}{\partial y'}(x_i,y(x_i),y'(x_i))
\frac{\partial y(x_i(0),0)}{\partial\varepsilon}
\\
\frac{\partial F}{\partial y'}(x_i,y(x_i),y'(x_i))
\end{pmatrix}
\]
mit dem Richtungsvektor  $\vec{r}_i(0)$ von
\eqref{buch:variation:allgemein:eqn:tangential} schreiben.
Mit dieser Notation erhält die allgemeinste Form der ersten Variation
die Form
\begin{equation}
\delta I(y)
=
\vec{f}_1\cdot \vec{r}_i(0)
-
\vec{f}_0\cdot \vec{r}_0(0)
+
\int_{x_0}^{x_1}
\biggl(
\frac{\partial F}{\partial y}(x,y(x),y'(x))
-
\frac{d}{dx}\frac{\partial F}{\partial y'}(x,y(x),y'(x))
\biggr)
\frac{\partial y}{\partial \varepsilon}(x,0)
\,dx.
\end{equation}

\begin{satz}
\label{buch:variation:allgemein:satz:allgemeinvariation1}
Sei $y(x,\varepsilon)$ eine Variation der zweimal stetig differenzierbaren
Funktion $y(x)$ und sei
\[
\vec{r}_i(0)
=
\frac{d}{d\varepsilon}
\begin{pmatrix}
x_i(\varepsilon)\\
y(x_i(\varepsilon),\varepsilon)
\end{pmatrix}\bigg|_{\varepsilon=0}
\]
der Tangentialvektor an die Kurve
$\varepsilon\mapsto P_i(\varepsilon) = (x_i(\varepsilon),y(x_i(\varepsilon)))$,
auf der sich der Endpunkt $P_i$ der Kurve während der Variation
bewegt.
Dann ist die Variation
\begin{equation}
\delta I(y)
=
\vec{f}_1\cdot\vec{r}_1(0)
-
\vec{f}_0\cdot\vec{r}_0(0)
+
\int_{x_0}^{x_1}
\biggl(
\frac{\partial F}{\partial y}(x,y(x),y'(x))
-
\frac{d}{dx}\frac{\partial F}{\partial y'}(x,y(x),y'(x))
\biggr)
\frac{\partial y}{\partial \varepsilon}(x,0)
\,dx
\label{buch:variation:allgemein:eqn:allgemeinvariation1}
\end{equation}
mit
\[
\vec{f}_i
=
\begin{pmatrix}
F(x_i,y(x_i),y'(x_i))
-
y'(x_i)
\frac{\partial F}{\partial y'}(x_i,y(x_i),y'(x_i))
\\
\frac{\partial F}{\partial y'}(x_i,y(x_i),y'(x_i))
\end{pmatrix}.
\]
\end{satz}

Beliebige Variationen der Endpunkte sind nicht sinnvoll, denn
wären die Endpunkte beliebig wählbar, dann lässt sich $I(y)$ kleiner
(oder grösser) machen, indem der Weg geeignet verkürzt wird.
Das einzig mögliche Minimum für das Funktional wird dann für eine
Funktion erreicht, bei der $x_0=x_1$ gilt.
Es sind daher nur Variationen zielführend, die tangential an eine
Kurve in der $x$-$y$-Ebene sind.
Die Skalarprodukte in 
\eqref{buch:variation:allgemein:eqn:allgemeinvariation1}
bedeuten liefern dann Randbedingungen für $y(x)$.

%
% Erste Variation für mehrere Funktionen 
%
\subsection{Erste Variation mehrere Funktionen
\label{buch:variation:allgemein:subsection:var1n}}
Die Entwicklung des letzten Abschnitts lässt sich auch für mehrere
Funktionen $y_1(x),\dots,y_n(x)$ durchführen.
Seien also $y_k(x,\varepsilon)$ Variationen der Funktionen $y_k(x)$
mit der Variation $x_0(\varepsilon)$ und $x_1(\varepsilon)$ der 
Intervallenden.
Die Endpunkte der Kurven
\[
x\mapsto
=
(x
y_1(x,\varepsilon),\dots
y_n(x,\varepsilon))
\in \mathbb{R}^{n+1}
\]
bewegen sich jetzt auf der Kurve
\[
\varepsilon
\mapsto
P_i(\varepsilon)
=
(x_i(\varepsilon),
y_1(x_i(\varepsilon),\varepsilon),
\dots
y_n(x_i(\varepsilon),\varepsilon))
\]
mit dem Tangentialvektor
\[
\vec{r}_i(\varepsilon)
=
\frac{d}{d\varepsilon}
\begin{pmatrix}
x_i(\varepsilon)\\
y_1(x_i(\varepsilon),\varepsilon)\\
\vdots\\
y_n(x_i(\varepsilon),\varepsilon)
\end{pmatrix}
\]
für $i=0,1$.

Dem Vektor $\vec{f}_i$ im Falle einer Funktion entspricht der Vektor
\[
\vec{f}_i
=
\begin{pmatrix}
F(x_i,y_1(x_i),y'_1(x_i),\dots,y_n(x_i),y_n'(x_i))
-
\sum_{k=1}^n y'_k(x_i) \frac{\partial F}{\partial y_k'}
(x_i,y_1(x_i),y'_1(x_i),\dots,y_n(x_i),y_n'(x_i))
\\
\frac{\partial F}{\partial y_1}
(x_i,y_1(x_i),y'_1(x_i),\dots,y_n(x_i),y_n'(x_i))
\\
\vdots
\\
\frac{\partial F}{\partial y_n}
(x_i,y_1(x_i),y'_1(x_i),\dots,y_n(x_i),y_n'(x_i))
\end{pmatrix}.
\]

Diese Notation ist nicht wirklich lesbar, wir greifen daher wieder
zurück auf die vektorielle Schreibweise, in der 
$y(x)$ und die Variation $y(x,\varepsilon)$ die $n$-dimensionalen
Vektoren
\[
y(x)
=
\begin{pmatrix}
y_1(x)\\
\vdots\\
y_n(x)
\end{pmatrix}
\qquad
\text{und}
\qquad
y(x,\varepsilon)
=
\begin{pmatrix}
y_1(x,\varepsilon)\\
\vdots\\
y_n(x,\varepsilon)
\end{pmatrix}
\]
sind.
Auch die Funktion $F$ können wir jetzt wie in Abschnitt
kompakter als
Funktion
\[
F\colon
\mathbb{R}\times \mathbb{R}^n\times\mathbb{R}^n
\to
\mathbb{R}
:
(x,y,y')
\mapsto
F(x,y,y')
\]
schreiben.
Die unteren $n$ Komponenten der Richtungsvektoren sind einfach
die Ableitung des Vektors
$y(x(\varepsilon),\varepsilon)$ nach $\varepsilon$.

Die unteren $n$ Komponenten von $\vec{f}_i$ sind die Ableitungen von
$F$ nach den $n$ Variablen $y_1,\dots,y_n$.
Dies sieht aus wie der Gradient, wenn man nur die $y$-Variablen 
berücksichtigt.
Wir schreiben dafür daher $\operatorname{grad}_yF(x,y,y')$.
Analog können wir für den Vektor der Ableitungen von $F$ nach den
$y_k'$ als $\operatorname{grad}_{y'}F(x,y,y')$ schreiben.
Mit dieser Notation bekommen wir die sehr kompakte Form
\[
\vec{f}_i
=
\begin{pmatrix}
F(x_i,y(x_i),y'(x_i)) - \operatorname{grad}_{y'}(x_i,y(x_i),y'(x_i))
\\
\operatorname{grad}_y F(x_i,y(x_i),y'(x_i))
\end{pmatrix}
\]
für $i=0,1$.

Für die Variation finden wir dann
\begin{align*}
\delta I(y)
&=
\vec{f}_1\cdot \vec{r}_1(0)
-
\vec{f}_0\cdot \vec{r}_0(0)
\\
&\qquad
+
\int_{x_0}^{x_1}
\biggl(
\operatorname{grad}_yF(x,y(x),y'(x))
-
\frac{d}{dx}
\operatorname{grad}_{y'}F(x,y(x),y'(x))
\biggr)
\cdot
\frac{\partial y}{\partial\varepsilon}(x,0)
\,dx
\end{align*}
Damit die Variation verschwindet, müssen also die
Euler-Lagrange-Differentialgleichungen erfüllt sein.
Zusätzlich müssen aber die Skalarprodukte
$\vec{f}_1\cdot \vec{r}_1(0)$
und
$\vec{f}_0\cdot \vec{r}_0(0)$
verschwinden.
Dies bedeutet, dass die Vektoren $\vec{f}_i$ auf den erlaubten
Tangentialvektoren $\vec{r}_i(0)$ orthogonal sind.


%
% Erste Variation mit höheren Ableitungen
%
\subsection{Erste Variation mit höheren Ableitungen
\label{buch:variation:allgemein:subsection:var2h}}



