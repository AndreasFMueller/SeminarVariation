Das Fermat-Prinzip besagt, dass Licht immer den schnellsten Weg wählt.
Die Ausbreitungsgeschwindigkeit $v=c/n$ in einem Medium hängt vom
Brechungsindex $n$ ab.
In der Atmosphäre nimmt die Dichte mit zunehmender Höhe ab und damit
wird auch der Brechungsindex kleiner und nähert sich schliesslich $1$.
Nehmen Sie also an, dass $n(y)$ nur von der Höhe abhängt.
\begin{teilaufgaben}
\item
Stellen Sie ein Variationsprinzip auf für einen Lichtstrahl, der sich
vom Punkt $(x_1,y_1)$ zum Punkt $(x_2,y_2)$ ausbreitet.
\item
Bestimmen Sie die Lagrange-Funktion des Variationsprinzips.
\item
Bestimmen Sie die Differentialgleichung für den Strahl von $(x_1,y_1)$
und $(x_2,y_2)$.
\item 
Lösen Sie die Differentialgleichung für den Fall konstanten Brechungsindexes.
\item
Zeigen Sie: wenn $n(y)$ streng monoton zunimmt, dann krümmt sich der Strahl
der Lichtstrahl `nach oben', er wird in Richtung zunehmender $y$-Werte
abgelenkt.
\end{teilaufgaben}
An einem heissen Sommertag kann die Luft in Bodennähe so stark aufgeheizt
werden, dass $n(y)$ mit zunehmender Höhe zunimmt.
Teilaufgabe e) erklärt dann, warum diese heisse Luft 
Spiegel wirken und daher wie Wasser aussehen kann.

\begin{loesung}
\begin{teilaufgaben}
\item
Wir beschreiben den Lichtstrahl als Funktion $y(x)$ mit den Randbedingungen
\(
y(x_1)=y_1
\)
und
\(
y(x_2)=y_2
\).
Die vom Strahl durcheilte Strecke ist $\!\sqrt{1+y'(x)^2}$.
Damit lässt sich die Zeit berechnen, die der Strahl entlang der
Kurve $y(x)$ braucht daher die Zeit
\[
T(y)
=
\int_{x_1}^{x_2}
\frac{\!\sqrt{1+y'(x)^2}}{c/n(y(x))}
\,dx
=
\frac{1}{c}
\int_{x_1}^{x_2}
n(y(x))
\sqrt{1+y'(x)^2}
\,dx
\]
für den Weg vom Punkt $(x_1,y_1)$ zum Punkt $(x_2,y_2)$.
Das gesuchte Variationsprinzip ist daher $\delta T=0$.
\item
Die Lagrange-Funktion ist
\(
L(x,y,y') = n(y)\sqrt{1+y^{\prime 2}}
\).
Dabei haben wir die Konstante $1/c$ weggelassen, die keinen
Einfluss auf das Extremum hat.
\item
Die Euler-Lagrange-Differentialgleichung benötigt die partiellen
Ableitungen der Lagrange-Funktion nach $y$ und $y'$:
\begin{align*}
\frac{\partial L}{\partial y}
&=
n'(y) \sqrt{1+y^{\prime 2}}
\\
\frac{\partial L}{\partial y'}
&=
n(y)\frac{y'}{\!\sqrt{1+y^{\prime 2}}}.
\end{align*}
In der Euler-Lagrange-Differentialgleichung wird in die Ableitung nach $y'$
die Funktion $y(x)$ eingesetzt und dann nach $x$ abgeleitet, dies ergibt
\begin{align*}
\frac{d}{dx}
\frac{\partial L}{\partial y'}
&=
\frac{n'(x) y'(x)^2}{\!\sqrt{1+y'(x)^2}}
+
n(y(x))
y''(x)
\frac{
(1+y'(x)^2) - y'(x)^2
}{
(1+y'(x)^2)^{\frac32}
}
\\
&=
\frac{n'(x) y'(x)^2}{\!\sqrt{1+y'(x)^2}}
+
n(y(x))
\frac{
y''(x)
}{
(1+y'(x)^2)^{\frac32}
}.
\end{align*}
Die Euler-Lagrange-Differentialgleichung wird damit zu
\begin{align*}
0
=
\frac{d}{dx}\frac{\partial L}{\partial y'}
-
\frac{\partial L}{\partial y}
&=
\frac{n'(x) y'(x)^2}{\!\sqrt{1+y'(x)^2}}
+
n(y(x))
\frac{
y''(x)
}{
(1+y'(x)^2)^{\frac32}
}
-
n'(y(x))\sqrt{1+y'(x)^2}
\\
&=
\frac{
n'(x) y'(x)^2
}{
\!\sqrt{1+y'(x)^2}
}
+
n(y(x))
\frac{
y''(x)
}{
(1+y'(x)^2)^{\frac32}
}
-
\frac{
n'(y(x))(1+y'(x)^2)
}{
\!\sqrt{1+y'(x)^2}
}
\\
&=
n(y(x))
\frac{
y''(x)
}{
(1+y'(x)^2)^{\frac32}
}
-
\frac{
n'(y(x))
}{
\!\sqrt{1+y'(x)^2}
}
\\
&=
\frac{
n(y(x))
y''(x)
-
n'(y(x))
(1+y'(x)^2)
}{
(1+y'(x)^2)^{\frac32}
}.
\end{align*}
Da der Nenner nicht verschwinden kann, muss der Zähler verschwinden,
die Differentialgleichung wird damit
\[
y''(x)
-
\frac{n'(y(x))}{n(y(x))} (1 + y'(x)^2)
=
0.
\]
\item
Im Falle konstanten Brechungsindexes ist $n'(y)=0$ und damit $y''(x)=0$.
Die Lösungen sind Geraden.
\item
Wenn $n(y)$ mit zunehmendem $y$ zunmimmt, dann ist $n'(y)>0$ für alle $y$
und daher
\[
y''(x)
=
\frac{n'(y(x))}{n(y(x))} (1 + y'(x)^2)
>
0,
\]
da $n(y(x))\ge 1$ und $1+y'(x)^2\ge 1$.
Die positive zweite Ableitung bedeutet, dass die Steigung $y'(x)$ zunimmt,
d.~h.~der Graph von $y(x)$ krümmt sich zu zunehmenden $y$-Werten hin.
\qedhere
\end{teilaufgaben}
\end{loesung}


