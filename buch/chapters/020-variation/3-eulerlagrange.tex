%
% 3-eulerlagrange.tex
%
% (c) 2023 Prof Dr Andreas Müller
%
\section{Die Euler-Lagrange Differentialgleichung
\label{buch:variation:section:eulerlagrange}}
\kopfrechts{Die Euler-Lagrange Differentialgleichung}
Das Neuartige an der Aufgabenstellung des Brachistochronenproblems
war, dass eine Funktion gesucht war, so dass ein damit gebildetes
Integral eine Minimaleigenschaft erfüllt.
Für die damalige Mathematik war die Aufgabe, eine Funktion zu finden,
nicht neu.
Die Theorie der Differentialgleichungen war bereits entwickelt,
Newton hat die Infinitesimalrechnung ja erfunden, um damit die
Bewegungsgleichungen der Physik zu formulieren und zu lösen.
In einer Differentialgleichung werden Werte und Ableitungen einer
Funktion an einer einzigen Stelle miteinander verbunden.
Etwas salop formuliert sagt die Differentialgleichung in jedem
Punkt, in welche Richtung und mit welcher Krümmung die Funktionskurve
weiter zu zeichnen ist.

Im Brachistochronenproblem tragen aber alle Werte der gesuchten
Funktion zum Integral bei, es scheint daher auf den ersten Blick
nicht möglich, das Problem durch schrittweise Konstruktion
``von Punkt zu Punkt'' der Lösungskurve zu konstruieren.

Bernoullis Lösung des Brachistochrononproblems beruht auf der
Beobachtung, dass sich die Bedinung für die schnellste Bahn
durch eine Bedingung ersetzen lässt, die in jedem einzelnen
Punkt ausgewertet werden kann.
Das von ihm verwendete Fermat-Prinzip wurde ursprünglich ebenfalls
als eine globale Eigenschaft eines Lichtstrahls formuliert.
Aus dem Fermat-Prinzip folgt aber das Brechungsgesetzt, welches
sagt, dass die Richtung eines Strahls in einem Punkt genau dann
ändert, wenn sich dort auch der Brechungsindex der beiden Medien
ändert.
Das Fermat-Prinzip ist also ein Beispiel dafür, wie eine globale
Bedingung erfüllt werden kann, indem einer lokalen Regel in jedem
Punkt gefolgt wird.

Es ist das Verdienst von Euler und Lagrange, zu erkennen, dass diese
Übersetzung eines globalen Variationsproblems in ein lokales 
Problem immer möglich ist.
Es entsteht dabei die Euler-Lagrange-Differentialgleichung, welche
die Problemstellung auf die Lösung einer Differentialgleichung
reduziert.
Damit ist ein allgemein anwendbares Lösungsverfahren gefunden.
Zu einem Variationsproblem lässt sich immer eine Differentialgleichung
finden, welche die gesuchte Funktion als Lösung hat.

In diesem Abschnitt soll dieser indirekte Weg der Lösung von
Variationsaufgaben dargestellt werden.
Wir werden später zeigen, dass diese Vorgehensweise nicht immer
erfolgreich sein kann.
Zum Beispiel werden wir in Kapitel~\ref{buch:chapter:nichtdiff}
Variationsprobleme kennenlernen, deren Lösungskurven nicht
differenzierbar sind und daher auch nicht von einer Differentialgleichung
gefunden werden können.
Im Kapitel~\ref{buch:chapter:direkt} werden daher die sogenannten
direkten Methodn vorgestellt, die den Umweg über eine
Differentialgleichung vermeiden.

%
% Die Lagrange-Funktion
%
\subsection{Die Lagrange-Funktion}
Wir betrachten Variationsproblem der folgenden Art.
Gesucht ist eine auf dem Intervall $[x_0,x_1]$ definirte
Funktion $y(x)$, die das Integral
\begin{equation}
I(y)
=
\int_{x_0}^{x_1}
F(x, y(x), y'(x))
\,dx
\label{buch:variation:eulerlagrange:eqn:funktional}
\end{equation}
maximiert oder minimiert.
Der Ausdruck~\eqref{buch:variation:eulerlagrange:eqn:funktional}
wird ein Funktional genannt.
Die Funktion
\[
F
\colon
\mathbb{R}\times
\mathbb{R}\times
\mathbb{R}
\to
\mathbb{R}
\]
von drei Variablen heisst die {\em Lagrange-Funktion}
des Funktionals \eqref{buch:variation:eulerlagrange:eqn:funktional}.

\begin{beispiel}
Die Lagrange-Funktion des Brachistochronenproblems ist
\[
L(x,y,y')
=
\sqrt{ \frac{1+y^{\prime 2}}{y} }.
\]
Die Funktion hängt nicht von $x$ ab, was bedeutet, dass eine
Verschiebung in $x$-Richtung die Form der Lösungsfunktion des
Variationsproblems nicht ändert.
\end{beispiel}

\begin{beispiel}
\label{buch:variation:eulerlagrange:beispiel:gerade}
Wir formulieren die Aufgabe, die kürzeste Verbindung der Punkte
$(x_0,y_0)$ und $(x_1,y_1)$ in einer Ebene zu finden, als Variationsproblem.
Die Länge einer Kurve $y(x)$ ist das Integral
\[
l(y)
=
\int_{x_0}^{x_1}
\sqrt{1+y'(x)^2}\,dx.
\]
Daraus lesen wir ab, dass die Lagrange-Funktion dieses Variationsproblems
\begin{equation}
L(x,y,y') = \sqrt{1+y^{\prime 2}}
\label{buch:variation:eulerlagrange:eqn:geradeL}
\end{equation}
ist.
Die Funktion hängt weder von $x$ noch von $y$ ab.
Dies ist auch zu erwarten, denn die Länge einer Kurve hängt nicht davon
ob, wo in der Ebene sie platziert ist.
Eine Verschiebung in $x$-Richtung würde das $x$-Argument ändern,
eine Verschiebung in $y$-Richtung die $y$-Werte.
Wäre $L$ von $x$ oder $y$ abhängig, könnte auch die Länge der Kurve
davon abhängen.
\end{beispiel}

%
% Euler-Lagrange_Differentialgleichung
%
\subsection{Euler-Lagrange-Differentialgleichung}
Das Maximum oder Minimum einer Funktionen mehrere Variablen wurde
gefunden, indem die Richtungsableitung berechnet und $=0$ gesetzt
wurde.
Um die Funktion zu bestimmen, die ein Funktional $I(y)$ zu einem
Maximum oder Minimum macht, versuchen wir, die Idee der Richtungsableitung
für ein Funktional nachzuahmen.
Wir nehmen daher an, dass $y(x)$ eine Funktion ist, die das Funktional
$I(y)$ zu einem Minimum macht.
Für die Richtungsableitung addieren wir ein Vielfaches einer
Funktion $\eta(x)$, die Summe $y(x)+\varepsilon\eta(x)$ entspricht
dann einer Geraden mit Richtung $\eta(x)$ im Funktionenraum.
Die Funktionen $y(x)+\varepsilon\eta(x)$ sind aber nur dann Kandidaten
für eine Lösung des Problems, wenn immer noch
\begin{align*}
y(x_0) + \varepsilon \eta(x_0) &= y_0
&&\text{und}&
y(x_1) + \varepsilon \eta(x_1) &= y_1
\end{align*}
gilt.
Dies ist nur möglich, wenn $\eta(x_0)=\eta(x_1)=0$ ist.

Wir berechnen jetzt die Ableitung der Funktion
$\varepsilon\mapsto I(y+\varepsilon\eta )$ an der Stelle $\varepsilon=0$.
Da die Intervallgrenzen nicht von $\varepsilon$ abhängen, können wir
die Ableitung unter das Integral nehmen:
\begin{align*}
\frac{d}{d\varepsilon}
I(y+\varepsilon\eta)
&=
\int_{x_0}^{x_1}
\frac{d}{d\varepsilon}
F(x,y(x)+\varepsilon\eta(x),y(x)+\varepsilon\eta'(x))
\,dx.
\intertext{Da $F$ differenzierbar ist, kann die Ableitung mit der
Kettenregel berechnet werden, sie ist}
&=
\int_{x_0}^{x_1}
\frac{\partial F}{\partial y}
(x,y(x)+\varepsilon\eta(x),y(x)+\varepsilon\eta'(x))
\eta(x)
\\
&\qquad
+
\frac{\partial F}{\partial y'}
(x,y(x)+\varepsilon\eta(x),y(x)+\varepsilon\eta'(x))
\eta'(x)
\,dx.
\intertext{Uns interessiert aber nur der Wert an der Stelle $\varepsilon=0$,
er ist}
\frac{d}{d\varepsilon}
I(y+\varepsilon\eta)
\bigg|_{\varepsilon=0}
&=
\int_{x_0}^{x_1}
\frac{\partial F}{\partial y}
(x,y(x),y'(x))
\,
\eta(x)
+
\frac{\partial F}{\partial y'}
(x,y(x),y'(x))
\,
\eta'(x)
\,dx
=0.
\end{align*}
Das Integral hängt von den verschiedenen Faktoren $\eta(x)$ und
von $\eta'(x)$ in den beiden Termen unter dem Integral ab.
Wir integrieren den zweiten Term partiell 
\begin{align*}
\int_{x_0}^{x_1}
\frac{\partial F}{\partial y'}(x,y(x),y'(x))\,\eta'(x)\,dx
&=
\biggl[
\frac{\partial F}{\partial y'}(x,y(x),y'(x))\,\eta(x)
\biggr]_{x_0}^{x_1}
\\
&\qquad
-
\int_{x_0}^{x_1}
\frac{d}{dx}
\frac{\partial F}{\partial y'}(x,y(x),y'(x))\,\eta(x)\,dx.
\end{align*}
Da $\eta(x_0)=\eta(x_1)=0$ verschwindet der erste Term
auf der rechten Seite, es bleibt
\[
\frac{d}{d\varepsilon}
I(y+\varepsilon\eta)
\bigg|_{\varepsilon=0}
=
\int_{x_0}^{x_1}
\biggl(
\frac{\partial F}{\partial y}
(x,y(x),y'(x))
-
\frac{d}{dx}
\frac{\partial F}{\partial y'}
(x,y(x),y'(x))
\biggr)
\eta(x)
\,dx.
\]
Dies kann auch als Skalarprodukt
\[
\biggl\langle 
\frac{\partial F}{\partial y}
(x,y(x),y'(x))
-
\frac{d}{dx}
\frac{\partial F}{\partial y'}
(x,y(x),y'(x))
,
\eta(x)
\biggr\rangle
=
0
\]
geschrieben werden.
Da dies für jede differenzierbare Funktion $\eta$ mit Randwerten
$\eta(x_0)=\eta(x_1)$ gelten muss, folgt nach dem
Fundamentallemma~\ref{buch:variation:fundamentallemma:satz:fundamentallemma},
der folgende Satz. 

\begin{satz}[Euler-Lagrange]
\label{buch:variation:eulerlagrange:satz:eulerlagrange}
Wenn die mindestens zweimal stetig differenzierbare Funktion $y(x)$
unter allen solchen Funktionen mit $y(x_0)=y_0$ und $y(x_1)=y_1$
das Funktional
\[
I(y)
=
\int_{x_0}^{x_1}
F(x,y(x),y'(x))\,dx
\]
zu einem Maximum oder Minimum macht, dann ist $y(x)$ eine Lösung der
{\em Euler-Lagrange-Differentialgleichung}
\begin{equation}
\frac{d}{dx}
\frac{\partial F}{\partial y'}(x,y(x),y'(x))
-
\frac{\partial F}{\partial y}(x,y(x),y'(x))
=
0.
\label{buch:variation:eulerlagrange:eqn:eulerlagrange}
\end{equation}
\end{satz}

Eine Lösung des Variationsproblems kann also als Lösung der
Euler-Lagrange-Dif\-fe\-ren\-tial\-glei\-chung mit den Randwerten
$y(x_0)=x_0$ und $y(x_1)=y_1$ gefunden werden.
Die Bedingung ist notwendig, aber nicht hinreichend.
Wie bei der Bestimmung eines Extremums bei Funktionen endlich
vieler Variablen garantiert das Verschwinden der Richtungsableitung
nicht, dass auch tatsächlich ein Extremum vorliegt.
Man sagt daher auch, dass eine Lösung $y(x)$ der
Euler-Lagrange-Differentialgleichung das Funktional $I(y)$
stationär macht.

Eine weitere Einschränkung ist, dass die Herleitung der
Euler-Lagrange-Differential\-gleichung vorausgesetzt hat,
dass die Lösungsfunktion $y(x)$ mindestens zweimal 
stetig differenzierbar ist.
Es gibt aber durchaus Variationsprobleme, deren Lösungen
nicht differenzierbar sind, dazu mehr im Kapitel~\ref{buch:chapter:nichtdiff}.

\begin{beispiel}
Wir lösen das Variationsproblem von Beispiel
\ref{buch:variation:eulerlagrange:beispiel:gerade}
mit der Lagrange-Funk\-tion
\eqref{buch:variation:eulerlagrange:eqn:geradeL}.
Da die Lagrange-Funktion nicht von $y$ abhängt, bleibt von der 
Euler-Lagrange-Gleichung nur
\[
\frac{d}{dx}
\frac{\partial L}{\partial y'}(x,y(x),y'(x))
=
0
\]
übrig.
Berechnung der Ableitung liefert
\[
\frac{\partial}{\partial y'}
\sqrt{1+y^{\prime 2}}
=
\frac{y'}{\sqrt{1+y^{\prime 2}}}.
\]
Die Ableitung nach $x$ ergibt
\begin{align*}
\frac{d}{dx}
\frac{\partial}{\partial y'}
\sqrt{1+y^{\prime 2}}
&=
\frac{d}{dx}
\frac{y'}{\sqrt{1+y^{\prime 2}}}
\\
&=
\frac{
y''\sqrt{1+y^{\prime 2}}-y'\cdot \frac{y'y''}{\sqrt{1+y^{\prime 2}}}
}{
1+y^{\prime 2}
}
\\
&=
y''
\frac{
1+y^{\prime 2}-y^{\prime 2}
}{
(1+y^{\prime 2})^{\frac32}
}.
\intertext{Die Euler-Lagrange-Differentialgleichung ist daher}
0
&=
\frac{y''}{(1+y^{\prime 2})^{\frac32}} .
\end{align*}
Der Nenner auf der rechten Seite ist immer $\ge 1$, die Gleichung kann
also nur erfüllt sein, wenn $y''=0$ ist.
Die Funktion $y(x)$ muss also eine lineare Funktion $y=ax+b$ sein.
Die Randbedingung wird erfüllt für die Geradengleichung
\[
y(x)
=
\frac{y_1-y_0}{x_1-x_0}(x-x_0) + y_0.
\]
Kürzeste Verbindungen in der Ebene sind daher Geraden.
\end{beispiel}

%
% Freie Randbedingungen
%
\subsection{Freie Randbedingungen}


