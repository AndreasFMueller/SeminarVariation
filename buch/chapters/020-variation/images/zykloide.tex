%
% zykloide.tex -- Parametrisierung der Zykloide
%
% (c) 2021 Prof Dr Andreas Müller, OST Ostschweizer Fachhochschule
%
\documentclass[tikz]{standalone}
\usepackage{amsmath}
\usepackage{times}
\usepackage{txfonts}
\usepackage{pgfplots}
\usepackage{csvsimple}
\usetikzlibrary{arrows,intersections,math,calc}
\definecolor{darkred}{rgb}{0.8,0,0}
\begin{document}
\def\skala{1}
\def\r{1.74}
\begin{tikzpicture}[>=latex,thick,scale=\skala,
declare function = {
	X(\t) = \r*(\t-sin(180*\t/3.14159));
	Y(\t) = \r*(1-cos(180*\t/3.14159));
}]
\clip (-0.51,-4) rectangle ({2*\r*3.14159+1.1},0.5);

\draw[->] (-0.1,0) -- ({2*3.14159*\r+1},0) coordinate[label={$x$}];
\draw[->] (0,0.1) -- (0,{-2*\r-0.4}) coordinate[label={right:$y$}];

\def\T{2}
\fill[color=blue!20] ({\r*\T},{-\r})
	-- ({\r*\T},0) arc (90:{90+180*\T/3.14159}:\r) -- cycle;
\draw[color=blue,line width=1.4pt] ({\r*\T},0) arc (90:{90+180*\T/3.14159}:\r);
\draw[color=blue,line width=0.5pt] ({\r*\T)},{-\r}) circle[radius=\r];
\node[color=blue] at ($({\r*\T},{-\r})+({90+90*\T/3.14159}:{0.3*\r})$)
	{$\varphi$};

\begin{scope}
	\clip (-0.45,{-2*\r-0.1}) rectangle ({2*\r*3.14159+0.7},0);
	\draw[color=darkred,line width=1.4pt]
		plot[domain=-1.9:8,samples=100] ({X(\x)},{-Y(\x)});
\end{scope}

\fill[color=blue] ({\r*\T)},{-\r}) circle[radius=0.08];
\fill[color=blue] ({X(\T)},{-Y(\T)}) circle[radius=0.08];
\draw[color=blue] (0,0) -- ({\r*\T},0);
\draw[color=blue] ({\r*\T)},{-\r}) -- ({X(\T)},{-Y(\T)});
\draw[color=blue] ({\r*\T)},{-\r}) -- ({\r*\T},0);
\draw ({\r*\T},-0.05) -- ({\r*\T},0.05);
\node at ({\r*\T},0.05) [above] {$r\varphi\mathstrut$};
\node at ({\r*\T},{-0.5*\r}) [right] {$r$};
\draw[line width=0.3pt] ({X(\T)},{-Y(\T)})
	-- ({\r*\T},{-Y(\T)}) -- ({\r*\T},{-\r});
\node at ({\r*\T},{-0.5*(Y(\T)+\r)}) [right] {$r\cos\varphi$};
\node at ({0.5*(X(\T)+\r*\T)+0.3},{-Y(\T)+0.05}) [below] {$r\sin\varphi$};

\draw ({2*\r*3.14159},-0.05) -- ({2*\r*3.14159},0.05);
\node at ({2*\r*3.14159},0.05) [above] {$2\pi r\mathstrut$};
%\node at ({\r*\T},{-\r}) [above right] {$(rt,-r)\mathstrut$};

\node at (0,0.05) [above] {$0\mathstrut$};

\draw (-0.05,{-\r}) -- (0.05,{-\r});
\node at (-0.05,{-\r}) [left] {$r\mathstrut$};
\draw (-0.05,{-2*\r}) -- (0.05,{-2*\r});
\node at (-0.05,{-2*\r}) [left] {$2r\mathstrut$};

\node at ({X(\T)},{-Y(\T)}) [below left] {$P(\varphi)\mathstrut$};

\end{tikzpicture}
\end{document}

