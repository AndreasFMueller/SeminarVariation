%
% bernoulli.tex -- Problem von Bernoulli
%
% (c) 2021 Prof Dr Andreas Müller, OST Ostschweizer Fachhochschule
%
\documentclass[tikz]{standalone}
\usepackage{amsmath}
\usepackage{times}
\usepackage{txfonts}
\usepackage{pgfplots}
\usepackage{csvsimple}
\usetikzlibrary{arrows,intersections,math}
\definecolor{darkred}{rgb}{0.8,0,0}
\begin{document}
\def\skala{1}
\def\r{2}
\def\m{4}
\begin{tikzpicture}[>=latex,thick,scale=\skala,
declare function = {
	X(\t) = \r*(-sin(180*\t/3.14159)+\t);
	Y(\t) = -\r*(1-cos(180*\t/3.14159));
	etaX(\t,\a) = 0;
	etaY(\t,\a) = 0.015*\a*(X(\t)*(X(\m)-X(\t));
}]

\coordinate (A) at (0,0);
\coordinate (B) at ({X(\m)},{Y(\m)});
\coordinate (M) at ({X(0.7*\m)},{Y(0.7*\m)});

\foreach \a in {-4,-3,...,7}{
	\draw[color=gray!40,line width=1.0pt] plot[domain=0:\m,samples=40]
		({X(\x)+etaX(\x,\a)},{Y(\x)+etaY(\x,\a)});
}

\draw[color=darkred,line width=1.4pt]
	plot[domain=0:\m,samples=40] ({X(\x)},{Y(\x)});

\draw[->] (-0.1,0) -- ({X(\m)+0.5},0) coordinate[label={$x$}];
\draw[->] (0,0.1) -- (0,{-2*\r-1.4}) coordinate[label={left:$y$}];

\fill[color=darkred] (A) circle[radius=0.08];
\fill[color=darkred] (B) circle[radius=0.08];
\fill[color=darkred] (M) circle[radius=0.08];
\node[color=darkred] at (A) [above] {$A$};
\node[color=darkred] at (B) [right] {$B$};
\node[color=darkred] at (M) [above right] {$M$};

\end{tikzpicture}
\end{document}

