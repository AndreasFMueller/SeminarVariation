%
% grundaufgabe.tex -- Grundaufgaben der Variationsrechnung
%
% (c) 2021 Prof Dr Andreas Müller, OST Ostschweizer Fachhochschule
%
\documentclass[tikz]{standalone}
\usepackage{amsmath}
\usepackage{times}
\usepackage{txfonts}
\usepackage{pgfplots}
\usepackage{csvsimple}
\usetikzlibrary{arrows,intersections,math}
\begin{document}
\definecolor{darkred}{rgb}{0.8,0,0}
\def\skala{1}
\begin{tikzpicture}[>=latex,thick,scale=\skala,
declare function = {
	leftx(\h,\a) = \a*(-0.8*\h*\h-\h*0.2)+1.2;
	lefty(\h,\a) = 1.5*\a*\h+1.6;
	rightx(\h,\a) = \a*(0.6*\h*\h-0.4*\h)+4.1;
	righty(\h,\a) = 1.5*\a*\h+2.6;
	geradex(\xl,\yl,\xr,\yr,\t) = \t*\xr+(1-\t)*\xl;
	geradey(\xl,\yl,\xr,\yr,\t) = \t*\yr+(1-\t)*\yl;
	X(\t,\h,\al,\ar) = geradex(leftx(\h,\al),lefty(\h,\al),rightx(\h,\ar),righty(\h,\ar),\t);
	Y(\t,\h,\al,\ar) = geradey(leftx(\h,\al),lefty(\h,\al),rightx(\h,\ar),righty(\h,\ar),\t)+\h*\t*(1-\t)-0.1*sin(360*\t);
}]

\def\kurven#1{
	\foreach \h in {-1,-0.8,...,1}{
		\draw[color=darkred!20,line width=1pt] plot[domain=0:1]
			({X(\x,\h*(#1),\al,\ar)},{Y(\x,\h*(#1),\al,\ar)});
	}
	\draw[color=darkred,line width=1.2pt] plot[domain=0:1]
		({X(\x,0,\al,\ar)},{Y(\x,0,\al,\ar)});
}

\def\koordinatensystem#1{
	\draw[->] (-0.1,0) -- (5.5,0) coordinate[label={$x$}];
	\draw[->] (0,-0.1) -- (0,4.4) coordinate[label={right:$y$}];
	\node at (2.7,0) [below] {#1\strut};
}

\def\randlinks{
	\begin{scope}
	\clip (0,0) rectangle (5.2,4.2);
	\fill[color=blue!20] (-1,-1)
		--
		plot[domain=-1.1:1.1] ({X(0,\x,\al,\ar)},{Y(0,\x,\al,\ar)})
		--
		(-1,5)
		--
		cycle;
	\draw[color=blue] plot[domain=-1.1:1.1]
		({X(0,\x,\al,\ar)},{Y(0,\x,\al,\ar)});
	\end{scope}
}

\def\punktlinks{
	\fill[color=blue] ({X(0,0,0,0)},{Y(0,0,0,0)}) circle[radius=0.08];
	\node[color=blue] at ({X(0,0,0,0)},{Y(0,0,0,0)}) [left] {$(x_1,y_1)$};
}

\def\punktrechts{
	\fill[color=blue] ({X(1,0,0,0)},{Y(1,0,0,0)}) circle[radius=0.08];
	\node[color=blue] at ({X(1,0,0,0)},{Y(1,0,0,0)}) [right] {$(x_2,y_2)$};
}

\def\randrechts{
	\begin{scope}
	\clip (0,0) rectangle (5.2,4.2);
	\fill[color=blue!20] (6,0)
		--
		plot[domain=-1.1:1.1] ({X(1,\x,\al,\ar)},{Y(1,\x,\al,\ar)})
		--
		(7,5)
		--
		cycle;
	\draw[color=blue] plot[domain=-1.1:1.1]
		({X(1,\x,\al,\ar)},{Y(1,\x,\al,\ar)});
	\end{scope}
}

\begin{scope}
\def\al{0}
\def\ar{0}
\kurven{5}
\punktlinks
\punktrechts
\koordinatensystem{a)}
\end{scope}

\begin{scope}[xshift=6.7cm,yshift=0cm]
\def\al{0}
\def\ar{1}
\kurven{1}
\randrechts
\punktlinks
\koordinatensystem{c)}
\end{scope}

\begin{scope}[xshift=6.7cm,yshift=-5cm]
\def\al{1}
\def\ar{0}
\kurven{1}
\randlinks
\punktrechts
\koordinatensystem{d)}
\end{scope}

\begin{scope}[xshift=0cm,yshift=-5cm]
\def\al{1}
\def\ar{1}
\kurven{1}
\randrechts
\randlinks
\koordinatensystem{b)}
\end{scope}

\end{tikzpicture}
\end{document}

