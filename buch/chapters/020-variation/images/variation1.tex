%
% variation1.tex -- Variation mit freiem rechtem Ende
%
% (c) 2021 Prof Dr Andreas Müller, OST Ostschweizer Fachhochschule
%
\documentclass[tikz]{standalone}
\usepackage{amsmath}
\usepackage{times}
\usepackage{txfonts}
\usepackage{pgfplots}
\usepackage{csvsimple}
\usetikzlibrary{arrows,intersections,math}
\definecolor{darkred}{rgb}{0.8,0,0}
\begin{document}
\def\skala{1}
\def\xzero{1}
\def\xone{10}
\def\yzero{1}
\def\yone{4}
\begin{tikzpicture}[>=latex,thick,scale=\skala,
declare function={
	x(\t)   = (1-\t)*\xzero+\t*\xone;
	t(\x)   = (\x-\xzero)/(\xone-\xzero);
	eta(\t) = \t*(1-\t)*2*cos(180*\t)+0.5*sin(810*\t)+1.8*\t;
	y(\t)   = (1-\t)*\yzero+\t*\yone+4*\t*(1-\t);
}]

\draw (\xzero,-0.05) -- (\xzero,0.05);
\draw (-0.05,\yzero) -- (0.05,\yzero);
\draw[line width=0.2pt] (\xzero,0) -- (\xzero,\yzero);
\draw[line width=0.2pt] (0,\yzero) -- (\xzero,\yzero);
\node at (\xzero,-0.05) [below] {$x_1\mathstrut$};
\node at (-0.05,\yzero) [left] {$y_1\mathstrut$};

\draw (\xone,-0.05) -- (\xone,0.05);
\draw[line width=0.2pt] (\xone,0) -- (\xone,{y(1)+1.6*eta(1)});
\node at (\xone,-0.05) [below] {$x_2\mathstrut$};

\fill[color=darkred] (\xzero,\yzero) circle[radius=0.08];
\fill[color=darkred] (\xone,\yone) circle[radius=0.08];
\node at (\xzero,\yzero) [above left] {$P_1$};
\node at (\xone,\yone) [right] {$P_2$};

\foreach \e in {-1.6,-1.2,...,1.7}{
	\draw[color=gray,smooth] plot[domain=0:1] ({x(\x)},{y(\x)+\e*eta(\x)});
}
\node[color=gray] at (3.2,3.75) {$y(x)+\varepsilon\eta(x)$};

\draw[color=darkred,smooth,line width=1.4pt] plot[domain=0:1] ({x(\x)},{y(\x)});
\node[color=darkred] at (6.2,3.98) {$y(x)$};

\draw[->] (-0.1,0) -- (11.6,0) coordinate[label={$x$}];
\draw[->] (0,-0.1) -- (0,8) coordinate[label={right:$y$}];

\end{tikzpicture}
\end{document}

