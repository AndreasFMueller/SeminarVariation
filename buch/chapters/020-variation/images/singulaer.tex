%
% singulaer.tex -- template for standalon tikz images
%
% (c) 2021 Prof Dr Andreas Müller, OST Ostschweizer Fachhochschule
%
\documentclass[tikz]{standalone}
\usepackage{amsmath}
\usepackage{times}
\usepackage{txfonts}
\usepackage{pgfplots}
\usepackage{csvsimple}
\usetikzlibrary{arrows,intersections,math}
\definecolor{darkred}{rgb}{0.8,0,0}
\begin{document}
\def\skala{0.96}
\def\rmax{1.8}

\pgfmathparse{2*3.14159*0.82}
\xdef\tend{\pgfmathresult}

\pgfmathparse{\rmax*(\tend-sin(180*\tend/3.14159))}
\xdef\xend{\pgfmathresult}

\pgfmathparse{\rmax*(1-cos(180*\tend/3.14159))}
\xdef\yend{\pgfmathresult}

\begin{tikzpicture}[>=latex,thick,scale=\skala,
declare function={
	X(\t,\r) = \r*(\t-sin(180*\t/3.14159));
	Y(\t,\r) = \r*(1-cos(180*\t/3.14159));
	a(\y,\r) = 2*asin(sqrt(\y/(2*\r)));
}]

\foreach \r in {0.6,0.8,...,\rmax}{
%\def\r{1}
	\pgfmathparse{sqrt(\yend/(2*\r))}
	\xdef\x{\pgfmathresult}

%\node at (0,-3) [right] {$x=\x$};

	\pgfmathparse{2*3.14159*atan(\x/sqrt(1-\x*\x))/180}
	\xdef\a{\pgfmathresult}

	\pgfmathparse{\r*(\a-sin(180*\a/3.14159))}
	\xdef\deltax{\pgfmathresult}

	\pgfmathparse{\xend+\deltax}
	\xdef\Xshift{\pgfmathresult}

	\draw[color=darkred,line width=1.2pt]
		plot[domain=0:{3.14159},smooth]
			({X(\x,\r)},{-Y(\x,\r)});
	\draw[color=darkred,line width=1.2pt]
		plot[domain={-3.14159}:0,smooth]
			({X(\x,\r)+\Xshift},{-Y(\x,\r)});
	\draw[color=blue,line width=1.2pt]
		({X(3.14159,\r)},{-Y(3.14159,\r)})
		--
		({X(-3.14159,\r)+\Xshift},{-Y(-3.14159,\r)});
}

\draw[->] (-0.1,0) -- (12.3,0) coordinate[label={$x$}];
\draw[->] (0,0.1) -- (0,-4.0) coordinate[label={left:$y$}];

\fill[color=darkred] (0,0) circle[radius=0.08];
\node[color=darkred] at (0,0) [above left] {$A$};
\fill[color=darkred] (\xend,{-\yend}) circle[radius=0.08];
\node[color=darkred] at (\xend,{-\yend}) [below right] {$B$};


\end{tikzpicture}
\end{document}

