%
% variation.tex -- allgemeine Theorie der ersten Variation
%
% (c) 2021 Prof Dr Andreas Müller, OST Ostschweizer Fachhochschule
%
\documentclass[tikz]{standalone}
\usepackage{amsmath}
\usepackage{times}
\usepackage{txfonts}
\usepackage{pgfplots}
\usepackage{csvsimple}
\usetikzlibrary{arrows,intersections,math}
\definecolor{darkred}{rgb}{0.8,0,0}
\definecolor{hellblau}{rgb}{0.2,0.6,1}
\definecolor{darkgreen}{rgb}{0,0.6,0}
\begin{document}
\def\skala{1}
\begin{tikzpicture}[>=latex,thick,scale=\skala,
declare function = {
	x0(\a) = 1.5-\a;
	y0(\a) = 3*sqrt(2+\a)-1.9;
	x1(\a) = 8+cosh(1.2*\a+0.5)-\a;
	y1(\a) = 3+2.3*sinh(\a+0.5);
	x(\t,\a) = x0(\a)+\t*(x1(\a)-x0(\a));
	t(\x,\a) = (\x-x0(\a))/(x1(\a)-x0(\a));
	y(\x,\a) = 0.3*(\a+1)*t(\x,\a)*(1-t(\x,\a))+0.03*sin(180*t(\x,\a))-0.08*sin(3*180*t(\x,\a));
	X(\x,\a) = \x;
	Y(\x,\a) = y0(\a)*(1-t(\x,\a))+y1(\a)*t(\x,\a)+y(\x,\a)*(x1(\a)-x0(\a));
	f(\x) = 1;
	dx(\t,\a) = (X(x(\t,\a+0.01),\a+0.01)-X(x(\t,\a),\a));
	dy(\t,\a) = (Y(x(\t,\a+0.01),\a+0.01)-Y(x(\t,\a),\a));
	rl(\t,\a) = sqrt(dx(\t,\a)*dx(\t,\a)+dy(\t,\a)*dy(\t,\a));
	rx(\t,\a) = dx(\t,\a)/rl(\t,\a);
	ry(\t,\a) = dy(\t,\a)/rl(\t,\a);
}]

\draw[->] (-0.1,0) -- (12.3,0) coordinate[label={$x$}];
\draw[->] (0,-0.1) -- (0,8.4) coordinate[label={right:$y$}];

\draw[color=red] plot[domain=-1:1] ({x0(\x)},{y0(\x)});
\draw[color=blue] plot[domain=-1:1] ({x1(\x)},{y1(\x)});

\foreach \a in {-1,-0.8,...,1}{
	\draw[smooth,color=gray!40]
		plot[domain={x0(\a)}:{x1(\a)}]
			({X(\x,\a)},{Y(\x,\a)});
}

%\foreach \t in {0,0.1,...,1}{
%	\draw[smooth,color=gray!40] plot[domain=-1:1] ({x(\t,\x)},{y(x(\t,\x),\x)});
%}

\begin{scope}
	\clip[smooth]
	plot[domain={x0(-1)}:{x1(-1)}] (\x,{Y(\x,-1)})
	--
	plot[domain=-1:1] ({x(1,\x)},{Y(x(1,\x),\x)})
	--
	plot[domain={x1(1)}:{x0(1)}] (\x,{Y(\x,1)})
	--
	plot[domain=-1:1] ({x(0,-\x)},{Y(x(0,-\x),-\x)})
	--
	cycle;
	%\fill[color=orange!10] (0,0) rectangle (12,8);
	\foreach \x in {1,...,12}{
		\draw[color=gray!40] (\x,0) -- (\x,8);
	}
\end{scope}

\node[color=gray!80] at (4.79,6.68) {$y(x,\varepsilon)$};

\draw[color=darkred,line width=1.4pt,smooth]
	plot[domain={x0(0)}:{x1(0)}] ({X(\x,0)},{Y(\x,0)});
\node[color=darkred] at (5.3,4.95) {$y(x)$};

\draw[color=blue,line width=1.4pt,smooth]
	plot[domain=-1:1] ({X(x(0,\x),\x)},{Y(x(0,\x),\x)});
\draw[color=blue,line width=1.4pt,smooth]
	plot[domain=-1:1] ({X(x(1,\x),\x)},{Y(x(1,\x),\x)});

\node[color=blue] at (2.0,1.3)
	[rotate=-53] {$(x_0(\varepsilon),y_0(\varepsilon))$};
\node[color=blue] at (9.9,2.9)
	[rotate=-62] {$(x_1(\varepsilon),y_1(\varepsilon))$};

\draw[->,color=hellblau,line width=2pt]
	({X(x(1,0.6),0.6)},{Y(x(1,0.6),0.6)}) -- +({2*rx(1,0.6)},{2*ry(1,0.6)});

\fill[color=hellblau] ({X(x(1,0.6),0.6)},{Y(x(1,0.6),0.6)}) circle[radius=0.08];
\node[color=hellblau] at (0.4,3.7) {$\vec{r}_0(\varepsilon)$};

\draw[->,color=hellblau,line width=2pt]
	({X(x(0,0.4),0.4)},{Y(x(0,0.4),0.4)}) -- +({rx(0,0.4)},{ry(0,0.4)});
\fill[color=hellblau] ({X(x(0,0.4),0.4)},{Y(x(0,0.4),0.4)}) circle[radius=0.08];
\node[color=hellblau] at (9.7,8.2) {$\vec{r}_1(\varepsilon)$};

\draw[->,color=darkgreen,line width=2pt]
	({X(x(1,0),0)},{Y(x(1,0),0)}) -- +({1.5*ry(1,0)},{-1.5*rx(1,0)});
\fill[color=darkgreen] ({X(x(1,0),0)},{Y(x(1,0),0)}) circle[radius=0.08];
\node[color=darkgreen] at (10.2,4.7) {$\vec{f}_1$};

\draw[->,color=darkgreen,line width=2pt]
	({X(x(0,0),0)},{Y(x(0,0),0)}) -- +({-1.5*ry(0,0)},{1.5*rx(0,0)});
\fill[color=darkgreen] ({X(x(0,0),0)},{Y(x(0,0),0)}) circle[radius=0.08];
\node[color=darkgreen] at (0.8,1.3) {$\vec{f}_0$};

\end{tikzpicture}
\end{document}

