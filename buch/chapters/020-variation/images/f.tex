%
% f.tex -- Plot der Funktion f(x)
%
% (c) 2021 Prof Dr Andreas Müller, OST Ostschweizer Fachhochschule
%
\documentclass[tikz]{standalone}
\usepackage{amsmath}
\usepackage{times}
\usepackage{txfonts}
\usepackage{pgfplots}
\usepackage{csvsimple}
\usetikzlibrary{arrows,intersections,math}
\begin{document}
\def\skala{2}
\begin{tikzpicture}[>=latex,thick,scale=\skala]

\draw[->] (-2.2,0) -- (3.2,0) coordinate[label={$x$}];
\draw[->] (0,-0.05) -- (0,1.1) coordinate[label={left:$y$}];

\draw[color=red,line width=1.4pt]
	plot[domain=0.33:10,samples=100] ({1/\x},{exp(-\x)})
	--
	(0,0) -- (-2,0);

\node at (-0.1,0.6) [left] {$\displaystyle
y=f(x) = \begin{cases}
e^{-1/x}&\quad x>0\\
0&\quad \text{sonst}
\end{cases}$};

\end{tikzpicture}
\end{document}

