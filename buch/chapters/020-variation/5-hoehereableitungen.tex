%
% 5-hoehereableitungen.tex
%
% (c) 2023 Prof Dr Andreas Müller
%
\section{Höhere Ableitungen
\label{buch:variation:section:hoehereableitungen}}
\kopfrechts{Höhere Ableitungen}
Das Beispiel der Spline-Interpolation hat gezeigt, dass es manchmal
nötig ist, höhere Ableitungen als die erste in einem Funktional
zu berücksichtigen.
In diesem Abschnitt werden weitere Beispiele dazu vorgestellt.

%
% Lagrange-Funktion mit höheren Ableitungen
%
\subsection{Lagrange-Funktion mit höheren Ableitungen}
Die Euler-Lagrange-Differentialgleichung wurde bisher für Funktionale
hergeletet, deren Lagrange-Funktion von $x$, der Funktion $y(x)$ und
der ersten Ableitung $y'(x)$ abhängen.

\begin{definition}
Eine Funktion $L(x,y,y',\dots,y^{(n)})$ heisst eine Lagrange-Funktion
der Ordnung $n$.
\index{Lagrange-Funktion $n$-ter Ordnung}%
\end{definition}

\begin{beispiel}
Das Variationsproblem für die Spline-Integration hat die Lagrange-Funktion
zweiter Ordnung
\[
L(x,y,y',y'') = y^{\prime\prime 2}
\]
verwendet.
\end{beispiel}

Ein elastischer Stab speichert bei Verbiegung Energie, deren Dichte
entlang des Stabes proportional zur Krümmung ist.
Ist $s\mapsto \gamma(s)\in\mathbb{R}^2$ eine differenzierbare
Parametrisierung einer ebenen Kurve, die die Form eines dünnen elastischen
Stabes beschreibt, dann ist die Gesamtenergie des Stabes gegeben durch
das Integral
\[
E
=
\int_a^b \kappa(s)^2 \,ds.
\]
Ist $s$ ein Bogenlängenparameter, also $|\dot{\gamma}(s)|=1$, dann ist
die Krümmung die zweite Ableitung, also
\[
I = \int_a^b \ddot{\gamma}(s)^2\,ds.
\]
Diese Parameterdarstellung ist aber nicht die Form, in der wir bis jetzt
Kurven in der Ebene beschreiben konnten.

Sei $y(x)$ eine Funktion, deren Graph einen elastisch verbogenen Stab
in der Ebene beschreibt.
Die Krümmung des Graphen kann nach
\[
\kappa(x)
=
\frac{y''(x)}{(1+y'(x)^2)^{\frac32}}
\]
berechnet werden.
Die Energie des Stabes wird dann
\[
E
=
\int_a^b \frac{y''(x)^2}{(1+y'(x)^2)^3}\,dx.
\]
Die Lagrange-Funktion des Problems der Biegung eines Stabes ist daher
\[
L(x,y,y',y'')
=
\frac{y^{\prime\prime 2}}{(1+y^{\prime 2})^3}.
\]

%
% Die verallgemeinerte Euler-Lagrange-Differentialgleichung
%
\subsection{Die verallgemeinerte Euler-Lagrange-Differentialgleichung}
Auch für ein Variationsproblem mit einer Lagrange-Funktion, die höhere
Ableitungen enthält, lässt sich mit der mehr oder weniger gleichen
Vorgehensweise eine Differentialgleichung für die gesuchte Funktion
$y(x)$ herleiten.
Wie schon im Beispiel zur Spline-Interpolation angedeutet, wird es
notwendig sein, mehrmals partiell zu integrieren.

Sei also $L(x,y,y',\dots,y^{(n)})$ eine Lagrange-Funktion $n$-ter Ordnung
und sei eine Funktion $y(x)$ gesucht, die ein kritischer Punkt des Integrals
\[
I
=
\int_a^b L(x,y(x),y'(x),\dots,y^{(n)}(x))\,dx
\]
ist.
Wir berechnen wieder die erste Variation mit Hilfe einer beliebig
oft differenzierbaren Funktion $\eta(x)$, welche in den Endpunkten
des Intervalls zusammen mit allen Ableitungen verschwindet.
Die Variation ist definiert als die Richtungsableitung in Richtung
von $\eta(x)$ als
\begin{align*}
\delta I
&=
\frac{d}{dt}
\int_a^b
L(x,y(x)+t\eta(x),y'(x)+t\eta'(x),\dots,y^{(n)}(x)+t\eta^{(n)}(x))
\,dx\biggl|_{t=0}.
\intertext{Durch Ableitung nach $t$ finden wir}
&=
\int_a^b
\frac{\partial L}{\partial y}(x,y(x),y'(x),\dots,y^{(n)}(x))
\eta(x)
+
\frac{\partial L}{\partial y'}(x,y(x),y'(x),\dots,y^{(n)}(x))
\eta'(x)
\\
&\qquad
+
\dots
+
\frac{\partial L}{\partial y^{(n)}}(x,y(x),y'(x),\dots,y^{(n)}(x))
\eta^{(n)}(x)
\,dx.
\end{align*}
Die Terme mit Ableitungen von $\eta(x)$ können durch partielle
Integration in Terme umgewandelt werden, die nur die Funktion 
$\eta(x)$ enthalten:
\begin{align*}
\delta I
&=
\int_a^b
\frac{\partial L}{\partial y^{(k)}}
(x,y(x),y'(x),\dots,y^{(n)}(x))
\eta^{(k)}(x)
\,dx
\\
&=
\biggl[
\frac{\partial L}{\partial y^{(k)}}
(x,y(x),y'(x),\dots,y^{(n)}(x))
\eta^{(k-1)}(x)
\biggr]_a^b
\\
&\qquad
-
\int_a^b
\frac{d}{dx}
\frac{\partial L}{\partial y^{(k)}}
(x,y(x),y'(x),\dots,y^{(n)}(x))
\eta^{(k-1)}(x)
\,dx.
\intertext{Da die Ableitungen von $\eta(x)$ in den Intevallenden
verschwinden, ist dies gleichbedeutend mit}
&=
-\int_a^b \frac{d}{dx} \frac{\partial L}{\partial y^{(k)}}
(x,y(x),y'(x),\dots,y^{(n)}(x))
\eta^{(k-1)}(x)
\,dx.
\intertext{Durch Iterieren dieser Rechnung erhalten wir}
&=
(-1)^{k}
\int_a^b
\frac{d^k}{dx^k}
\frac{\partial L}{\partial y^{(k)}}
(x,y(x),y'(x),\dots,y^{(n)}(x))
\eta(x)
\,dx.
\end{align*}
Die Variation kann jetzt als
\begin{align*}
\delta I
&=
\int_a^b
\biggl(
\frac{\partial L}{\partial y}
-
\frac{d}{dx}
\frac{\partial L}{\partial y'}
+
\frac{d^2}{dx^2}
\frac{\partial L}{\partial y''}
-
\dots
+
(-1)^n
\frac{d^n}{dx^n}
\frac{\partial L}{\partial y^{(n)}}
\biggr)
\eta(x)\,dx
\end{align*}
geschrieben werden.
Die Variation $\delta I$ muss für jede Wahl von $\eta(x)$ verschwinden,
daher folgt aus dem Fundamentallemma, dass $y(x)$ die Differentialgleichung
\begin{equation}
\frac{\partial L}{\partial y}
-
\frac{d}{dx}
\frac{\partial L}{\partial y'}
+
\frac{d^2}{dx^2}
\frac{\partial L}{\partial y''}
-
\dots
+
(-1)^n
\frac{d^n}{dx^n}
\frac{\partial L}{\partial y^{(n)}}
=
0
\label{buch:variation:hohere:eqn:eulerlagrange}
\end{equation}
erfüllen muss.
Man beachte, dass wir in dieser Rechnung stillschweigend annehmen,
dass die Funktion $y(x)$ genügend oft stetig differenzierbar ist,
so dass die einzelnen Terme der Differentialgleichung
\eqref{buch:variation:hohere:eqn:eulerlagrange} wohldefiniert sind.

\begin{satz}[Euler-Lagrange-Differentialgleichung]
Eine genügend oft differenzierbare Funktion $y(x)$ ist ein stationärer
Punkt des Integrals
\[
I
=
\int_a^b
L(x,y(x),y'(x),\dots,y^{(n)}(x))
\,dx
\]
mit der Lagrange-Funktion $n$-ter Ordnung $L(x,y,y',\dots,y^{(n)})$,
wenn sie die die Euler-Lagrange-Differentialgleichung
\eqref{buch:variation:hohere:eqn:eulerlagrange} erfüllt.
\end{satz}


