%
% 6-mehrerefunktionen.tex
%
% (c) 2023 Prof Dr Andreas Müller
%
\section{Varationsproblem für mehrere Funktionen
\label{buch:variation:section:mehrerefunktionen}}
\kopfrechts{Mehrere Funktionen}
Nur sehr spezielle Kurven können als Graphen einer Funktion $y(x)$
dargestellt werden.
Als Lösung des isoperimetrischen Problems wird ein Kreis erwartet,
der sich sicher nicht so darstellen lässt.
Die natürliche Darstellung eines Kreises ist eine Parameterdarstellung
$t\mapsto(\cos t,\sin t)$, auf die die bisherige Theorie nicht
vorbereitet ist.

%
% Lagrange-Funktion für mehrere Funktionen
%
\subsection{Lagrange-Funktion für mehrere Funktionen
\label{buch:variation:mehrerefunction:subsection:lagrangefunktion}}
Eine Parameterdarstellung einer Kurve ist ein Vektor von Funktionen
$y_1(x),\dots,y_r(x)$.
Wir schreiben auch
\[
y(x)
=
\begin{pmatrix}
y_1(x)\\
\vdots\\
y_r(x)
\end{pmatrix}
\qquad\text{und}\qquad
y'(x)
=
\frac{d}{dx}
\begin{pmatrix}
y_1(x)\\
\vdots\\
y_r(x)
\end{pmatrix}
=
\begin{pmatrix}
y_1'(x)\\
\vdots\\
y_r'(x)
\end{pmatrix}
\]
für die Vektorfunktion und ihre erste Ableitung.

Eine {\em Lagrange-Funktion} für ein Variationsproblem wird von
der unabhängigen Variablen $x$, den Funktionswerten aller Funktionen
$y_1(x),\dots,y_r(x)$ und den Ableitungen $y'_1(x),\dots,y'_r(x)$
abhängen.
Sie ist also eine Funktion von $2r+1$ Variablen, die wir als
\begin{equation*}
F
\colon
\mathbb{R}^{2r+1}\to\mathbb{R}
:
(x,y_1,\dots,y_r,y'_1,\dots,y'_r)\mapsto F(x,y_1,\dots,y_r,y'_1,\dots,y'_r)
\end{equation*}
Mit dieser Schreibweise wird
\[
I(y)
=
\int_{x_0}^{x_1}
F\bigl(x,y_1(x),\dots,y_r(x),y'_1(x),\dots,y'_r(x)\bigr)\,dx
\]
das Funktional, das extremal gemacht werden soll.
Der Fall $r=1$ ist der bereits früher behandelte.

Besonders elegant lässt sich die Theorie formulieren, wenn wir
die Lagrange-Funktion als Funktion der vektorwertigen Argumente
$y$ und $y'$ schreiben:
\begin{equation}
F
\colon
\mathbb{R}\times\mathbb{R}^r\times\mathbb{R}^r
\to
\mathbb{R}
:
(x,y,y')
\mapsto F(x,y,y').
\label{buch:variation:mehrerefunktionen:eqn:Fvektor}
\end{equation}
Das zu variierende Integral wird dann
\[
I(y)
=
\int_{x_0}^{x_1}
F(x,y(x),y'(x))
\,dx
\]
In dieser Schreibweise unterscheidet sich das Problem formal
nicht mehr vom bereits behandelten.
Es kann in dieser Form aber nicht mit der bereits hergeleiteten
Euler-Lagrange-Differentialgleichung gelöst werden, da die
Ableitung $\partial F/\partial y$ nach einem Vektor $y$ nicht
definiert ist.

%
% Ableitungen nach den Vektorargumenten
%
\subsection{Ableitungen nach den Vektorargumenten
\label{buch:variation:mehrerefunktionen:subsection:vektorableitung}}
Sei $F$ eine Lagrange-Funktion der Form
\eqref{buch:variation:mehrerefunktionen:eqn:Fvektor}.
Wir möchten die Ableitung nach den Vektorargument $y$ und $y'$ 
definieren, damit wir später im
Abschnitt~\ref{buch:variation:mehrerefunktionen:subsection:eulerlagrange}
die Euler-Lagrange-Gleichungen so kompakt wie möglich schreiben können.

Da der Vektor $y$ aus den Variablen $y_1,\dots,y_r$ besteht und $y'$
aus den $y'_1,\dots,y'_r$, ist jede der Ableitungen
\[
\frac{\partial F}{\partial y_k}
\qquad\text{und}\qquad
\frac{\partial F}{\partial y'_k}
\]
wohldefiniert.
Sie bilden zwei Vektoren, die wir als
\begin{equation}
\frac{\partial}{\partial y}
F(x,y,y')
=
\begin{pmatrix}
\displaystyle
\frac{\partial}{\partial y_1}F(x,y,y')\\
\vdots\\
\displaystyle
\frac{\partial}{\partial y_r}F(x,y,y')
\end{pmatrix}
\qquad\text{und}\qquad
\frac{\partial}{\partial y'} F(x,y,y')
=
\begin{pmatrix}
\displaystyle
\frac{\partial}{\partial y'_1}F(x,y,y')\\
\vdots\\
\displaystyle
\frac{\partial}{\partial y'_r}F(x,y,y')
\end{pmatrix}
\end{equation}
schreiben wollen.
Ist $\eta(x)$ eine vektorwertige Funktion mit Komponenten
$\eta_k(x)$, dann kann man die in der Variation von
$f(\varepsilon)
=
F\bigl(x,y(x)+\varepsilon\eta(x),y(x)+\varepsilon\eta(x)\bigr)$
benötigte Ableitung nach $\varepsilon$ als
\begin{align*}
\frac{d}{d\varepsilon}f(\varepsilon)
&=
\sum_{k=1}^n
\frac{\partial}{\partial y_k}
F\bigl(x,y(x)+\varepsilon\eta(x),y'(x)+\varepsilon\eta'(x)\bigr)
\,
\eta_k(x)
\\
&\qquad
+
\frac{\partial}{\partial y'_k}
F\bigl(x,y(x)+\varepsilon\eta(x),y'(x)+\varepsilon\eta'(x)\bigr)
\,
\eta_k'(x)
\\
&=
\frac{\partial}{\partial y}
F\bigl(x,y(x),y'(x)\bigr)\cdot \eta(x)
+
\frac{\partial}{\partial y'}
F\bigl(x,y(x),y'(x)\bigr)\cdot \eta'(x)
\end{align*}
schreiben.
Der einzige Unterschied in der Notation gegenüber dem skalaren Fall
ist, dass jeweils das Skalarprodukt zur Multiplikation mit $\eta(x)$
bzw.~$\eta'(x)$ verwendet werden muss.

%
% Die Euler-Lagrange-Differentialgleichung
%
\subsection{Die Euler-Lagrange-Differentialgleichung
\label{buch:variation:mehrerefunktionen:subsection:eulerlagrange}}
Für eine Lagrange-Funktion für $r$ Funktionen $y_1(x),\dots,y_r(x)$
lässt sich die Variation des Integrals
\[
I
=
\int_a^b L(x,y_1(x),y_1'(x),\dots,y_r(x),y_r'(x))\,dx
\]
ganz analog zur einer Lagrange-Funktion
mit nur einer Funktion berechnen.
Dazu verwenden wir Funktionen $\eta_1(x),\dots,\eta_r(x)$, die
in den Endpunkten verschwinden und berechnen die Variation
\begin{align}
\delta I
&=
\frac{d}{dt}
\int_a^b
L\bigl(x,y_1(x)+t\eta_1(x),y_1'(x)+t\eta_1'(x),\dots,
y_r(x)+t\eta_r(x),y_r'(x)+t\eta_r'(x)\bigr)
\,dx
\bigg|_{t=0}
\notag
\\
&=
\int_a^b
\frac{\partial L}{\partial y_1}\,\eta_1(x)
+
\frac{\partial L}{\partial y'_1}\,\eta'_1(x)
+
\dots
\frac{\partial L}{\partial y_r}\,\eta_r(x)
+
\frac{\partial L}{\partial y'_r}\,\eta'_r(x)
\,dx.
\notag
\intertext{Die Terme mit Ableitungen von $\eta'_i(x)$ können durch partielle
Integration umgeformt werden:}
&=
\int_a^b
\frac{\partial L}{\partial y_1}\,\eta_1(x)
+\dots+
\frac{\partial L}{\partial y_r}\,\eta_r(x)
\,dx
+
\biggl[
\frac{\partial L}{\partial y'_1}\,\eta_1(x)
+
\frac{\partial L}{\partial y'_r}\,\eta_r(x)
\biggr]_a^b
\\
&\qquad
-
\int_a^b
\frac{d}{dx}
\frac{\partial L}{\partial y'_1}
\,
\eta_1(x)
+
\dots
+
\frac{d}{dx}
\frac{\partial L}{\partial y'_r}
\,
\eta_r(x).
\notag
\intertext{Der mittlere Term verschwindet, weil die Funktionen
$\eta_i(x)$ an den Intervallenden verschwinden.
Die Variation ist daher}
\delta I
&=
\int_a^b
\biggl(
\frac{\partial L}{\partial y_1}-\frac{d}{dx}\frac{\partial L}{\partial y'_1}
\biggr)\,\eta_1(x)
\,dx
+
\dots
+
\int_a^b
\biggl(
\frac{\partial L}{\partial y_r}-\frac{d}{dx}\frac{\partial L}{\partial y'_r}
\biggr)\,\eta_r(x)
\,dx.
\label{buch:variation:mehrere:eqn:summe}
\end{align}
Da die Funktionen $\eta_i(x)$ alle bis auf eine $=0$ gewählt werden können,
muss jedes der Integrale in \eqref{buch:variation:mehrere:eqn:summe}
verschwinden muss.
Nach dem Fundamentallemma folgt daher der folgende Satz.

\begin{satz}
\label{buch:variation:mehrere:satz:rfunktionen}
Das Integral
\[
\int_a^b L\bigl(x,y_1(x),y_1'(x),\dots,y_r(x),y'_r(x)\bigr)\,dx
\]
mit einer Lagrange-Funktion für $r$ Funktionen $y_1(x),\dots,y_r(x)$
nimmt einen stationären Wert an für Funktionen,
welche das Differentialgiechungssystem
\begin{equation}
\frac{\partial L}{\partial y_k}\bigl(x,y_1(x),y'_1(x),\dots,y_r(x),y'_r(x)\bigr)
-
\frac{d}{dx}
\frac{\partial L}{\partial y'_k}\bigl(x,y_1(x),y'_1(x),\dots,y_r(x),y'_r(x)\bigr)
=
0,
\label{buch:variation:mehrerefunktionen:eqn:reulerlagrange}
\end{equation}
$k=1,\dots,r$ erfüllen.
\end{satz}

In einen Variationsproblem sind im Allgemeinen geeignete Randbedingungen
notwendig, die die Lösung des Differentialgleichungssystems
\eqref{buch:variation:mehrerefunktionen:eqn:reulerlagrange}
eindeutig festlegen.

%
% Vektorform der Euler-Lagrange-Differentialgleichung
%
\subsubsection{Vektorform der Euler-Lagrange-Differentialgleichung}
Die Lagrange-Funktion $L(x,y_1,y'_1,\dots,y_r,y'_r)$ kann auch als
eine Funktion
\[
L\colon
\mathbb{R}\times\mathbb{R}^r \times \mathbb{R}^r
\to
\mathbb{R}
\]
geschrieben werden.
Die Ableitung $D_2L$ ist die Ableitung nach den Variablen $y_1,\dots,y_r$
während $D_3L$ die Ableitung nach den Variablen $y'_1,\dots,y'_r$ ist.
Gesucht ist wie früher ein kritischer Punkt des Integrals
\[
I
=
\int_a^b L\bigl(x,y(x),y'(x)\bigr)\,dx,
\]
wobei $y\colon[a,b]\to\mathbb{R}^r$ eine vektorwertige Funktion ist.
Um die Variation zu bilden, brauchen wir eine vektorwertige Funktion
$\eta\colon[a,b]\to\mathbb{R}^r$, deren Komponenten in den Endpunkten
des Intervalls verschwinden.
Die Variation ist dann
\begin{align*}
\delta I
&=
\frac{d}{dx}
\int_a^b L\bigl(x, y(x)+t\eta(x), y'(x)+t\eta'(x)\bigr)\,dx
\bigg|_{t=0}
\\
&=
\int_a^b
D_2L\bigl(x,y(x),y'(x)\bigr)\, \eta(x)
+
D_3L\bigl(x,y(x),y'(x)\bigr)\, \eta'(x)
\,dx.
\intertext{$D_2L$ ist eine Linearform, die auf den Vektor $\eta(x)$ 
angewendet wird, und analog für $D_3L$.
Für den Term mit $\eta'(x)$ verwenden wir wieder partielle Integration
und erhalten}
&=
\int_a^b D_2L\bigl(x,y(x),y'(x)\bigr)\,\eta(x)\,dx
+
\biggl[L\bigl(x,y(x),y'(x)\bigr)\,\eta(x)\biggr]_a^b
\\
&\qquad
-
\int_a^b \frac{d}{dx}D_eL\bigl(x,y(x),y'(x)\bigr)\, \eta(x)\,dx.
\intertext{Da die Komponenten von $\eta(x)$ an den Intervallenden
verschwinden, fällt der mittlere Term weg und es bleibt}
&=
\int_a^b
\bigl(
D_2L\bigl(x,y(x),y'(x)\bigr)-\frac{d}{dx}D_3L\bigl(x,y(x),y'(x)\bigr)
\bigr)
\,
\eta(x)\,dx.
\end{align*}
Nach dem Fundamentallemma folgt die $r$-dimensionale
Vektordifferentialgleichung
\[
D_2L\bigl(x,y(x),y'(x)\bigr) - \frac{d}{dx}D_3L\bigl(x,y(x),y'(x)\bigr) = 0.
\]
Beide Terme sind Linearformen auf $\mathbb{R}^r$, ihre Koeffizienten
sind die einzelnen Differentialgleichungen von
Satz~\ref{buch:variation:mehrere:satz:rfunktionen}.

%
% Parameterdarstellung von Kurven
%
\subsection{Parameterdarstellung von Kurven
\label{buch:variation:mehrerefunktionen:subsection:kurven}}
Ein Kurve im $r$-dimensionalen Raum $\mathbb{R}^r$ ist eine differenzierbare
Abbildung
\[
\gamma
\colon
\mathbb{R} \to \mathbb{R}^r
:
t \mapsto \gamma(t).
\]
Die Koordinaten von $\gamma(t)$ bilden $r$ Funktionen $\gamma_i(t)$
für $i=1,\dots,r$.
Die Ableitung nach dem Kurvenparameter $t$ ist der Tangentialvektor
\[
\frac{d}{dt}\gamma(t) = \dot{\gamma}(t),
\]
er wird mit dem Punkt bezeichnet.

Ein Variationsproblem für eine Kurve basiert auf einer Lagrange-Funktion
\[
L
\colon
\mathbb{R}\times\mathbb{R}^r\times\mathbb{R}^r
:
(t,x,\dot{x})
\mapsto
L(t,x,\dot{x}),
\]
aus der das Funktional
\begin{equation}
I(\gamma)
=
\int_{t_0}^{t_1}
L\bigl(t,\gamma(t),\dot{\gamma}(t)\bigr)
\,dt
\label{buch:variation:mehrerefunktionen:eqn:funktionalkurve}
\end{equation}
abgeleitet werden kann.
Dies ist natürlich ein Variationsproblem für die $r$ Komponentenfunktionen
des Vektors $\gamma(t)$.

Die Euler-Lagrange-Differentialgleichung für das Funktional
\eqref{buch:variation:mehrerefunktionen:eqn:funktionalkurve}
ist die $r$-di\-men\-sio\-na\-le Vektordifferentialgleichung
\begin{equation}
\frac{\partial L}{\partial x}\bigl(t,\gamma(t), \dot{\gamma}(t)\bigr)
-
\frac{d}{dt}
\frac{\partial L}{\partial \dot{x}}\bigl(t,\gamma(t),\dot{\gamma}(t)\bigr)
=
0.
\label{buch:variation:mehrerefunktionen:eqn:eulerlagrangekurve}
\end{equation}
Zusätzlich sind Randbedingungen notwendig, um die Lösung 
eindeutig festzulegen.
Lösungen der Differentialgleichung
\eqref{buch:variation:mehrerefunktionen:eqn:eulerlagrangekurve}
sind kritische Punkte des Funktionals
\eqref{buch:variation:mehrerefunktionen:eqn:funktionalkurve}.

%
% Das homogene Problem
%
%\subsection{Das homogene Variationsproblem
%\label{buch:variation:mehrerefunktionen:subsection:homogen}}






