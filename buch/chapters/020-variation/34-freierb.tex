%
% 33-freierb.tex
%
% (c) 2023 Prof Dr Andreas Müller
%

%
% Freie Randbedingungen
%
\subsection{Freie Randbedingungen
\label{buch:variation:eulerlagrange:subsection:freierb}}
In der Herleitung der Euler-Lagrange-Differentialgleichung wurde angenommen,
dass die Endpunkte der Lösungsfunktion durch $y(x_1)=y_0$ und $y(x_2)=y_1$
fest vorgegeben sind.
Diese Voraussetzung soll in diesem Abschnitt abgeschwächt werden.
Die Funktionswerte in den Endpunkten sollen also nicht mehr fest
vorgegeben sein.

\begin{beispiel}
\label{buch:variation:eulerlagrange:beispiel:freiegerade}
Im Beispiel~\ref{buch:variation:eulerlagrange:beispiel:gerade}
wurde die kürzeste Kurve zwischen zwei Punkten in der Ebene
gesucht und wie erwartet eine Gerade als Lösung gefunden.
Wenn die Werte $y_0$ und $y_1$ jetzt nicht mehr vorgegeben sind,
wird die kürzeste Verbindung zwischen den beiden Geraden
$x=x_1$ und $x=x_2$ gesucht.
Die Lösung dieses Problems ist nicht eindeutig, jede horizontale
Strecke mit $y_0=y_1$ ist eine Lösung.
\end{beispiel}

Das Beispiel zeigt, dass es im Allgemeinen immer noch die Vorgabe
eines der beiden Randwerte braucht, um die Lösung eindeutig zu
bestimmen.
Wir lösen daher die folgende Aufgabe, die nach
Abschnitt \ref{buch:variation:subsection:grundaufgaben}
ein Anfangspunkt-Endkurve-Problem ist.

\begin{aufgabe}
Gesucht ist eine zweimal stetig differnzierbare Funktion $y(x)$ auf
dem Intervall $[x_1,x_2]$ mit $y(x_1)=y_0$, die das Integral
\[
I(y)
=
\int_{x_1}^{x_2} F\bigl(x,y(x),y'(x)\bigr)\,dx
\]
zu einem Extremum macht.
Am rechten Ende des Intervalls ist der Funktion $y(x)$ keine
Randbedingung auferlegt.
\end{aufgabe}

\begin{proof}[Lösung]
%
% variation1.tex -- Variation mit freiem rechtem Ende
%
% (c) 2021 Prof Dr Andreas Müller, OST Ostschweizer Fachhochschule
%
\documentclass[tikz]{standalone}
\usepackage{amsmath}
\usepackage{times}
\usepackage{txfonts}
\usepackage{pgfplots}
\usepackage{csvsimple}
\usetikzlibrary{arrows,intersections,math}
\definecolor{darkred}{rgb}{0.8,0,0}
\begin{document}
\def\skala{1}
\def\xzero{1}
\def\xone{10}
\def\yzero{1}
\def\yone{4}
\begin{tikzpicture}[>=latex,thick,scale=\skala,
declare function={
	x(\t)   = (1-\t)*\xzero+\t*\xone;
	t(\x)   = (\x-\xzero)/(\xone-\xzero);
	eta(\t) = \t*(1-\t)*2*cos(180*\t)+0.5*sin(810*\t)+1.8*\t;
	y(\t)   = (1-\t)*\yzero+\t*\yone+4*\t*(1-\t);
}]

\draw (\xzero,-0.05) -- (\xzero,0.05);
\draw (-0.05,\yzero) -- (0.05,\yzero);
\draw[line width=0.2pt] (\xzero,0) -- (\xzero,\yzero);
\draw[line width=0.2pt] (0,\yzero) -- (\xzero,\yzero);
\node at (\xzero,-0.05) [below] {$x_0\mathstrut$};
\node at (-0.05,\yzero) [left] {$y_0\mathstrut$};

\draw (\xone,-0.05) -- (\xone,0.05);
\draw[line width=0.2pt] (\xone,0) -- (\xone,{y(1)+1.6*eta(1)});
\node at (\xone,-0.05) [below] {$x_1\mathstrut$};

\fill[color=darkred] (\xzero,\yzero) circle[radius=0.08];
\fill[color=darkred] (\xone,\yone) circle[radius=0.08];
\node at (\xzero,\yzero) [above left] {$P_0$};
\node at (\xone,\yone) [right] {$P_1$};

\foreach \e in {-1.6,-1.2,...,1.7}{
	\draw[color=gray,smooth] plot[domain=0:1] ({x(\x)},{y(\x)+\e*eta(\x)});
}
\node[color=gray] at (3.2,3.75) {$y(x)+\varepsilon\eta(x)$};

\draw[color=darkred,smooth,line width=1.4pt] plot[domain=0:1] ({x(\x)},{y(\x)});
\node[color=darkred] at (6.2,3.98) {$y(x)$};

\draw[->] (-0.1,0) -- (11.6,0) coordinate[label={$x$}];
\draw[->] (0,-0.1) -- (0,8) coordinate[label={right:$y$}];

\end{tikzpicture}
\end{document}


Sei $y(x)$ eine Lösung der Aufgabe und sei $y_1:=y(x_2)$ der Wert
der Lösung am rechten Rand des Intervalls.
Wir berechnen wieder die Variation von $I(y)$ mit Hilfe von
stetig differenzierbaren Funktionen $\eta(x)$, die jetzt aber 
nur noch die Bedingungn $\eta(x_1)=0$ erfüllen müssen
(Abbildung~\ref{buch:variation:fig:variation1}).
Die Richtungsableitung ist wie früher
\begin{align*}
\frac{d}{d\varepsilon}
I(y+\varepsilon\eta)
\bigg|_{\varepsilon=0}
&=
\frac{d}{d\varepsilon}
\int_{x_1}^{x_2}
F\bigl(x,y(x)+\varepsilon\eta(x),y'(x)+\varepsilon\eta'(x)\bigr)\,dx
\\
&=
\int_{x_1}^{x_2}
\frac{\partial F}{\partial y}\bigl(x,y(x),y'(x)\bigr) 
\eta(x)
+
\frac{\partial F}{\partial y'}
\bigl(x,y(x),y'(x)\bigr)
\eta'(x)
\,dx
\intertext{und mit partieller Integration}
&=
\biggl[
\frac{\partial F}{\partial y'}\bigl(x,y(x),y'(x)\bigr) \, \eta(x)
\biggr]_{x_1}^{x_2}
\\
&\qquad
+
\int_{x_1}^{x_2}
\biggl(
\frac{\partial F}{\partial y}\bigl(x,y(x),y'(x)\bigr)
-
\frac{d}{dx}
\frac{\partial F}{\partial y'}\bigl(x,y(x),y'(x)\bigr)
\biggr)
\,
\eta(x)
\,dx.
\end{align*}
Im Gegensatz zu früher können wir jetzt aber nicht mehr
schliessen, dass der erste Term verschwindet, da $y(x_2)$ nicht
mehr als $=0$ vorausgesetzt wird.
Vielmehr erhalten wir für die erste Variation
\begin{align*}
\delta I(y)
&=
\frac{\partial F}{\partial y'} \bigl(x_2,y(x_2),y'(x_2)\bigr)\, \eta(x_2)
\\
&\qquad
+
\int_{x_1}^{x_2}
\biggl(
\frac{\partial F}{\partial y}\bigl(x,y(x),y'(x)\bigr)
-
\frac{d}{dx}
\frac{\partial F}{\partial y'}\bigl(x,y(x),y'(x)\bigr)
\biggr)
\,
\eta(x)
\,dx.
\end{align*}
Die Klammer im Integral ist von der Euler-Lagrange-Differentialgleichung
her bekannt.
Es ist aber ein weiterer Term hinzugekommen, der genau dann
verschwindet, wenn auch $\eta(x_2)=0$ ist.

Ist $\eta(x_2)=0$,
dann ist $y(x)$ natürlich erst recht eine Lösung des Problems, das
Funktional $I(y)$ mit den {\em zwei} Randbedingungen
$y(x_1)=y_0$ und $y(x_2)=y_1$ zu einem Extremum zu machen, also
muss die Funktion $y(x)$ sicher die Euler-Lagrange-Differentialgleichung
erfüllen.
Die Klammer im Integral wird daher verschwinden, die Variation
reduziert sich auf den ersten Term
\[
\delta I(y)
=
\frac{\partial F}{\partial y'} \bigl(x_2,y(x_2),y'(x_2)\bigr)\, \eta(x_2)
=
0.
\]
Sie verschwindet nur dann für alle zulässigen Funktionen $\eta(x)$, wenn
\begin{equation*}
\frac{\partial F}{\partial y'}\bigl(x_2,y(x_2),y'(x_2)\bigr)=0
\end{equation*}
gilt.
Dies ist eine zusätzliche Randbedingung für die Funktion $y(x)$, geschrieben
in einer impliziten Form.
\end{proof}

Wir halten das Resultat der Aufgabenlösung als Satz fest:

\begin{satz}
\label{buch:variation:eulerlagrange:satz:zusaetzlicherb}
Wenn die zweimal stetig differenzierbare Funktion $y(x)$ mit dem
Randwert $y(x_1)=y_0$ das Integral
\[
I(y)
=
\int_{x_1}^{x_2} F\bigl(x,y(x),y'(x)\bigr)\,dx
\]
zu einem Extremum macht, dann erfüllt sie am rechten Intervallende
die Randbedingung
\begin{equation}
\frac{\partial F}{\partial y'}\bigl(x_2,y(x_2),y'(x_2)\bigr)=0.
\label{buch:variation:eulerlagrange:eqn:zusaetzlicherb}
\end{equation}
zusätzlich zur Euler-Lagrange-Gleichung für die Lagrange-Funktion $F$.
\end{satz}

\begin{beispiel}
\label{buch:variation:eulerlagrange:beispiel:einseitigegerade}
Wir betrachten wieder das Funktional
\[
I(y)
=
\int_{x_1}^{x_2}
\sqrt{1+y^{\prime 2}(x)}
\,dx
\]
mit der einzigen Randbedingung $y(x_1)=y_0$, der Funktionswert auf 
der rechten Seite ist nicht vorgebeben.
Der Satz~\eqref{buch:variation:eulerlagrange:satz:zusaetzlicherb}
besagt zunächst, dass die Lösungsfunktion wieder eine Gerade sein
muss, da die Euler-Lagrange-Gleichung erfüllt sein muss.
Zusätzlich muss aber auch die Randbedingung
\eqref{buch:variation:eulerlagrange:eqn:zusaetzlicherb}
am rechten Ende des Intervalls erfüllt sein.
Die Ableitung der Lagrange-Funktion ist in diesem Fall durch
\eqref{buch:variation:eulerlagrange:eqn:ableitungFyp}
gegeben, es muss also
\[
\frac{y'(x_2)}{\sqrt{1+y'(x_2)^2}}
=
0
\qquad\Rightarrow\qquad y'(x_2)=0
\]
gelten.
Die Lösung ist daher wie erwartet eine horizontale Strecke.
\end{beispiel}


