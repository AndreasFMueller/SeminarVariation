%
% 2-hamilton.tex
%
% (c) 2023 Prof Dr Andreas Müller
%
\section{Hamilton-Mechanik
\label{buch:mechanik:section:hamilton}}
\kopfrechts{Hamilton-Mechanik}
In Abschnitt~\ref{buch:hamiltonjacobi:section:kanonisch} wurde gezeigt,
wie sich für ein Variationsprinzip ein Differentialgleichungssystem
erster Ordnung finden lässt.
Die zu $x$ konjugierte Koordinaten $p$ bekommt damit die physikalische
Bedeutung des Impulses und der $x$-$p$-Raum ist der Phasenraum.

%
%  Kanonische Koordinaten
%
\subsection{Kanonische Koordinaten}
Wir gehen von einem mechanischen System aus, welches durch das
Variationsprinzip 
\begin{equation}
\delta
\int
L(t, q(t), \dot{q}(t))\,dt
=
0
\label{buch:mechanik:hamilton:eqn:variation}
\end{equation}
mit der Lagrange-Funktion $L(t,q,\dot{q})$, von $n$ Variablen
$q=(q_1,\dots,q_n)$ beschrieben wird.
Im Beispiel des Kapitza-Pendels waren dies die Koordinaten $\varphi(t)$
und $h(t)$.
Nach dem Prinzip von Maupertuis ist $L$ die Differenz von kinetischer
und potenzieller Energie.

Die Verwendung der verallgemeinerten Koordinaten $q_i$ vereinfacht
die Herleitung der Bewegungsgleichungen.
Es ist oft einfach, die kartesischen Koordinaten, in denen wir Ausdrücke
für die kinetische und die potenzielle Energie bereits kennen, durch
die verallegemeinerte Koordinaten auszudrücken.
Das Beispiel des Kapiza-Pendels von
Abschnitt \ref{buch:mechanik:lagrange:subsection:kapiza}
demonstriert dies ebenso wie das Kapitel~\ref{chapter:doppelpendel}
über das Doppelpendel.

Die Verwendung der Variablen $q_i$ haben jedoch zwei Nachteile.
Zunächst ist die Euler-Lagrange-Differentialgleichung eine
Differentialgleichung zweiter Ordnung für die $n$ Funktion 
$q_i(t)$.
Die Verwendung des Impulses ermöglicht, die Bewegungsgleichungen
als Differentialgleichungen erster Ordnung zu schreiben.
Die Idee des Phasenraumes mit den Orts- und Impulskoordinaten
und der Dimension $2n$ ermöglicht eine geometrische und sehr erfolgreiche
Beschreibung der Lösungen als Fluss.
Es ist daher wünschbar, auch für ein allgemeines System mit Koordinaten
$q_i$ Bewegungsgleichungen erster Ordnung zu finden.

Aus rein mathematischer Sicht kann man jedes Differentialgleichungssystem
\[
y'' = f(x,y,y')
\]
durch die Substition $v=y'$ als Differentialgleichung
\[
\frac{d}{dx}
\begin{pmatrix}
y\\ v
\end{pmatrix}
=
\begin{pmatrix}
v\\
f(x,y,y')
\end{pmatrix}
\]
zweiter Ordnung schreiben.
Bereits in der newtonschen Mechanik hat sich die rein kinematische
Grösse $v=y'$ als nicht unbedingt zweckmässig erwiesen.

Der Impuls der newtonschen Mechanik ermöglicht, Erhaltungssätze auf besonders
gut nachvollziehbare Weise zu formulieren.
Solche Erhaltungssätze können auch in komplizierteren Systemen
vorhanden sein, die gewählten verallgemeinerten Koordinaten machen
es jedoch schwierig, die Erhaltungsgrössen zu finden.
Der Satz~\ref{buch:symmetrien:noether:satz:noether} von Emmy Noether
konstruiert zwar solche Erhaltungsgrössen, sie sind aber bei weitem nicht
so offensichtlich wie der Impuls in kartesischen Koordinaten.

Die Euler-Lagrange-Differentialgleichungen für das Variationsproblem
\eqref{buch:mechanik:hamilton:eqn:variation} haben die Form
\[
\frac{d}{dt}
\frac{\partial L}{\partial \dot{q}_i}(t,q(t),\dot{q}(t))
=
\frac{\partial L}{\partial q_i}(t,q(t),\dot{q}(t)),\qquad
\text{mit $i=1,\dots,n$}.
\]
Schreibt man
\[
p_i(t)
= 
\frac{\partial L}{\partial \dot{q}_i}(t,q_i(t),\dot{q}_i(t)),
\]
entsteht also automatisch die Differentialgleichung
\[
\frac{d}{dt} p_i(t)
=
\frac{\partial L}{\partial \dot{q}_i}(t,q(t),\dot{q}(t))
\]
erster
Ordnung, allerdings in den $3n$ Variablen $q$, $\dot{q}$ und
$\dot{p}$.
Es müssen also aus $p$, $q$ und $\dot{q}$ die Variablen $\dot{q}$
eliminiert werden.
Dass die Koordinaten $\dot{q}$ die Ableitung von $q$ ist, ist eine
der ursprünglichen Bewegungsgleichungen, diese müssen ebenfalls durch
$p$ und $q$ ausgedrückt werden.

%
% Hamilton-Funktion
%
\subsection{Hamilton-Funktion}
Wir betrachten den Ausdruck
\[
H
=
p\dot{q}
-
L(t,q,\dot{q}).
\]
Der letzte Term ist als Skalarprodukt der beiden Vektoren $p$ und $\dot{q}$
zu lesen.
Die Ableitung nach $q$ ist
\[
\frac{\partial H}{\partial \dot{q}_i}
=
p_i
-
\frac{\partial L}{\partial \dot{q}_i}(t,q,\dot{q}).
\]
Nach der Definition von $p_i$ ist der erste Term auf der rechten Seite
ebenfalls $p_i$, es folgt also
\[
\frac{\partial H}{\partial \dot{q}_i} = 0.
\]
Als Funktion von $q$, $\dot{q}$ und $p$ hängt $H$ also nicht von $\dot{q}_i$
ab, kann also als eine Funktion von $q$ und $p$ allein geschrieben werden.

\begin{definition}[Hamilton-Funktion]
Die {\em Hamilton-Funktion} zur Lagrange-Funktion $L(t,q,\dot{q})$ ist
\[
H(t,q,p)
=
p\dot{q}
-
L(t,q,\dot{q}).
\]
\end{definition}

Für einen freien Massepunkt ist $p_i = m\dot{q}_i$ und damit damit
\[
H=
m\dot{x}_i^2
-
L
=
2T
-
(V-T)
=
T+V,
\]
die Hamilton-Funktion ist also die Gesamtenergie des Systems.

%
% Bewegungsgleichungen
%
\subsection{Bewegungsgleichungen}
Da im Produkt $p\dot{q}$ die Variable $q_i$ nicht vorkommt, ist
Ableitung von $H$ nach $q_i$
\[
\frac{\partial H}{\partial q_i}
=
-
\frac{\partial L}{\partial q_i}.
\]
Die Euler-Lagrange-Differentialgleichung ist
\[
\frac{\partial L}{\partial q_i}
=
\frac{d}{dt}\frac{\partial L}{\partial \dot{q}_i}
=
\frac{d}{dt} p_i.
\]
Damit kann di Bewegungsgleichung für den verallemeinerten Impuls $p_i$ auch als
\[
\frac{d}{dt}p_i
=
-
\frac{\partial H}{\partial q_i}(t,q(t),p(t))
\]
geschrieben werden.

Die Abhängigkeit der Hamilton-Funktion $H(t,q,p)$ äusert sich nur im Term
$-p\dot{q}$, so dass die Ableitung von $H$ nach $p_i$ sofort als
\[
\frac{\partial H}{\partial p_i}
=
\dot{q}_i
\]
berechnen lässt.
Damit sind jetzt die gesuchten Bewegungsgleichungen erster Ordnung gefunden.
Ist $q(t)$ eine Lösung der Euler-Lagrange-Differentialgleichung, dann sind
$q(t)$ und $p(t)$ eine Lösung der {\em Hamilton-Differentialgleichung}
\begin{align*}
\frac{d}{dt} q_i
&=
\phantom{-}\frac{\partial H}{\partial p_i}
\\
\frac{d}{dt} p_i
&=
-
\frac{\partial H}{\partial q_i}.
\end{align*}
Damit ist die versprochene Umwandlung in ein Differentialgleichungssystem
erster Ordnung vollzogen.


