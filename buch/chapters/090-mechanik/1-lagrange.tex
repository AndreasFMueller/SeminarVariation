%
% 1-lagrange.tex
%
% (c) 2023 Prof Dr Andreas Müller
%
\section{Klassische Mechanik nach Lagrange
\label{buch:mechanik:section:lagrange}}
\kopfrechts{Klassische Mechanik nach Lagrange}
Die Basis der klassischen Mechanik wurde von Newton in der
Principia gelegt.
Insbesondere das zweite newtonsche Gesetz
\(
F=ma
\)
definiert Kräfte die Ursache für die Beschleunigung.
Es ermöglicht, für jedes mechanische System eine Bewegungsgleichung
in der Form einer Differentialgleichung zweiter Ordnung für die
Koordinaten aufzustellen.
Der Prozess kann etwas schwerfällig sein, vor allem unter dem Einfluss
externer Zwangskräfte.
Das Maupertuissche Prinzip der geringsten Wirkung erlaubt, dies wesentlich
zu vereinfachen.
Wir führen die Vorgehensweise in
Abschnitt~\ref{buch:mechanik:lagrange:subsection:massepunkt}
ein und zeigen in
Abschnitt~\ref{buch:mechanik:lagrange:subsection:koordinaten},
wie sich damit die Bewegungsgleichungen sehr leicht in beliebigen
Koordinaten herleiten lassen.


%
% Ein Massepunkt in einem Potential
%
\subsection{Ein Massepunkt in einem Potential
\label{buch:mechanik:lagrange:subsection:massepunkt}}
Wir betrachten ein Teilchen mit der Masse $m$, welches sich in
einem dreidimensionalen Koordinatensystem $\mathbb{R}^3$ bewegt.
Die Bahn eines Teilchens ist eine Kurve
$\mathbb{R}\to\mathbb{R}^3:t\to x(t)$.
Die kinetische Energie hängt von der Geschwindigkeit 
\[
\dot{x}(t) = \frac{dx}{dt}(t)
\]
ab, sie ist
\[
T=
E_{\text{kin}}
=
\frac12 m\dot{x}^2.
\]
Das Teilchen bewege sich ausserdem unter dem Einfluss von Kräften,
die durch ein Potential $V(t,x)$ gegeben sind.
Die Kräft sind dann durch den Gradienten $-\operatorname{grad}V(t,x)$ 
gegeben.

Aus der kinetischen Energie $T$ und dem Potential $V(t,x)$ lässt sich
die Lagrange-Funktion $L(t,x,\dot{x}) = \frac12m\dot{x}^2-V(t,x)$ 
zusammensetzen.
Das Funktional
\begin{equation}
S
=
\int_{t_0}^{t_1}
L(t,x(t),\dot{x}(t))
\,dt
\label{buch:mechanik:lagrange:eqn:wirkung}
\end{equation}
heisst die Wirkung, sie hat die Dimension $\text{Energie}\times\text{Zeit}$.
Für die Euler-Lagrange-Dif\-fe\-ren\-tialgleichung des
Funktionals~\eqref{buch:mechanik:lagrange:eqn:wirkung} brauchen wir
zunächst die partiellen Ableitungen
\begin{align*}
\frac{\partial L}{\partial x}
&=
-\frac{\partial V}{\partial x}(t,x)
=
-\operatorname{grad}V(t,x)
&&\text{und}&
\frac{\partial V}{\partial\dot{x}}
&=
m\dot{x}
\end{align*}
von $L$, aus denen sich dann die Bewegungsgleichung
\[
\frac{d}{dx}\frac{\partial L}{\partial\dot{x}}
=
\frac{\partial L}{\partial x}
\qquad\Rightarrow\qquad
\frac{d}{dt}m\dot{x}
=
-\operatorname{grad}V(t,x)
\]
ergeben. 
Dies ist tatsächlich die Bewegungsgleichung, die sich aus dem
zweiten newtonschen Gesetz ergibt.

%
% Verallgemeinerte Koordinaten
%
\subsection{Verallgemeinerte Koordinaten
\label{buch:mechanik:lagrange:subsection:koordinaten}}
Das Prinzip der kleinsten Wirkung ermöglicht, die Bewegungsgleichungen
in beliebigen Koordinaten auszudrücken, solange sich die kinetische
Energie und die potentielle Energie in diesen Koordinaten ausdrücken
lässt.
Dies ist oft ein einfaches Problem, wenn sich kartesische Koordinaten
durch die verallgemeinerten Koordinaten ausdrücken lassen.

Als Beispiel sei ein Massepunkt $m$ betrachtet, der sich in einem
Schwerefeld auf auf einer Kurve $y=f(x)$ bewegen kann.
Die Position des Punktes kann durch die Koordinaten $q=x$ beschrieben
werden.
Die potentielle Energie ist in diesem Fall $E_{\text{pot}} = mgf(q)$.
Die kinetische Energie setzt sich zusammen aus den $x$- und
$y$-Geschwindigkeiten
\[
E_{\text{kin}}
=
\frac12m(\dot{x}^2 + \dot{y}^2)
=
\frac12m(\dot{q}^2 + f'(q)^2 \dot{q}^2)
=
\frac12m(1+f'(q)^2)\dot{q}^2.
\]
Die Lagrange-Funktion dieses Systems ist daher 
\[
L(t,q,\dot{q})
=
\frac12m(1+f'(q)^2)\dot{q}^2 - mgf(q).
\]
Daraus lässt sich die Bewegungsgleichung als die
Euler-Lagrange-Differentialgleichung aus den partiellen Ableitungen
\begin{align*}
\frac{\partial L}{\partial q}
&=
mf'(q)f''(q)\dot{q}^2 - mgf'(q)
\\
\frac{\partial L}{\partial\dot{q}}
&=
m(1+f'(q)^2)\dot{q}
\end{align*}
als
\begin{align*}
0
=
\frac{\partial L}{\partial q}-\frac{d}{dt}\frac{\partial L}{\partial\dot{q}}
&=
mf'(q)f''(q)\dot{q}^2
-
mgf'(q)
-
m(1+f'(q)^2)\ddot{q}
-
2mf'(q)f''(q)\dot{q}^2
\\
&=
-
mf'(q)f''(q)\dot{q}^2
-
mgf'(q)
-
m(1+f'(q)^2)\ddot{q}
\end{align*}
erhalten.
Nach Division durch $m(1+f'(q)^2)$ erhält man die Form
\begin{align*}
\ddot{q}
=
-
\frac{f'(q)(g+f''(q)\dot{q}^2)}{1+f'(q)^2}.
\end{align*}


%
% Das Kapiza-Pendel
%
\subsection{Das Kapiza-Pendel}
%
% kapiza.tex -- template for standalon tikz images
%
% (c) 2021 Prof Dr Andreas Müller, OST Ostschweizer Fachhochschule
%
\documentclass[tikz]{standalone}
\usepackage{amsmath}
\usepackage{times}
\usepackage{txfonts}
\usepackage{pgfplots}
\usepackage{csvsimple}
\usetikzlibrary{arrows,intersections,math,calc}
\begin{document}
\definecolor{darkred}{rgb}{0.8,0,0}
\def\skala{1}
\def\a{75}
\def\r{5}
\begin{tikzpicture}[>=latex,thick,scale=\skala]

\draw[line width=0.3pt] (-1,0) -- (-2,0);
\draw[<->] (-1.8,-1) -- (-1.8,1);
\node at (-1.9,0) [left] {$h(t)$};

\fill[color=brown!20] (-1,-1) rectangle (1,0);
\draw[color=brown] (-1,0) -- (1,0);

\fill[color=gray!20] (0,0) -- (\a:3) arc(\a:90:3) -- cycle;
\draw[line width=0.3pt] (0,0) -- (0,4);

\draw[->,color=darkred] (\a:\r) -- +(0,-2);
\node[color=darkred] at ($(\a:\r)+(0,-2)$) [right] {$g$};

\draw[color=blue,line width=1.4pt] (0,0) -- (\a:5);

\fill[color=blue] (\a:\r) circle[radius=0.26];
\node[color=blue] at (\a:{0.75*\r}) [left] {$l$};

\node[color=white] at (\a:\r) {$m$};
\fill (0,0) circle[radius=0.08];
\node at ({0.5*(\a+90)}:2.5) {$\varphi$};

\node at (0,0) [below] {$O$};

\end{tikzpicture}
\end{document}


Ein invertiertes Pendel mit der Länge $l$ und Masse $m$ wird im
Drehpunkt mit der Kreisfrequenz $\omega$ vertikal bewegt
(Abbildung~\ref{buch:mechanik:lagrange:fig:kapiza}).
Mit Hilfe des Prinzips der kleinsten Wirkung lässt sich die
Bewegungsgleichung finden, indem man zunächst die Lagrange-Funktion
aufstellt und dann die zugehörige Euler-Lagrange-Differentialgleichung
ableitet.
Wir werden die Bewegungsgleichungen etwas allgemeiner für eine
vertikale Bewegung mit $h(t)$ berechnen und erst später auf eine
Bewegung mit Kreisbewegung spezialisieren.

Das Pendel kann beschrieben werden durch den Ablenkungswinkel $\varphi$
von der Vertikalen.
Gesucht ist eine Bewegungsgleichung für $\varphi$ als Funktion der Zeit.
Dazu muss die Lagrange-Funktion durch den Winkel $\varphi$ und die
Winkelgeschwindigkeit $\dot{\varphi}$ ausgedrückt werden.
Sowohl die kinetische Energie wie auch die potentielle Energie lässt
sich leichter in kartesischen Koordinaten ausdrücken.
Die Koordinaten sind
\begin{align*}
x(t)
&=
l \sin\varphi(t)
\\
y(t)
&=
l \cos\varphi(t) + h(t).
\end{align*}
Die Geschwindigkeitskomponenten sind dann
\begin{align*}
\dot{x}(t) &= \phantom{-}l \cos\varphi(t)\cdot \dot{\varphi}(t) \\
\dot{y}(t) &=          - l \sin\varphi(t)\cdot \dot{\varphi}(t) + \dot{h}(t).
\end{align*}
Daraus ergibt sich die kinetische Energie
\begin{align*}
T
&=
\frac12
m(\dot{x}(t)^2 + \dot{y}(t)^2)
\\
&=
\frac12m\bigl(
l^2 \cos^2 \varphi(t)\cdot\dot{\varphi}(t)^2
+
(-l\sin\varphi(t)\cdot\dot{\varphi}(t) + \dot{h}(t))^2
\bigr)
\\
&=
\frac12m\bigl(
l^2 \cos^2\varphi(t)\cdot\dot{\varphi}(t)^2
+
l^2 \sin^2\varphi(t)\cdot\dot{\varphi}(t)^2
-
2l\sin\varphi(t)\dot{\varphi}(t)^2
+
\dot{h}(t)^2
\bigr)
\\
&=
\frac12m\bigl(
l^2
-
l\sin \varphi(t)\cdot \dot{\varphi}(t)\dot{h}(t) + \dot{h}(t)^2
\bigr)
.
\end{align*}
Für die potentielle Energie gilt
\[
V(t,\varphi)
=
mgy(t)
=
mgl\cos\varphi(t) + mgh(t)
\]
Insbesamt erhalten wir die zeitabhängige Lagrange-Funktion
\[
L(t,\varphi,\dot{\varphi})
=
\frac12m\bigl(
l^2\dot{\varphi}(t)^2
-
l\sin \varphi\cdot \dot{\varphi}\dot{h}(t) + \dot{h}(t)^2
\bigr)
-
mgl\cos\varphi - mgh(t).
\]

%
% Die Bewegungsgleichung
%
\subsubsection{Die Bewegungsgleichung}
Die partiellen Ableitungen der Lagrange-Funktion nach $\varphi$ und
$\dot{\varphi}$ sind
\begin{align*}
\frac{\partial L}{\partial\varphi}
&=
-ml
\dot{\varphi}\dot{h}(t)
\cos\varphi
+
mgl\sin\varphi
\\
\frac{\partial L}{\partial\dot{\varphi}}
&=
ml^2\dot{\varphi}
-ml\dot{h}(t)\sin\varphi.
\end{align*}
Für die Bewegungsgleichung brauchen wir die Ableitung des zweiten
Ausdrucks nach $t$, dafür bekommen wir
\begin{align*}
\frac{d}{dt}\frac{\partial L}{\partial\dot{\varphi}}
&=
ml^2 \ddot{\varphi}
-ml\ddot{h}\sin\varphi
-ml\dot{h}\dot{\varphi}\cos\varphi
\end{align*}
Die Euler-Lagrange-Differentialgleichung ist jetzt
\begin{align*}
0
=
\frac{\partial L}{\partial \varphi}
-
\frac{d}{dt}\frac{\partial L}{\partial\dot{\varphi}}
&=
-ml
\dot{\varphi}\dot{h}(t)
\cos\varphi
+
mgl\sin\varphi
-
ml^2 \ddot{\varphi}
+ml\ddot{h}(t)\sin\varphi
+ml\dot{h}(t)\dot{\varphi}\cos\varphi
\intertext{Der erste und der letzte Term heben sich weg:}
&=
mgl\sin\varphi
-
ml^2 \ddot{\varphi}
+ml\ddot{h}(t)\sin\varphi.
\end{align*}
Division durch $ml$ ergibt die Bewegungsgleichung
\begin{equation}
l\ddot{\varphi}
=
g\sin\varphi
+
\ddot{h}\sin\varphi.
\label{buch:mechanik:lagrange:eqn:kapiza-bewegungsgleichung}
\end{equation}
Ohne den Term $\ddot{h}\sin\varphi$ auf der rechten Seite von
\eqref{buch:mechanik:lagrange:eqn:kapiza-bewegungsgleichung}
entsteht wie erwartet die Bewegungsgleichung des mathematischen Pendels.
Allerdings ist die Kraft nicht rücktreibend, das Pendel entfernt sich
aus der Nulllage.

Wählt man für $h(t)$ eine harmonische Schwingung in der Form
$h(t)=A\sin\omega t$, dann ist $\ddot{h}=-\omega^2 A \sin\omega t$
und die Bewegungsgleichung wird
\[
l\ddot{\varphi}
=
(g-\omega^2 A \sin\omega t)\sin\varphi.
\]
Die Differentialgleichung kann mit der Mathieu-Differentialgleichung
oder mit Floquet-Theorie behandelt werden.
Für hohe Frequenzen kann nach Pyotr Kapiza durch zeitliche Mittelung
gezeigt werden, dass die Lage $\varphi=0$ eine stabile Lage wird.




