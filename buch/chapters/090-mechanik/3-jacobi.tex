%
% 3-jacobi.tex
%
% (c) 2023 Prof Dr Andreas Müller
%
\section{Hamilton-Jacobi-Differentialgleichung
\label{buch:mechanik:section:jacobi}}
\kopfrechts{Hamilton-Jacobi-Differentialgleichungen}
In Abschnitt~\ref{buch:hamiltonjacobi:section:jacobi} wurde gezeigt,
dass die Fundamentalfunktion $S(x,y)$ eines Variationsprinzips 
die Hamilton-Jacobi-Differentialgleichung erfüllt, und dass umgekehrt
aus einem allgemeinen Integral der Hamilton-Jacobi-Integral Lösungen
der kanonischen Differentialgleichungen gefunden werden können.
Dies ermöglicht Bahnen von mechanischen Systeme auf eine kanonische
Art zu charakterisierugen.
In diesem Abschnitt zeigen wir die Vorgehensweise an einer Reihe
von Beispielen.

%
% Der schiefe Wurf
%
\subsection{Der schiefe Wurf
\label{buch:mechanik:jacobi:schieferwurf}}
Wir betrachten einen Massepunkt, der zur Zeit $t=0$ ausgehend von einem
Punkt $\vec{p}$ mit Anfangsgeschwindigkeit $\vec{v}_0$ abgeschossen
wird und sich allein unter der Wirkung der in negativer $y$-Richtung
wirkenden Gravitation bewegt.

%
% Lagrange-Funktion
%
\subsubsection{Lagrange-Funktion}
Die Lagrange-Funktion dieses Systems ist
\[
\left.
\begin{aligned}
E_{\text{kin}}
&=
\frac12 mv^2
\\
E_{\text{pot}}
&=
mgh
\end{aligned}
\right\}
\quad\Rightarrow\quad
L(t,x,y,\dot{x},\dot{y})
=
\frac{m}2 (\dot{x}^2 + \dot{y}^2)
-
mgy.
\]
Für die Euler-Lagrange-Differentialgleichung erhalten wir
\begin{equation}
\left.
\begin{aligned}
\frac{\partial L}{\partial x}
&=
0
&
\frac{\partial L}{\partial \dot{x}}
&=
m\dot{x}
\\
\frac{\partial L}{\partial y}
&=
-g
&
\frac{\partial L}{\partial \dot{y}}
&=
m\dot{y}
\end{aligned}
\;\right\}
\quad\Rightarrow\quad
\left\{\;
\begin{aligned}
0
&=
\frac{\partial L}{\partial x}-\frac{d}{dt}\frac{\partial L}{\partial\dot{x}}
=
-\frac{d}{dt}m\dot{x}
&&\Rightarrow&\frac{d}{dt}m\dot{x}&=0
\\
0
&=
\frac{\partial L}{\partial y}-\frac{d}{dt}\frac{\partial L}{\partial\dot{y}}
=
-g
-\frac{d}{dt}m\dot{y}
&&\Rightarrow&\frac{d}{dt}m\dot{y}&=-g.
\end{aligned}
\right.
\label{buch:mechanik:jacobi:wurf:eqn:bewegungsgleichung}
\end{equation}

%
% kanonische Koordinaten
%
\subsubsection{Konjugierte Koordinaten}
Die konjugierten Koordinaten sind
\begin{align*}
p_x &= \frac{\partial L}{\partial \dot{x}} = m\dot{x}
&
\dot{x}
&=
\frac{p_x}{m}
\\
p_y &= \frac{\partial L}{\partial \dot{y}} = m\dot{y}
&
\dot{y}
&=
\frac{p_y}{m}.
\end{align*}
Dies ist der Impuls, was im Nachhinein die Notation für die
konjugierten Koordinaten rechtfertigt.
Mit diesen Koordinaten lassen sich die Bewegungslgeichungen 
\eqref{buch:mechanik:jacobi:wurf:eqn:bewegungsgleichung} in der
Form
\begin{equation}
\begin{aligned}
\frac{d}{dt}p_x &= 0\\
\frac{d}{dt}p_y &= -g
\end{aligned}
\label{buch:mechanik:jacobi:wurf:eqn:bewegungsgleichung2}
\end{equation}
schreiben.

%
% Hamilton-Funktion
%
\subsubsection{Hamilton-Funktion}
Die Hamilton-Funktion ist dann
\begin{align*}
H(t,x,y,p_x,p_y)
&=
\dot{x}p_x
+
\dot{y}p_y
-
L(t,x,y,\dot{x},\dot{y}
\\
&=
\dot{x}p_x
+
\dot{y}p_y
-
\frac{m}2(\dot{x}^2+\dot{y}^2)
+mgy
\\
&=
\frac{p_x^2}{2}
+
\frac{p_y^2}{2}
-
\frac{m}2\biggl(\frac{p_x^2}{m^2} + \frac{p_y^2}{m^2}\biggr) + mgy
\\
&=
\frac{1}{2m}(p_x^2+p_y^2) + mgy
=
E_{\text{kin}} + E_{\text{pot}}.
\end{align*}
Die Hamilton-Funktion hängt nicht von der Zeit ab, sie ist die erhaltene
Gesamtenergie.

%
% Bewegungsgleichungen
%
\subsubsection{Bewegungsgleichungen}
Die kanonischen Differentialgleichungen werden jetzt
\begin{align*}
\frac{dx}{dt}
&=
\frac{\partial H}{\partial p_x}
=
\frac{p_x}{m}
&&\text{und}&
\frac{dy}{dt}
&=
\frac{\partial H}{\partial p_y}
=
\frac{p_y}{m},
\\
\frac{dp_x}{dt}
&=
-\frac{\partial H}{\partial x}
=
0
&&\text{und}&
\frac{dp_y}{dt}
&=
-\frac{\partial H}{\partial y}
=
-g,
\end{align*}
die mit den bekannten Bewegungsgleichungen
\eqref{buch:mechanik:jacobi:wurf:eqn:bewegungsgleichung2}
übereinstimmen.

%
% Hamilton-Jacobi-Differentialgleichung
%
\subsubsection{Hamilton-Jacobi-Differentialgleichung}
Mit der Hamilton-Funktion können wir jetzt die
Hamilton-Jacobi-Differentialgleichung 
\begin{align*}
\frac{\partial S}{\partial t}
&=
H\biggl(
t,x,y,\frac{\partial S}{\partial x},\frac{\partial S}{\partial y}
\biggr)
=
\frac1{2m}\biggl(
\biggl(\frac{\partial S}{\partial x}\biggr)^2
+
\biggl(\frac{\partial S}{\partial y}\biggr)^2
\biggr)
+
mgy
\end{align*}
aufstellen.

Wir versuchen, die Differentialgleichung mit einem Separationsansatz
der Form
\[
S(t,x,y)
=
X(x)
+
Y(y)
+
T(t),
\]
den wir in die Differentialgleichung einsetzen und
\[
T'(t)
=
\frac1{2m}(X'(x)^2 + Y'(y)^2)
+
gmy
\]
erhalten.
Da die rechte Seite nicht von der Zeit abhängt, muss $T'(t) = a_1$ konstant
sein.

Die Terme, die $y$ enthalten, können auf die andere Seite gebracht
werden, es entsteht die Gleichung
\[
\frac{1}{2m}X'(x)^2
=
a_1-\frac{1}{2m}Y'(y)^2 -mgy.
\]
Auf der linken Seite kommt nur die Variable $x$ vor, auf der rechten
nur die Variable $y$, also müssen beide Seiten konstant sein,
wir bezeichnen die Konstante mit $a_2$ und erhalten die Gleichungen
\begin{align*}
\frac1{2m}X'(x)^2&= a_2
&&\Rightarrow&
X'(x) &= \sqrt{2ma_2}
\\
\frac1{2m}Y'(x)^2&=a_1-a_2-mgy
&&\Rightarrow&
Y'(y) &= \sqrt{2m(a_1-a_2-mgy)}
\end{align*}
mit den Lösungen
\begin{align*}
X(x) &= \sqrt{2ma_2}x
\\
Y(y) &=
-\frac{1}{3gm^2}
\bigl(2m(a_1-a_2-mgy)\bigr)^{\frac32}.
\end{align*}
Die Fundamentalfunktion ist daher
\[
S(t,x,y,a_1,a_2)
=
a_1t
+
\sqrt{2ma_2}x
-
\frac{1}{3gm^2}
\bigl(2m(a_1-a_2-mgy)\bigr)^{\frac32}
\]
Die beiden Konstanten $a_1$ und $a_2$ haben die Dimension einer
Energie.

%
% Regularität
%
\subsubsection{Regularität}
Wir müssen die Regularität überprüfen, dazu ist die Matrix
\begin{align*}
A
&=
\begin{pmatrix}
\displaystyle
\frac{\partial^2 S}{\partial x\,\partial a_1}
&
\displaystyle
\frac{\partial^2 S}{\partial x\,\partial a_2}
\\[8pt]
\displaystyle
\frac{\partial^2 S}{\partial y\,\partial a_1}
&
\displaystyle
\frac{\partial^2 S}{\partial y\,\partial a_2}
\end{pmatrix}
\\
&=
\begin{pmatrix}
0
&
\displaystyle
\frac{m}{\sqrt{2m a_2}}
\\[8pt]
\displaystyle
\frac{m}{\sqrt{2m(a_1-a_2-mgy)}}
&
\displaystyle
-\frac{m}{\sqrt{2m(a_1-a_2-mgy)}}
\end{pmatrix}
\intertext{mit der Determinanten}
\det A
&=
-\frac{m}{2\sqrt{a_2(a_1-a_2-mgy)}}
\end{align*}
untersuchen.
Sie ist offenbar für $a_2\ne 0$ und $a_1-a_2-mgy\ne 0$ von $0$
verschieden, wir können also erwarten, dass der Satz von Jacobi
eine Lösung liefert.

%
% Lösungen
%
\subsubsection{Lösung}
Zur Bestimmung der Lösung müssen jetzt die Ableitungen nach
den Parametern berechnet werden:
%(%i4) ratsimp(diff(S,a1))
%                                                 2
%                   sqrt(2) sqrt((a1 - a2) m - g m  y) - g m t
%(%o4)            - ------------------------------------------
%                                      g m
%(%i5) ratsimp(diff(S,a2))
%                                                  2         2
%               2 sqrt(a2 m) sqrt((a1 - a2) m - g m  y) + g m  x
%(%o5)          ------------------------------------------------
%                            sqrt(2) g m sqrt(a2 m)
%
\begin{align*}
b_1
=
\frac{\partial S}{\partial a_1}
&=
-\frac{\sqrt{2m(a_1-a_2-gmy)}}{gm}
+t
\\
b_2
=
\frac{\partial S}{\partial a_2}
&=
\frac{1}{\sqrt{2ma_2}}
\biggl(
\sqrt{2ma_2}
\frac{\sqrt{2m(a_1-a_2-gmy)}}{gm}
+mx
\biggr).
\end{align*}
Die beiden Konstanten $b_1$ und $b_2$ haben die Dimension einer Zeit.
Die erste Gleichung können wir auch in der Form
\begin{align*}
t-b_1
&=
\frac{\sqrt{2m(a_1-a_2-gmy)}}{gm}
\intertext{schreiben, womit wir die zweite Gleichung etwas übersichtlicher
in}
b_2
&=
t-b_1
+
\frac{m}{\sqrt{2ma_2}}x
\end{align*}
umformen können.
Die beiden Gleichungen können nach $x(t)$ und $y(t)$ aufgelöst werden
und ergeben
\begin{align*}
x(t)
&=
\frac{\sqrt{2ma_2}}{m}
(b_1 + b_2 - t)
\\
y(t)
&=
\frac{a_1-a_2}{gm}
-
\frac{g}{2}(t-b_1)^2.
\end{align*}
Die physikalische Bedeutung der Konstanten lässt sich jetzt aus den 
Lösungsfunktionen ablesen.
$b_1$ ist die Zeit, zu der sich der Masspunkt im Scheitelpunkt
der Bahn befindet.
$b_2$ ist die Zeit zwischen dem Scheitelpunkt der Bahn und dem Überqueren
der $y$-Achse.
Die Differenz $a_1-a_2$ ist die potentielle Energie, die der Massepunkt
im Bahnscheitel hat und $a_2$ ist die kinetische Energie im Bahnscheitel.
Somit ist $a_1$ die Gesamtenergie.


