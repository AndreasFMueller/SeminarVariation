%
% kapiza.tex -- template for standalon tikz images
%
% (c) 2021 Prof Dr Andreas Müller, OST Ostschweizer Fachhochschule
%
\documentclass[tikz]{standalone}
\usepackage{amsmath}
\usepackage{times}
\usepackage{txfonts}
\usepackage{pgfplots}
\usepackage{csvsimple}
\usetikzlibrary{arrows,intersections,math,calc}
\begin{document}
\definecolor{darkred}{rgb}{0.8,0,0}
\def\skala{1}
\def\a{75}
\def\r{5}
\begin{tikzpicture}[>=latex,thick,scale=\skala]

\draw[line width=0.3pt] (-1,0) -- (-2,0);
\draw[<->] (-1.8,-1) -- (-1.8,1);
\node at (-1.9,0) [left] {$h(t)$};

\fill[color=brown!20] (-1,-1) rectangle (1,0);
\draw[color=brown] (-1,0) -- (1,0);

\fill[color=gray!20] (0,0) -- (\a:3) arc(\a:90:3) -- cycle;
\draw[line width=0.3pt] (0,0) -- (0,4);

\draw[->,color=darkred] (\a:\r) -- +(0,-2);
\node[color=darkred] at ($(\a:\r)+(0,-2)$) [right] {$g$};

\draw[color=blue,line width=1.4pt] (0,0) -- (\a:5);

\fill[color=blue] (\a:\r) circle[radius=0.26];
\node[color=blue] at (\a:{0.75*\r}) [left] {$l$};

\node[color=white] at (\a:\r) {$m$};
\fill (0,0) circle[radius=0.08];
\node at ({0.5*(\a+90)}:2.5) {$\varphi$};

\node at (0,0) [below] {$O$};

\end{tikzpicture}
\end{document}

