Finden Sie die Funktion $y(x)$ zwischen den Punkten $(0,0)$ und $(1,0)$,
das Integral
\[
A(y)
=
\int_{0}^{1} y(x)^2\,dx
\]
maximiert und für die ausserdem die Bedingung
\[
B(y)
=
\int_{0}^{1}
y'(x)^2
\,dx
=
C
\]
erfüllt.

\begin{loesung}
Die Lagrange-Funktionen der beiden Funktionale sind
\begin{align*}
F(x,y,y') &= y^2          &&\text{für $A(y)$}\\
G(x,y,y') &= y^{\prime 2} &&\text{für $B(y)$.}
\end{align*}
Von beiden muss jetzt der Gradient gebildet werden, also die linke
Seite der Euler-Lagrange-Differentialgleichung.
Für $F$ ist dies
\[
\frac{\partial F}{\partial y}(x,y(x),y'(x))
-
\frac{d}{dx}\frac{\partial F}{\partial y'}(x,y(x),y'(x))
=
2y(x),
\]
für $G$ ist es
\[
\frac{\partial G}{\partial y}(x,y(x),y'(x))
-
\frac{d}{dx}\frac{\partial G}{\partial y'}(x,y(x),y'(x))
=
\frac{d}{dx}(2y'(x))
=
2y''(x).
\]
Nach der Methode der Lagrange-Multiplikatoren muss jetzt die Funktion
${\color{darkred}y(x)}$ und eine Zahl ${\color{darkred}\lambda}\in\mathbb{R}$
so gefunden werden, dass
\begin{align}
2{\color{darkred}y(x)}&=2{\color{darkred}\lambda y}''{\color{darkred}(x)}
\label{buch:aufgabe:501:eqn:dgl}
\\
\int_{x_1}^{x_2} {\color{darkred}y(x)}^2\,dx &= C
\label{buch:aufgabe:501:eqn:integral}
\\
{\color{darkred}y}(0) &= 0
\label{buch:aufgabe:501:eqn:rb1}
\\
{\color{darkred}y}(1) &= 0
\label{buch:aufgabe:501:eqn:rb2}
\end{align}
Als Erstes lösen wir die Differentialgleichung~\eqref{buch:aufgabe:501:eqn:dgl}
in der Form
\[
y''(x) = \frac{1}{\lambda} y(x).
\]
Je nach Vorzeichen von $\lambda$ gibt es verschieden mögliche Lösungen.

Wir versuchen zunächst den Ansatz $y(x)=e^{ax}$ und bekommen
\[
a^2e^{ax} = \frac{1}{\lambda} e^{ax},
\]
und damit $a=\pm1/\!\sqrt{\lambda}$.
Diese Lösung existiert nur dann, wenn $\lambda>0$ ist.
Die Funktion $y(x)$ ist daher eine Linearkombination 
\[
y(x) = A_+ e^{x/\!\sqrt{\lambda}} + A_- e^{-x/\!\sqrt{\lambda}}.
\]
Diese Funktion muss jetzt ausserdem die Randbedingungen erfüllen wir setzen
daher $0$ und $1$ ein und erhalten das homogene lineare Gleichungssystem
\[
\renewcommand{\arraycolsep}{3pt}
\begin{array}{lclcl}
A_+                       &+& A_-                        &=& 0 \\
A_+e^{1/\!\sqrt{\lambda}} &+& A_-e^{-1/\!\sqrt{\lambda}} &=& 0
\end{array}
\]
Für die Koeffizienten $A_+$ und $A_-$.
Es hat die Determinante
\[
A_+A_-(e^{1/\!\sqrt{\lambda}}-e^{-1/\!\sqrt{\lambda}}\ne 0,
\]
daher kann es nur die triviale Lösung $A_+=A_-=0$ haben.
Dann ist aber die Funktion $y(x)=0$, d.~h.~die Nebenbedingung
\eqref{buch:aufgabe:501:eqn:integral} kann nicht erfüllt werden, 
der Ansatz mit Exponentialfunktionen führt nicht auf eine Lösung.
Folglich muss $\lambda<0$ sein.

Wir versuchen daher den Fall $\lambda<0$, in diesem Fall sind die
Funktionen
\[
\sin x/\!\sqrt{-\lambda}
\qquad\text{und}\qquad
\cos x/\!\sqrt{-\lambda}
\]
Lösungen der Differentialgleichung.
Die allgemeine Lösung ist daher eine Linearkombination
\[
y(x)
=
A \cos  x/\!\sqrt{-\lambda}
+
B \sin  x/\!\sqrt{-\lambda}.
\]
Jetzt setzen wir die Randbedingung an der Stelle $x=0$ ein und erhalten
\[
0
=
y(0)
=
A.
\]
An der Stelle $x=1$ folgt dann
\[
0=y(1)
=
B\sin 1/\!\sqrt{-\lambda}.
\]
Da die Nullstellen der Sinusfunktion die ganzzahligen Vielfachen von $\pi$
sind, folgt
\[
1/\!\sqrt{-\lambda} = k\pi
\qquad\Rightarrow\qquad
\lambda = -\frac{1}{k^2\pi^2}
\]
für $k\in\mathbb{N}$.
Damit ist jetzt gefunden, dass
\[
y(x) = B \sin k\pi x
\]
ist.
Jetzt muss $B$ nur noch so bestimmt werden, dass die
Nebenbedingung~\ref{buch:aufgabe:501:eqn:integral} erfüllt ist.
Dies ist die Gleichung
\begin{align*}
C
&=
\int_0^1 y'(x)^2\,dx
\\
&=
\int_0^1 B k\pi \cos^2 k\pi x\,dx
\\
&=
Bk\pi \int_0^1 \cos^2 \pi x\,dx
=
\frac{Bk\pi}2
\qquad
\Rightarrow
\qquad
B
=
\frac{2C}{k\pi}.
\end{align*}
Der Wert des Integrals $A(y)$ kann jetzt ebenfalls bestimmt werden, es
ist
\begin{align*}
A(y)
&=
\int_0^1 y(x)^2\,dx
=
\int_0^1 B\sin^2 k\pi x\,dx
=
\frac{2C}{k\pi} \int_0^1 \sin^2k\pi x\,dx
=
\frac{C}{k\pi}.
\end{align*}
Da $k\in\mathbb{N}$ sein muss, wird das Maximum bei $k=1$ erreicht.
\end{loesung}
