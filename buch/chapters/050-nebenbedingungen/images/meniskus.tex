%
% meniskus.tex -- Meniskus shapes
%
% (c) 2021 Prof Dr Andreas Müller, OST Ostschweizer Fachhochschule
%
\documentclass[tikz]{standalone}
\usepackage{amsmath}
\usepackage{times}
\usepackage{txfonts}
\usepackage{pgfplots}
\usepackage{csvsimple}
\usetikzlibrary{arrows,intersections,math}
\definecolor{darkred}{rgb}{0.8,0,0}
\begin{document}
\def\skala{1}
\begin{tikzpicture}[>=latex,thick,scale=\skala]

% add image content here
\def\dx{5}
\def\dy{6}

\def\kurveone{
	({-1.0000*\dx},{0.1545*\dy})
	-- ({-0.9899*\dx},{0.1533*\dy})
	-- ({-0.9798*\dx},{0.1521*\dy})
	-- ({-0.9697*\dx},{0.1510*\dy})
	-- ({-0.9596*\dx},{0.1498*\dy})
	-- ({-0.9495*\dx},{0.1487*\dy})
	-- ({-0.9394*\dx},{0.1476*\dy})
	-- ({-0.9293*\dx},{0.1465*\dy})
	-- ({-0.9192*\dx},{0.1454*\dy})
	-- ({-0.9091*\dx},{0.1444*\dy})
	-- ({-0.8990*\dx},{0.1433*\dy})
	-- ({-0.8889*\dx},{0.1423*\dy})
	-- ({-0.8788*\dx},{0.1413*\dy})
	-- ({-0.8687*\dx},{0.1403*\dy})
	-- ({-0.8586*\dx},{0.1393*\dy})
	-- ({-0.8485*\dx},{0.1383*\dy})
	-- ({-0.8384*\dx},{0.1373*\dy})
	-- ({-0.8283*\dx},{0.1364*\dy})
	-- ({-0.8182*\dx},{0.1355*\dy})
	-- ({-0.8081*\dx},{0.1345*\dy})
	-- ({-0.7980*\dx},{0.1336*\dy})
	-- ({-0.7879*\dx},{0.1327*\dy})
	-- ({-0.7778*\dx},{0.1319*\dy})
	-- ({-0.7677*\dx},{0.1310*\dy})
	-- ({-0.7576*\dx},{0.1301*\dy})
	-- ({-0.7475*\dx},{0.1293*\dy})
	-- ({-0.7374*\dx},{0.1285*\dy})
	-- ({-0.7273*\dx},{0.1277*\dy})
	-- ({-0.7172*\dx},{0.1269*\dy})
	-- ({-0.7071*\dx},{0.1261*\dy})
	-- ({-0.6970*\dx},{0.1253*\dy})
	-- ({-0.6869*\dx},{0.1246*\dy})
	-- ({-0.6768*\dx},{0.1238*\dy})
	-- ({-0.6667*\dx},{0.1231*\dy})
	-- ({-0.6566*\dx},{0.1224*\dy})
	-- ({-0.6465*\dx},{0.1217*\dy})
	-- ({-0.6364*\dx},{0.1210*\dy})
	-- ({-0.6263*\dx},{0.1203*\dy})
	-- ({-0.6162*\dx},{0.1196*\dy})
	-- ({-0.6061*\dx},{0.1190*\dy})
	-- ({-0.5960*\dx},{0.1183*\dy})
	-- ({-0.5859*\dx},{0.1177*\dy})
	-- ({-0.5758*\dx},{0.1171*\dy})
	-- ({-0.5657*\dx},{0.1164*\dy})
	-- ({-0.5556*\dx},{0.1158*\dy})
	-- ({-0.5455*\dx},{0.1153*\dy})
	-- ({-0.5354*\dx},{0.1147*\dy})
	-- ({-0.5253*\dx},{0.1141*\dy})
	-- ({-0.5152*\dx},{0.1136*\dy})
	-- ({-0.5051*\dx},{0.1130*\dy})
	-- ({-0.4949*\dx},{0.1125*\dy})
	-- ({-0.4848*\dx},{0.1120*\dy})
	-- ({-0.4747*\dx},{0.1115*\dy})
	-- ({-0.4646*\dx},{0.1110*\dy})
	-- ({-0.4545*\dx},{0.1105*\dy})
	-- ({-0.4444*\dx},{0.1100*\dy})
	-- ({-0.4343*\dx},{0.1096*\dy})
	-- ({-0.4242*\dx},{0.1091*\dy})
	-- ({-0.4141*\dx},{0.1087*\dy})
	-- ({-0.4040*\dx},{0.1083*\dy})
	-- ({-0.3939*\dx},{0.1079*\dy})
	-- ({-0.3838*\dx},{0.1075*\dy})
	-- ({-0.3737*\dx},{0.1071*\dy})
	-- ({-0.3636*\dx},{0.1067*\dy})
	-- ({-0.3535*\dx},{0.1063*\dy})
	-- ({-0.3434*\dx},{0.1060*\dy})
	-- ({-0.3333*\dx},{0.1056*\dy})
	-- ({-0.3232*\dx},{0.1053*\dy})
	-- ({-0.3131*\dx},{0.1049*\dy})
	-- ({-0.3030*\dx},{0.1046*\dy})
	-- ({-0.2929*\dx},{0.1043*\dy})
	-- ({-0.2828*\dx},{0.1040*\dy})
	-- ({-0.2727*\dx},{0.1037*\dy})
	-- ({-0.2626*\dx},{0.1035*\dy})
	-- ({-0.2525*\dx},{0.1032*\dy})
	-- ({-0.2424*\dx},{0.1030*\dy})
	-- ({-0.2323*\dx},{0.1027*\dy})
	-- ({-0.2222*\dx},{0.1025*\dy})
	-- ({-0.2121*\dx},{0.1023*\dy})
	-- ({-0.2020*\dx},{0.1020*\dy})
	-- ({-0.1919*\dx},{0.1018*\dy})
	-- ({-0.1818*\dx},{0.1017*\dy})
	-- ({-0.1717*\dx},{0.1015*\dy})
	-- ({-0.1616*\dx},{0.1013*\dy})
	-- ({-0.1515*\dx},{0.1012*\dy})
	-- ({-0.1414*\dx},{0.1010*\dy})
	-- ({-0.1313*\dx},{0.1009*\dy})
	-- ({-0.1212*\dx},{0.1007*\dy})
	-- ({-0.1111*\dx},{0.1006*\dy})
	-- ({-0.1010*\dx},{0.1005*\dy})
	-- ({-0.0909*\dx},{0.1004*\dy})
	-- ({-0.0808*\dx},{0.1003*\dy})
	-- ({-0.0707*\dx},{0.1003*\dy})
	-- ({-0.0606*\dx},{0.1002*\dy})
	-- ({-0.0505*\dx},{0.1001*\dy})
	-- ({-0.0404*\dx},{0.1001*\dy})
	-- ({-0.0303*\dx},{0.1000*\dy})
	-- ({-0.0202*\dx},{0.1000*\dy})
	-- ({-0.0101*\dx},{0.1000*\dy})
	-- ({-0.0000*\dx},{0.1000*\dy})
	-- ({0.0101*\dx},{0.1000*\dy}) % 0.0010
	-- ({0.0202*\dx},{0.1000*\dy}) % 0.0020
	-- ({0.0303*\dx},{0.1000*\dy}) % 0.0030
	-- ({0.0404*\dx},{0.1001*\dy}) % 0.0040
	-- ({0.0505*\dx},{0.1001*\dy}) % 0.0051
	-- ({0.0606*\dx},{0.1002*\dy}) % 0.0061
	-- ({0.0707*\dx},{0.1003*\dy}) % 0.0071
	-- ({0.0808*\dx},{0.1003*\dy}) % 0.0081
	-- ({0.0909*\dx},{0.1004*\dy}) % 0.0091
	-- ({0.1010*\dx},{0.1005*\dy}) % 0.0101
	-- ({0.1111*\dx},{0.1006*\dy}) % 0.0111
	-- ({0.1212*\dx},{0.1007*\dy}) % 0.0122
	-- ({0.1313*\dx},{0.1009*\dy}) % 0.0132
	-- ({0.1414*\dx},{0.1010*\dy}) % 0.0142
	-- ({0.1515*\dx},{0.1012*\dy}) % 0.0152
	-- ({0.1616*\dx},{0.1013*\dy}) % 0.0162
	-- ({0.1717*\dx},{0.1015*\dy}) % 0.0173
	-- ({0.1818*\dx},{0.1017*\dy}) % 0.0183
	-- ({0.1919*\dx},{0.1018*\dy}) % 0.0193
	-- ({0.2020*\dx},{0.1020*\dy}) % 0.0203
	-- ({0.2121*\dx},{0.1023*\dy}) % 0.0214
	-- ({0.2222*\dx},{0.1025*\dy}) % 0.0224
	-- ({0.2323*\dx},{0.1027*\dy}) % 0.0234
	-- ({0.2424*\dx},{0.1030*\dy}) % 0.0245
	-- ({0.2525*\dx},{0.1032*\dy}) % 0.0255
	-- ({0.2626*\dx},{0.1035*\dy}) % 0.0266
	-- ({0.2727*\dx},{0.1037*\dy}) % 0.0276
	-- ({0.2828*\dx},{0.1040*\dy}) % 0.0287
	-- ({0.2929*\dx},{0.1043*\dy}) % 0.0297
	-- ({0.3030*\dx},{0.1046*\dy}) % 0.0308
	-- ({0.3131*\dx},{0.1049*\dy}) % 0.0318
	-- ({0.3232*\dx},{0.1053*\dy}) % 0.0329
	-- ({0.3333*\dx},{0.1056*\dy}) % 0.0340
	-- ({0.3434*\dx},{0.1060*\dy}) % 0.0350
	-- ({0.3535*\dx},{0.1063*\dy}) % 0.0361
	-- ({0.3636*\dx},{0.1067*\dy}) % 0.0372
	-- ({0.3737*\dx},{0.1071*\dy}) % 0.0383
	-- ({0.3838*\dx},{0.1075*\dy}) % 0.0394
	-- ({0.3939*\dx},{0.1079*\dy}) % 0.0405
	-- ({0.4040*\dx},{0.1083*\dy}) % 0.0415
	-- ({0.4141*\dx},{0.1087*\dy}) % 0.0426
	-- ({0.4242*\dx},{0.1091*\dy}) % 0.0438
	-- ({0.4343*\dx},{0.1096*\dy}) % 0.0449
	-- ({0.4444*\dx},{0.1100*\dy}) % 0.0460
	-- ({0.4545*\dx},{0.1105*\dy}) % 0.0471
	-- ({0.4646*\dx},{0.1110*\dy}) % 0.0482
	-- ({0.4747*\dx},{0.1115*\dy}) % 0.0493
	-- ({0.4848*\dx},{0.1120*\dy}) % 0.0505
	-- ({0.4949*\dx},{0.1125*\dy}) % 0.0516
	-- ({0.5051*\dx},{0.1130*\dy}) % 0.0528
	-- ({0.5152*\dx},{0.1136*\dy}) % 0.0539
	-- ({0.5253*\dx},{0.1141*\dy}) % 0.0551
	-- ({0.5354*\dx},{0.1147*\dy}) % 0.0562
	-- ({0.5455*\dx},{0.1153*\dy}) % 0.0574
	-- ({0.5556*\dx},{0.1158*\dy}) % 0.0586
	-- ({0.5657*\dx},{0.1164*\dy}) % 0.0597
	-- ({0.5758*\dx},{0.1171*\dy}) % 0.0609
	-- ({0.5859*\dx},{0.1177*\dy}) % 0.0621
	-- ({0.5960*\dx},{0.1183*\dy}) % 0.0633
	-- ({0.6061*\dx},{0.1190*\dy}) % 0.0645
	-- ({0.6162*\dx},{0.1196*\dy}) % 0.0657
	-- ({0.6263*\dx},{0.1203*\dy}) % 0.0670
	-- ({0.6364*\dx},{0.1210*\dy}) % 0.0682
	-- ({0.6465*\dx},{0.1217*\dy}) % 0.0694
	-- ({0.6566*\dx},{0.1224*\dy}) % 0.0707
	-- ({0.6667*\dx},{0.1231*\dy}) % 0.0719
	-- ({0.6768*\dx},{0.1238*\dy}) % 0.0732
	-- ({0.6869*\dx},{0.1246*\dy}) % 0.0744
	-- ({0.6970*\dx},{0.1253*\dy}) % 0.0757
	-- ({0.7071*\dx},{0.1261*\dy}) % 0.0770
	-- ({0.7172*\dx},{0.1269*\dy}) % 0.0783
	-- ({0.7273*\dx},{0.1277*\dy}) % 0.0796
	-- ({0.7374*\dx},{0.1285*\dy}) % 0.0809
	-- ({0.7475*\dx},{0.1293*\dy}) % 0.0822
	-- ({0.7576*\dx},{0.1301*\dy}) % 0.0835
	-- ({0.7677*\dx},{0.1310*\dy}) % 0.0848
	-- ({0.7778*\dx},{0.1319*\dy}) % 0.0862
	-- ({0.7879*\dx},{0.1327*\dy}) % 0.0875
	-- ({0.7980*\dx},{0.1336*\dy}) % 0.0889
	-- ({0.8081*\dx},{0.1345*\dy}) % 0.0903
	-- ({0.8182*\dx},{0.1355*\dy}) % 0.0917
	-- ({0.8283*\dx},{0.1364*\dy}) % 0.0930
	-- ({0.8384*\dx},{0.1373*\dy}) % 0.0944
	-- ({0.8485*\dx},{0.1383*\dy}) % 0.0959
	-- ({0.8586*\dx},{0.1393*\dy}) % 0.0973
	-- ({0.8687*\dx},{0.1403*\dy}) % 0.0987
	-- ({0.8788*\dx},{0.1413*\dy}) % 0.1001
	-- ({0.8889*\dx},{0.1423*\dy}) % 0.1016
	-- ({0.8990*\dx},{0.1433*\dy}) % 0.1031
	-- ({0.9091*\dx},{0.1444*\dy}) % 0.1045
	-- ({0.9192*\dx},{0.1454*\dy}) % 0.1060
	-- ({0.9293*\dx},{0.1465*\dy}) % 0.1075
	-- ({0.9394*\dx},{0.1476*\dy}) % 0.1090
	-- ({0.9495*\dx},{0.1487*\dy}) % 0.1106
	-- ({0.9596*\dx},{0.1498*\dy}) % 0.1121
	-- ({0.9697*\dx},{0.1510*\dy}) % 0.1137
	-- ({0.9798*\dx},{0.1521*\dy}) % 0.1152
	-- ({0.9899*\dx},{0.1533*\dy}) % 0.1168
	-- ({1.0000*\dx},{0.1545*\dy}) % 0.1184
}
\def\kurvetwo{
	({-1.0000*\dx},{0.3101*\dy})
	-- ({-0.9899*\dx},{0.3077*\dy})
	-- ({-0.9798*\dx},{0.3053*\dy})
	-- ({-0.9697*\dx},{0.3029*\dy})
	-- ({-0.9596*\dx},{0.3006*\dy})
	-- ({-0.9495*\dx},{0.2983*\dy})
	-- ({-0.9394*\dx},{0.2960*\dy})
	-- ({-0.9293*\dx},{0.2938*\dy})
	-- ({-0.9192*\dx},{0.2916*\dy})
	-- ({-0.9091*\dx},{0.2894*\dy})
	-- ({-0.8990*\dx},{0.2873*\dy})
	-- ({-0.8889*\dx},{0.2852*\dy})
	-- ({-0.8788*\dx},{0.2831*\dy})
	-- ({-0.8687*\dx},{0.2811*\dy})
	-- ({-0.8586*\dx},{0.2791*\dy})
	-- ({-0.8485*\dx},{0.2771*\dy})
	-- ({-0.8384*\dx},{0.2752*\dy})
	-- ({-0.8283*\dx},{0.2732*\dy})
	-- ({-0.8182*\dx},{0.2713*\dy})
	-- ({-0.8081*\dx},{0.2695*\dy})
	-- ({-0.7980*\dx},{0.2676*\dy})
	-- ({-0.7879*\dx},{0.2658*\dy})
	-- ({-0.7778*\dx},{0.2641*\dy})
	-- ({-0.7677*\dx},{0.2623*\dy})
	-- ({-0.7576*\dx},{0.2606*\dy})
	-- ({-0.7475*\dx},{0.2589*\dy})
	-- ({-0.7374*\dx},{0.2572*\dy})
	-- ({-0.7273*\dx},{0.2556*\dy})
	-- ({-0.7172*\dx},{0.2540*\dy})
	-- ({-0.7071*\dx},{0.2524*\dy})
	-- ({-0.6970*\dx},{0.2509*\dy})
	-- ({-0.6869*\dx},{0.2493*\dy})
	-- ({-0.6768*\dx},{0.2478*\dy})
	-- ({-0.6667*\dx},{0.2463*\dy})
	-- ({-0.6566*\dx},{0.2449*\dy})
	-- ({-0.6465*\dx},{0.2435*\dy})
	-- ({-0.6364*\dx},{0.2421*\dy})
	-- ({-0.6263*\dx},{0.2407*\dy})
	-- ({-0.6162*\dx},{0.2393*\dy})
	-- ({-0.6061*\dx},{0.2380*\dy})
	-- ({-0.5960*\dx},{0.2367*\dy})
	-- ({-0.5859*\dx},{0.2355*\dy})
	-- ({-0.5758*\dx},{0.2342*\dy})
	-- ({-0.5657*\dx},{0.2330*\dy})
	-- ({-0.5556*\dx},{0.2318*\dy})
	-- ({-0.5455*\dx},{0.2306*\dy})
	-- ({-0.5354*\dx},{0.2294*\dy})
	-- ({-0.5253*\dx},{0.2283*\dy})
	-- ({-0.5152*\dx},{0.2272*\dy})
	-- ({-0.5051*\dx},{0.2261*\dy})
	-- ({-0.4949*\dx},{0.2251*\dy})
	-- ({-0.4848*\dx},{0.2240*\dy})
	-- ({-0.4747*\dx},{0.2230*\dy})
	-- ({-0.4646*\dx},{0.2220*\dy})
	-- ({-0.4545*\dx},{0.2211*\dy})
	-- ({-0.4444*\dx},{0.2201*\dy})
	-- ({-0.4343*\dx},{0.2192*\dy})
	-- ({-0.4242*\dx},{0.2183*\dy})
	-- ({-0.4141*\dx},{0.2174*\dy})
	-- ({-0.4040*\dx},{0.2166*\dy})
	-- ({-0.3939*\dx},{0.2157*\dy})
	-- ({-0.3838*\dx},{0.2149*\dy})
	-- ({-0.3737*\dx},{0.2142*\dy})
	-- ({-0.3636*\dx},{0.2134*\dy})
	-- ({-0.3535*\dx},{0.2126*\dy})
	-- ({-0.3434*\dx},{0.2119*\dy})
	-- ({-0.3333*\dx},{0.2112*\dy})
	-- ({-0.3232*\dx},{0.2106*\dy})
	-- ({-0.3131*\dx},{0.2099*\dy})
	-- ({-0.3030*\dx},{0.2093*\dy})
	-- ({-0.2929*\dx},{0.2086*\dy})
	-- ({-0.2828*\dx},{0.2081*\dy})
	-- ({-0.2727*\dx},{0.2075*\dy})
	-- ({-0.2626*\dx},{0.2069*\dy})
	-- ({-0.2525*\dx},{0.2064*\dy})
	-- ({-0.2424*\dx},{0.2059*\dy})
	-- ({-0.2323*\dx},{0.2054*\dy})
	-- ({-0.2222*\dx},{0.2050*\dy})
	-- ({-0.2121*\dx},{0.2045*\dy})
	-- ({-0.2020*\dx},{0.2041*\dy})
	-- ({-0.1919*\dx},{0.2037*\dy})
	-- ({-0.1818*\dx},{0.2033*\dy})
	-- ({-0.1717*\dx},{0.2030*\dy})
	-- ({-0.1616*\dx},{0.2026*\dy})
	-- ({-0.1515*\dx},{0.2023*\dy})
	-- ({-0.1414*\dx},{0.2020*\dy})
	-- ({-0.1313*\dx},{0.2017*\dy})
	-- ({-0.1212*\dx},{0.2015*\dy})
	-- ({-0.1111*\dx},{0.2012*\dy})
	-- ({-0.1010*\dx},{0.2010*\dy})
	-- ({-0.0909*\dx},{0.2008*\dy})
	-- ({-0.0808*\dx},{0.2007*\dy})
	-- ({-0.0707*\dx},{0.2005*\dy})
	-- ({-0.0606*\dx},{0.2004*\dy})
	-- ({-0.0505*\dx},{0.2003*\dy})
	-- ({-0.0404*\dx},{0.2002*\dy})
	-- ({-0.0303*\dx},{0.2001*\dy})
	-- ({-0.0202*\dx},{0.2000*\dy})
	-- ({-0.0101*\dx},{0.2000*\dy})
	-- ({-0.0000*\dx},{0.2000*\dy})
	-- ({0.0101*\dx},{0.2000*\dy}) % 0.0020
	-- ({0.0202*\dx},{0.2000*\dy}) % 0.0040
	-- ({0.0303*\dx},{0.2001*\dy}) % 0.0061
	-- ({0.0404*\dx},{0.2002*\dy}) % 0.0081
	-- ({0.0505*\dx},{0.2003*\dy}) % 0.0101
	-- ({0.0606*\dx},{0.2004*\dy}) % 0.0121
	-- ({0.0707*\dx},{0.2005*\dy}) % 0.0142
	-- ({0.0808*\dx},{0.2007*\dy}) % 0.0162
	-- ({0.0909*\dx},{0.2008*\dy}) % 0.0182
	-- ({0.1010*\dx},{0.2010*\dy}) % 0.0202
	-- ({0.1111*\dx},{0.2012*\dy}) % 0.0223
	-- ({0.1212*\dx},{0.2015*\dy}) % 0.0243
	-- ({0.1313*\dx},{0.2017*\dy}) % 0.0263
	-- ({0.1414*\dx},{0.2020*\dy}) % 0.0284
	-- ({0.1515*\dx},{0.2023*\dy}) % 0.0304
	-- ({0.1616*\dx},{0.2026*\dy}) % 0.0325
	-- ({0.1717*\dx},{0.2030*\dy}) % 0.0345
	-- ({0.1818*\dx},{0.2033*\dy}) % 0.0366
	-- ({0.1919*\dx},{0.2037*\dy}) % 0.0386
	-- ({0.2020*\dx},{0.2041*\dy}) % 0.0407
	-- ({0.2121*\dx},{0.2045*\dy}) % 0.0428
	-- ({0.2222*\dx},{0.2050*\dy}) % 0.0449
	-- ({0.2323*\dx},{0.2054*\dy}) % 0.0469
	-- ({0.2424*\dx},{0.2059*\dy}) % 0.0490
	-- ({0.2525*\dx},{0.2064*\dy}) % 0.0511
	-- ({0.2626*\dx},{0.2069*\dy}) % 0.0532
	-- ({0.2727*\dx},{0.2075*\dy}) % 0.0553
	-- ({0.2828*\dx},{0.2081*\dy}) % 0.0574
	-- ({0.2929*\dx},{0.2086*\dy}) % 0.0595
	-- ({0.3030*\dx},{0.2093*\dy}) % 0.0617
	-- ({0.3131*\dx},{0.2099*\dy}) % 0.0638
	-- ({0.3232*\dx},{0.2106*\dy}) % 0.0659
	-- ({0.3333*\dx},{0.2112*\dy}) % 0.0681
	-- ({0.3434*\dx},{0.2119*\dy}) % 0.0702
	-- ({0.3535*\dx},{0.2126*\dy}) % 0.0724
	-- ({0.3636*\dx},{0.2134*\dy}) % 0.0745
	-- ({0.3737*\dx},{0.2142*\dy}) % 0.0767
	-- ({0.3838*\dx},{0.2149*\dy}) % 0.0789
	-- ({0.3939*\dx},{0.2157*\dy}) % 0.0811
	-- ({0.4040*\dx},{0.2166*\dy}) % 0.0833
	-- ({0.4141*\dx},{0.2174*\dy}) % 0.0855
	-- ({0.4242*\dx},{0.2183*\dy}) % 0.0878
	-- ({0.4343*\dx},{0.2192*\dy}) % 0.0900
	-- ({0.4444*\dx},{0.2201*\dy}) % 0.0922
	-- ({0.4545*\dx},{0.2211*\dy}) % 0.0945
	-- ({0.4646*\dx},{0.2220*\dy}) % 0.0968
	-- ({0.4747*\dx},{0.2230*\dy}) % 0.0990
	-- ({0.4848*\dx},{0.2240*\dy}) % 0.1013
	-- ({0.4949*\dx},{0.2251*\dy}) % 0.1036
	-- ({0.5051*\dx},{0.2261*\dy}) % 0.1060
	-- ({0.5152*\dx},{0.2272*\dy}) % 0.1083
	-- ({0.5253*\dx},{0.2283*\dy}) % 0.1106
	-- ({0.5354*\dx},{0.2294*\dy}) % 0.1130
	-- ({0.5455*\dx},{0.2306*\dy}) % 0.1154
	-- ({0.5556*\dx},{0.2318*\dy}) % 0.1177
	-- ({0.5657*\dx},{0.2330*\dy}) % 0.1201
	-- ({0.5758*\dx},{0.2342*\dy}) % 0.1225
	-- ({0.5859*\dx},{0.2355*\dy}) % 0.1250
	-- ({0.5960*\dx},{0.2367*\dy}) % 0.1274
	-- ({0.6061*\dx},{0.2380*\dy}) % 0.1299
	-- ({0.6162*\dx},{0.2393*\dy}) % 0.1323
	-- ({0.6263*\dx},{0.2407*\dy}) % 0.1348
	-- ({0.6364*\dx},{0.2421*\dy}) % 0.1373
	-- ({0.6465*\dx},{0.2435*\dy}) % 0.1399
	-- ({0.6566*\dx},{0.2449*\dy}) % 0.1424
	-- ({0.6667*\dx},{0.2463*\dy}) % 0.1450
	-- ({0.6768*\dx},{0.2478*\dy}) % 0.1475
	-- ({0.6869*\dx},{0.2493*\dy}) % 0.1501
	-- ({0.6970*\dx},{0.2509*\dy}) % 0.1527
	-- ({0.7071*\dx},{0.2524*\dy}) % 0.1554
	-- ({0.7172*\dx},{0.2540*\dy}) % 0.1580
	-- ({0.7273*\dx},{0.2556*\dy}) % 0.1607
	-- ({0.7374*\dx},{0.2572*\dy}) % 0.1634
	-- ({0.7475*\dx},{0.2589*\dy}) % 0.1661
	-- ({0.7576*\dx},{0.2606*\dy}) % 0.1688
	-- ({0.7677*\dx},{0.2623*\dy}) % 0.1716
	-- ({0.7778*\dx},{0.2641*\dy}) % 0.1744
	-- ({0.7879*\dx},{0.2658*\dy}) % 0.1772
	-- ({0.7980*\dx},{0.2676*\dy}) % 0.1800
	-- ({0.8081*\dx},{0.2695*\dy}) % 0.1829
	-- ({0.8182*\dx},{0.2713*\dy}) % 0.1857
	-- ({0.8283*\dx},{0.2732*\dy}) % 0.1886
	-- ({0.8384*\dx},{0.2752*\dy}) % 0.1915
	-- ({0.8485*\dx},{0.2771*\dy}) % 0.1945
	-- ({0.8586*\dx},{0.2791*\dy}) % 0.1975
	-- ({0.8687*\dx},{0.2811*\dy}) % 0.2005
	-- ({0.8788*\dx},{0.2831*\dy}) % 0.2035
	-- ({0.8889*\dx},{0.2852*\dy}) % 0.2065
	-- ({0.8990*\dx},{0.2873*\dy}) % 0.2096
	-- ({0.9091*\dx},{0.2894*\dy}) % 0.2127
	-- ({0.9192*\dx},{0.2916*\dy}) % 0.2159
	-- ({0.9293*\dx},{0.2938*\dy}) % 0.2190
	-- ({0.9394*\dx},{0.2960*\dy}) % 0.2222
	-- ({0.9495*\dx},{0.2983*\dy}) % 0.2255
	-- ({0.9596*\dx},{0.3006*\dy}) % 0.2287
	-- ({0.9697*\dx},{0.3029*\dy}) % 0.2320
	-- ({0.9798*\dx},{0.3053*\dy}) % 0.2353
	-- ({0.9899*\dx},{0.3077*\dy}) % 0.2387
	-- ({1.0000*\dx},{0.3101*\dy}) % 0.2421
}
\def\kurvethree{
	({-1.0000*\dx},{0.4681*\dy})
	-- ({-0.9899*\dx},{0.4643*\dy})
	-- ({-0.9798*\dx},{0.4606*\dy})
	-- ({-0.9697*\dx},{0.4569*\dy})
	-- ({-0.9596*\dx},{0.4533*\dy})
	-- ({-0.9495*\dx},{0.4497*\dy})
	-- ({-0.9394*\dx},{0.4462*\dy})
	-- ({-0.9293*\dx},{0.4428*\dy})
	-- ({-0.9192*\dx},{0.4394*\dy})
	-- ({-0.9091*\dx},{0.4360*\dy})
	-- ({-0.8990*\dx},{0.4327*\dy})
	-- ({-0.8889*\dx},{0.4295*\dy})
	-- ({-0.8788*\dx},{0.4263*\dy})
	-- ({-0.8687*\dx},{0.4231*\dy})
	-- ({-0.8586*\dx},{0.4200*\dy})
	-- ({-0.8485*\dx},{0.4170*\dy})
	-- ({-0.8384*\dx},{0.4140*\dy})
	-- ({-0.8283*\dx},{0.4110*\dy})
	-- ({-0.8182*\dx},{0.4081*\dy})
	-- ({-0.8081*\dx},{0.4053*\dy})
	-- ({-0.7980*\dx},{0.4025*\dy})
	-- ({-0.7879*\dx},{0.3997*\dy})
	-- ({-0.7778*\dx},{0.3970*\dy})
	-- ({-0.7677*\dx},{0.3943*\dy})
	-- ({-0.7576*\dx},{0.3917*\dy})
	-- ({-0.7475*\dx},{0.3891*\dy})
	-- ({-0.7374*\dx},{0.3866*\dy})
	-- ({-0.7273*\dx},{0.3841*\dy})
	-- ({-0.7172*\dx},{0.3816*\dy})
	-- ({-0.7071*\dx},{0.3792*\dy})
	-- ({-0.6970*\dx},{0.3768*\dy})
	-- ({-0.6869*\dx},{0.3745*\dy})
	-- ({-0.6768*\dx},{0.3722*\dy})
	-- ({-0.6667*\dx},{0.3700*\dy})
	-- ({-0.6566*\dx},{0.3678*\dy})
	-- ({-0.6465*\dx},{0.3656*\dy})
	-- ({-0.6364*\dx},{0.3635*\dy})
	-- ({-0.6263*\dx},{0.3614*\dy})
	-- ({-0.6162*\dx},{0.3593*\dy})
	-- ({-0.6061*\dx},{0.3573*\dy})
	-- ({-0.5960*\dx},{0.3554*\dy})
	-- ({-0.5859*\dx},{0.3534*\dy})
	-- ({-0.5758*\dx},{0.3515*\dy})
	-- ({-0.5657*\dx},{0.3497*\dy})
	-- ({-0.5556*\dx},{0.3479*\dy})
	-- ({-0.5455*\dx},{0.3461*\dy})
	-- ({-0.5354*\dx},{0.3443*\dy})
	-- ({-0.5253*\dx},{0.3426*\dy})
	-- ({-0.5152*\dx},{0.3410*\dy})
	-- ({-0.5051*\dx},{0.3393*\dy})
	-- ({-0.4949*\dx},{0.3377*\dy})
	-- ({-0.4848*\dx},{0.3362*\dy})
	-- ({-0.4747*\dx},{0.3346*\dy})
	-- ({-0.4646*\dx},{0.3331*\dy})
	-- ({-0.4545*\dx},{0.3317*\dy})
	-- ({-0.4444*\dx},{0.3303*\dy})
	-- ({-0.4343*\dx},{0.3289*\dy})
	-- ({-0.4242*\dx},{0.3275*\dy})
	-- ({-0.4141*\dx},{0.3262*\dy})
	-- ({-0.4040*\dx},{0.3249*\dy})
	-- ({-0.3939*\dx},{0.3237*\dy})
	-- ({-0.3838*\dx},{0.3225*\dy})
	-- ({-0.3737*\dx},{0.3213*\dy})
	-- ({-0.3636*\dx},{0.3201*\dy})
	-- ({-0.3535*\dx},{0.3190*\dy})
	-- ({-0.3434*\dx},{0.3179*\dy})
	-- ({-0.3333*\dx},{0.3169*\dy})
	-- ({-0.3232*\dx},{0.3158*\dy})
	-- ({-0.3131*\dx},{0.3149*\dy})
	-- ({-0.3030*\dx},{0.3139*\dy})
	-- ({-0.2929*\dx},{0.3130*\dy})
	-- ({-0.2828*\dx},{0.3121*\dy})
	-- ({-0.2727*\dx},{0.3112*\dy})
	-- ({-0.2626*\dx},{0.3104*\dy})
	-- ({-0.2525*\dx},{0.3096*\dy})
	-- ({-0.2424*\dx},{0.3089*\dy})
	-- ({-0.2323*\dx},{0.3081*\dy})
	-- ({-0.2222*\dx},{0.3074*\dy})
	-- ({-0.2121*\dx},{0.3068*\dy})
	-- ({-0.2020*\dx},{0.3061*\dy})
	-- ({-0.1919*\dx},{0.3055*\dy})
	-- ({-0.1818*\dx},{0.3050*\dy})
	-- ({-0.1717*\dx},{0.3044*\dy})
	-- ({-0.1616*\dx},{0.3039*\dy})
	-- ({-0.1515*\dx},{0.3035*\dy})
	-- ({-0.1414*\dx},{0.3030*\dy})
	-- ({-0.1313*\dx},{0.3026*\dy})
	-- ({-0.1212*\dx},{0.3022*\dy})
	-- ({-0.1111*\dx},{0.3019*\dy})
	-- ({-0.1010*\dx},{0.3015*\dy})
	-- ({-0.0909*\dx},{0.3012*\dy})
	-- ({-0.0808*\dx},{0.3010*\dy})
	-- ({-0.0707*\dx},{0.3008*\dy})
	-- ({-0.0606*\dx},{0.3006*\dy})
	-- ({-0.0505*\dx},{0.3004*\dy})
	-- ({-0.0404*\dx},{0.3002*\dy})
	-- ({-0.0303*\dx},{0.3001*\dy})
	-- ({-0.0202*\dx},{0.3001*\dy})
	-- ({-0.0101*\dx},{0.3000*\dy})
	-- ({-0.0000*\dx},{0.3000*\dy})
	-- ({0.0101*\dx},{0.3000*\dy}) % 0.0030
	-- ({0.0202*\dx},{0.3001*\dy}) % 0.0061
	-- ({0.0303*\dx},{0.3001*\dy}) % 0.0091
	-- ({0.0404*\dx},{0.3002*\dy}) % 0.0121
	-- ({0.0505*\dx},{0.3004*\dy}) % 0.0152
	-- ({0.0606*\dx},{0.3006*\dy}) % 0.0182
	-- ({0.0707*\dx},{0.3008*\dy}) % 0.0212
	-- ({0.0808*\dx},{0.3010*\dy}) % 0.0243
	-- ({0.0909*\dx},{0.3012*\dy}) % 0.0273
	-- ({0.1010*\dx},{0.3015*\dy}) % 0.0304
	-- ({0.1111*\dx},{0.3019*\dy}) % 0.0334
	-- ({0.1212*\dx},{0.3022*\dy}) % 0.0365
	-- ({0.1313*\dx},{0.3026*\dy}) % 0.0395
	-- ({0.1414*\dx},{0.3030*\dy}) % 0.0426
	-- ({0.1515*\dx},{0.3035*\dy}) % 0.0457
	-- ({0.1616*\dx},{0.3039*\dy}) % 0.0488
	-- ({0.1717*\dx},{0.3044*\dy}) % 0.0518
	-- ({0.1818*\dx},{0.3050*\dy}) % 0.0549
	-- ({0.1919*\dx},{0.3055*\dy}) % 0.0580
	-- ({0.2020*\dx},{0.3061*\dy}) % 0.0611
	-- ({0.2121*\dx},{0.3068*\dy}) % 0.0642
	-- ({0.2222*\dx},{0.3074*\dy}) % 0.0674
	-- ({0.2323*\dx},{0.3081*\dy}) % 0.0705
	-- ({0.2424*\dx},{0.3089*\dy}) % 0.0736
	-- ({0.2525*\dx},{0.3096*\dy}) % 0.0768
	-- ({0.2626*\dx},{0.3104*\dy}) % 0.0800
	-- ({0.2727*\dx},{0.3112*\dy}) % 0.0831
	-- ({0.2828*\dx},{0.3121*\dy}) % 0.0863
	-- ({0.2929*\dx},{0.3130*\dy}) % 0.0895
	-- ({0.3030*\dx},{0.3139*\dy}) % 0.0927
	-- ({0.3131*\dx},{0.3149*\dy}) % 0.0959
	-- ({0.3232*\dx},{0.3158*\dy}) % 0.0992
	-- ({0.3333*\dx},{0.3169*\dy}) % 0.1024
	-- ({0.3434*\dx},{0.3179*\dy}) % 0.1057
	-- ({0.3535*\dx},{0.3190*\dy}) % 0.1089
	-- ({0.3636*\dx},{0.3201*\dy}) % 0.1122
	-- ({0.3737*\dx},{0.3213*\dy}) % 0.1155
	-- ({0.3838*\dx},{0.3225*\dy}) % 0.1188
	-- ({0.3939*\dx},{0.3237*\dy}) % 0.1222
	-- ({0.4040*\dx},{0.3249*\dy}) % 0.1255
	-- ({0.4141*\dx},{0.3262*\dy}) % 0.1289
	-- ({0.4242*\dx},{0.3275*\dy}) % 0.1323
	-- ({0.4343*\dx},{0.3289*\dy}) % 0.1357
	-- ({0.4444*\dx},{0.3303*\dy}) % 0.1391
	-- ({0.4545*\dx},{0.3317*\dy}) % 0.1425
	-- ({0.4646*\dx},{0.3331*\dy}) % 0.1460
	-- ({0.4747*\dx},{0.3346*\dy}) % 0.1495
	-- ({0.4848*\dx},{0.3362*\dy}) % 0.1530
	-- ({0.4949*\dx},{0.3377*\dy}) % 0.1565
	-- ({0.5051*\dx},{0.3393*\dy}) % 0.1601
	-- ({0.5152*\dx},{0.3410*\dy}) % 0.1636
	-- ({0.5253*\dx},{0.3426*\dy}) % 0.1672
	-- ({0.5354*\dx},{0.3443*\dy}) % 0.1709
	-- ({0.5455*\dx},{0.3461*\dy}) % 0.1745
	-- ({0.5556*\dx},{0.3479*\dy}) % 0.1782
	-- ({0.5657*\dx},{0.3497*\dy}) % 0.1819
	-- ({0.5758*\dx},{0.3515*\dy}) % 0.1856
	-- ({0.5859*\dx},{0.3534*\dy}) % 0.1893
	-- ({0.5960*\dx},{0.3554*\dy}) % 0.1931
	-- ({0.6061*\dx},{0.3573*\dy}) % 0.1969
	-- ({0.6162*\dx},{0.3593*\dy}) % 0.2008
	-- ({0.6263*\dx},{0.3614*\dy}) % 0.2046
	-- ({0.6364*\dx},{0.3635*\dy}) % 0.2085
	-- ({0.6465*\dx},{0.3656*\dy}) % 0.2125
	-- ({0.6566*\dx},{0.3678*\dy}) % 0.2164
	-- ({0.6667*\dx},{0.3700*\dy}) % 0.2204
	-- ({0.6768*\dx},{0.3722*\dy}) % 0.2244
	-- ({0.6869*\dx},{0.3745*\dy}) % 0.2285
	-- ({0.6970*\dx},{0.3768*\dy}) % 0.2326
	-- ({0.7071*\dx},{0.3792*\dy}) % 0.2368
	-- ({0.7172*\dx},{0.3816*\dy}) % 0.2409
	-- ({0.7273*\dx},{0.3841*\dy}) % 0.2451
	-- ({0.7374*\dx},{0.3866*\dy}) % 0.2494
	-- ({0.7475*\dx},{0.3891*\dy}) % 0.2537
	-- ({0.7576*\dx},{0.3917*\dy}) % 0.2580
	-- ({0.7677*\dx},{0.3943*\dy}) % 0.2624
	-- ({0.7778*\dx},{0.3970*\dy}) % 0.2668
	-- ({0.7879*\dx},{0.3997*\dy}) % 0.2713
	-- ({0.7980*\dx},{0.4025*\dy}) % 0.2758
	-- ({0.8081*\dx},{0.4053*\dy}) % 0.2804
	-- ({0.8182*\dx},{0.4081*\dy}) % 0.2850
	-- ({0.8283*\dx},{0.4110*\dy}) % 0.2896
	-- ({0.8384*\dx},{0.4140*\dy}) % 0.2944
	-- ({0.8485*\dx},{0.4170*\dy}) % 0.2991
	-- ({0.8586*\dx},{0.4200*\dy}) % 0.3039
	-- ({0.8687*\dx},{0.4231*\dy}) % 0.3088
	-- ({0.8788*\dx},{0.4263*\dy}) % 0.3137
	-- ({0.8889*\dx},{0.4295*\dy}) % 0.3187
	-- ({0.8990*\dx},{0.4327*\dy}) % 0.3238
	-- ({0.9091*\dx},{0.4360*\dy}) % 0.3289
	-- ({0.9192*\dx},{0.4394*\dy}) % 0.3340
	-- ({0.9293*\dx},{0.4428*\dy}) % 0.3393
	-- ({0.9394*\dx},{0.4462*\dy}) % 0.3446
	-- ({0.9495*\dx},{0.4497*\dy}) % 0.3499
	-- ({0.9596*\dx},{0.4533*\dy}) % 0.3554
	-- ({0.9697*\dx},{0.4569*\dy}) % 0.3609
	-- ({0.9798*\dx},{0.4606*\dy}) % 0.3665
	-- ({0.9899*\dx},{0.4643*\dy}) % 0.3721
	-- ({1.0000*\dx},{0.4681*\dy}) % 0.3779
}
\def\kurvefour{
	({-1.0000*\dx},{0.6302*\dy})
	-- ({-0.9899*\dx},{0.6248*\dy})
	-- ({-0.9798*\dx},{0.6196*\dy})
	-- ({-0.9697*\dx},{0.6144*\dy})
	-- ({-0.9596*\dx},{0.6093*\dy})
	-- ({-0.9495*\dx},{0.6043*\dy})
	-- ({-0.9394*\dx},{0.5994*\dy})
	-- ({-0.9293*\dx},{0.5945*\dy})
	-- ({-0.9192*\dx},{0.5898*\dy})
	-- ({-0.9091*\dx},{0.5851*\dy})
	-- ({-0.8990*\dx},{0.5805*\dy})
	-- ({-0.8889*\dx},{0.5760*\dy})
	-- ({-0.8788*\dx},{0.5715*\dy})
	-- ({-0.8687*\dx},{0.5672*\dy})
	-- ({-0.8586*\dx},{0.5629*\dy})
	-- ({-0.8485*\dx},{0.5587*\dy})
	-- ({-0.8384*\dx},{0.5545*\dy})
	-- ({-0.8283*\dx},{0.5504*\dy})
	-- ({-0.8182*\dx},{0.5464*\dy})
	-- ({-0.8081*\dx},{0.5425*\dy})
	-- ({-0.7980*\dx},{0.5386*\dy})
	-- ({-0.7879*\dx},{0.5348*\dy})
	-- ({-0.7778*\dx},{0.5311*\dy})
	-- ({-0.7677*\dx},{0.5274*\dy})
	-- ({-0.7576*\dx},{0.5238*\dy})
	-- ({-0.7475*\dx},{0.5203*\dy})
	-- ({-0.7374*\dx},{0.5168*\dy})
	-- ({-0.7273*\dx},{0.5134*\dy})
	-- ({-0.7172*\dx},{0.5100*\dy})
	-- ({-0.7071*\dx},{0.5067*\dy})
	-- ({-0.6970*\dx},{0.5035*\dy})
	-- ({-0.6869*\dx},{0.5003*\dy})
	-- ({-0.6768*\dx},{0.4972*\dy})
	-- ({-0.6667*\dx},{0.4942*\dy})
	-- ({-0.6566*\dx},{0.4912*\dy})
	-- ({-0.6465*\dx},{0.4882*\dy})
	-- ({-0.6364*\dx},{0.4853*\dy})
	-- ({-0.6263*\dx},{0.4825*\dy})
	-- ({-0.6162*\dx},{0.4797*\dy})
	-- ({-0.6061*\dx},{0.4770*\dy})
	-- ({-0.5960*\dx},{0.4743*\dy})
	-- ({-0.5859*\dx},{0.4717*\dy})
	-- ({-0.5758*\dx},{0.4692*\dy})
	-- ({-0.5657*\dx},{0.4667*\dy})
	-- ({-0.5556*\dx},{0.4642*\dy})
	-- ({-0.5455*\dx},{0.4618*\dy})
	-- ({-0.5354*\dx},{0.4595*\dy})
	-- ({-0.5253*\dx},{0.4571*\dy})
	-- ({-0.5152*\dx},{0.4549*\dy})
	-- ({-0.5051*\dx},{0.4527*\dy})
	-- ({-0.4949*\dx},{0.4505*\dy})
	-- ({-0.4848*\dx},{0.4484*\dy})
	-- ({-0.4747*\dx},{0.4464*\dy})
	-- ({-0.4646*\dx},{0.4444*\dy})
	-- ({-0.4545*\dx},{0.4424*\dy})
	-- ({-0.4444*\dx},{0.4405*\dy})
	-- ({-0.4343*\dx},{0.4386*\dy})
	-- ({-0.4242*\dx},{0.4368*\dy})
	-- ({-0.4141*\dx},{0.4351*\dy})
	-- ({-0.4040*\dx},{0.4333*\dy})
	-- ({-0.3939*\dx},{0.4316*\dy})
	-- ({-0.3838*\dx},{0.4300*\dy})
	-- ({-0.3737*\dx},{0.4284*\dy})
	-- ({-0.3636*\dx},{0.4269*\dy})
	-- ({-0.3535*\dx},{0.4254*\dy})
	-- ({-0.3434*\dx},{0.4239*\dy})
	-- ({-0.3333*\dx},{0.4225*\dy})
	-- ({-0.3232*\dx},{0.4212*\dy})
	-- ({-0.3131*\dx},{0.4199*\dy})
	-- ({-0.3030*\dx},{0.4186*\dy})
	-- ({-0.2929*\dx},{0.4173*\dy})
	-- ({-0.2828*\dx},{0.4162*\dy})
	-- ({-0.2727*\dx},{0.4150*\dy})
	-- ({-0.2626*\dx},{0.4139*\dy})
	-- ({-0.2525*\dx},{0.4129*\dy})
	-- ({-0.2424*\dx},{0.4118*\dy})
	-- ({-0.2323*\dx},{0.4109*\dy})
	-- ({-0.2222*\dx},{0.4099*\dy})
	-- ({-0.2121*\dx},{0.4090*\dy})
	-- ({-0.2020*\dx},{0.4082*\dy})
	-- ({-0.1919*\dx},{0.4074*\dy})
	-- ({-0.1818*\dx},{0.4066*\dy})
	-- ({-0.1717*\dx},{0.4059*\dy})
	-- ({-0.1616*\dx},{0.4052*\dy})
	-- ({-0.1515*\dx},{0.4046*\dy})
	-- ({-0.1414*\dx},{0.4040*\dy})
	-- ({-0.1313*\dx},{0.4035*\dy})
	-- ({-0.1212*\dx},{0.4029*\dy})
	-- ({-0.1111*\dx},{0.4025*\dy})
	-- ({-0.1010*\dx},{0.4020*\dy})
	-- ({-0.0909*\dx},{0.4017*\dy})
	-- ({-0.0808*\dx},{0.4013*\dy})
	-- ({-0.0707*\dx},{0.4010*\dy})
	-- ({-0.0606*\dx},{0.4007*\dy})
	-- ({-0.0505*\dx},{0.4005*\dy})
	-- ({-0.0404*\dx},{0.4003*\dy})
	-- ({-0.0303*\dx},{0.4002*\dy})
	-- ({-0.0202*\dx},{0.4001*\dy})
	-- ({-0.0101*\dx},{0.4000*\dy})
	-- ({-0.0000*\dx},{0.4000*\dy})
	-- ({0.0101*\dx},{0.4000*\dy}) % 0.0040
	-- ({0.0202*\dx},{0.4001*\dy}) % 0.0081
	-- ({0.0303*\dx},{0.4002*\dy}) % 0.0121
	-- ({0.0404*\dx},{0.4003*\dy}) % 0.0162
	-- ({0.0505*\dx},{0.4005*\dy}) % 0.0202
	-- ({0.0606*\dx},{0.4007*\dy}) % 0.0243
	-- ({0.0707*\dx},{0.4010*\dy}) % 0.0283
	-- ({0.0808*\dx},{0.4013*\dy}) % 0.0324
	-- ({0.0909*\dx},{0.4017*\dy}) % 0.0364
	-- ({0.1010*\dx},{0.4020*\dy}) % 0.0405
	-- ({0.1111*\dx},{0.4025*\dy}) % 0.0446
	-- ({0.1212*\dx},{0.4029*\dy}) % 0.0487
	-- ({0.1313*\dx},{0.4035*\dy}) % 0.0527
	-- ({0.1414*\dx},{0.4040*\dy}) % 0.0568
	-- ({0.1515*\dx},{0.4046*\dy}) % 0.0610
	-- ({0.1616*\dx},{0.4052*\dy}) % 0.0651
	-- ({0.1717*\dx},{0.4059*\dy}) % 0.0692
	-- ({0.1818*\dx},{0.4066*\dy}) % 0.0733
	-- ({0.1919*\dx},{0.4074*\dy}) % 0.0775
	-- ({0.2020*\dx},{0.4082*\dy}) % 0.0816
	-- ({0.2121*\dx},{0.4090*\dy}) % 0.0858
	-- ({0.2222*\dx},{0.4099*\dy}) % 0.0900
	-- ({0.2323*\dx},{0.4109*\dy}) % 0.0942
	-- ({0.2424*\dx},{0.4118*\dy}) % 0.0984
	-- ({0.2525*\dx},{0.4129*\dy}) % 0.1026
	-- ({0.2626*\dx},{0.4139*\dy}) % 0.1069
	-- ({0.2727*\dx},{0.4150*\dy}) % 0.1111
	-- ({0.2828*\dx},{0.4162*\dy}) % 0.1154
	-- ({0.2929*\dx},{0.4173*\dy}) % 0.1197
	-- ({0.3030*\dx},{0.4186*\dy}) % 0.1240
	-- ({0.3131*\dx},{0.4199*\dy}) % 0.1284
	-- ({0.3232*\dx},{0.4212*\dy}) % 0.1327
	-- ({0.3333*\dx},{0.4225*\dy}) % 0.1371
	-- ({0.3434*\dx},{0.4239*\dy}) % 0.1415
	-- ({0.3535*\dx},{0.4254*\dy}) % 0.1459
	-- ({0.3636*\dx},{0.4269*\dy}) % 0.1504
	-- ({0.3737*\dx},{0.4284*\dy}) % 0.1548
	-- ({0.3838*\dx},{0.4300*\dy}) % 0.1593
	-- ({0.3939*\dx},{0.4316*\dy}) % 0.1639
	-- ({0.4040*\dx},{0.4333*\dy}) % 0.1684
	-- ({0.4141*\dx},{0.4351*\dy}) % 0.1730
	-- ({0.4242*\dx},{0.4368*\dy}) % 0.1776
	-- ({0.4343*\dx},{0.4386*\dy}) % 0.1822
	-- ({0.4444*\dx},{0.4405*\dy}) % 0.1869
	-- ({0.4545*\dx},{0.4424*\dy}) % 0.1916
	-- ({0.4646*\dx},{0.4444*\dy}) % 0.1963
	-- ({0.4747*\dx},{0.4464*\dy}) % 0.2011
	-- ({0.4848*\dx},{0.4484*\dy}) % 0.2059
	-- ({0.4949*\dx},{0.4505*\dy}) % 0.2107
	-- ({0.5051*\dx},{0.4527*\dy}) % 0.2156
	-- ({0.5152*\dx},{0.4549*\dy}) % 0.2205
	-- ({0.5253*\dx},{0.4571*\dy}) % 0.2255
	-- ({0.5354*\dx},{0.4595*\dy}) % 0.2305
	-- ({0.5455*\dx},{0.4618*\dy}) % 0.2355
	-- ({0.5556*\dx},{0.4642*\dy}) % 0.2406
	-- ({0.5657*\dx},{0.4667*\dy}) % 0.2457
	-- ({0.5758*\dx},{0.4692*\dy}) % 0.2509
	-- ({0.5859*\dx},{0.4717*\dy}) % 0.2561
	-- ({0.5960*\dx},{0.4743*\dy}) % 0.2614
	-- ({0.6061*\dx},{0.4770*\dy}) % 0.2667
	-- ({0.6162*\dx},{0.4797*\dy}) % 0.2721
	-- ({0.6263*\dx},{0.4825*\dy}) % 0.2775
	-- ({0.6364*\dx},{0.4853*\dy}) % 0.2830
	-- ({0.6465*\dx},{0.4882*\dy}) % 0.2885
	-- ({0.6566*\dx},{0.4912*\dy}) % 0.2941
	-- ({0.6667*\dx},{0.4942*\dy}) % 0.2997
	-- ({0.6768*\dx},{0.4972*\dy}) % 0.3054
	-- ({0.6869*\dx},{0.5003*\dy}) % 0.3112
	-- ({0.6970*\dx},{0.5035*\dy}) % 0.3170
	-- ({0.7071*\dx},{0.5067*\dy}) % 0.3230
	-- ({0.7172*\dx},{0.5100*\dy}) % 0.3289
	-- ({0.7273*\dx},{0.5134*\dy}) % 0.3350
	-- ({0.7374*\dx},{0.5168*\dy}) % 0.3411
	-- ({0.7475*\dx},{0.5203*\dy}) % 0.3473
	-- ({0.7576*\dx},{0.5238*\dy}) % 0.3536
	-- ({0.7677*\dx},{0.5274*\dy}) % 0.3599
	-- ({0.7778*\dx},{0.5311*\dy}) % 0.3664
	-- ({0.7879*\dx},{0.5348*\dy}) % 0.3729
	-- ({0.7980*\dx},{0.5386*\dy}) % 0.3795
	-- ({0.8081*\dx},{0.5425*\dy}) % 0.3862
	-- ({0.8182*\dx},{0.5464*\dy}) % 0.3930
	-- ({0.8283*\dx},{0.5504*\dy}) % 0.3999
	-- ({0.8384*\dx},{0.5545*\dy}) % 0.4069
	-- ({0.8485*\dx},{0.5587*\dy}) % 0.4140
	-- ({0.8586*\dx},{0.5629*\dy}) % 0.4212
	-- ({0.8687*\dx},{0.5672*\dy}) % 0.4285
	-- ({0.8788*\dx},{0.5715*\dy}) % 0.4360
	-- ({0.8889*\dx},{0.5760*\dy}) % 0.4435
	-- ({0.8990*\dx},{0.5805*\dy}) % 0.4512
	-- ({0.9091*\dx},{0.5851*\dy}) % 0.4590
	-- ({0.9192*\dx},{0.5898*\dy}) % 0.4669
	-- ({0.9293*\dx},{0.5945*\dy}) % 0.4750
	-- ({0.9394*\dx},{0.5994*\dy}) % 0.4832
	-- ({0.9495*\dx},{0.6043*\dy}) % 0.4916
	-- ({0.9596*\dx},{0.6093*\dy}) % 0.5001
	-- ({0.9697*\dx},{0.6144*\dy}) % 0.5088
	-- ({0.9798*\dx},{0.6196*\dy}) % 0.5177
	-- ({0.9899*\dx},{0.6248*\dy}) % 0.5267
	-- ({1.0000*\dx},{0.6302*\dy}) % 0.5359
}
\def\kurvefive{
	({-1.0000*\dx},{0.7991*\dy})
	-- ({-0.9899*\dx},{0.7917*\dy})
	-- ({-0.9798*\dx},{0.7845*\dy})
	-- ({-0.9697*\dx},{0.7775*\dy})
	-- ({-0.9596*\dx},{0.7706*\dy})
	-- ({-0.9495*\dx},{0.7638*\dy})
	-- ({-0.9394*\dx},{0.7572*\dy})
	-- ({-0.9293*\dx},{0.7506*\dy})
	-- ({-0.9192*\dx},{0.7443*\dy})
	-- ({-0.9091*\dx},{0.7380*\dy})
	-- ({-0.8990*\dx},{0.7319*\dy})
	-- ({-0.8889*\dx},{0.7259*\dy})
	-- ({-0.8788*\dx},{0.7200*\dy})
	-- ({-0.8687*\dx},{0.7142*\dy})
	-- ({-0.8586*\dx},{0.7085*\dy})
	-- ({-0.8485*\dx},{0.7030*\dy})
	-- ({-0.8384*\dx},{0.6975*\dy})
	-- ({-0.8283*\dx},{0.6922*\dy})
	-- ({-0.8182*\dx},{0.6869*\dy})
	-- ({-0.8081*\dx},{0.6818*\dy})
	-- ({-0.7980*\dx},{0.6767*\dy})
	-- ({-0.7879*\dx},{0.6718*\dy})
	-- ({-0.7778*\dx},{0.6669*\dy})
	-- ({-0.7677*\dx},{0.6621*\dy})
	-- ({-0.7576*\dx},{0.6574*\dy})
	-- ({-0.7475*\dx},{0.6528*\dy})
	-- ({-0.7374*\dx},{0.6483*\dy})
	-- ({-0.7273*\dx},{0.6439*\dy})
	-- ({-0.7172*\dx},{0.6396*\dy})
	-- ({-0.7071*\dx},{0.6354*\dy})
	-- ({-0.6970*\dx},{0.6312*\dy})
	-- ({-0.6869*\dx},{0.6271*\dy})
	-- ({-0.6768*\dx},{0.6231*\dy})
	-- ({-0.6667*\dx},{0.6192*\dy})
	-- ({-0.6566*\dx},{0.6153*\dy})
	-- ({-0.6465*\dx},{0.6116*\dy})
	-- ({-0.6364*\dx},{0.6079*\dy})
	-- ({-0.6263*\dx},{0.6042*\dy})
	-- ({-0.6162*\dx},{0.6007*\dy})
	-- ({-0.6061*\dx},{0.5972*\dy})
	-- ({-0.5960*\dx},{0.5938*\dy})
	-- ({-0.5859*\dx},{0.5905*\dy})
	-- ({-0.5758*\dx},{0.5872*\dy})
	-- ({-0.5657*\dx},{0.5840*\dy})
	-- ({-0.5556*\dx},{0.5809*\dy})
	-- ({-0.5455*\dx},{0.5779*\dy})
	-- ({-0.5354*\dx},{0.5749*\dy})
	-- ({-0.5253*\dx},{0.5719*\dy})
	-- ({-0.5152*\dx},{0.5691*\dy})
	-- ({-0.5051*\dx},{0.5663*\dy})
	-- ({-0.4949*\dx},{0.5636*\dy})
	-- ({-0.4848*\dx},{0.5609*\dy})
	-- ({-0.4747*\dx},{0.5583*\dy})
	-- ({-0.4646*\dx},{0.5558*\dy})
	-- ({-0.4545*\dx},{0.5533*\dy})
	-- ({-0.4444*\dx},{0.5509*\dy})
	-- ({-0.4343*\dx},{0.5485*\dy})
	-- ({-0.4242*\dx},{0.5462*\dy})
	-- ({-0.4141*\dx},{0.5440*\dy})
	-- ({-0.4040*\dx},{0.5418*\dy})
	-- ({-0.3939*\dx},{0.5397*\dy})
	-- ({-0.3838*\dx},{0.5377*\dy})
	-- ({-0.3737*\dx},{0.5357*\dy})
	-- ({-0.3636*\dx},{0.5337*\dy})
	-- ({-0.3535*\dx},{0.5318*\dy})
	-- ({-0.3434*\dx},{0.5300*\dy})
	-- ({-0.3333*\dx},{0.5282*\dy})
	-- ({-0.3232*\dx},{0.5265*\dy})
	-- ({-0.3131*\dx},{0.5249*\dy})
	-- ({-0.3030*\dx},{0.5233*\dy})
	-- ({-0.2929*\dx},{0.5217*\dy})
	-- ({-0.2828*\dx},{0.5202*\dy})
	-- ({-0.2727*\dx},{0.5188*\dy})
	-- ({-0.2626*\dx},{0.5174*\dy})
	-- ({-0.2525*\dx},{0.5161*\dy})
	-- ({-0.2424*\dx},{0.5148*\dy})
	-- ({-0.2323*\dx},{0.5136*\dy})
	-- ({-0.2222*\dx},{0.5124*\dy})
	-- ({-0.2121*\dx},{0.5113*\dy})
	-- ({-0.2020*\dx},{0.5103*\dy})
	-- ({-0.1919*\dx},{0.5093*\dy})
	-- ({-0.1818*\dx},{0.5083*\dy})
	-- ({-0.1717*\dx},{0.5074*\dy})
	-- ({-0.1616*\dx},{0.5066*\dy})
	-- ({-0.1515*\dx},{0.5058*\dy})
	-- ({-0.1414*\dx},{0.5050*\dy})
	-- ({-0.1313*\dx},{0.5043*\dy})
	-- ({-0.1212*\dx},{0.5037*\dy})
	-- ({-0.1111*\dx},{0.5031*\dy})
	-- ({-0.1010*\dx},{0.5026*\dy})
	-- ({-0.0909*\dx},{0.5021*\dy})
	-- ({-0.0808*\dx},{0.5016*\dy})
	-- ({-0.0707*\dx},{0.5013*\dy})
	-- ({-0.0606*\dx},{0.5009*\dy})
	-- ({-0.0505*\dx},{0.5006*\dy})
	-- ({-0.0404*\dx},{0.5004*\dy})
	-- ({-0.0303*\dx},{0.5002*\dy})
	-- ({-0.0202*\dx},{0.5001*\dy})
	-- ({-0.0101*\dx},{0.5000*\dy})
	-- ({-0.0000*\dx},{0.5000*\dy})
	-- ({0.0101*\dx},{0.5000*\dy}) % 0.0051
	-- ({0.0202*\dx},{0.5001*\dy}) % 0.0101
	-- ({0.0303*\dx},{0.5002*\dy}) % 0.0152
	-- ({0.0404*\dx},{0.5004*\dy}) % 0.0202
	-- ({0.0505*\dx},{0.5006*\dy}) % 0.0253
	-- ({0.0606*\dx},{0.5009*\dy}) % 0.0303
	-- ({0.0707*\dx},{0.5013*\dy}) % 0.0354
	-- ({0.0808*\dx},{0.5016*\dy}) % 0.0405
	-- ({0.0909*\dx},{0.5021*\dy}) % 0.0456
	-- ({0.1010*\dx},{0.5026*\dy}) % 0.0507
	-- ({0.1111*\dx},{0.5031*\dy}) % 0.0558
	-- ({0.1212*\dx},{0.5037*\dy}) % 0.0609
	-- ({0.1313*\dx},{0.5043*\dy}) % 0.0660
	-- ({0.1414*\dx},{0.5050*\dy}) % 0.0711
	-- ({0.1515*\dx},{0.5058*\dy}) % 0.0763
	-- ({0.1616*\dx},{0.5066*\dy}) % 0.0814
	-- ({0.1717*\dx},{0.5074*\dy}) % 0.0866
	-- ({0.1818*\dx},{0.5083*\dy}) % 0.0918
	-- ({0.1919*\dx},{0.5093*\dy}) % 0.0970
	-- ({0.2020*\dx},{0.5103*\dy}) % 0.1022
	-- ({0.2121*\dx},{0.5113*\dy}) % 0.1075
	-- ({0.2222*\dx},{0.5124*\dy}) % 0.1127
	-- ({0.2323*\dx},{0.5136*\dy}) % 0.1180
	-- ({0.2424*\dx},{0.5148*\dy}) % 0.1233
	-- ({0.2525*\dx},{0.5161*\dy}) % 0.1287
	-- ({0.2626*\dx},{0.5174*\dy}) % 0.1340
	-- ({0.2727*\dx},{0.5188*\dy}) % 0.1394
	-- ({0.2828*\dx},{0.5202*\dy}) % 0.1448
	-- ({0.2929*\dx},{0.5217*\dy}) % 0.1502
	-- ({0.3030*\dx},{0.5233*\dy}) % 0.1557
	-- ({0.3131*\dx},{0.5249*\dy}) % 0.1612
	-- ({0.3232*\dx},{0.5265*\dy}) % 0.1667
	-- ({0.3333*\dx},{0.5282*\dy}) % 0.1723
	-- ({0.3434*\dx},{0.5300*\dy}) % 0.1779
	-- ({0.3535*\dx},{0.5318*\dy}) % 0.1835
	-- ({0.3636*\dx},{0.5337*\dy}) % 0.1892
	-- ({0.3737*\dx},{0.5357*\dy}) % 0.1949
	-- ({0.3838*\dx},{0.5377*\dy}) % 0.2006
	-- ({0.3939*\dx},{0.5397*\dy}) % 0.2064
	-- ({0.4040*\dx},{0.5418*\dy}) % 0.2122
	-- ({0.4141*\dx},{0.5440*\dy}) % 0.2181
	-- ({0.4242*\dx},{0.5462*\dy}) % 0.2240
	-- ({0.4343*\dx},{0.5485*\dy}) % 0.2300
	-- ({0.4444*\dx},{0.5509*\dy}) % 0.2360
	-- ({0.4545*\dx},{0.5533*\dy}) % 0.2420
	-- ({0.4646*\dx},{0.5558*\dy}) % 0.2482
	-- ({0.4747*\dx},{0.5583*\dy}) % 0.2543
	-- ({0.4848*\dx},{0.5609*\dy}) % 0.2605
	-- ({0.4949*\dx},{0.5636*\dy}) % 0.2668
	-- ({0.5051*\dx},{0.5663*\dy}) % 0.2732
	-- ({0.5152*\dx},{0.5691*\dy}) % 0.2796
	-- ({0.5253*\dx},{0.5719*\dy}) % 0.2860
	-- ({0.5354*\dx},{0.5749*\dy}) % 0.2926
	-- ({0.5455*\dx},{0.5779*\dy}) % 0.2992
	-- ({0.5556*\dx},{0.5809*\dy}) % 0.3058
	-- ({0.5657*\dx},{0.5840*\dy}) % 0.3126
	-- ({0.5758*\dx},{0.5872*\dy}) % 0.3194
	-- ({0.5859*\dx},{0.5905*\dy}) % 0.3263
	-- ({0.5960*\dx},{0.5938*\dy}) % 0.3333
	-- ({0.6061*\dx},{0.5972*\dy}) % 0.3404
	-- ({0.6162*\dx},{0.6007*\dy}) % 0.3475
	-- ({0.6263*\dx},{0.6042*\dy}) % 0.3548
	-- ({0.6364*\dx},{0.6079*\dy}) % 0.3621
	-- ({0.6465*\dx},{0.6116*\dy}) % 0.3695
	-- ({0.6566*\dx},{0.6153*\dy}) % 0.3771
	-- ({0.6667*\dx},{0.6192*\dy}) % 0.3847
	-- ({0.6768*\dx},{0.6231*\dy}) % 0.3925
	-- ({0.6869*\dx},{0.6271*\dy}) % 0.4003
	-- ({0.6970*\dx},{0.6312*\dy}) % 0.4083
	-- ({0.7071*\dx},{0.6354*\dy}) % 0.4164
	-- ({0.7172*\dx},{0.6396*\dy}) % 0.4246
	-- ({0.7273*\dx},{0.6439*\dy}) % 0.4330
	-- ({0.7374*\dx},{0.6483*\dy}) % 0.4415
	-- ({0.7475*\dx},{0.6528*\dy}) % 0.4501
	-- ({0.7576*\dx},{0.6574*\dy}) % 0.4588
	-- ({0.7677*\dx},{0.6621*\dy}) % 0.4678
	-- ({0.7778*\dx},{0.6669*\dy}) % 0.4768
	-- ({0.7879*\dx},{0.6718*\dy}) % 0.4861
	-- ({0.7980*\dx},{0.6767*\dy}) % 0.4955
	-- ({0.8081*\dx},{0.6818*\dy}) % 0.5051
	-- ({0.8182*\dx},{0.6869*\dy}) % 0.5149
	-- ({0.8283*\dx},{0.6922*\dy}) % 0.5248
	-- ({0.8384*\dx},{0.6975*\dy}) % 0.5350
	-- ({0.8485*\dx},{0.7030*\dy}) % 0.5454
	-- ({0.8586*\dx},{0.7085*\dy}) % 0.5560
	-- ({0.8687*\dx},{0.7142*\dy}) % 0.5668
	-- ({0.8788*\dx},{0.7200*\dy}) % 0.5779
	-- ({0.8889*\dx},{0.7259*\dy}) % 0.5893
	-- ({0.8990*\dx},{0.7319*\dy}) % 0.6009
	-- ({0.9091*\dx},{0.7380*\dy}) % 0.6127
	-- ({0.9192*\dx},{0.7443*\dy}) % 0.6249
	-- ({0.9293*\dx},{0.7506*\dy}) % 0.6374
	-- ({0.9394*\dx},{0.7572*\dy}) % 0.6502
	-- ({0.9495*\dx},{0.7638*\dy}) % 0.6634
	-- ({0.9596*\dx},{0.7706*\dy}) % 0.6769
	-- ({0.9697*\dx},{0.7775*\dy}) % 0.6908
	-- ({0.9798*\dx},{0.7845*\dy}) % 0.7051
	-- ({0.9899*\dx},{0.7917*\dy}) % 0.7198
	-- ({1.0000*\dx},{0.7991*\dy}) % 0.7350
}
\def\kurvesix{
	({-1.0000*\dx},{0.9794*\dy})
	-- ({-0.9899*\dx},{0.9692*\dy})
	-- ({-0.9798*\dx},{0.9594*\dy})
	-- ({-0.9697*\dx},{0.9498*\dy})
	-- ({-0.9596*\dx},{0.9404*\dy})
	-- ({-0.9495*\dx},{0.9313*\dy})
	-- ({-0.9394*\dx},{0.9224*\dy})
	-- ({-0.9293*\dx},{0.9137*\dy})
	-- ({-0.9192*\dx},{0.9052*\dy})
	-- ({-0.9091*\dx},{0.8970*\dy})
	-- ({-0.8990*\dx},{0.8889*\dy})
	-- ({-0.8889*\dx},{0.8810*\dy})
	-- ({-0.8788*\dx},{0.8733*\dy})
	-- ({-0.8687*\dx},{0.8658*\dy})
	-- ({-0.8586*\dx},{0.8584*\dy})
	-- ({-0.8485*\dx},{0.8513*\dy})
	-- ({-0.8384*\dx},{0.8442*\dy})
	-- ({-0.8283*\dx},{0.8373*\dy})
	-- ({-0.8182*\dx},{0.8306*\dy})
	-- ({-0.8081*\dx},{0.8240*\dy})
	-- ({-0.7980*\dx},{0.8176*\dy})
	-- ({-0.7879*\dx},{0.8113*\dy})
	-- ({-0.7778*\dx},{0.8051*\dy})
	-- ({-0.7677*\dx},{0.7991*\dy})
	-- ({-0.7576*\dx},{0.7932*\dy})
	-- ({-0.7475*\dx},{0.7874*\dy})
	-- ({-0.7374*\dx},{0.7817*\dy})
	-- ({-0.7273*\dx},{0.7762*\dy})
	-- ({-0.7172*\dx},{0.7707*\dy})
	-- ({-0.7071*\dx},{0.7654*\dy})
	-- ({-0.6970*\dx},{0.7602*\dy})
	-- ({-0.6869*\dx},{0.7551*\dy})
	-- ({-0.6768*\dx},{0.7502*\dy})
	-- ({-0.6667*\dx},{0.7453*\dy})
	-- ({-0.6566*\dx},{0.7405*\dy})
	-- ({-0.6465*\dx},{0.7358*\dy})
	-- ({-0.6364*\dx},{0.7313*\dy})
	-- ({-0.6263*\dx},{0.7268*\dy})
	-- ({-0.6162*\dx},{0.7224*\dy})
	-- ({-0.6061*\dx},{0.7181*\dy})
	-- ({-0.5960*\dx},{0.7139*\dy})
	-- ({-0.5859*\dx},{0.7098*\dy})
	-- ({-0.5758*\dx},{0.7058*\dy})
	-- ({-0.5657*\dx},{0.7019*\dy})
	-- ({-0.5556*\dx},{0.6981*\dy})
	-- ({-0.5455*\dx},{0.6943*\dy})
	-- ({-0.5354*\dx},{0.6907*\dy})
	-- ({-0.5253*\dx},{0.6871*\dy})
	-- ({-0.5152*\dx},{0.6836*\dy})
	-- ({-0.5051*\dx},{0.6802*\dy})
	-- ({-0.4949*\dx},{0.6769*\dy})
	-- ({-0.4848*\dx},{0.6736*\dy})
	-- ({-0.4747*\dx},{0.6705*\dy})
	-- ({-0.4646*\dx},{0.6674*\dy})
	-- ({-0.4545*\dx},{0.6644*\dy})
	-- ({-0.4444*\dx},{0.6614*\dy})
	-- ({-0.4343*\dx},{0.6586*\dy})
	-- ({-0.4242*\dx},{0.6558*\dy})
	-- ({-0.4141*\dx},{0.6531*\dy})
	-- ({-0.4040*\dx},{0.6504*\dy})
	-- ({-0.3939*\dx},{0.6479*\dy})
	-- ({-0.3838*\dx},{0.6454*\dy})
	-- ({-0.3737*\dx},{0.6430*\dy})
	-- ({-0.3636*\dx},{0.6406*\dy})
	-- ({-0.3535*\dx},{0.6383*\dy})
	-- ({-0.3434*\dx},{0.6361*\dy})
	-- ({-0.3333*\dx},{0.6340*\dy})
	-- ({-0.3232*\dx},{0.6319*\dy})
	-- ({-0.3131*\dx},{0.6299*\dy})
	-- ({-0.3030*\dx},{0.6280*\dy})
	-- ({-0.2929*\dx},{0.6261*\dy})
	-- ({-0.2828*\dx},{0.6243*\dy})
	-- ({-0.2727*\dx},{0.6226*\dy})
	-- ({-0.2626*\dx},{0.6209*\dy})
	-- ({-0.2525*\dx},{0.6193*\dy})
	-- ({-0.2424*\dx},{0.6178*\dy})
	-- ({-0.2323*\dx},{0.6163*\dy})
	-- ({-0.2222*\dx},{0.6149*\dy})
	-- ({-0.2121*\dx},{0.6136*\dy})
	-- ({-0.2020*\dx},{0.6123*\dy})
	-- ({-0.1919*\dx},{0.6111*\dy})
	-- ({-0.1818*\dx},{0.6100*\dy})
	-- ({-0.1717*\dx},{0.6089*\dy})
	-- ({-0.1616*\dx},{0.6079*\dy})
	-- ({-0.1515*\dx},{0.6069*\dy})
	-- ({-0.1414*\dx},{0.6060*\dy})
	-- ({-0.1313*\dx},{0.6052*\dy})
	-- ({-0.1212*\dx},{0.6044*\dy})
	-- ({-0.1111*\dx},{0.6037*\dy})
	-- ({-0.1010*\dx},{0.6031*\dy})
	-- ({-0.0909*\dx},{0.6025*\dy})
	-- ({-0.0808*\dx},{0.6020*\dy})
	-- ({-0.0707*\dx},{0.6015*\dy})
	-- ({-0.0606*\dx},{0.6011*\dy})
	-- ({-0.0505*\dx},{0.6008*\dy})
	-- ({-0.0404*\dx},{0.6005*\dy})
	-- ({-0.0303*\dx},{0.6003*\dy})
	-- ({-0.0202*\dx},{0.6001*\dy})
	-- ({-0.0101*\dx},{0.6000*\dy})
	-- ({-0.0000*\dx},{0.6000*\dy})
	-- ({0.0101*\dx},{0.6000*\dy}) % 0.0061
	-- ({0.0202*\dx},{0.6001*\dy}) % 0.0121
	-- ({0.0303*\dx},{0.6003*\dy}) % 0.0182
	-- ({0.0404*\dx},{0.6005*\dy}) % 0.0243
	-- ({0.0505*\dx},{0.6008*\dy}) % 0.0303
	-- ({0.0606*\dx},{0.6011*\dy}) % 0.0364
	-- ({0.0707*\dx},{0.6015*\dy}) % 0.0425
	-- ({0.0808*\dx},{0.6020*\dy}) % 0.0486
	-- ({0.0909*\dx},{0.6025*\dy}) % 0.0547
	-- ({0.1010*\dx},{0.6031*\dy}) % 0.0608
	-- ({0.1111*\dx},{0.6037*\dy}) % 0.0670
	-- ({0.1212*\dx},{0.6044*\dy}) % 0.0731
	-- ({0.1313*\dx},{0.6052*\dy}) % 0.0793
	-- ({0.1414*\dx},{0.6060*\dy}) % 0.0854
	-- ({0.1515*\dx},{0.6069*\dy}) % 0.0916
	-- ({0.1616*\dx},{0.6079*\dy}) % 0.0979
	-- ({0.1717*\dx},{0.6089*\dy}) % 0.1041
	-- ({0.1818*\dx},{0.6100*\dy}) % 0.1104
	-- ({0.1919*\dx},{0.6111*\dy}) % 0.1166
	-- ({0.2020*\dx},{0.6123*\dy}) % 0.1230
	-- ({0.2121*\dx},{0.6136*\dy}) % 0.1293
	-- ({0.2222*\dx},{0.6149*\dy}) % 0.1357
	-- ({0.2323*\dx},{0.6163*\dy}) % 0.1421
	-- ({0.2424*\dx},{0.6178*\dy}) % 0.1485
	-- ({0.2525*\dx},{0.6193*\dy}) % 0.1550
	-- ({0.2626*\dx},{0.6209*\dy}) % 0.1615
	-- ({0.2727*\dx},{0.6226*\dy}) % 0.1680
	-- ({0.2828*\dx},{0.6243*\dy}) % 0.1746
	-- ({0.2929*\dx},{0.6261*\dy}) % 0.1812
	-- ({0.3030*\dx},{0.6280*\dy}) % 0.1879
	-- ({0.3131*\dx},{0.6299*\dy}) % 0.1946
	-- ({0.3232*\dx},{0.6319*\dy}) % 0.2013
	-- ({0.3333*\dx},{0.6340*\dy}) % 0.2081
	-- ({0.3434*\dx},{0.6361*\dy}) % 0.2150
	-- ({0.3535*\dx},{0.6383*\dy}) % 0.2219
	-- ({0.3636*\dx},{0.6406*\dy}) % 0.2288
	-- ({0.3737*\dx},{0.6430*\dy}) % 0.2358
	-- ({0.3838*\dx},{0.6454*\dy}) % 0.2429
	-- ({0.3939*\dx},{0.6479*\dy}) % 0.2500
	-- ({0.4040*\dx},{0.6504*\dy}) % 0.2572
	-- ({0.4141*\dx},{0.6531*\dy}) % 0.2645
	-- ({0.4242*\dx},{0.6558*\dy}) % 0.2719
	-- ({0.4343*\dx},{0.6586*\dy}) % 0.2793
	-- ({0.4444*\dx},{0.6614*\dy}) % 0.2867
	-- ({0.4545*\dx},{0.6644*\dy}) % 0.2943
	-- ({0.4646*\dx},{0.6674*\dy}) % 0.3019
	-- ({0.4747*\dx},{0.6705*\dy}) % 0.3097
	-- ({0.4848*\dx},{0.6736*\dy}) % 0.3175
	-- ({0.4949*\dx},{0.6769*\dy}) % 0.3254
	-- ({0.5051*\dx},{0.6802*\dy}) % 0.3334
	-- ({0.5152*\dx},{0.6836*\dy}) % 0.3415
	-- ({0.5253*\dx},{0.6871*\dy}) % 0.3497
	-- ({0.5354*\dx},{0.6907*\dy}) % 0.3580
	-- ({0.5455*\dx},{0.6943*\dy}) % 0.3664
	-- ({0.5556*\dx},{0.6981*\dy}) % 0.3749
	-- ({0.5657*\dx},{0.7019*\dy}) % 0.3836
	-- ({0.5758*\dx},{0.7058*\dy}) % 0.3924
	-- ({0.5859*\dx},{0.7098*\dy}) % 0.4013
	-- ({0.5960*\dx},{0.7139*\dy}) % 0.4103
	-- ({0.6061*\dx},{0.7181*\dy}) % 0.4195
	-- ({0.6162*\dx},{0.7224*\dy}) % 0.4288
	-- ({0.6263*\dx},{0.7268*\dy}) % 0.4383
	-- ({0.6364*\dx},{0.7313*\dy}) % 0.4479
	-- ({0.6465*\dx},{0.7358*\dy}) % 0.4577
	-- ({0.6566*\dx},{0.7405*\dy}) % 0.4677
	-- ({0.6667*\dx},{0.7453*\dy}) % 0.4778
	-- ({0.6768*\dx},{0.7502*\dy}) % 0.4882
	-- ({0.6869*\dx},{0.7551*\dy}) % 0.4987
	-- ({0.6970*\dx},{0.7602*\dy}) % 0.5095
	-- ({0.7071*\dx},{0.7654*\dy}) % 0.5204
	-- ({0.7172*\dx},{0.7707*\dy}) % 0.5316
	-- ({0.7273*\dx},{0.7762*\dy}) % 0.5431
	-- ({0.7374*\dx},{0.7817*\dy}) % 0.5547
	-- ({0.7475*\dx},{0.7874*\dy}) % 0.5667
	-- ({0.7576*\dx},{0.7932*\dy}) % 0.5789
	-- ({0.7677*\dx},{0.7991*\dy}) % 0.5914
	-- ({0.7778*\dx},{0.8051*\dy}) % 0.6042
	-- ({0.7879*\dx},{0.8113*\dy}) % 0.6174
	-- ({0.7980*\dx},{0.8176*\dy}) % 0.6308
	-- ({0.8081*\dx},{0.8240*\dy}) % 0.6447
	-- ({0.8182*\dx},{0.8306*\dy}) % 0.6589
	-- ({0.8283*\dx},{0.8373*\dy}) % 0.6735
	-- ({0.8384*\dx},{0.8442*\dy}) % 0.6885
	-- ({0.8485*\dx},{0.8513*\dy}) % 0.7040
	-- ({0.8586*\dx},{0.8584*\dy}) % 0.7200
	-- ({0.8687*\dx},{0.8658*\dy}) % 0.7365
	-- ({0.8788*\dx},{0.8733*\dy}) % 0.7535
	-- ({0.8889*\dx},{0.8810*\dy}) % 0.7711
	-- ({0.8990*\dx},{0.8889*\dy}) % 0.7894
	-- ({0.9091*\dx},{0.8970*\dy}) % 0.8083
	-- ({0.9192*\dx},{0.9052*\dy}) % 0.8279
	-- ({0.9293*\dx},{0.9137*\dy}) % 0.8483
	-- ({0.9394*\dx},{0.9224*\dy}) % 0.8696
	-- ({0.9495*\dx},{0.9313*\dy}) % 0.8917
	-- ({0.9596*\dx},{0.9404*\dy}) % 0.9148
	-- ({0.9697*\dx},{0.9498*\dy}) % 0.9390
	-- ({0.9798*\dx},{0.9594*\dy}) % 0.9644
	-- ({0.9899*\dx},{0.9692*\dy}) % 0.9911
	-- ({1.0000*\dx},{0.9794*\dy}) % 1.0191
}
\def\kurveseven{
	({-1.0000*\dx},{1.1825*\dy})
	-- ({-0.9899*\dx},{1.1674*\dy})
	-- ({-0.9798*\dx},{1.1529*\dy})
	-- ({-0.9697*\dx},{1.1390*\dy})
	-- ({-0.9596*\dx},{1.1257*\dy})
	-- ({-0.9495*\dx},{1.1128*\dy})
	-- ({-0.9394*\dx},{1.1005*\dy})
	-- ({-0.9293*\dx},{1.0885*\dy})
	-- ({-0.9192*\dx},{1.0770*\dy})
	-- ({-0.9091*\dx},{1.0659*\dy})
	-- ({-0.8990*\dx},{1.0550*\dy})
	-- ({-0.8889*\dx},{1.0446*\dy})
	-- ({-0.8788*\dx},{1.0344*\dy})
	-- ({-0.8687*\dx},{1.0245*\dy})
	-- ({-0.8586*\dx},{1.0149*\dy})
	-- ({-0.8485*\dx},{1.0056*\dy})
	-- ({-0.8384*\dx},{0.9965*\dy})
	-- ({-0.8283*\dx},{0.9877*\dy})
	-- ({-0.8182*\dx},{0.9791*\dy})
	-- ({-0.8081*\dx},{0.9707*\dy})
	-- ({-0.7980*\dx},{0.9626*\dy})
	-- ({-0.7879*\dx},{0.9546*\dy})
	-- ({-0.7778*\dx},{0.9468*\dy})
	-- ({-0.7677*\dx},{0.9393*\dy})
	-- ({-0.7576*\dx},{0.9319*\dy})
	-- ({-0.7475*\dx},{0.9247*\dy})
	-- ({-0.7374*\dx},{0.9177*\dy})
	-- ({-0.7273*\dx},{0.9108*\dy})
	-- ({-0.7172*\dx},{0.9041*\dy})
	-- ({-0.7071*\dx},{0.8976*\dy})
	-- ({-0.6970*\dx},{0.8912*\dy})
	-- ({-0.6869*\dx},{0.8849*\dy})
	-- ({-0.6768*\dx},{0.8788*\dy})
	-- ({-0.6667*\dx},{0.8729*\dy})
	-- ({-0.6566*\dx},{0.8671*\dy})
	-- ({-0.6465*\dx},{0.8614*\dy})
	-- ({-0.6364*\dx},{0.8558*\dy})
	-- ({-0.6263*\dx},{0.8504*\dy})
	-- ({-0.6162*\dx},{0.8451*\dy})
	-- ({-0.6061*\dx},{0.8399*\dy})
	-- ({-0.5960*\dx},{0.8349*\dy})
	-- ({-0.5859*\dx},{0.8299*\dy})
	-- ({-0.5758*\dx},{0.8251*\dy})
	-- ({-0.5657*\dx},{0.8204*\dy})
	-- ({-0.5556*\dx},{0.8158*\dy})
	-- ({-0.5455*\dx},{0.8114*\dy})
	-- ({-0.5354*\dx},{0.8070*\dy})
	-- ({-0.5253*\dx},{0.8027*\dy})
	-- ({-0.5152*\dx},{0.7985*\dy})
	-- ({-0.5051*\dx},{0.7945*\dy})
	-- ({-0.4949*\dx},{0.7905*\dy})
	-- ({-0.4848*\dx},{0.7867*\dy})
	-- ({-0.4747*\dx},{0.7829*\dy})
	-- ({-0.4646*\dx},{0.7792*\dy})
	-- ({-0.4545*\dx},{0.7757*\dy})
	-- ({-0.4444*\dx},{0.7722*\dy})
	-- ({-0.4343*\dx},{0.7688*\dy})
	-- ({-0.4242*\dx},{0.7655*\dy})
	-- ({-0.4141*\dx},{0.7623*\dy})
	-- ({-0.4040*\dx},{0.7592*\dy})
	-- ({-0.3939*\dx},{0.7562*\dy})
	-- ({-0.3838*\dx},{0.7532*\dy})
	-- ({-0.3737*\dx},{0.7504*\dy})
	-- ({-0.3636*\dx},{0.7476*\dy})
	-- ({-0.3535*\dx},{0.7449*\dy})
	-- ({-0.3434*\dx},{0.7423*\dy})
	-- ({-0.3333*\dx},{0.7398*\dy})
	-- ({-0.3232*\dx},{0.7374*\dy})
	-- ({-0.3131*\dx},{0.7350*\dy})
	-- ({-0.3030*\dx},{0.7328*\dy})
	-- ({-0.2929*\dx},{0.7306*\dy})
	-- ({-0.2828*\dx},{0.7285*\dy})
	-- ({-0.2727*\dx},{0.7264*\dy})
	-- ({-0.2626*\dx},{0.7245*\dy})
	-- ({-0.2525*\dx},{0.7226*\dy})
	-- ({-0.2424*\dx},{0.7208*\dy})
	-- ({-0.2323*\dx},{0.7191*\dy})
	-- ({-0.2222*\dx},{0.7175*\dy})
	-- ({-0.2121*\dx},{0.7159*\dy})
	-- ({-0.2020*\dx},{0.7144*\dy})
	-- ({-0.1919*\dx},{0.7130*\dy})
	-- ({-0.1818*\dx},{0.7116*\dy})
	-- ({-0.1717*\dx},{0.7104*\dy})
	-- ({-0.1616*\dx},{0.7092*\dy})
	-- ({-0.1515*\dx},{0.7081*\dy})
	-- ({-0.1414*\dx},{0.7070*\dy})
	-- ({-0.1313*\dx},{0.7061*\dy})
	-- ({-0.1212*\dx},{0.7052*\dy})
	-- ({-0.1111*\dx},{0.7043*\dy})
	-- ({-0.1010*\dx},{0.7036*\dy})
	-- ({-0.0909*\dx},{0.7029*\dy})
	-- ({-0.0808*\dx},{0.7023*\dy})
	-- ({-0.0707*\dx},{0.7018*\dy})
	-- ({-0.0606*\dx},{0.7013*\dy})
	-- ({-0.0505*\dx},{0.7009*\dy})
	-- ({-0.0404*\dx},{0.7006*\dy})
	-- ({-0.0303*\dx},{0.7003*\dy})
	-- ({-0.0202*\dx},{0.7001*\dy})
	-- ({-0.0101*\dx},{0.7000*\dy})
	-- ({-0.0000*\dx},{0.7000*\dy})
	-- ({0.0101*\dx},{0.7000*\dy}) % 0.0071
	-- ({0.0202*\dx},{0.7001*\dy}) % 0.0141
	-- ({0.0303*\dx},{0.7003*\dy}) % 0.0212
	-- ({0.0404*\dx},{0.7006*\dy}) % 0.0283
	-- ({0.0505*\dx},{0.7009*\dy}) % 0.0354
	-- ({0.0606*\dx},{0.7013*\dy}) % 0.0425
	-- ({0.0707*\dx},{0.7018*\dy}) % 0.0496
	-- ({0.0808*\dx},{0.7023*\dy}) % 0.0567
	-- ({0.0909*\dx},{0.7029*\dy}) % 0.0639
	-- ({0.1010*\dx},{0.7036*\dy}) % 0.0710
	-- ({0.1111*\dx},{0.7043*\dy}) % 0.0782
	-- ({0.1212*\dx},{0.7052*\dy}) % 0.0854
	-- ({0.1313*\dx},{0.7061*\dy}) % 0.0926
	-- ({0.1414*\dx},{0.7070*\dy}) % 0.0998
	-- ({0.1515*\dx},{0.7081*\dy}) % 0.1071
	-- ({0.1616*\dx},{0.7092*\dy}) % 0.1144
	-- ({0.1717*\dx},{0.7104*\dy}) % 0.1217
	-- ({0.1818*\dx},{0.7116*\dy}) % 0.1290
	-- ({0.1919*\dx},{0.7130*\dy}) % 0.1364
	-- ({0.2020*\dx},{0.7144*\dy}) % 0.1438
	-- ({0.2121*\dx},{0.7159*\dy}) % 0.1513
	-- ({0.2222*\dx},{0.7175*\dy}) % 0.1588
	-- ({0.2323*\dx},{0.7191*\dy}) % 0.1664
	-- ({0.2424*\dx},{0.7208*\dy}) % 0.1739
	-- ({0.2525*\dx},{0.7226*\dy}) % 0.1816
	-- ({0.2626*\dx},{0.7245*\dy}) % 0.1893
	-- ({0.2727*\dx},{0.7264*\dy}) % 0.1970
	-- ({0.2828*\dx},{0.7285*\dy}) % 0.2048
	-- ({0.2929*\dx},{0.7306*\dy}) % 0.2127
	-- ({0.3030*\dx},{0.7328*\dy}) % 0.2206
	-- ({0.3131*\dx},{0.7350*\dy}) % 0.2286
	-- ({0.3232*\dx},{0.7374*\dy}) % 0.2366
	-- ({0.3333*\dx},{0.7398*\dy}) % 0.2447
	-- ({0.3434*\dx},{0.7423*\dy}) % 0.2529
	-- ({0.3535*\dx},{0.7449*\dy}) % 0.2612
	-- ({0.3636*\dx},{0.7476*\dy}) % 0.2695
	-- ({0.3737*\dx},{0.7504*\dy}) % 0.2780
	-- ({0.3838*\dx},{0.7532*\dy}) % 0.2865
	-- ({0.3939*\dx},{0.7562*\dy}) % 0.2951
	-- ({0.4040*\dx},{0.7592*\dy}) % 0.3038
	-- ({0.4141*\dx},{0.7623*\dy}) % 0.3126
	-- ({0.4242*\dx},{0.7655*\dy}) % 0.3215
	-- ({0.4343*\dx},{0.7688*\dy}) % 0.3305
	-- ({0.4444*\dx},{0.7722*\dy}) % 0.3397
	-- ({0.4545*\dx},{0.7757*\dy}) % 0.3489
	-- ({0.4646*\dx},{0.7792*\dy}) % 0.3583
	-- ({0.4747*\dx},{0.7829*\dy}) % 0.3678
	-- ({0.4848*\dx},{0.7867*\dy}) % 0.3774
	-- ({0.4949*\dx},{0.7905*\dy}) % 0.3872
	-- ({0.5051*\dx},{0.7945*\dy}) % 0.3971
	-- ({0.5152*\dx},{0.7985*\dy}) % 0.4072
	-- ({0.5253*\dx},{0.8027*\dy}) % 0.4174
	-- ({0.5354*\dx},{0.8070*\dy}) % 0.4278
	-- ({0.5455*\dx},{0.8114*\dy}) % 0.4384
	-- ({0.5556*\dx},{0.8158*\dy}) % 0.4492
	-- ({0.5657*\dx},{0.8204*\dy}) % 0.4601
	-- ({0.5758*\dx},{0.8251*\dy}) % 0.4713
	-- ({0.5859*\dx},{0.8299*\dy}) % 0.4826
	-- ({0.5960*\dx},{0.8349*\dy}) % 0.4942
	-- ({0.6061*\dx},{0.8399*\dy}) % 0.5061
	-- ({0.6162*\dx},{0.8451*\dy}) % 0.5181
	-- ({0.6263*\dx},{0.8504*\dy}) % 0.5304
	-- ({0.6364*\dx},{0.8558*\dy}) % 0.5430
	-- ({0.6465*\dx},{0.8614*\dy}) % 0.5559
	-- ({0.6566*\dx},{0.8671*\dy}) % 0.5691
	-- ({0.6667*\dx},{0.8729*\dy}) % 0.5826
	-- ({0.6768*\dx},{0.8788*\dy}) % 0.5965
	-- ({0.6869*\dx},{0.8849*\dy}) % 0.6106
	-- ({0.6970*\dx},{0.8912*\dy}) % 0.6252
	-- ({0.7071*\dx},{0.8976*\dy}) % 0.6402
	-- ({0.7172*\dx},{0.9041*\dy}) % 0.6556
	-- ({0.7273*\dx},{0.9108*\dy}) % 0.6714
	-- ({0.7374*\dx},{0.9177*\dy}) % 0.6877
	-- ({0.7475*\dx},{0.9247*\dy}) % 0.7046
	-- ({0.7576*\dx},{0.9319*\dy}) % 0.7220
	-- ({0.7677*\dx},{0.9393*\dy}) % 0.7399
	-- ({0.7778*\dx},{0.9468*\dy}) % 0.7585
	-- ({0.7879*\dx},{0.9546*\dy}) % 0.7777
	-- ({0.7980*\dx},{0.9626*\dy}) % 0.7977
	-- ({0.8081*\dx},{0.9707*\dy}) % 0.8185
	-- ({0.8182*\dx},{0.9791*\dy}) % 0.8401
	-- ({0.8283*\dx},{0.9877*\dy}) % 0.8626
	-- ({0.8384*\dx},{0.9965*\dy}) % 0.8860
	-- ({0.8485*\dx},{1.0056*\dy}) % 0.9106
	-- ({0.8586*\dx},{1.0149*\dy}) % 0.9363
	-- ({0.8687*\dx},{1.0245*\dy}) % 0.9634
	-- ({0.8788*\dx},{1.0344*\dy}) % 0.9918
	-- ({0.8889*\dx},{1.0446*\dy}) % 1.0218
	-- ({0.8990*\dx},{1.0550*\dy}) % 1.0535
	-- ({0.9091*\dx},{1.0659*\dy}) % 1.0872
	-- ({0.9192*\dx},{1.0770*\dy}) % 1.1230
	-- ({0.9293*\dx},{1.0885*\dy}) % 1.1613
	-- ({0.9394*\dx},{1.1005*\dy}) % 1.2023
	-- ({0.9495*\dx},{1.1128*\dy}) % 1.2464
	-- ({0.9596*\dx},{1.1257*\dy}) % 1.2942
	-- ({0.9697*\dx},{1.1390*\dy}) % 1.3461
	-- ({0.9798*\dx},{1.1529*\dy}) % 1.4030
	-- ({0.9899*\dx},{1.1674*\dy}) % 1.4656
	-- ({1.0000*\dx},{1.1825*\dy}) % 1.5351
}
\def\kurveeight{
	({-1.0000*\dx},{1.4606*\dy})
	-- ({-0.9899*\dx},{1.4258*\dy})
	-- ({-0.9798*\dx},{1.3964*\dy})
	-- ({-0.9697*\dx},{1.3705*\dy})
	-- ({-0.9596*\dx},{1.3471*\dy})
	-- ({-0.9495*\dx},{1.3257*\dy})
	-- ({-0.9394*\dx},{1.3059*\dy})
	-- ({-0.9293*\dx},{1.2874*\dy})
	-- ({-0.9192*\dx},{1.2700*\dy})
	-- ({-0.9091*\dx},{1.2536*\dy})
	-- ({-0.8990*\dx},{1.2380*\dy})
	-- ({-0.8889*\dx},{1.2232*\dy})
	-- ({-0.8788*\dx},{1.2090*\dy})
	-- ({-0.8687*\dx},{1.1955*\dy})
	-- ({-0.8586*\dx},{1.1825*\dy})
	-- ({-0.8485*\dx},{1.1699*\dy})
	-- ({-0.8384*\dx},{1.1579*\dy})
	-- ({-0.8283*\dx},{1.1463*\dy})
	-- ({-0.8182*\dx},{1.1351*\dy})
	-- ({-0.8081*\dx},{1.1242*\dy})
	-- ({-0.7980*\dx},{1.1138*\dy})
	-- ({-0.7879*\dx},{1.1036*\dy})
	-- ({-0.7778*\dx},{1.0938*\dy})
	-- ({-0.7677*\dx},{1.0842*\dy})
	-- ({-0.7576*\dx},{1.0750*\dy})
	-- ({-0.7475*\dx},{1.0660*\dy})
	-- ({-0.7374*\dx},{1.0573*\dy})
	-- ({-0.7273*\dx},{1.0488*\dy})
	-- ({-0.7172*\dx},{1.0405*\dy})
	-- ({-0.7071*\dx},{1.0325*\dy})
	-- ({-0.6970*\dx},{1.0247*\dy})
	-- ({-0.6869*\dx},{1.0171*\dy})
	-- ({-0.6768*\dx},{1.0097*\dy})
	-- ({-0.6667*\dx},{1.0025*\dy})
	-- ({-0.6566*\dx},{0.9954*\dy})
	-- ({-0.6465*\dx},{0.9886*\dy})
	-- ({-0.6364*\dx},{0.9819*\dy})
	-- ({-0.6263*\dx},{0.9754*\dy})
	-- ({-0.6162*\dx},{0.9691*\dy})
	-- ({-0.6061*\dx},{0.9629*\dy})
	-- ({-0.5960*\dx},{0.9569*\dy})
	-- ({-0.5859*\dx},{0.9511*\dy})
	-- ({-0.5758*\dx},{0.9454*\dy})
	-- ({-0.5657*\dx},{0.9398*\dy})
	-- ({-0.5556*\dx},{0.9344*\dy})
	-- ({-0.5455*\dx},{0.9291*\dy})
	-- ({-0.5354*\dx},{0.9239*\dy})
	-- ({-0.5253*\dx},{0.9189*\dy})
	-- ({-0.5152*\dx},{0.9140*\dy})
	-- ({-0.5051*\dx},{0.9092*\dy})
	-- ({-0.4949*\dx},{0.9046*\dy})
	-- ({-0.4848*\dx},{0.9001*\dy})
	-- ({-0.4747*\dx},{0.8957*\dy})
	-- ({-0.4646*\dx},{0.8914*\dy})
	-- ({-0.4545*\dx},{0.8873*\dy})
	-- ({-0.4444*\dx},{0.8832*\dy})
	-- ({-0.4343*\dx},{0.8793*\dy})
	-- ({-0.4242*\dx},{0.8754*\dy})
	-- ({-0.4141*\dx},{0.8717*\dy})
	-- ({-0.4040*\dx},{0.8681*\dy})
	-- ({-0.3939*\dx},{0.8646*\dy})
	-- ({-0.3838*\dx},{0.8612*\dy})
	-- ({-0.3737*\dx},{0.8579*\dy})
	-- ({-0.3636*\dx},{0.8547*\dy})
	-- ({-0.3535*\dx},{0.8516*\dy})
	-- ({-0.3434*\dx},{0.8486*\dy})
	-- ({-0.3333*\dx},{0.8457*\dy})
	-- ({-0.3232*\dx},{0.8429*\dy})
	-- ({-0.3131*\dx},{0.8402*\dy})
	-- ({-0.3030*\dx},{0.8376*\dy})
	-- ({-0.2929*\dx},{0.8351*\dy})
	-- ({-0.2828*\dx},{0.8326*\dy})
	-- ({-0.2727*\dx},{0.8303*\dy})
	-- ({-0.2626*\dx},{0.8281*\dy})
	-- ({-0.2525*\dx},{0.8259*\dy})
	-- ({-0.2424*\dx},{0.8239*\dy})
	-- ({-0.2323*\dx},{0.8219*\dy})
	-- ({-0.2222*\dx},{0.8200*\dy})
	-- ({-0.2121*\dx},{0.8182*\dy})
	-- ({-0.2020*\dx},{0.8165*\dy})
	-- ({-0.1919*\dx},{0.8149*\dy})
	-- ({-0.1818*\dx},{0.8133*\dy})
	-- ({-0.1717*\dx},{0.8119*\dy})
	-- ({-0.1616*\dx},{0.8105*\dy})
	-- ({-0.1515*\dx},{0.8092*\dy})
	-- ({-0.1414*\dx},{0.8080*\dy})
	-- ({-0.1313*\dx},{0.8069*\dy})
	-- ({-0.1212*\dx},{0.8059*\dy})
	-- ({-0.1111*\dx},{0.8050*\dy})
	-- ({-0.1010*\dx},{0.8041*\dy})
	-- ({-0.0909*\dx},{0.8033*\dy})
	-- ({-0.0808*\dx},{0.8026*\dy})
	-- ({-0.0707*\dx},{0.8020*\dy})
	-- ({-0.0606*\dx},{0.8015*\dy})
	-- ({-0.0505*\dx},{0.8010*\dy})
	-- ({-0.0404*\dx},{0.8007*\dy})
	-- ({-0.0303*\dx},{0.8004*\dy})
	-- ({-0.0202*\dx},{0.8002*\dy})
	-- ({-0.0101*\dx},{0.8000*\dy})
	-- ({-0.0000*\dx},{0.8000*\dy})
	-- ({0.0101*\dx},{0.8000*\dy}) % 0.0081
	-- ({0.0202*\dx},{0.8002*\dy}) % 0.0162
	-- ({0.0303*\dx},{0.8004*\dy}) % 0.0243
	-- ({0.0404*\dx},{0.8007*\dy}) % 0.0323
	-- ({0.0505*\dx},{0.8010*\dy}) % 0.0405
	-- ({0.0606*\dx},{0.8015*\dy}) % 0.0486
	-- ({0.0707*\dx},{0.8020*\dy}) % 0.0567
	-- ({0.0808*\dx},{0.8026*\dy}) % 0.0649
	-- ({0.0909*\dx},{0.8033*\dy}) % 0.0730
	-- ({0.1010*\dx},{0.8041*\dy}) % 0.0812
	-- ({0.1111*\dx},{0.8050*\dy}) % 0.0894
	-- ({0.1212*\dx},{0.8059*\dy}) % 0.0977
	-- ({0.1313*\dx},{0.8069*\dy}) % 0.1059
	-- ({0.1414*\dx},{0.8080*\dy}) % 0.1142
	-- ({0.1515*\dx},{0.8092*\dy}) % 0.1226
	-- ({0.1616*\dx},{0.8105*\dy}) % 0.1310
	-- ({0.1717*\dx},{0.8119*\dy}) % 0.1394
	-- ({0.1818*\dx},{0.8133*\dy}) % 0.1478
	-- ({0.1919*\dx},{0.8149*\dy}) % 0.1564
	-- ({0.2020*\dx},{0.8165*\dy}) % 0.1649
	-- ({0.2121*\dx},{0.8182*\dy}) % 0.1735
	-- ({0.2222*\dx},{0.8200*\dy}) % 0.1822
	-- ({0.2323*\dx},{0.8219*\dy}) % 0.1909
	-- ({0.2424*\dx},{0.8239*\dy}) % 0.1997
	-- ({0.2525*\dx},{0.8259*\dy}) % 0.2086
	-- ({0.2626*\dx},{0.8281*\dy}) % 0.2175
	-- ({0.2727*\dx},{0.8303*\dy}) % 0.2265
	-- ({0.2828*\dx},{0.8326*\dy}) % 0.2356
	-- ({0.2929*\dx},{0.8351*\dy}) % 0.2448
	-- ({0.3030*\dx},{0.8376*\dy}) % 0.2540
	-- ({0.3131*\dx},{0.8402*\dy}) % 0.2633
	-- ({0.3232*\dx},{0.8429*\dy}) % 0.2728
	-- ({0.3333*\dx},{0.8457*\dy}) % 0.2823
	-- ({0.3434*\dx},{0.8486*\dy}) % 0.2919
	-- ({0.3535*\dx},{0.8516*\dy}) % 0.3017
	-- ({0.3636*\dx},{0.8547*\dy}) % 0.3116
	-- ({0.3737*\dx},{0.8579*\dy}) % 0.3215
	-- ({0.3838*\dx},{0.8612*\dy}) % 0.3316
	-- ({0.3939*\dx},{0.8646*\dy}) % 0.3419
	-- ({0.4040*\dx},{0.8681*\dy}) % 0.3523
	-- ({0.4141*\dx},{0.8717*\dy}) % 0.3628
	-- ({0.4242*\dx},{0.8754*\dy}) % 0.3735
	-- ({0.4343*\dx},{0.8793*\dy}) % 0.3843
	-- ({0.4444*\dx},{0.8832*\dy}) % 0.3953
	-- ({0.4545*\dx},{0.8873*\dy}) % 0.4065
	-- ({0.4646*\dx},{0.8914*\dy}) % 0.4179
	-- ({0.4747*\dx},{0.8957*\dy}) % 0.4294
	-- ({0.4848*\dx},{0.9001*\dy}) % 0.4412
	-- ({0.4949*\dx},{0.9046*\dy}) % 0.4532
	-- ({0.5051*\dx},{0.9092*\dy}) % 0.4654
	-- ({0.5152*\dx},{0.9140*\dy}) % 0.4778
	-- ({0.5253*\dx},{0.9189*\dy}) % 0.4905
	-- ({0.5354*\dx},{0.9239*\dy}) % 0.5035
	-- ({0.5455*\dx},{0.9291*\dy}) % 0.5167
	-- ({0.5556*\dx},{0.9344*\dy}) % 0.5302
	-- ({0.5657*\dx},{0.9398*\dy}) % 0.5441
	-- ({0.5758*\dx},{0.9454*\dy}) % 0.5583
	-- ({0.5859*\dx},{0.9511*\dy}) % 0.5728
	-- ({0.5960*\dx},{0.9569*\dy}) % 0.5877
	-- ({0.6061*\dx},{0.9629*\dy}) % 0.6030
	-- ({0.6162*\dx},{0.9691*\dy}) % 0.6187
	-- ({0.6263*\dx},{0.9754*\dy}) % 0.6348
	-- ({0.6364*\dx},{0.9819*\dy}) % 0.6514
	-- ({0.6465*\dx},{0.9886*\dy}) % 0.6685
	-- ({0.6566*\dx},{0.9954*\dy}) % 0.6862
	-- ({0.6667*\dx},{1.0025*\dy}) % 0.7044
	-- ({0.6768*\dx},{1.0097*\dy}) % 0.7233
	-- ({0.6869*\dx},{1.0171*\dy}) % 0.7428
	-- ({0.6970*\dx},{1.0247*\dy}) % 0.7630
	-- ({0.7071*\dx},{1.0325*\dy}) % 0.7840
	-- ({0.7172*\dx},{1.0405*\dy}) % 0.8058
	-- ({0.7273*\dx},{1.0488*\dy}) % 0.8285
	-- ({0.7374*\dx},{1.0573*\dy}) % 0.8522
	-- ({0.7475*\dx},{1.0660*\dy}) % 0.8770
	-- ({0.7576*\dx},{1.0750*\dy}) % 0.9030
	-- ({0.7677*\dx},{1.0842*\dy}) % 0.9302
	-- ({0.7778*\dx},{1.0938*\dy}) % 0.9588
	-- ({0.7879*\dx},{1.1036*\dy}) % 0.9890
	-- ({0.7980*\dx},{1.1138*\dy}) % 1.0209
	-- ({0.8081*\dx},{1.1242*\dy}) % 1.0547
	-- ({0.8182*\dx},{1.1351*\dy}) % 1.0907
	-- ({0.8283*\dx},{1.1463*\dy}) % 1.1291
	-- ({0.8384*\dx},{1.1579*\dy}) % 1.1703
	-- ({0.8485*\dx},{1.1699*\dy}) % 1.2146
	-- ({0.8586*\dx},{1.1825*\dy}) % 1.2625
	-- ({0.8687*\dx},{1.1955*\dy}) % 1.3146
	-- ({0.8788*\dx},{1.2090*\dy}) % 1.3716
	-- ({0.8889*\dx},{1.2232*\dy}) % 1.4344
	-- ({0.8990*\dx},{1.2380*\dy}) % 1.5041
	-- ({0.9091*\dx},{1.2536*\dy}) % 1.5823
	-- ({0.9192*\dx},{1.2700*\dy}) % 1.6710
	-- ({0.9293*\dx},{1.2874*\dy}) % 1.7729
	-- ({0.9394*\dx},{1.3059*\dy}) % 1.8919
	-- ({0.9495*\dx},{1.3257*\dy}) % 2.0337
	-- ({0.9596*\dx},{1.3471*\dy}) % 2.2073
	-- ({0.9697*\dx},{1.3705*\dy}) % 2.4272
	-- ({0.9798*\dx},{1.3964*\dy}) % 2.7199
	-- ({0.9899*\dx},{1.4258*\dy}) % 3.1393
	-- ({1.0000*\dx},{1.4606*\dy}) % 3.8198
}
\def\hzero{0.54773}
\def\kurvefinal{
	({-1.0000*\dx},{0.8833*\dy})
	-- ({-0.9899*\dx},{0.8747*\dy})
	-- ({-0.9798*\dx},{0.8664*\dy})
	-- ({-0.9697*\dx},{0.8582*\dy})
	-- ({-0.9596*\dx},{0.8503*\dy})
	-- ({-0.9495*\dx},{0.8425*\dy})
	-- ({-0.9394*\dx},{0.8348*\dy})
	-- ({-0.9293*\dx},{0.8274*\dy})
	-- ({-0.9192*\dx},{0.8201*\dy})
	-- ({-0.9091*\dx},{0.8130*\dy})
	-- ({-0.8990*\dx},{0.8060*\dy})
	-- ({-0.8889*\dx},{0.7991*\dy})
	-- ({-0.8788*\dx},{0.7924*\dy})
	-- ({-0.8687*\dx},{0.7859*\dy})
	-- ({-0.8586*\dx},{0.7794*\dy})
	-- ({-0.8485*\dx},{0.7732*\dy})
	-- ({-0.8384*\dx},{0.7670*\dy})
	-- ({-0.8283*\dx},{0.7609*\dy})
	-- ({-0.8182*\dx},{0.7550*\dy})
	-- ({-0.8081*\dx},{0.7492*\dy})
	-- ({-0.7980*\dx},{0.7435*\dy})
	-- ({-0.7879*\dx},{0.7380*\dy})
	-- ({-0.7778*\dx},{0.7325*\dy})
	-- ({-0.7677*\dx},{0.7272*\dy})
	-- ({-0.7576*\dx},{0.7219*\dy})
	-- ({-0.7475*\dx},{0.7168*\dy})
	-- ({-0.7374*\dx},{0.7117*\dy})
	-- ({-0.7273*\dx},{0.7068*\dy})
	-- ({-0.7172*\dx},{0.7020*\dy})
	-- ({-0.7071*\dx},{0.6972*\dy})
	-- ({-0.6970*\dx},{0.6926*\dy})
	-- ({-0.6869*\dx},{0.6880*\dy})
	-- ({-0.6768*\dx},{0.6836*\dy})
	-- ({-0.6667*\dx},{0.6792*\dy})
	-- ({-0.6566*\dx},{0.6749*\dy})
	-- ({-0.6465*\dx},{0.6707*\dy})
	-- ({-0.6364*\dx},{0.6666*\dy})
	-- ({-0.6263*\dx},{0.6626*\dy})
	-- ({-0.6162*\dx},{0.6587*\dy})
	-- ({-0.6061*\dx},{0.6548*\dy})
	-- ({-0.5960*\dx},{0.6511*\dy})
	-- ({-0.5859*\dx},{0.6474*\dy})
	-- ({-0.5758*\dx},{0.6438*\dy})
	-- ({-0.5657*\dx},{0.6402*\dy})
	-- ({-0.5556*\dx},{0.6368*\dy})
	-- ({-0.5455*\dx},{0.6334*\dy})
	-- ({-0.5354*\dx},{0.6301*\dy})
	-- ({-0.5253*\dx},{0.6269*\dy})
	-- ({-0.5152*\dx},{0.6237*\dy})
	-- ({-0.5051*\dx},{0.6206*\dy})
	-- ({-0.4949*\dx},{0.6176*\dy})
	-- ({-0.4848*\dx},{0.6147*\dy})
	-- ({-0.4747*\dx},{0.6118*\dy})
	-- ({-0.4646*\dx},{0.6090*\dy})
	-- ({-0.4545*\dx},{0.6063*\dy})
	-- ({-0.4444*\dx},{0.6036*\dy})
	-- ({-0.4343*\dx},{0.6010*\dy})
	-- ({-0.4242*\dx},{0.5985*\dy})
	-- ({-0.4141*\dx},{0.5960*\dy})
	-- ({-0.4040*\dx},{0.5936*\dy})
	-- ({-0.3939*\dx},{0.5913*\dy})
	-- ({-0.3838*\dx},{0.5891*\dy})
	-- ({-0.3737*\dx},{0.5869*\dy})
	-- ({-0.3636*\dx},{0.5847*\dy})
	-- ({-0.3535*\dx},{0.5827*\dy})
	-- ({-0.3434*\dx},{0.5807*\dy})
	-- ({-0.3333*\dx},{0.5787*\dy})
	-- ({-0.3232*\dx},{0.5768*\dy})
	-- ({-0.3131*\dx},{0.5750*\dy})
	-- ({-0.3030*\dx},{0.5733*\dy})
	-- ({-0.2929*\dx},{0.5716*\dy})
	-- ({-0.2828*\dx},{0.5699*\dy})
	-- ({-0.2727*\dx},{0.5683*\dy})
	-- ({-0.2626*\dx},{0.5668*\dy})
	-- ({-0.2525*\dx},{0.5654*\dy})
	-- ({-0.2424*\dx},{0.5640*\dy})
	-- ({-0.2323*\dx},{0.5626*\dy})
	-- ({-0.2222*\dx},{0.5614*\dy})
	-- ({-0.2121*\dx},{0.5601*\dy})
	-- ({-0.2020*\dx},{0.5590*\dy})
	-- ({-0.1919*\dx},{0.5579*\dy})
	-- ({-0.1818*\dx},{0.5568*\dy})
	-- ({-0.1717*\dx},{0.5558*\dy})
	-- ({-0.1616*\dx},{0.5549*\dy})
	-- ({-0.1515*\dx},{0.5540*\dy})
	-- ({-0.1414*\dx},{0.5532*\dy})
	-- ({-0.1313*\dx},{0.5525*\dy})
	-- ({-0.1212*\dx},{0.5518*\dy})
	-- ({-0.1111*\dx},{0.5511*\dy})
	-- ({-0.1010*\dx},{0.5505*\dy})
	-- ({-0.0909*\dx},{0.5500*\dy})
	-- ({-0.0808*\dx},{0.5495*\dy})
	-- ({-0.0707*\dx},{0.5491*\dy})
	-- ({-0.0606*\dx},{0.5487*\dy})
	-- ({-0.0505*\dx},{0.5484*\dy})
	-- ({-0.0404*\dx},{0.5482*\dy})
	-- ({-0.0303*\dx},{0.5480*\dy})
	-- ({-0.0202*\dx},{0.5478*\dy})
	-- ({-0.0101*\dx},{0.5478*\dy})
	-- ({-0.0000*\dx},{0.5477*\dy})
	-- ({0.0101*\dx},{0.5478*\dy}) % 0.0055
	-- ({0.0202*\dx},{0.5478*\dy}) % 0.0111
	-- ({0.0303*\dx},{0.5480*\dy}) % 0.0166
	-- ({0.0404*\dx},{0.5482*\dy}) % 0.0221
	-- ({0.0505*\dx},{0.5484*\dy}) % 0.0277
	-- ({0.0606*\dx},{0.5487*\dy}) % 0.0332
	-- ({0.0707*\dx},{0.5491*\dy}) % 0.0388
	-- ({0.0808*\dx},{0.5495*\dy}) % 0.0444
	-- ({0.0909*\dx},{0.5500*\dy}) % 0.0499
	-- ({0.1010*\dx},{0.5505*\dy}) % 0.0555
	-- ({0.1111*\dx},{0.5511*\dy}) % 0.0611
	-- ({0.1212*\dx},{0.5518*\dy}) % 0.0667
	-- ({0.1313*\dx},{0.5525*\dy}) % 0.0723
	-- ({0.1414*\dx},{0.5532*\dy}) % 0.0780
	-- ({0.1515*\dx},{0.5540*\dy}) % 0.0836
	-- ({0.1616*\dx},{0.5549*\dy}) % 0.0893
	-- ({0.1717*\dx},{0.5558*\dy}) % 0.0949
	-- ({0.1818*\dx},{0.5568*\dy}) % 0.1006
	-- ({0.1919*\dx},{0.5579*\dy}) % 0.1064
	-- ({0.2020*\dx},{0.5590*\dy}) % 0.1121
	-- ({0.2121*\dx},{0.5601*\dy}) % 0.1179
	-- ({0.2222*\dx},{0.5614*\dy}) % 0.1237
	-- ({0.2323*\dx},{0.5626*\dy}) % 0.1295
	-- ({0.2424*\dx},{0.5640*\dy}) % 0.1353
	-- ({0.2525*\dx},{0.5654*\dy}) % 0.1412
	-- ({0.2626*\dx},{0.5668*\dy}) % 0.1471
	-- ({0.2727*\dx},{0.5683*\dy}) % 0.1530
	-- ({0.2828*\dx},{0.5699*\dy}) % 0.1590
	-- ({0.2929*\dx},{0.5716*\dy}) % 0.1650
	-- ({0.3030*\dx},{0.5733*\dy}) % 0.1710
	-- ({0.3131*\dx},{0.5750*\dy}) % 0.1771
	-- ({0.3232*\dx},{0.5768*\dy}) % 0.1832
	-- ({0.3333*\dx},{0.5787*\dy}) % 0.1893
	-- ({0.3434*\dx},{0.5807*\dy}) % 0.1955
	-- ({0.3535*\dx},{0.5827*\dy}) % 0.2017
	-- ({0.3636*\dx},{0.5847*\dy}) % 0.2080
	-- ({0.3737*\dx},{0.5869*\dy}) % 0.2143
	-- ({0.3838*\dx},{0.5891*\dy}) % 0.2207
	-- ({0.3939*\dx},{0.5913*\dy}) % 0.2271
	-- ({0.4040*\dx},{0.5936*\dy}) % 0.2335
	-- ({0.4141*\dx},{0.5960*\dy}) % 0.2401
	-- ({0.4242*\dx},{0.5985*\dy}) % 0.2466
	-- ({0.4343*\dx},{0.6010*\dy}) % 0.2533
	-- ({0.4444*\dx},{0.6036*\dy}) % 0.2600
	-- ({0.4545*\dx},{0.6063*\dy}) % 0.2667
	-- ({0.4646*\dx},{0.6090*\dy}) % 0.2735
	-- ({0.4747*\dx},{0.6118*\dy}) % 0.2804
	-- ({0.4848*\dx},{0.6147*\dy}) % 0.2874
	-- ({0.4949*\dx},{0.6176*\dy}) % 0.2944
	-- ({0.5051*\dx},{0.6206*\dy}) % 0.3015
	-- ({0.5152*\dx},{0.6237*\dy}) % 0.3087
	-- ({0.5253*\dx},{0.6269*\dy}) % 0.3160
	-- ({0.5354*\dx},{0.6301*\dy}) % 0.3233
	-- ({0.5455*\dx},{0.6334*\dy}) % 0.3308
	-- ({0.5556*\dx},{0.6368*\dy}) % 0.3383
	-- ({0.5657*\dx},{0.6402*\dy}) % 0.3459
	-- ({0.5758*\dx},{0.6438*\dy}) % 0.3536
	-- ({0.5859*\dx},{0.6474*\dy}) % 0.3614
	-- ({0.5960*\dx},{0.6511*\dy}) % 0.3693
	-- ({0.6061*\dx},{0.6548*\dy}) % 0.3773
	-- ({0.6162*\dx},{0.6587*\dy}) % 0.3855
	-- ({0.6263*\dx},{0.6626*\dy}) % 0.3937
	-- ({0.6364*\dx},{0.6666*\dy}) % 0.4021
	-- ({0.6465*\dx},{0.6707*\dy}) % 0.4106
	-- ({0.6566*\dx},{0.6749*\dy}) % 0.4192
	-- ({0.6667*\dx},{0.6792*\dy}) % 0.4280
	-- ({0.6768*\dx},{0.6836*\dy}) % 0.4369
	-- ({0.6869*\dx},{0.6880*\dy}) % 0.4459
	-- ({0.6970*\dx},{0.6926*\dy}) % 0.4551
	-- ({0.7071*\dx},{0.6972*\dy}) % 0.4645
	-- ({0.7172*\dx},{0.7020*\dy}) % 0.4740
	-- ({0.7273*\dx},{0.7068*\dy}) % 0.4837
	-- ({0.7374*\dx},{0.7117*\dy}) % 0.4936
	-- ({0.7475*\dx},{0.7168*\dy}) % 0.5036
	-- ({0.7576*\dx},{0.7219*\dy}) % 0.5139
	-- ({0.7677*\dx},{0.7272*\dy}) % 0.5244
	-- ({0.7778*\dx},{0.7325*\dy}) % 0.5351
	-- ({0.7879*\dx},{0.7380*\dy}) % 0.5460
	-- ({0.7980*\dx},{0.7435*\dy}) % 0.5571
	-- ({0.8081*\dx},{0.7492*\dy}) % 0.5685
	-- ({0.8182*\dx},{0.7550*\dy}) % 0.5802
	-- ({0.8283*\dx},{0.7609*\dy}) % 0.5921
	-- ({0.8384*\dx},{0.7670*\dy}) % 0.6043
	-- ({0.8485*\dx},{0.7732*\dy}) % 0.6168
	-- ({0.8586*\dx},{0.7794*\dy}) % 0.6296
	-- ({0.8687*\dx},{0.7859*\dy}) % 0.6428
	-- ({0.8788*\dx},{0.7924*\dy}) % 0.6563
	-- ({0.8889*\dx},{0.7991*\dy}) % 0.6702
	-- ({0.8990*\dx},{0.8060*\dy}) % 0.6845
	-- ({0.9091*\dx},{0.8130*\dy}) % 0.6992
	-- ({0.9192*\dx},{0.8201*\dy}) % 0.7143
	-- ({0.9293*\dx},{0.8274*\dy}) % 0.7299
	-- ({0.9394*\dx},{0.8348*\dy}) % 0.7460
	-- ({0.9495*\dx},{0.8425*\dy}) % 0.7627
	-- ({0.9596*\dx},{0.8503*\dy}) % 0.7799
	-- ({0.9697*\dx},{0.8582*\dy}) % 0.7977
	-- ({0.9798*\dx},{0.8664*\dy}) % 0.8162
	-- ({0.9899*\dx},{0.8747*\dy}) % 0.8354
	-- ({1.0000*\dx},{0.8833*\dy}) % 0.8553
}


\fill[color=blue!20] \kurvefinal -- ({\dx},0) -- ({-\dx},0) -- cycle;

\draw[color=darkred!20] ({-\dx},0) -- ({\dx},0);
\draw[color=darkred!20] \kurveone;
\draw[color=darkred!20] \kurvetwo;
\draw[color=darkred!20] \kurvethree;
\draw[color=darkred!20] \kurvefour;
\draw[color=darkred!20] \kurvefive;
\draw[color=darkred!20] \kurvesix;
\draw[color=darkred!20] \kurveseven;
\draw[color=darkred!20] \kurveeight;
\draw[color=darkred,line width=1.4pt] \kurvefinal;

\draw[color=blue] ({-\dx},0) -- ({-\dx},{1.5*\dy});
\draw[color=blue] ({\dx},0) -- ({\dx},{1.5*\dy});

\draw[->] ({-\dx-0.1},0) -- ({\dx+0.5},0) coordinate[label={$x$}];
\draw[->] (0,-0.1) -- (0,{1.55*\dy}) coordinate[label={right:$y$}];

\node at ({-\dx},0) [below] {$-l\mathstrut$};
\node at ({\dx},0) [below] {$l\mathstrut$};

\end{tikzpicture}
\end{document}

