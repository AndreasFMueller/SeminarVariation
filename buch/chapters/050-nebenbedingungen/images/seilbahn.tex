%
% seilbahn.tex -- Seilbahn
%
% (c) 2021 Prof Dr Andreas Müller, OST Ostschweizer Fachhochschule
%
\documentclass[tikz]{standalone}
\usepackage{amsmath}
\usepackage{times}
\usepackage{txfonts}
\usepackage{pgfplots}
\usepackage{csvsimple}
\usetikzlibrary{arrows,intersections,math}
\begin{document}
\def\skala{1}


\begin{tikzpicture}[>=latex,thick,scale=\skala]

\pgfmathparse{sqrt(9+sinh(-1.5)*sinh(-1.5)}
\xdef\l{\pgfmathresult}
\draw (-4.5,{cosh(-1.5)}) -- +({sinh(-1.5)/\l},{-3/\l});
\fill[color=gray!20] ({-4.5+sinh(-1.5)/\l},{cosh(-1.5)-3/\l}) circle[radius=1];

\pgfmathparse{sqrt(9+sinh(0.5)*sinh(0.5)}
\xdef\l{\pgfmathresult}
\draw (1.5,{cosh(0.5)}) -- +({sinh(0.5)/\l},{-3/\l});
\fill[color=gray!20] ({1.5+sinh(0.5)/\l},{cosh(0.5)-3/\l}) circle[radius=1];

\draw[color=red,line width=1.4pt] plot[domain=-1.5:0.5] ({3*\x},{cosh(\x)});

\draw[line width=0.3pt] (-7,-2) -- (-7,3);
\draw[line width=0.3pt] (4,-2) -- (4,3);

\draw[->] (-7.6,-2) -- (4.6,-2) coordinate[label={$x$}];
\draw[->] (-7.5,-2.1) -- (-7.5,3.3) coordinate[label={right:$y$}];

\draw (-7,-2.05) -- (-7,-1.95);
\node at (-7,-2) [below] {$x_0\mathstrut$};
\draw (4,-2.05) -- (4,-1.95);
\node at (4,-2) [below] {$x_1\mathstrut$};

\end{tikzpicture}
\end{document}

