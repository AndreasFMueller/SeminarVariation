%
% einseitig.tex -- Variation für einseitige Bindungen
%
% (c) 2021 Prof Dr Andreas Müller, OST Ostschweizer Fachhochschule
%
\documentclass[tikz]{standalone}
\usepackage{amsmath}
\usepackage{times}
\usepackage{txfonts}
\usepackage{pgfplots}
\usepackage{csvsimple}
\usetikzlibrary{arrows,intersections,math}
\begin{document}
\definecolor{darkred}{rgb}{0.8,0,0}
\definecolor{darkgreen}{rgb}{0,0.6,0}
\def\skala{1}
\xdef\yzero{6}
\begin{tikzpicture}[>=latex,thick,scale=\skala,
declare function={
	Phi(\t) = 5-4*((\t-9)/5)*((\t-9)/5);
	X(\t,\xend) = \t*\xend;
	Y(\t,\yend) = (1-\t)*\yzero+\t*(\yend+sin(180*\t)+0.3*sin(540*\t);
}]

%\draw (0,0) rectangle (10,9);
\def\l{0.7}

\fill[color=darkgreen!20] (6.5,0) rectangle (10,7.5);
\node[color=darkgreen] at (8.25,0) [above] {$A$};
\node[color=white] at (8.25,7.5) [below] {$A$};

\begin{scope}
\clip (0,0) rectangle (10,7.5);

\foreach \u in {-3,-2,...,3}{
	\pgfmathparse{6.5+\u*\l}
	\xdef\xend{\pgfmathresult}
	\pgfmathparse{Phi(\xend)}
	\xdef\yend{\pgfmathresult}
	\draw[color=darkred]
		plot[domain=0:1,smooth] ({X(\x,\xend)},{Y(\x,\yend)});
}
\pgfmathparse{Phi(6.5)}
\xdef\yend{\pgfmathresult}
\draw[color=darkred,line width=1.4pt]
	plot[domain=0:1,smooth] ({X(\x,6.5)},{Y(\x,\yend)});

\fill[color=blue!40,opacity=0.5]
	plot[domain=-1.25:0.2] ({9+5*\x},{5-4*\x*\x)}) -- (10,0) -- cycle;

\draw[color=blue]
	plot[domain={6.5-3*\l}:{6.5+3*\l},smooth] ({\x},{Phi(\x)});
\draw[color=blue,line width=1.4pt]
	plot[domain=6.5:10,smooth] ({\x},{Phi(\x)});

\end{scope}


\draw[line width=0.3pt] (0,0) -- (0,7.5);
\draw[line width=0.3pt] (10,0) -- (10,7.5);
\draw[line width=0.3pt] ({6.5-2*\l},{Phi(6.5-2*\l)}) -- ({6.5-2*\l},0);
\draw[line width=0.3pt] (6.5,{Phi(6.5)}) -- (6.5,0);
\node at ({6.5-2*\l},{Phi(6.5-2*\l)}) [right]
{${\color{darkred}y({\color{black}x_*(\varepsilon)},\varepsilon)}
=
{\color{blue}\varphi({\color{black}x_*(\varepsilon)})}\mathstrut$};
\node at (6.2,{Phi(6.5)-0.3}) [right]
{${\color{darkred}y(}x_*{\color{darkred})}
=
{\color{darkred}y({\color{black}x_*})}
=
{\color{blue}\varphi({\color{black}x_*})}
\mathstrut$};
\draw[->] (-1.1,0) -- (10.8,0) coordinate[label={$x$}];
\draw[->] (-1,-0.1) -- (-1,7.7) coordinate[label={right:$y$}];
\draw ({6.5-2*\l},0.05) -- ({6.5-2*\l},-0.05);
\node at ({6.5-2*\l},-0.05) [below] {$x_*(\varepsilon)\mathstrut$};
\draw (6.5,0.05) -- (6.5,-0.05);
\node at (6.5,-0.05) [below] {$x_*=x_*(0)\mathstrut$};
\draw (0,0.05) -- (0,-0.05);
\node at (0,-0.05) [below] {$x_0\mathstrut$};
\draw (10,0.05) -- (10,-0.05);
\node at (10,-0.05) [below] {$x_1\mathstrut$};
\node at (0,\yzero) [left] {$y_0$};

\fill[color=violet] (6.5,{Phi(6.5)}) circle[radius=0.05];
\fill[color=darkred] (0,\yzero) circle[radius=0.05];
\node[color=blue] at (8,4.5) {$\varphi(x)$};
\node[color=darkred] at (4,6.13) {$y(x,\varepsilon)$};
\node[color=darkred] at (6.02,4.83) [rotate=-40] {$y(x)=y(x,0)$};

\end{tikzpicture}
\end{document}

