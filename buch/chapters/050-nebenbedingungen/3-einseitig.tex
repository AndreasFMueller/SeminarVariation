%
% 3-einseitig.tex
%
% (c) 2024 Prof Dr Andreas Müller
%
\section{Einseitige Bindungen
\label{buch:nebenbedingungen:section:einseitigebindungen}}
\kopfrechts{Einseitige Bindungen}
Die bisher untersuchten Nebenbedingungen waren alle von der
Form einer Gleichung, zum Beispiel 
\[
\int_{x_0}^{x_1} G(x,y(x),y'(x))\,dx = 0
\qquad\text{oder}\qquad
y'(x_2) = c \quad\text{mit $x_2\in (x_0,x_1)$.}
\]
Im ersten Fall liefert die Idee der Lagrange-Multiplikatoren
eine Lösungsmethode.
Das zweite Problem kann aufgeteilt werden in zwei Probleme auf
den Teilintervallen $[x_0,x_2]$ und $[x_2,x_1]$ mit einem zusätzlichen
Randterm an der Stelle $x=x_2$.
Es sind aber auch Aufgabenstellungen denkbar, in denen Nebenbedingungen
in der Form einer Ungleichung gegeben sind.

%
% Ungleichungen als Nebenbedingungen
%
\subsection{Ungleichungen als Nebenbedingungen
\label{buch:nebenbedingungen:einseitig:subsection:ungleichungen}}
Wir motivieren die Problemstellung mit zwei Anwendungsbeispielen.

%
% Ein Seil über eine Stütze leiten
%
\subsubsection{Ein Seil über eine Stütze leiten}
%
% seilbahn.tex -- Seilbahn
%
% (c) 2021 Prof Dr Andreas Müller, OST Ostschweizer Fachhochschule
%
\documentclass[tikz]{standalone}
\usepackage{amsmath}
\usepackage{times}
\usepackage{txfonts}
\usepackage{pgfplots}
\usepackage{csvsimple}
\usetikzlibrary{arrows,intersections,math}
\begin{document}
\def\skala{1}


\begin{tikzpicture}[>=latex,thick,scale=\skala]

\pgfmathparse{sqrt(9+sinh(-1.5)*sinh(-1.5)}
\xdef\l{\pgfmathresult}
\draw (-4.5,{cosh(-1.5)}) -- +({sinh(-1.5)/\l},{-3/\l});
\fill[color=gray!20] ({-4.5+sinh(-1.5)/\l},{cosh(-1.5)-3/\l}) circle[radius=1];

\pgfmathparse{sqrt(9+sinh(0.5)*sinh(0.5)}
\xdef\l{\pgfmathresult}
\draw (1.5,{cosh(0.5)}) -- +({sinh(0.5)/\l},{-3/\l});
\fill[color=gray!20] ({1.5+sinh(0.5)/\l},{cosh(0.5)-3/\l}) circle[radius=1];

\draw[color=red,line width=1.4pt] plot[domain=-1.5:0.5] ({3*\x},{cosh(\x)});

\draw[line width=0.3pt] (-7,-2) -- (-7,3);
\draw[line width=0.3pt] (4,-2) -- (4,3);

\draw[->] (-7.6,-2) -- (4.6,-2) coordinate[label={$x$}];
\draw[->] (-7.5,-2.1) -- (-7.5,3.3) coordinate[label={right:$y$}];

\draw (-7,-2.05) -- (-7,-1.95);
\node at (-7,-2) [below] {$x_0\mathstrut$};
\draw (4,-2.05) -- (4,-1.95);
\node at (4,-2) [below] {$x_1\mathstrut$};

\end{tikzpicture}
\end{document}


\index{Tragseil}%
\index{Seilbahn}%
Das Tragseil einer Seilbahn trägt die Gondel, die vom Zugseil
hochgezogen wird.
\index{Zugseil}%
Normalerweise wird das Trageseil nicht einfach zwischen Berg- und Talstation
gespannt sondern über mehrere Zwischstützen geleitet.
Das Seil liegt dort in einer gekrümmten Seilauflagerung, in der es sich
in vielen Fällen in Längsrichtung bewegen kann.
Die Gesamtlänge des Seils bleibt gleich, aber in der Seilauflagerung
\index{Seilauflagerung}%
folgt es mindestens ein Stück weit deren Krümmung
(Abbildung~\ref{buch:nebenbedingungen:fig:seilbahn}).

Gesucht ist die Funktion $y(x)$, definiert auf dem Intervall $[x_0,x_1]$,
welche das Seil beschreibt.
Die Gesamtlänge des Seils ist 
\[
l(y) = \int_{x_0}^{x_1} \sqrt{1+y'(x)^2}\,dx,
\]
wie wir in früheren Beispielen gesehen haben.
Im Kapitel~\ref{chapter:kettenlinie} wird dargelegt, dass das Seil
versucht, die potentielle Energie
\begin{equation}
E(y) = \mu g \int_{x_0}^{x_1} y\sqrt{1+y'(x)^2}\,dx
\label{buch:nebenbedingungen:einseitig:eqn:seilE}
\end{equation}
zu minimieren, wobei $g$ die Erdbeschleunigung und $\mu$ die
lineare Massendichte ist.
Der Faktor $\mu g$ ändert die Lösung des Extremalproblems nicht und 
kann daher auch weggelassen werden.
In Kapitel~\ref{chapter:kettenlinie} wird als Lösung für ein Seil
ohne Zwischenstüzen eine Kettenlinie gefunden.

Die Zwischenstützen erzwingen nun, dass sich das Seil oberhalb der 
Auflagerungen bewegen muss.
Es gibt daher eine zusätzliche Nebenbedingung der Form
\[
y(x) \ge \varphi(x),
\]
der die Funktion $y(x)$ genügen muss.

%
% Einen Schlauch über einen Schlauchanschluss ziehen
%
\subsubsection{Einen Schlauch über einen Schlauchanschluss ziehen}
Wenn ein Schlauch über einen Schlauchanschluss gezogen wird, spannt
\index{Schlauch}%
\index{Schlauchanschluss}%
er sich und versucht eine Minimalfläche zu bilden.
\index{Minimalflache@Minimalfläche}%
Der Schlauch bildet daher eine Rotationsfläche minimalen Inhalts.
\index{Rotationsfläche}%
In Kapitel~\ref{chapter:minimalflaechen} wird dieses Problem 
studiert.
Beschreibt man den Radius des Schlauchs in Abhängigkeit von der
Position $x$ entlang der Achse durch die Funktion $y(x)$, dann ist
der Flächeninhalt
\begin{equation}
F(y)
=
2\pi
\int_{x_0}^{x_1}
y(x)\sqrt{1+y'(x)^2}\,dx.
\label{buch:nebenbedingungen:einseitig:eqn:schlauchF}
\end{equation}
Der Faktor $2\pi$ hat keinen Einfluss auf das Minimum.
Das Funktional $F(y)$ von
\eqref{buch:nebenbedingungen:einseitig:eqn:seilE}
ist äquivalent zum Funktional $E(y)$ von
\eqref{buch:nebenbedingungen:einseitig:eqn:schlauchF}
im Seilbahnproblem,
die Lösungsfunktion $y(x)$ ist daher eine $\cosh$-Kurve,
die Fläche eine sogenannte {\em Katenoide}.
\index{Katenoide}

Der Schlauch muss aber auch auf Teilen der Oberfläche des
Schlauchanschlusses aufliegen.
Dies bedeutet, dass der Innenradius $y(x)$ des Schlauchs immer
mindestens so gross sein muss wie der Radius $r(x)$ des Schlauchanschlusses.
Die Funktion $y(x)$ muss daher eine Nebenbedingung der Form
\[
y(x) \ge r(x),\quad x\in[x_0,x_1]
\]
erfüllen.

%
% Einseitige Bindungen
%
\subsubsection{Einseitige Bindungen}
Eine Nebenbedingung in der Form einer Ungleichung der Art
\[
y(x) \ge \varphi(x)
\]
mit einer vorgegebenen Funktion $\varphi(x)$ heisst auch
ein {\em Problem mit einseitiger Bindung}
\index{einseitige Bindung}%
\index{Bindung, einseitig}%
oder ein Problem mit einer {\em Gebietseinschränkung}.
\index{Gebietseinschrankung@Gebietseinschränkung}%

%
% Euler-Lgrange-Differentialgleichung
%
\subsection{Euler-Lagrange-Differentialungleichung
\label{buch:nebenbedingungen:einseitig:subsection:eldgl}}
Die Euler-Lagrange-Differentialgleichung war bis anhing die Standardmethode,
ein Variationsproblem in ein lokales Problem umzuwandeln.
Die Herleitung verlangte aber, dass die Lösungsfunktion $y(x)$ beliebig
variiert werden konnte, was beim Vorhandensein einseitiger Bindungen
oder Gebietseinschränkungen nicht mehr möglich ist.

%
% Aufgabenstellung
%
\subsubsection{Aufgabenstellung}
Gegeben ist ein Lagrange-Funktion
$F\colon [x_0,x_1]\times\mathbb{R}\times\mathbb{R}\to\mathbb{R}:(x,y,y')\mapsto F(x,y,y')$ 
und eine Funktion $\varphi\colon[x_0,x_1]\to\mathbb{R}$.

\begin{aufgabe}
\label{buch:nebenbedingungen:einseitig:aufgabe}
Gesucht ist eine Funktion $y(x)$, definiert auf dem Intervall $[x_0,x_1]$,
mit Randwerten $y(x_0) = y_0$ und $y(x_1)=y_1$, welche das Integral
\[
I(y)
=
\int_{x_0}^{x_1} F(x,y(x),y'(x))\,dx
\]
minimiert und ausserdem die Gebietseinschränkung
\[
y(x) \ge \varphi(x), \quad x\in[x_0,x_1]
\]
erfüllt.
\end{aufgabe}

%
% Variation mit einseitiger Bindung
%
\subsubsection{Variation mit einseitiger Bindung}
Auf einer Teilmenge $A$ des Definitionsbereichs $[x_0,x_1]$ folgt
die Lösung der Funktion $\varphi(x)$, auf dem Rest ist die Differenz
$y(x)-\varphi(x)>0$.
Mit einer Variationsfunktionen $\eta(x)$ wird die erste Variation
\[
\delta I
=
\frac{d}{d\varepsilon}I(y+\varepsilon\eta)\bigg|_{\varepsilon=0}
\]
bestimmt.
$\eta(x)$ kann ausserhalb der Menge $A$
beliebige, nicht zu grosse positive und negative Werte annehmen.
Innerhalb von $A$ aber muss $\eta(x)\ge 0$ sein.
Ausserdem kann der Parameter $\varepsilon$ nur positive Werte annehmen.
Das Kriterium für ein Minimum ist dann aber auch nicht mehr, dass die
Ableitung verschwinden muss, sondern dass die Ableitung $\ge 0$ ist.
Innerhalb von $A$ muss also gelten
\[
\frac{d}{d\varepsilon}I(y+\varepsilon\eta)\bigg|_{\varepsilon=0} \ge 0.
\]
Eingesetzt in das Integral erhalten wir
\begin{align*}
\delta I
&=
\frac{d}{d\varepsilon}
\int_{x_0}^{x_1}
F(x,y(x)+\varepsilon \eta(x),y'(x)+\varepsilon \eta'(x))\,dx
\bigg|_{\varepsilon=0}
\\
&=
\int_{x_0}^{x_1}
\frac{\partial F}{\partial y}(x,y(x),y'(x))\eta(x)
+
\frac{\partial F}{\partial y'}(x,y(x),y'(x))\eta'(x)
\,dx
\\
&=
\int_{x_0}^{x_1}
\biggl(
\frac{\partial F}{\partial y}(x,y(x),y'(x))
+
\frac{d}{dx}
\frac{\partial F}{\partial y'}(x,y(x),y'(x))
\biggr)
\eta(x)
\,dx
\intertext{Dieses Integral können wir zerlegen in ein Integral über $A$
und eines über $[x_0,x_1]\setminus A$:}
&=
\int_A
\biggl(
\frac{\partial F}{\partial y} + \frac{d}{dx}\frac{\partial F}{\partial y'}
\biggr)
\eta(x)
\,dx
+
\int_{[x_0,x_1]\setminus A}
\biggl(
\frac{\partial F}{\partial y} + \frac{d}{dx}\frac{\partial F}{\partial y'}
\biggr)
\eta(x)
\,dx
\ge 0.
\end{align*}
Durch geeignete Wahl der Funktionen $\eta$ können wir den Beitrag der
beiden Integrale zur Variation auseinanderhalten, was in den nachfolgenden
Abschnitten diskutiert werden soll.

%
% Variationen in $A$
%
\subsubsection{Variationen in $A$}
Zunächst betrachten wir Funktionen $\eta(x)$, die ausserhalb von $A$
verschwinden.
Die Werte $\eta(x)$ müssen natürlich im Inneren von $A$ immmer $\ge 0$
sein.
Wir brauchen daher die folgende Variante des Fundamentallemmas.
\index{Fundamentallemma!einseitig}%

\begin{satz}[Einseitiges Fundamentallemma]
Sei $f\colon A\to\mathbb{R}$ eine stetige Funktion und sei
\[
\int_A f(x)\eta(x)\ge 0
\]
für alle beliebig oft differenzierbaren Funktionen $\eta\colon A\to\mathbb{R}$
mit $\eta(x)\ge 0$.
\end{satz}

\begin{proof}
Wir führen den Beweis mit Widerspruch.
Sei $x_*\in A$ eine Stelle, an der $f(x_*)<0$ ist.
Dann gibt es wegen der Stetigkeit von $f$ auch eine kleine Umgebung $U$ von
$x_*$, in der $f(x)$ immer noch $<\frac12f(x_*)<0$ ist.
Mit einer deliebig oft differenzierbaren, positiven Funktion $\eta(x)$,
die nur in der Umgebung $U$ Werte $\ge 0$ hat, folgt wie im
Fundamentallemma \ref{buch:variation:fundamentallemma:satz:fundamentallemma},
dass
\[
\int_A f(x)\eta(x)\,dx < 0
\]
wird, im Widerspruch zu den Voraussetzungen.
Folglich muss $f(x)\ge 0$ sein.
\end{proof}

Aus dem Lemma folgt also, dass
\begin{equation}
\frac{\partial F}{\partial y}(x,y(x),y'(x))
-
\frac{d}{dx}
\frac{\partial F}{\partial y'}(x,y(x),y'(x))
\ge 0
\end{equation}
gilt für $x\in A$.

%
% Variation ausserhalb von $A$
%
\subsubsection{Variationen ausserhalb von $A$}
Ausserhalb von $A$ darf $\eta(x)$ positive und negative Werte annehmen.
Wir betrachten daher jetzt Funktionen $\eta(x)$, die innerhalb von $A$
verschwinden.
Wir schreiben $B=[x_0,x_1]\setminus A$.
Für eine solche Funktion muss
\[
\int_B \biggl(
\frac{\partial F}{\partial y}(x,y(x),y'(x))
-
\frac{d}{dx}
\frac{\partial F}{\partial y'}(x,y(x),y'(x))
\biggr)
\eta(x)
\,dx
\ge 0
\]
gelten.
Dasselbe muss auch für $-\eta(x)$ gelten, also
\[
0
\le
\int_B \biggl(
\frac{\partial F}{\partial y}
-
\frac{d}{dx}
\frac{\partial F}{\partial y'}
\biggr)
(-\eta(x))
\,dx
=
-
\int_B \biggl(
\frac{\partial F}{\partial y}
-
\frac{d}{dx}
\frac{\partial F}{\partial y'}
\biggr)
\eta(x)
\,dx
\le 0,
\]
dies ist nur möglich, wenn das Integral verschwindet.
Nach dem Fundamentallemma muss dann auch der Klammerausdruck
im Integranden verschwinden.
Ausserhalb von $A$ gilt daher weiterhin die
Euler-Lagrange-Differentialgleichung.

\begin{satz}[Euler-Lagrange-Differentialungleichung]
\label{buch:nebenbedingungen:einseitig:satz:elungleichung}
Eine Funktion $y(x)$, die die
Aufgabe~\ref{buch:nebenbedingungen:einseitig:aufgabe}
löst, erfüllt dort, wo $y(x)=\varphi(x)$ ist, die Ungleichung
\index{Euler-Lagrange-Differentialungleichung}%
\index{Ungleichung, Differential-}%
\begin{equation}
\frac{\partial F}{\partial y'}(x,\varphi(x),\varphi'(x))
-
\frac{d}{dx}\frac{\partial F}{\partial y'}(x,\varphi(x),\varphi'(x))
\ge
0
\label{buch:nebenbedingungen:einseitig:eqn:eldglu}
\end{equation}
und im Rest des Definitionsbereichs die Euler-Lagrange-Differentialgleichung
\begin{equation}
\frac{\partial F}{\partial y}(x,y(x),y'(x))
-
\frac{d}{dx}
\frac{\partial F}{\partial y'}(x,y(x),y'(x))
=
0.
\label{buch:nebenbedingungen:einseitig:eqn:eldgl}
\end{equation}
\end{satz}

%
% Anschlussbedingungen
%
\subsection{Anschlussbedingungen
\label{buch:nebenbedingungen:einseitig:subsection:anschluss}}
%
% einseitig.tex -- Variation für einseitige Bindungen
%
% (c) 2021 Prof Dr Andreas Müller, OST Ostschweizer Fachhochschule
%
\documentclass[tikz]{standalone}
\usepackage{amsmath}
\usepackage{times}
\usepackage{txfonts}
\usepackage{pgfplots}
\usepackage{csvsimple}
\usetikzlibrary{arrows,intersections,math}
\begin{document}
\definecolor{darkred}{rgb}{0.8,0,0}
\definecolor{darkgreen}{rgb}{0,0.6,0}
\def\skala{1}
\xdef\yzero{6}
\begin{tikzpicture}[>=latex,thick,scale=\skala,
declare function={
	Phi(\t) = 5-4*((\t-9)/5)*((\t-9)/5);
	X(\t,\xend) = \t*\xend;
	Y(\t,\yend) = (1-\t)*\yzero+\t*(\yend+sin(180*\t)+0.3*sin(540*\t);
}]

%\draw (0,0) rectangle (10,9);
\def\l{0.7}

\fill[color=darkgreen!20] (6.5,0) rectangle (10,7.5);
\node[color=darkgreen] at (8.25,0) [above] {$A$};
\node[color=white] at (8.25,7.5) [below] {$A$};

\begin{scope}
\clip (0,0) rectangle (10,7.5);

\foreach \u in {-3,-2,...,3}{
	\pgfmathparse{6.5+\u*\l}
	\xdef\xend{\pgfmathresult}
	\pgfmathparse{Phi(\xend)}
	\xdef\yend{\pgfmathresult}
	\draw[color=darkred]
		plot[domain=0:1,smooth] ({X(\x,\xend)},{Y(\x,\yend)});
}
\pgfmathparse{Phi(6.5)}
\xdef\yend{\pgfmathresult}
\draw[color=darkred,line width=1.4pt]
	plot[domain=0:1,smooth] ({X(\x,6.5)},{Y(\x,\yend)});

\fill[color=blue!40,opacity=0.5]
	plot[domain=-1.25:0.2] ({9+5*\x},{5-4*\x*\x)}) -- (10,0) -- cycle;

\draw[color=blue]
	plot[domain={6.5-3*\l}:{6.5+3*\l},smooth] ({\x},{Phi(\x)});
\draw[color=blue,line width=1.4pt]
	plot[domain=6.5:10,smooth] ({\x},{Phi(\x)});

\end{scope}


\draw[line width=0.3pt] (0,0) -- (0,7.5);
\draw[line width=0.3pt] (10,0) -- (10,7.5);
\draw[line width=0.3pt] ({6.5-2*\l},{Phi(6.5-2*\l)}) -- ({6.5-2*\l},0);
\draw[line width=0.3pt] (6.5,{Phi(6.5)}) -- (6.5,0);
\node at ({6.5-2*\l},{Phi(6.5-2*\l)}) [right]
{${\color{darkred}y({\color{black}x_*(\varepsilon)},\varepsilon)}
=
{\color{blue}\varphi({\color{black}x_*(\varepsilon)})}\mathstrut$};
\node at (6.2,{Phi(6.5)-0.3}) [right]
{${\color{darkred}y(}x_*{\color{darkred})}
=
{\color{darkred}y({\color{black}x_*})}
=
{\color{blue}\varphi({\color{black}x_*})}
\mathstrut$};
\draw[->] (-1.1,0) -- (10.8,0) coordinate[label={$x$}];
\draw[->] (-1,-0.1) -- (-1,7.7) coordinate[label={right:$y$}];
\draw ({6.5-2*\l},0.05) -- ({6.5-2*\l},-0.05);
\node at ({6.5-2*\l},-0.05) [below] {$x_*(\varepsilon)\mathstrut$};
\draw (6.5,0.05) -- (6.5,-0.05);
\node at (6.5,-0.05) [below] {$x_*=x_*(0)\mathstrut$};
\draw (0,0.05) -- (0,-0.05);
\node at (0,-0.05) [below] {$x_1\mathstrut$};
\draw (10,0.05) -- (10,-0.05);
\node at (10,-0.05) [below] {$x_2\mathstrut$};
\node at (0,\yzero) [left] {$y_1$};

\fill[color=violet] (6.5,{Phi(6.5)}) circle[radius=0.05];
\fill[color=darkred] (0,\yzero) circle[radius=0.05];
\node[color=blue] at (8,4.5) {$\varphi(x)$};
\node[color=darkred] at (4,6.13) {$y(x,\varepsilon)$};
\node[color=darkred] at (6.02,4.83) [rotate=-40] {$y(x)=y(x,0)$};

\end{tikzpicture}
\end{document}


Eine Lösung der
Aufgabe~\ref{buch:nebenbedingungen:einseitig:aufgabe}
setzt sich aus Teilstücken zusammen, auf denen die Lösungsfunktion mit
$\varphi(x)$ übereinstimmt und wo die Bedingung
\eqref{buch:nebenbedingungen:einseitig:eqn:eldglu}
erfüllt sein muss, und ``freien'' Teilstücken, wo die
Euler-Lagrange-Differentialgleichung
\eqref{buch:nebenbedingungen:einseitig:eqn:eldgl} gilt.
An einer Übergangsstelle $x_*$ muss $y(x_*)=\varphi(x_*)$ sein,
aber die Steigung von $y(x)$ kann in $x_*$ einen Sprung
erleiden.
Wir suchen nach einer Bedingung für die die beiden Steigungen 
$y'(x_*+)$ und $y'(x_*-)$ ähnlich wie die Transversalitätsbedingungen.
\index{Anschlussbedingungen}%

\begin{satz}[Anschlussbedingungen]
\label{buch:nebenbedingungen:einseitig:satz:anschlussbedingungen}
Ist $x_*$ ein Punkt in $[x_0,x_1]$, links von dem eine Lösung $y(x)$ der
Aufgabe~\ref{buch:nebenbedingungen:einseitig:aufgabe}
frei ist, während sie rechts davon an $\varphi(x)$ anliegt, also
$y(x) = \varphi(x)$ für $x>x_*$ und nahe bei $x_*$, dann gilt
für die linksseitige Steigung $y'(x_*-)$ von $y(x_*)$ die
Bedingung
\[
F(x_*,y(x_*),y'(x_*-)) = F(x_*,\varphi(x_*),\varphi'(x_*)).
\]
\end{satz}

\begin{proof}
Indem wir uns nötigenfalls auf eine Umgebung von $x_*$ beschränken,
können wir eine Variation $x(\varepsilon)=x_*+\varepsilon$ und
$y(x,\varepsilon)$ derart wählen, dass die Funktion $x\mapsto y(x,\varepsilon)$
für $x\in[x_0,x_*+\varepsilon]$ eine Lösung der
Euler-Lagrange-Differentialgleichung ist, und in $x\in[x_*+\varepsilon,x_1]$
mit der Funktion $\varphi(x)$ übereinstimmt.
Eine solche Variation ist in
Abbildung~\ref{buch:nebenbedingungen:einseitig:fig:einseitig}
dargestellt.
Damit $y(x,0)$ eine Extremale ist, muss die Ableitung des Funktionals
$I(y(x,\varepsilon))$ nach $\varepsilon$ an der Stelle $0$ verschwinden.

Wir berechnen jetzt das Funktional
\begin{align}
I(y(x,\varepsilon))
&=
\int_{x_0}^{x_1} F(x,y(x,\varepsilon),y'(x,\varepsilon))\,dx
\\
&=
\int_{x_0}^{x_*+\varepsilon} F(x,y(x,\varepsilon),y'(x,\varepsilon))\,dx
+
\int_{x_*+\varepsilon}^{x_1} F(x,\varphi(x),\varphi'(x))\,dx.
\notag
\intertext{Die Ableitung des zweiten Integrals gibt den Wert an der unteren
Grenze, also}
&=
\int_{x_0}^{x_*+\varepsilon} F(x,y(x,\varepsilon),y'(x,\varepsilon))\,dx
-
F(x_*,\varphi(x_*),\varphi'(x_*).
\notag
\intertext{Die Ableitung des ersten Integrals ist durch die allgemeine
Variationsformel von
Satz~\ref{buch:variation:allgemein:satz:allgemeinvariation1} gegeben, die}
&=
\vec{r}_* \cdot \vec{f}_*
+
\int_{x_0}^{x_*}
\biggl(
\frac{\partial F}{\partial y}-\frac{d}{dx}\frac{\partial F}{\partial y}
\biggr)
\frac{\partial y}{\partial\varepsilon}(x,0)\,dx
-
F(x_*,\varphi(x_*),\varphi'(x_*)
\label{buch:nebenbedingungen:einseitig:eqn:allgvariation}
\end{align}
ergibt.
Darin ist $\vec{r}_*$ der Tangentialvektor an die Randkurve, also
\[
\vec{r}_* 
=
\begin{pmatrix}
1\\
\varphi'(x_*)
\end{pmatrix},
\]
und $\vec{f}_*$ ist gegeben durch die
Formel~\eqref{buch:nebenbedingungen:allgemein:eqn:fi}, also
\[
\vec{f}_*
=
\begin{pmatrix}
\displaystyle
F(x_*,y(x_*),y'(x_*))
-
\frac{\partial F}{\partial y'}(x_*,y(x_*),y'(x_*))
\frac{\partial y}{\partial \varepsilon}(x_*,0)\\
\displaystyle
\frac{\partial F}{\partial y'}(x_*,y(x_*),y'(x_*))
\end{pmatrix}.
\]
Da $y(x,\varepsilon)$ links von $x_*$ eine Lösung der
Euler-Lagrange-Differentiagleichung ist, verschwindet der Integrand
in \eqref{buch:nebenbedingungen:einseitig:eqn:allgvariation}
und das Integral leistet keinen Beitrag.
Damit kann jetzt die Ableitung nach $\varepsilon$ an der Stelle
$\varepsilon=0$ ausgerechnet werden, sie ist
\begin{align*}
\delta I(y)
&=
F(x_*,y(x_*),y'(x_*))
-
\frac{\partial F}{\partial y'}(x_*,y(x_*),y'(x_*))
\frac{\partial y}{\partial \varepsilon}(x_*,0)
\\
&\qquad
+
\frac{\partial F}{\partial y'}(x_*,y(x_*),y'(x_*))\varphi'(x_*)
-
F(x_*,\varphi(x_*),\varphi'(x_*)).
\intertext{Wegen $\partial y(x_*,0)/\partial \varepsilon=\varphi'(x_*)$
heben sich der zweite und dritte Term weg und es bleibt}
&=
F(x_*,y(x_*),y'(x_*))
-
F(x_*,\varphi(x_*),\varphi'(x_*)).
\end{align*}
Da die Ableitung $\delta I(y)$ verschwindet, folgt
\[
F(x_*,y(x_*),y'(x_*))
=
F(x_*,\varphi(x_*),\varphi'(x_*)),
\]
womit die Aussage bewiesen ist.
\end{proof}

Die Bedingung ist natürlich erfüllt, wenn $y'(x_*-)=\varphi'(x_*)$ ist.
Dies ist der einfachste Fall, in diesem Fall ist $y'(x)$ stetig,
die Lösungskurve $y(x)$ schmiegt sich an der Stelle $x_*$  tangential
an die Nebenbedingung $\varphi(x)$ an.

Die Lösungsfunktion kann nur dann eine Sprungstelle bei $x_*$ haben,
wenn die Gleichung
\[
F(x_*\varphi(x_*),p)
=
F(x_*,\varphi(x_*),\varphi'(x_*))
\]
für $p$ mehr als eine Lösung hat.
Falls $p$ eine weitere Lösung ist, sind auch Extremalen mit
\(
y'(x_*-)=p
\)
denkbar.

