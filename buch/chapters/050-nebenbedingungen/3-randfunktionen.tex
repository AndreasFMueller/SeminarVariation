%
% 3-randfunktionen.tex
%
% (c) 2024 Prof Dr Andreas Müller
%
\section{Randfunktionen
\label{buch:nebenbedingungen:section:randfunktionen}}
In Abschnitt~\ref{buch:nebenbedingungen:section:transversal} wurden
Bedingungen am Rand des Definitionsgebietes betrachtet und dabei
die Transversalitätsbedingungen aus der allgemeinen Variationsformel
abgeleitet.
In den weiteren Untersuchungen war dann aber das Definitionsgebiet
fest.
Das zu minimierende Funktional enthielt ausser dem Integral auch
noch weitere Terme, die aber nur von den Werten $y(x_0)$ und
$y(x_1)$ der Funktion $y$ an den Endpunktend es Intervalls
abhingen.
Die allgemeine Variationsformel ermöglicht aber auch die Behandlung
von Randfunktionen, die auch von den möglicherweise variablen
Intervallgrenzen abhängen können.
Dieser Fall wird zum Beispiel in Kapitel~\ref{chapter:widerstand}
benötigt.

%
% Aufgabenstellung
%
\subsection{Aufgabenstellung
\label{buch:nebenbedingungen:randfunktionen:subsection:aufgabenstellung}}

\begin{equation}
I(y)
=
\int_{x_0}^{x_1}
F(x,y(x),y'(x))
\,dx
+
g_0(x_0,y(x_0))
+
g_1(x_1,y(x_1))
\label{buch:nebenbedingungen:randfunktionen:eqn:funktional}
\end{equation}

%
% Lösung
%
\subsection{Lösung
\label{buch:nebenbedingungen:randfunktionen:subsection:loesung}}


