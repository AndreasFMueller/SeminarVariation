%
% 3-randfunktionen.tex
%
% (c) 2024 Prof Dr Andreas Müller
%
\section{Randfunktionen
\label{buch:nebenbedingungen:section:randfunktionen}}
In Abschnitt~\ref{buch:nebenbedingungen:section:transversal} wurden
Bedingungen am Rand des Definitionsgebietes betrachtet und dabei
die Transversalitätsbedingungen aus der allgemeinen Variationsformel
abgeleitet.
In den weiteren Untersuchungen war dann aber das Definitionsgebiet
fest.
Das zu minimierende Funktional enthielt ausser dem Integral auch
noch weitere Terme, die aber nur von den Werten $y(x_0)$ und
$y(x_1)$ der Funktion $y$ an den Endpunktend es Intervalls
abhingen.
Die allgemeine Variationsformel ermöglicht aber auch die Behandlung
von Randfunktionen, die auch von den möglicherweise variablen
Intervallgrenzen abhängen können.
Dieser Fall wird zum Beispiel in Kapitel~\ref{chapter:widerstand}
benötigt.

%
% Aufgabenstellung
%
\subsection{Aufgabenstellung
\label{buch:nebenbedingungen:randfunktionen:subsection:aufgabenstellung}}
Wir betrachten wieder ein Variationsproblem für ein Integral mit einer
Lagrange-Funktion $F(x,y,y')$, aber das Definitionsintervall soll
jetzt variabel sein.
Wir bezeichnen das Integral miut
\begin{equation*}
I_0(y)
=
\int_{x_0}^{x_1}
F(x,y(x),y(x))\,dx
\end{equation*}
Wir verlangen nur dass sich die Endpunkte $(x_0,y_0=y(x_1))$ und 
$(x_1,y_1=y(x_1))$ auf vorgegebenen Randkurven bewegen, die zum
Beispiel durch eine Parameterdarstellung gegeben sein können.
Zusätzlich soll aber der Wert des Funktionals nicht nur vom ``freien''
Teil der Funktionskurve abhängen, sondern auch davon, wo die Kurve auf
die Ränder trifft.
Sind $\alpha\mapsto \gamma_i(\alpha) = (x_i(\alpha),y_i(\alpha))$
Parametrisierungen der Randkurven, dann muss es Beiträge $g_i(\alpha)$
zum Funktional geben, die von $\alpha$ abhängen.
Es ist jedoch einfacher, wenn sich diese Beträge durch die Koordinaten
$(x_i,y_i)$ der Endpunkte ausdrücken alssen, also durch ein Funktionen
$g_i(x,y)$, für die
\[
g_i(\alpha) = g_i(x_i(\alpha), y_i(\alpha))
\]
gilt.

\begin{aufgabe}
\label{buch:nebenbedingungen:randfunktionen:aufgabe}
Sei $F\colon\mathbb{R}\times\mathbb{R}^n \times\mathbb{R}^n\to\mathbb{R}$
eine stetig differenzierbare Funktion und seien die Funktionen 
$g_i\colon\mathbb{R}\to\mathbb{R}$ stetig differenzierbar.
Man löse das Anfangskurve-Endkurve-Problem für das Funktional
\begin{align}
I(y)
&=
I_0(y)
+
g_1(x_1,y(x_1))
-
g_0(x_0,y(x_0))
\notag
\\
&=
\int_{x_0}^{x_1}
F(x,y(x),y'(x))
\,dx
+
g_1(x_1,y(x_1))
-
g_0(x_0,y(x_0)),
\label{buch:nebenbedingungen:randfunktionen:eqn:funktional}
\end{align}
wobei die Funktion $y(x)$ die Randbedingungen
\(
y(x_i(\alpha)) = y_i(\alpha)
\)
für Parametrisierungen der $(x_i(\alpha),y_i(\alpha))$ der Anfangs-
bzw.~Endkurve erfüllt.
\end{aufgabe}

Die Wahl der Vorzeichen in 
\eqref{buch:nebenbedingungen:randfunktionen:eqn:funktional}
ist so, dass sich für die Bedingungen einfachere Formeln ergeben.

%
% Lösung
%
\subsection{Lösung
\label{buch:nebenbedingungen:randfunktionen:subsection:loesung}}
Die allgemeine Varitiationsformel ermöglicht, die Veränderung des
Integrals im Funktional
\eqref{buch:nebenbedingungen:randfunktionen:eqn:funktional}
zu berechnen.
Wenn sich die Endpunkte der Funktion auf vorgegebenen
Kurven bewegen, ergeben sich in
Satz~\ref{buch:nebenbedingungen:transversal:satz:transversal}
die Transversalitätsbedingungen.
Der Einfluss der Wahl der Endpunkte kam dabei ausschliesslich vom
Integral.
Die zusätzlichen Terme in
\eqref{buch:nebenbedingungen:randfunktionen:eqn:funktional}
verlangen daher, dass die Transversalitätsbedingungen modifiziert
werden müssen, um ihren Einfluss auf den Wert des Funktionals
adäquat zu berücksichtigen.

Sei jetzt jetzt also $y(x,\varepsilon)$ wieder eine allgemeine Variation
einer Lösung des Extremalproblems für das
Funktional~\eqref{buch:nebenbedingungen:randfunktionen:eqn:funktional}.
Die Endpunkte sind wieder die Punkte
$(x_i(\varepsilon),y(x_i(\varepsilon),\varepsilon))$ mit $i=0,1$.
Die Variation des Integrals ist nach
Satz~\ref{buch:variation:allgemein:satz:allgemeinvariation1} 
gegeben durch
\begin{align*}
\delta I
&=
\vec{f}_i\cdot \vec{r}_i(0) - \vec{f}_0\cdot\vec{r}_0(0)
+
\delta I_0(y)
\\
&=
\vec{f}_i\cdot \vec{r}_i(0) - \vec{f}_0\cdot\vec{r}_0(0)
+
\int_{x_0}^{x_1}
\biggl(
\frac{\partial F}{\partial y}(x,y(x),y'(x))
-
\frac{d}{dx}\frac{\partial F}{\partial y'}(x,y(x),y'(x))
\biggr)
\frac{\partial y}{\partial\varepsilon}(x,0)
\,dx
\end{align*}
gegeben.
Dazu kommen jetzt noch die Ableitungen der Randfunktionen nach
$\varepsilon$ an der Stelle $\varepsilon=0$.
Sie sind
\begin{align*}
\frac{\partial}{\partial\varepsilon}
g_i(x_i(\varepsilon),y(x_i(\varepsilon),\varepsilon))
\bigg|_{\varepsilon=0}
&=
\frac{\partial g_i}{\partial x}(x_i,y(x_i))
\frac{d x_i}{d\varepsilon}\bigg|_{\varepsilon=0}
+
\frac{\partial g_i}{\partial y}(x_i,y(x_i))
\frac{d}{d\varepsilon}y(x_i(\varepsilon),\varepsilon)
\bigg|_{\varepsilon=0}
\\
&=
\bgroup
\renewcommand\arraystretch{2}
\begin{pmatrix}
\displaystyle
\frac{\partial g_i}{\partial x}(x_i,y(x_i))
\\
\displaystyle
\frac{\partial g_i}{\partial y}(x_i,y(x_i))
\end{pmatrix}
\egroup
\cdot
\frac{d}{d\varepsilon}
\begin{pmatrix}
x_i(\varepsilon)\\
y(x_i(\varepsilon),\varepsilon)
\end{pmatrix}
\bigg|_{\varepsilon=0}.
\end{align*}
Der zweite Faktor ist genau der Vektor $\vec{r}_i$. 
Für die Variation des Funktionals muss also der Vektor $\vec{f}_i$ durch
den Vektor
\begin{equation}
\vec{f}_i
=
\renewcommand\arraystretch{2}
\begin{pmatrix}
\displaystyle
F(x_i,y(x_i),y'(x_i))
-
y'(x_i) \frac{\partial F}{\partial y'}(x_i,y(x_i),y'(x_i))
+
\frac{\partial g_i}{\partial x}(x_i,y(x_i))
\\
\displaystyle
\frac{\partial F}{\partial y'}(x_i,y(x_i),y'(x_i))
+
\frac{\partial g_i}{\partial y}(x_i,y(x_i))
\end{pmatrix}
\label{buch:nebenbedingungen:randfunktionen:eqn:fnew}
\end{equation}
ersetzt werden.

Aus der Variationsbedingung $\delta I=0$ folgt jetzt, dass eine
Lösung des Variationsproblems im Inneren des Definitionsintervalls
die Euler-Lagrange-Differentialgleichung erfüllt.
Ausserdem müssen an den Endpunkten die Randbedingungen
\begin{equation}
\vec{f}_i\cdot\vec{r}_i
=
0,\quad i=0,1,
\end{equation}
mit den Vektoren $\vec{f}_i$ von
\eqref{buch:nebenbedingungen:randfunktionen:eqn:fnew}
erfüllt sein.

\begin{satz}[Lösung eines Variationsproblems mit Randfunktionen]
\label{buch:nebenbedingungen:randfunktionen:satz:loesung}
Eine Lösung $y(x)$ der
Aufgabe~\ref{buch:nebenbedingungen:randfunktionen:aufgabe}
erfüllt die Euler-Lagrange-Differentialgleichung
\begin{equation}
\frac{\partial F}{\partial y}(x,y(x),y'(x))
-
\frac{d}{dx}
\frac{\partial F}{\partial y'}(x,y(x),y'(x))
=
0
\end{equation}
und die Randbedingungen
\[
\vec{f}_i\cdot \vec{r}_i =  0
\qquad\Rightarrow\qquad
\vec{f}_i\perp\vec{r}_i
\]
mit den Vektoren $\vec{f}_i$ von
\eqref{buch:nebenbedingungen:randfunktionen:eqn:fnew}
und den Vektoren $\vec{r}_i$, die Tangentialvektoren an die Randkurven
sind.
\end{satz}


