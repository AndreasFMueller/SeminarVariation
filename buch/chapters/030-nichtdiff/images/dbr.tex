%
% dbr.tex -- Beweis des Lemmas von du Bois-Reymond
%
% (c) 2024 Prof Dr Andreas Müller, OST Ostschweizer Fachhochschule
%
\documentclass[tikz]{standalone}
\usepackage{amsmath}
\usepackage{times}
\usepackage{txfonts}
\usepackage{pgfplots}
\usepackage{csvsimple}
\definecolor{darkgreen}{rgb}{0,0.6,0}
\definecolor{darkred}{rgb}{0.8,0,0}
\usetikzlibrary{arrows,intersections,math,calc}
\begin{document}
\def\skala{1}
\begin{tikzpicture}[>=latex,thick,scale=\skala,
declare function={ f(\x) = 2*cos(10*\x-20)+0.5*sin(50*\x)+1; },
]

\def\a{2}
\def\b{8}
\def\e{0.8}
\def\h{3}
\def\d{0.45}

\fill[color=gray!20] ({\a-\e},-3.2) rectangle ({\a+\e},4.3);
\fill[color=gray!20] ({\b-\e},-3.2) rectangle ({\b+\e},4.3);


\fill[color=orange!20] ({\a-\e},{f(\a)-\d}) rectangle ({\a+\e},{f(\a)+\d});
\fill[color=orange!20] ({\b-\e},{f(\b)-\d}) rectangle ({\b+\e},{f(\b)+\d});
\fill[color=blue!20]
	plot[domain=-180:180] ({\a-\e*\x/180},{\h*(1+cos(\x))/2}) -- cycle;
\fill[color=darkred!20]
	plot[domain=-180:180] ({\b-\e*\x/180},{-\h*(1+cos(\x))/2}) -- cycle;

\draw[color=orange,line width=1.4pt] ({\a-\e},{f(\a)}) -- ({\a+\e},{f(\a)});
\draw[color=orange,line width=1.4pt] ({\b-\e},{f(\b)}) -- ({\b+\e},{f(\b)});
\draw[color=orange,line width=0.4pt]
	({\a-\e},{f(\a-\e)}) -- ({\a-\e},{f(\a)}) -- ({\a+\e},{f(\a)})
	-- ({\a+\e},{f(\a+\e)});
\draw[color=orange,line width=0.4pt]
	({\b-\e},{f(\b-\e)}) -- ({\b-\e},{f(\b)}) -- ({\b+\e},{f(\b)})
	-- ({\b+\e},{f(\b+\e)});

\draw[color=darkgreen,line width=1.4pt]
	plot[domain=0:11,samples=100] ({\x},{f(\x)});
\draw[color=darkgreen,line width=0.3pt] (\a,0) -- (\a,{f(\a)});
\draw[color=darkgreen,line width=0.3pt] (\b,0) -- (\b,{f(\b)});
\fill[color=darkgreen] (\a,{f(\a)}) circle[radius=0.08];
\fill[color=darkgreen] (\b,{f(\b)}) circle[radius=0.08];
\node[color=darkgreen] at (\a,{f(\a)}) [above] {$f(a)$};
\node[color=darkgreen] at (\b,{f(\b)}) [above] {$f(b)$};

\node[color=orange] at ({\a+\e},{f(\a)+\d}) [right] {$f(a)+\delta\mathstrut$};
\node[color=orange] at ({\a+\e},{f(\a)-\d}) [right] {$f(a)-\delta\mathstrut$};
\node[color=orange] at ({\b+\e},{f(\b)+\d}) [right] {$f(b)+\delta\mathstrut$};
\node[color=orange] at ({\b+\e},{f(\b)-\d}) [right] {$f(b)-\delta\mathstrut$};

\coordinate (A) at (6,3.9);

\node[color=orange] at (A) {$\tilde{f}(x)$};
\draw[color=orange,line width=0.3pt,shorten >= 0.3cm]
	({\a+0.8*\e},{f(\a)}) -- (A);
\draw[color=orange,line width=0.3pt,shorten >= 0.3cm]
	({\b-0.8*\e},{f(\b)}) -- (A);

\node[color=darkgreen] at (5,{f(5)}) [below] {$f(x)$};

\draw[color=blue,line width=1.1pt]
	plot[domain=-180:180] ({\a-\e*\x/180},{\h*(1+cos(\x))/2});

\draw[color=darkred,line width=1.1pt]
	plot[domain=-180:180] ({\b-\e*\x/180},{-\h*(1+cos(\x))/2});

\draw[->] (-0.1,0) -- (11.5,0) coordinate[label={$x$}];
\draw[->] (0,-3.2) -- (0,4.3) coordinate[label={left:$y$}];

\draw (\a,-0.05) -- (\a,0.05);
\node at (\a,0) [below] {$a\mathstrut$};
\draw (\b,-0.05) -- (\b,0.05);
\node at (\b,0) [below] {$b\mathstrut$};

\draw ({\a-\e},-0.05) -- ({\a-\e},0.05);
\node at ({\a-\e},0) [below left] {$a-\varepsilon\mathstrut$};
\draw ({\a+\e},-0.05) -- ({\a+\e},0.05);
\node at ({\a+\e},0) [below right] {$a+\varepsilon\mathstrut$};

\node[color=blue] at (\a,{0.3*\h}) {$g_a(x)$};
\node[color=darkred] at (\b,{-0.3*\h}) {$-g_b(x)$};

\draw ({\b-\e},-0.05) -- ({\b-\e},0.05);
\node at ({\b-\e},0) [below left] {$b-\varepsilon\mathstrut$};
\draw ({\b+\e},-0.05) -- ({\b+\e},0.05);
\node at ({\b+\e},0) [below right] {$b+\varepsilon\mathstrut$};

\end{tikzpicture}
\end{document}

