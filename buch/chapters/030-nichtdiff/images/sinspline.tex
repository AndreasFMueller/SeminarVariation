%
% sinspline.tex -- template for standalon tikz images
%
% (c) 2021 Prof Dr Andreas Müller, OST Ostschweizer Fachhochschule
%
\documentclass[tikz]{standalone}
\usepackage{amsmath}
\usepackage{times}
\usepackage{txfonts}
\usepackage{pgfplots}
\usepackage{csvsimple}
\usetikzlibrary{arrows,intersections,math}
\definecolor{darkred}{rgb}{0.8,0,0}
\definecolor{darkgreen}{rgb}{0,0.6,0}
\begin{document}
\input{sinpath.tex}
\def\skala{1}
\begin{tikzpicture}[>=latex,thick,scale=\skala]

\draw[->] (-0.1,0) -- ({3*3.14159+0.4},0) coordinate[label={$x$}];
\draw[->] (0,-2.1) -- (0,7.6) coordinate[label={right:$y$}];

\draw (-0.05,6) -- (0.05,6);
\node at (-0.05,6) [left] {$1\mathstrut$};
\draw[line width=0.3pt] (0,6) -- ({3*3.14159},6);
\draw ({3*3.14159},-0.05) -- ({3*3.14159},0.05);
\node at ({3*3.14159},-0.05) [below] {$\displaystyle\frac{\pi}2$};
\node at (-0.05,0) [left] {$0\mathstrut$};

\draw[color=darkgreen] \fehlerspline;

\draw[color=orange] \fehlerhermite;

\draw[color=blue] plot[domain=0:90,samples=100]
	({6*3.14159*\x/180},{6*sin(\x)});

\draw[color=darkred,line width=1.4pt] \spline;

\foreach \x in {0,9,...,90}{
	\fill[color=blue,line width=1.4pt]
		({6*3.14159*\x/180},{6*sin(\x)}) circle[radius=0.08];
}

\end{tikzpicture}
\end{document}

