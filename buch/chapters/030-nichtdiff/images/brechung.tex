%
% brechung.tex -- Herleitung des Brechungsgesetztes
%
% (c) 2021 Prof Dr Andreas Müller, OST Ostschweizer Fachhochschule
%
\documentclass[tikz]{standalone}
\usepackage{amsmath}
\usepackage{times}
\usepackage{txfonts}
\usepackage{pgfplots}
\usepackage{csvsimple}
\definecolor{darkred}{rgb}{0.8,0,0}
\usetikzlibrary{arrows,intersections,math,calc}
\begin{document}
\def\skala{1}
\begin{tikzpicture}[>=latex,thick,scale=\skala]

\def\xnull{-3.5}
\def\ynull{1}
\def\xeins{4.0}
\def\yeins{5}
\def\ystern{3.5}

\pgfmathparse{atan2(\yeins-\ystern,\xeins)}
\xdef\alphaeins{\pgfmathresult}
\pgfmathparse{atan2(\ynull-\ystern,\xnull)}
\xdef\alphanull{\pgfmathresult}

\coordinate (A) at (\xnull,\ynull);
\coordinate (B) at (0,\ystern);
\coordinate (C) at (\xeins,\yeins);

\fill[color=blue!10] (-5,-0.1) rectangle (0,6.1);
\fill[color=blue!30] (0,-0.1) rectangle (5,6.1);

\fill[color=gray!40,opacity=0.5]
	(B) -- +(0:2) arc (0:\alphaeins:2) -- cycle;
\node at ($(B)+({\alphaeins/2}:1.5)$) {$\alpha(0+)$};
\fill[color=gray!40,opacity=0.5]
	(B) -- +(-180:2) arc (-180:\alphanull:2) -- cycle;
\node at ($(B)+({-180+(180+\alphanull)/2}:{1.5})$) {$\alpha(0-)$};

\draw[->] (0,-0.1) -- (0,6.5) coordinate[label={right:$y$}];
\draw[->] (-5.1,0) -- (5.5,0) coordinate[label={$x$}];

\draw[line width=0.3pt] (-3,\ystern) -- (3,\ystern);

\draw[line width=0.3pt] (\xnull,0) -- (A);
\draw (\xnull,0.05) -- (\xnull,-0.05);
\node at (\xnull,-0.05) [below] {$x_1\mathstrut$};
\draw[line width=0.3pt] (\xeins,0) -- (C);
\draw (\xeins,0.05) -- (\xeins,-0.05);
\node at (\xeins,-0.05) [below] {$x_2\mathstrut$};
\node at (0,-0.05) [below] {$0\mathstrut$};

\draw[color=darkred,line width=1.4pt] (A) -- (B) -- (C);
\fill[color=darkred] (A) circle[radius=0.08];
\node at (A) [left] {$(x_1,y_1)$};
\fill[color=darkred] (C) circle[radius=0.08];
\node at (C) [above] {$(x_2,y_2)$};

\node[color=blue] at (-4.5,5.5) {$n_-$};
\node[color=blue] at (4.5,0.5) {$n_+$};

\node[color=darkred] at (B) [below right] {$(0,y_*)$};
\fill[color=darkred] (B) circle[radius=0.08];

\end{tikzpicture}
\end{document}

