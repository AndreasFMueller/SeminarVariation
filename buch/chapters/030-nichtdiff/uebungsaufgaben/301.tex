Man finde die kürzeste Verbindung zwischen zwei Punkten in der oberen
Halbebene, die durch einen Punkt auf der $x$-Achse verläuft
(Abbildung~\ref{buch:301:fig:reflexion}).
\begin{figure}
\centering
\def\a{40}
\begin{tikzpicture}[>=latex,thick]
\draw[->] (-5.1,0) -- (5.5,0) coordinate[label={$x$}];
\draw[->] (-5,-0.8) -- (-5,3.8) coordinate[label={right:$y$}];
\fill[color=darkred!20]
	(0,0) -- (1.5,0) arc (0:\a:1.5) -- cycle;
\fill[color=darkred!20]
	(0,0) -- ({180-\a}:1.5) arc ({180-\a}:180:1.5) -- cycle;
\fill[color=blue!20] (-5,-0.7) rectangle (5,0);
\draw[color=blue] (-5,0) -- (5,0);
\fill (\a:5) circle[radius=0.08];
\fill ({180-\a}:3) circle[radius=0.08];
\fill (0,0) circle[radius=0.08];
\draw ({180-\a}:3) -- (0,0) -- (\a:5);
\node at (\a:5) [right] {$P_1$};
\node at ({180-\a}:3) [left] {$P_0$};
\node at (0,0) [below] {$P_*$};
\end{tikzpicture}
\caption{Reflexionsgesetz als Resultat der weierstrass-erdmannschen
Eckenbedingung für das Funktional mit der Lagrange-Funktion
$F(x,y,y') = \sqrt{1+y^{\prime 2}}$ (zu Aufgabe~\ref{301}).
\label{buch:301:fig:reflexion}}
\end{figure}

\begin{loesung}
Seien die Punkt $P_0=(x_0,y_0)$ und $P_1=(x_1,y_1)$ gegeben mit $y_0>0$
und $y_1>0$.
Gesucht ist eine Funktion $y(x)$ mit den Randbedingungen $y(x_0)=y_0$
und $y(x_1)=y_1)$, welche das Integral
\begin{equation}
I(y)
=
\int_{x_0}^{x_1} \sqrt{1+y'(x)^2}\,dx
\label{buch:301:eqn:funktional}
\end{equation}
mit der Lagrange-Funktion $F(x,y,y') = \sqrt{1+y^{\prime 2}}$
minimiert.
Es ist bereits bekannt, dass die Euler-Lagrange-Differentialgleichung
für das Funktional~\eqref{buch:301:eqn:funktional} Geraden als
Lösungen hat.
Die kürzeste Verbindung ist daher die Gerade durch die Punkte $P_0$ und
$P_1$.
Zusätzlich wird jetzt aber verlangt, dass die Kurve durch einen Punkt
auf der $x$-Achse verläuft.
Sei $P_*=(x_*,0)$ dieser Punkt.
Die Funktion besteht dann aus zwei Geradensegmenten von $P_0$ nach
$P_*$ und von dort zum Punkt $P_1$ mit einer Ecke im Punkt $P_*$.
Die Eckenbedingung von
Satz~\ref{buch:nichtdiff:splines:satz:weierstrass-erdmann}
besagt, dass  die Funktion
\begin{align}
F(x,y(x),y'(x))
-
y'(x)
\frac{\partial F}{\partial y'}
(x,y(x),y'(x))
&=
\sqrt{1+y'(x)^2}
-
\frac{y'(x)^2}{\sqrt{1+y'(x)^2}}
\notag
\\
&=
\frac{1+y'(x)^2-y'(x)^2}{\sqrt{1+y'(x)^2}}
=
\frac{1}{\sqrt{1+y'(x)^2}}
\label{buch:301:eqn:eckenbedingung}
\end{align}
eine stetige Funktion sein muss.
Dies ist nur möglich, wenn die Steigungen $y'(x_*-)$ und $y'(x_*+)$ dem
Ausdruck~\eqref{buch:301:eqn:eckenbedingung} den gleichen Wert geben.
Dies ist nur dann möglich, wenn $y'(x_*-)^2=y'(x_*+)^2$ ist, die
beiden Steigungen müssen also den gleichen Betrag haben.
Dies ist das Reflexionsgesetz.

Man beachte, dass die zweite Eckenbedingung, die Stetigkeit von
$\partial F/\partial y'$ nicht erfüllt sein muss, weil die Grenzlinie,
entlang der $x_*$ varieren kann, keine vertikale Komponente hat.
\end{loesung}



