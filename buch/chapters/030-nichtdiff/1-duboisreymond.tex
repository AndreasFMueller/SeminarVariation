%
% 1-duboisreymond.tex
%
% (c) 2024 Prof Dr Andreas Müller
%
\section{Das Lemma von du~Bois-Reymond
\label{buch:nichtdiff:section:duboisreymond}}
\kopfrechts{Das Lemma von du~Bois-Reymond}
Wir gehen wieder von einem Variationsproblem der Form
\begin{equation}
\delta
\int_{x_1}^{x_2}
F(x,y(x),y'(x))\,dx
=
0
\label{buch:nichtdiff:duboisreymond:eqn:problem}
\end{equation}
mit festen Randwerten $y(x_1)=y_0$ und $y(x_2)=y_1$ aus.
Die Herleitung der Euler-Lagrange-Differentialgleichung für dieses
Problem hat angenommen, dass die Lösungsfunktion $y(x)$ zweimal
stetig differenzierbar ist.
Paul du~Bois-Reymond%
\footnote{Paul du~Bois-Reymond, 1831--1889, deutscher
Mathematiker, studierte erst in Zürich Medizin, dann in Mathematik
in Königsberg und an der Universität Berlin.
Er promovierte bei Ernst Kummer und arbeitete anschliessend als
Gymnasiallehrer.
Er habilitierte sich 1865 in Heidelberg und 1869 als Professor nach
Freiburg im  Breisgau berufen.
Ab 1874 wirkte er in Tübingen und ab 1884 an der
Technischen Hochschule Berlin.}
hat darauf hingewiesen, dass dadurch die Menge der akzeptablen
Lösungsfunktionen unnötig eingeschränkt wird.
Die folgende Rechnung und das weiter unten formulierte 
Lemma~\ref{buch:nichtdiff:duboisreymond:lemma:duboisreymond}
zeigen, dass die Euler-Lagrange-Gleichung auch gilt, wenn $y(x)$
als nur einmal stetig differenzierbar angenommen wird.

Die erste Variation ist die Ableitung
\[
\frac{d}I(y+\varepsilon\eta)\bigg|_{\varepsilon=0}
=
\frac{d}{d\varepsilon}
\int_{x_1}^{x_2}
F\bigl(x,y(x)+\varepsilon\eta(x), y'(x)+\varepsilon\eta'(x)\bigr)\,dx
\bigg|_{\varepsilon}
\]
für beliebige Variationen $\eta(x)$ der Funktion $y(x)$.
Wie früher nehmen wir an, dass $F$ nach $y$ und $y'$ differenzierbar
ist, und verwenden dies, um die Ableitung zu berechnen:
\begin{equation}
\frac{d}{d\varepsilon}I(y+\varepsilon\eta)
=
\int_{x_1}^{x_2}
\frac{\partial F}{\partial y}\bigl(x,y(x),y'(x)\bigr)\cdot \eta(x)
+
\frac{\partial F}{\partial y'}\bigl(x,y(x),y'(x)\bigr)\cdot \eta'(x)
\,dx.
\label{nichtdiff:duboisreymon:eqn:ableitung}
\end{equation}
Bei der Herleitung der Euler-Lagrange-Gleichung in
Abschnitt~\ref{buch:variation:section:eulerlagrange}
haben wir uns nicht gescheut, den zweiten Term im Integral
partiell zu integrieren, und aus dem Faktor $\eta'(x)$ einen
beiden Termen gemeinsamen Faktor $\eta(x)$ zu machen.
Falls jedoch $y(x)$ nicht differenzierbar ist, oder wenn $F$ nur
einmal stetig differenzierbar ist, ist dies nicht zulässig.

Es gibt aber keine solchen Einschränkungen für die Funktionen $\eta(x)$.
Wir können uns bei der Wahl von $\eta(x)$ auf beliebig oft differenzierbare
Funktionen einschränken.
Wir versuchen daher den ersten Term
in~\eqref{nichtdiff:duboisreymon:eqn:ableitung}
partiell zu integrieren.
Die dazu benötigte Stammfunktion des ersten Terms schreiben wir
\[
\int_{x_1}^x
\frac{\partial F}{\partial y}\bigl(\xi, y(\xi), y'(\xi)\bigr)\,d\xi.
\]
Das erste Integral in~\eqref{nichtdiff:duboisreymon:eqn:ableitung}
wird damit zu
\begin{align*}
\int_{x_1}^{x_2}
\frac{\partial F}{\partial y}\bigl(x,y(x),y'(x)\bigr)\,\eta(x)\,dx
&=
\biggl[
\int_{x_1}^{x}
\frac{\partial F}{\partial y}\bigl(\xi,y(\xi),y'(\xi)\bigr)
\,d\xi
\cdot 
\eta(x)
\biggr]_{x_1}^{x_2}
\\
&\qquad
-
\int_{x_1}^{x_2}
\int_{x_1}^{x}
\frac{\partial F}{\partial y}\bigl(\xi,y(\xi),y'(\xi)\bigr)
\,d\xi
\cdot\eta'(x)
\,dx.
\end{align*}
Da in den Randpunkten $\eta(x_1)=\eta(x_2)=0$ gilt, 
fällt der erste Term weg.
Einsetzen in~\eqref{nichtdiff:duboisreymon:eqn:ableitung}
liefert jetzt die Bedingung
\begin{equation}
\int_{x_1}^{x_2}
\biggl(
-\int_{x_1}^x\frac{\partial F}{\partial y}\bigl(\xi, y(\xi),y'(\xi)\bigr)
\,d\xi
+
\frac{\partial F}{\partial y'}\bigl(x,y(x),y'(x)\bigr)
\biggr)
\cdot
\eta'(x)\,dx
=
0
\label{buch:nichtdiff:duboisreymond:eqn:integralgleichung}
\end{equation}
für alle einmal stetig differenzierbaren Funktionen $\eta(x)$, die
an den Endpunkten des Intervalls verschwinden.
Daraus können wir aber nicht schliessen, dass die Klammer im
Integral verschwindet, denn dies folgt nach dem Fundamentallemma
nur, wenn man mit beliebige Funktionen $\eta(x)$ multipliziert.
Die Funktionen $\eta'(x)$ haben die einschränkende Eigenschaft,
dass ihr Integral zwischen $x_1$ und $x_2$
\[
\int_{x_1}^{x_2}
\eta'(x)\,dx
=
\biggl[
\eta(x)
\biggr]_{x_1}^{x_2}
=
\eta(x_2) - \eta(x_1)
=
0-0
=
0
\]
ist.
Für beliebige Funktionen $\eta(x)$ kann das Integral zwischen $x_1$
und $x_2$ bliebige Werte annehmen, selbst wenn $\eta(x_1)=\eta(x_2)=0$
sind.
Es gilt aber die folgende Verallgemeinerung des Fundamentallemmas.

\begin{lemma}[du Bois-Reymond\footnote{Paul du~Bois-Reymond, 1831--1889,
deutscher Mathematiker, studierte erst in Zürich Medizin, dann in Mathematik
in Königsberg und an der Universität Berlin.
Er promovierte bei Ernst Kummer und arbeitete anschliessend als
Gymnasiallehrer.
Er habilitierte sich 1865 in Heidelber und 1869 als Professor nach
Freiburg im  Breisgau berufen.
Ab 1874 wirkte er in Tübingen und ab 1884 an der
Technischen Hochschule Berlin.}]
\label{buch:nichtdiff:duboisreymond:lemma:duboisreymond}
Sei $f(x)$ eine stetige Funktion auf dem Intervall $[x_1,x_2]$, für die
\[
\int_{x_1}^{x_2}
f(x)\,\eta'(x)\,dx
=
0
\]
gilt für alle einmal stetig differenzierbaren Funktionen auf $[x_1,x_2]$,
die in den Intervallenden verschwinden.
Dann ist die Funktion $f(x)$ konstant.
\end{lemma}

%
% dbr.tex -- Beweis des Lemmas von du Bois-Reymond
%
% (c) 2024 Prof Dr Andreas Müller, OST Ostschweizer Fachhochschule
%
\documentclass[tikz]{standalone}
\usepackage{amsmath}
\usepackage{times}
\usepackage{txfonts}
\usepackage{pgfplots}
\usepackage{csvsimple}
\definecolor{darkgreen}{rgb}{0,0.6,0}
\definecolor{darkred}{rgb}{0.8,0,0}
\usetikzlibrary{arrows,intersections,math,calc}
\begin{document}
\def\skala{1}
\begin{tikzpicture}[>=latex,thick,scale=\skala,
declare function={ f(\x) = 2*cos(10*\x-20)+0.5*sin(50*\x)+1; },
]

\def\a{2}
\def\b{8}
\def\e{0.8}
\def\h{3}
\def\d{0.45}

\fill[color=orange!20] ({\a-\e},{f(\a)-\d}) rectangle ({\a+\e},{f(\a)+\d});
\fill[color=orange!20] ({\b-\e},{f(\b)-\d}) rectangle ({\b+\e},{f(\b)+\d});
\fill[color=blue!20]
	plot[domain=-180:180] ({\a-\e*\x/180},{\h*(1+cos(\x))/2}) -- cycle;
\fill[color=darkred!20]
	plot[domain=-180:180] ({\b-\e*\x/180},{-\h*(1+cos(\x))/2}) -- cycle;

\draw[color=orange,line width=1.4pt] ({\a-\e},{f(\a)}) -- ({\a+\e},{f(\a)});
\draw[color=orange,line width=1.4pt] ({\b-\e},{f(\b)}) -- ({\b+\e},{f(\b)});
\draw[color=orange,line width=0.4pt]
	({\a-\e},0) -- ({\a-\e},{f(\a)}) -- ({\a+\e},{f(\a)}) -- ({\a+\e},0);
\draw[color=orange,line width=0.4pt]
	({\b-\e},0) -- ({\b-\e},{f(\b)}) -- ({\b+\e},{f(\b)}) -- ({\b+\e},0);

\draw[color=darkgreen,line width=1.4pt]
	plot[domain=0:11,samples=100] ({\x},{f(\x)});
\draw[color=darkgreen,line width=0.3pt] (\a,0) -- (\a,{f(\a)});
\draw[color=darkgreen,line width=0.3pt] (\b,0) -- (\b,{f(\b)});
\fill[color=darkgreen] (\a,{f(\a)}) circle[radius=0.08];
\fill[color=darkgreen] (\b,{f(\b)}) circle[radius=0.08];
\node[color=darkgreen] at (\a,{f(\a)}) [above] {$f(a)$};
\node[color=darkgreen] at (\b,{f(\b)}) [above] {$f(b)$};

\node[color=orange] at ({\a+\e},{f(\a)+\d}) [right] {$f(a)+\delta\mathstrut$};
\node[color=orange] at ({\a+\e},{f(\a)-\d}) [right] {$f(a)-\delta\mathstrut$};
\node[color=orange] at ({\b+\e},{f(\b)+\d}) [right] {$f(b)+\delta\mathstrut$};
\node[color=orange] at ({\b+\e},{f(\b)-\d}) [right] {$f(b)-\delta\mathstrut$};

\coordinate (A) at (6,3.9);

\node[color=orange] at (A) {$\tilde{f}(x)$};
\draw[color=orange,line width=0.3pt,shorten >= 0.3cm]
	({\a+0.8*\e},{f(\a)}) -- (A);
\draw[color=orange,line width=0.3pt,shorten >= 0.3cm]
	({\b-0.8*\e},{f(\b)}) -- (A);

\node[color=darkgreen] at (5,{f(5)}) [below] {$f(x)$};

\draw[color=blue,line width=1.1pt]
	plot[domain=-180:180] ({\a-\e*\x/180},{\h*(1+cos(\x))/2});

\draw[color=darkred,line width=1.1pt]
	plot[domain=-180:180] ({\b-\e*\x/180},{-\h*(1+cos(\x))/2});

\draw[->] (-0.1,0) -- (11.5,0) coordinate[label={$x$}];
\draw[->] (0,-3.2) -- (0,4.3) coordinate[label={left:$y$}];

\draw (\a,-0.05) -- (\a,0.05);
\node at (\a,0) [below] {$a\mathstrut$};
\draw (\b,-0.05) -- (\b,0.05);
\node at (\b,0) [below] {$b\mathstrut$};

\draw ({\a-\e},-0.05) -- ({\a-\e},0.05);
\node at ({\a-\e},0) [below left] {$a-\varepsilon\mathstrut$};
\draw ({\a+\e},-0.05) -- ({\a+\e},0.05);
\node at ({\a+\e},0) [below right] {$a+\varepsilon\mathstrut$};

\node[color=blue] at (\a,{0.3*\h}) {$g_a(x)$};
\node[color=darkred] at (\b,{-0.3*\h}) {$-g_b(x)$};

\draw ({\b-\e},-0.05) -- ({\b-\e},0.05);
\node at ({\b-\e},0) [below left] {$b-\varepsilon\mathstrut$};
\draw ({\b+\e},-0.05) -- ({\b+\e},0.05);
\node at ({\b+\e},0) [below right] {$b+\varepsilon\mathstrut$};

\end{tikzpicture}
\end{document}



\begin{proof}
Wir müssen zeigen, dass die Funktionswerte von $f$ in den Punkten $a$ und $b$
im Inneren des Intervalls $[x_1,x_2]$ gleich gross sind.
Sei $\varepsilon>0$ so klein, dass die Intervalle
$(a-\varepsilon,a+\varepsilon)$ und $(b-\varepsilon,b+\varepsilon)$ ebenfalls
vollständig in $[a,b]$ enthalten sind.
Nach Satz~\ref{buch:variation:satz:gabeins} gibt es zwei nichtnegative
Funktionen $g_a$ und $g_b$ mit
\[
\operatorname{supp} g_a = (a-\varepsilon,a+\varepsilon)
\qquad\text{und}\qquad
\operatorname{supp} g_b = (b-\varepsilon,b+\varepsilon),
\]
deren Integral ausserdem den Wert $1$ haben.
Wir wählen für $\eta$ die Stammfunktion
\[
\eta(x)
=
\int_{x_1}^x g_a(\xi)-g_b(\xi)\,d\xi
\]
von $x\mapsto g_a(x) -g_b(x)$.
Am linken Rand $x_1$ verschwindet sie, am rechten Rand gilt
\[
\eta(x_2)
=
\int_{x_1}^{x_2}
g_a(\xi)-g_b(\xi)
\,d\xi
=
\int_{x_1}^{x_2}
g_a(\xi)
\,d\xi
-
\int_{x_1}^{x_2}
g_b(\xi)
\,d\xi
=
1-1
=
0.
\]
Die Ableitung $\eta'(x)=g_a(x)-g_b(x)$ ist somit eine Funktion,
auf die die Voraussetzungen des Satzes zutreffend sind.
Daher gilt $\langle f,\eta'\rangle=0$.

Da die Funktion $f(x)$ stetig ist, können wir zu jedem beliebigen 
$\delta>0$ das $\varepsilon>0$ so klein wählen, dass 
$|f(x)-f(a)|<\delta$ für $|x-a|<\varepsilon$ und
$|f(x)-f(b)|<\delta$ für $|x-b|<\varepsilon$.
Wir approximieren die $f$ durch die Funktion
\[
\tilde{f}(x)
=
\begin{cases}
f(a)&\qquad x\in[a-\varepsilon,a+\varepsilon]\\
f(b)&\qquad x\in[b-\varepsilon,b+\varepsilon]\\
f(x)&\qquad\text{sonst}.
\end{cases}
\]
Die Funktion $\tilde{f}$ ist immer noch integrierbar und es gilt
\[
|f(x)-\tilde{f}(x)| < \delta.
\]
Für das Integral $\langle f-\tilde{f}, \eta'\rangle$ folgt jetzt
\begin{align*}
|\langle f-\tilde{f},\eta'\rangle|
&=
\biggl|
\int_{x_1}^{x_2}
(f(x)-\tilde{f}(x))\eta'(x)
\,dx
\biggr|
\\
&\le
\int_{x_1}^{x_2}
|f(x)-\tilde{f}(x)| \cdot |\eta'(x)|
\,dx
\\
&\le 
\delta
\,
\biggl(
\int_{x_1}^{x_2} g_a(x)\,dx
+
\int_{x_1}^{x_2} g_b(x)\,dx
\biggr)
=
2\delta.
\end{align*}
Die beiden Integrale $\langle f,\eta'\rangle$ und 
$\langle \tilde{f},\eta'\rangle$ unterscheiden sich höchstens um
\begin{equation}
|\langle f,\eta'\rangle - \langle \tilde{f},\eta'\rangle|
=
|\langle f-\tilde{f},\eta' \rangle|
\le
2\delta,
\label{buch:nichtdiff:duboisreymond:eqn:ftildef}
\end{equation}
sie sind also beliebig nahe.

Die Funktion $\tilde{f}$ ist auf den beiden Teilintervallen
$(a-\varepsilon,a+\varepsilon)$ und $(b-\varepsilon,b+\varepsilon)$
konstant, das Integral kann daher sofort berechnet werden:
\begin{align*}
\langle \tilde{f},g_a\rangle
&=
\int_{a-\varepsilon}^{a+\varepsilon}
f(a) g_a(x)
\,dx
=
f(a)
&&\text{und}&
\langle \tilde{f},g_b\rangle
&=
\int_{b-\varepsilon}^{b+\varepsilon}
f(b) g_b(x)
\,dx
=
f(b),
\end{align*}
und damit
\[
\langle \tilde{f},\eta'\rangle
=
\langle \tilde{f},g_a-g_b\rangle
=
f(a)-f(b).
\]
Nach Voraussetzung ist $\langle f,\eta'\rangle=0$ und damit ist
wegen \eqref{buch:nichtdiff:duboisreymond:eqn:ftildef}
\[
|f(a)-f(b)|
=
|\langle\tilde{f},\eta'\rangle|
=
|
\underbrace{\langle f,\eta'\rangle}_{\displaystyle =0}
-
\langle \tilde{f},\eta'\rangle
|
=
|\langle f-\tilde{f},\eta'\rangle|
\le
2\delta
\]
für jedes beliebige $\delta > 0$.
Dies ist nur möglich, wenn $f(a)-f(b)=0$ ist oder $f(a)=f(b)$.
Somit hat $f$ überall im Intervall den gleichen Wert.
\end{proof}

Mit dem Lemma~\ref{buch:nichtdiff:duboisreymond:lemma:duboisreymond}
von du~Bois-Reymond kann man jetzt schliessen, dass der Integrand in
\eqref{buch:nichtdiff:duboisreymond:eqn:integralgleichung}
konstant sein muss.
Es gilt also
\begin{equation}
\int_{x_1}^x\frac{\partial F}{\partial y}\bigl(\xi, y(\xi),y'(\xi)\bigr)
\,d\xi
=
\frac{\partial F}{\partial y'}\bigl(x,y(x),y'(x)\bigr)
+
C
\label{buch:nichtdiff:duboisreymond:gleichung}
\end{equation}
Da die Funktionen $y(\xi)$ und $y'(\xi)$ stetig sind und auch
die Ableitung $\partial F/\partial y$ als stetig in allen Argumenten
vorausgesetzt ist, ist der Integrand auf der linken Seite eine
stetige Funktion von $\xi$.
Nach dem Hauptsatz der Infinitesimalrechnung ist das Integral
stetig differenzierbar und die Ableitung ist der Integrand.
Daher muss auch die rechte Seite stetig differenzierbar sein,
es ist also zulässig den Ableitungsoperator $d/dx$ darauf anzuwenden,
was die Gleichung
\begin{equation}
\frac{\partial F}{\partial y}\bigl(x,y(x),y'(x)\bigr)
=
\frac{d}{dx}
\frac{\partial F}{\partial y'}\bigl(x,y(x),y'(x)\bigr)
\label{buch:nichtdiff:duboisreymond:eqn:eulerlagrange}
\end{equation}
ergibt.
Die Gleichung
\eqref{buch:nichtdiff:duboisreymond:eqn:eulerlagrange}
ist aber wieder die Euler-Lgrange-Gleichung.
Die Voraussetzung, dass $y(x)$ zweimal stetig differenzierbar sein
muss, war also gar nicht notwendig für die Gültigkeit der
Euler-Lagrange-Gleichung.
Sie folgt auch, wenn man nur voraussetzt, dass $y(x)$ einmal stetig
differenzierbar ist.

Aus der Euler-Lagrange-Differentialgleichung folgt noch nicht, dass
die zweite Ableitung $y''(x)$ der Lösung stetig ist.
David Hilbert hat aber den folgenden Satz bewiesen.

\begin{satz}
Ist $\frac{\partial F}{\partial y'}$ differenzierbar und ist $y(x)$ eine
Lösung des Variationsproblems~\eqref{buch:nichtdiff:duboisreymond:eqn:problem},
dann ist $y''(x)$ überall dort definiert, wo
\begin{equation}
\frac{\partial^2 F}{\partial y^{\prime 2}}\bigl(x,y(x),y'(x)\bigr) \ne 0
\label{buch:nichtdiff:duboysreymond:eqn:positiv}
\end{equation}
gilt.
\end{satz}

\begin{proof}
Nach Voraussetzung ist die Ableitung von $F$ nach $y'$ differenzierbar.
Setzt man die Funktion $y(x)$ ein, ist nach 
\eqref{buch:nichtdiff:duboisreymond:gleichung}
die Funktion
\begin{equation}
x\mapsto
\frac{\partial F}{\partial y'}\bigl(x,y(x),y'(x)\bigr)
\label{buch:nichtdiff:eqn:hilbertf}
\end{equation}
stetig differenzierbar.
Zur Berechnung der Ableitung von \eqref{buch:nichtdiff:eqn:hilbertf}
betrachten wir die Differenz
\begin{align*}
\Delta F
&=
\frac{\partial F}{\partial y'}\bigl(x+h,y(x+h),y'(x+h)\bigr)
-
\frac{\partial F}{\partial y'}\bigl(x,y(x),y'(x)\bigr).
\end{align*}
Die Ableitung von \eqref{buch:nichtdiff:eqn:hilbertf} ist
nach \eqref{buch:nichtdiff:duboisreymond:gleichung}
\begin{align*}
\frac{d}{dx}
\frac{\partial F}{\partial y'}\bigl(x,y(x),y'(x)\bigr)
&=
\lim_{h\to 0}
\frac{\Delta F}{h}
=
\frac{\partial F}{\partial y}\bigl(x,y(x),y'(x)\bigr).
\end{align*}
Wir berechnen den Grenzwert auf eine andere Art.
Da $\partial F/\partial y'$ differenzierbar ist, kann man die Differenz
$\Delta F$ als
\begin{align*}
\Delta F
=
\frac{\partial^2 F}{\partial y'\partial x}\bigl(x,y(x),y'(x)\bigr) h
&+
\frac{\partial^2 F}{\partial y'\partial y}\bigl(x,y(x),y'(x)\bigr)
\cdot
(y(x+h)-y(x))
\\
&+
\frac{\partial^2 F}{\partial y^{\prime 2}}\bigl(x,y(x),y'(x)\bigr)
\cdot
(y'(x+h)+y'(x))
+
o(h)
\end{align*}
schreiben.
Teilen duch $h$ ergibt
\begin{align}
\frac{\Delta F}{h}
=
\frac{\partial^2 F}{\partial y'\partial x}\bigl(x,y(x),y'(x)\bigr)
&+
\frac{\partial^2 F}{\partial y'\partial y}\bigl(x,y(x),y'(x)\bigr)
\cdot
\frac{y(x+h)-y(x)}{h}
\label{buch:nichtdiff:eqn:hilbertlim1}
\\
&+
\frac{\partial^2 F}{\partial y^{\prime 2}}\bigl(x,y(x),y'(x)\bigr)
\cdot
\frac{y'(x+h)-y'(x)}{h}
+
\frac{o(h)}{h}.
\label{buch:nichtdiff:eqn:hilbertlim2}
\end{align}
Die linke Seite konvergiert gegen die Ableitung von
\eqref{buch:nichtdiff:eqn:hilbertf}, somit muss auch die rechte Seite
konvergieren.

Die Voraussetzung \eqref{buch:nichtdiff:duboysreymond:eqn:positiv}
garantiert, dass nach dem Differenzenquotienten
in \eqref{buch:nichtdiff:eqn:hilbertlim2} aufgelöst werden kann.
\begin{align*}
\frac{y'(x+h)-y'(x)}{h}
&=
\frac{1}{\displaystyle
\frac{\partial^2 F}{\partial y^{\prime 2}}\bigl(x,y(x),y'(x)\bigr)
}
\biggl(
\frac{\partial F}{\partial y}\bigl(x,y(x),y'(x)\bigr)
-
\frac{\partial^2 F}{\partial y'\partial x}\bigl(x,y(x),y'(x)\bigr)
\\
&\qquad\qquad
-
\frac{\partial F}{\partial y'}\bigl(x,y(x),y'(x)\bigr)
\cdot
\frac{y(x+h)-y(x)}{h}
-
\frac{o(h)}{h}
\biggr).
\intertext{Der Differenzenquotient für $y(x)$ konvergiert gegen
$y'(x)$ und $o(h)/h$ konvergiert gegen $0$.
Somit konvergiert auch die linke Seite, sie ist die zweite Ableitung
}
y''(x)
&=
\frac{1}{\displaystyle
\frac{\partial^2 F}{\partial y^{\prime 2}}\bigl(x,y(x),y'(x)\bigr)
}
\biggl(
\frac{\partial F}{\partial y}\bigl(x,y(x),y'(x)\bigr)
-
\frac{\partial^2 F}{\partial y'\partial x}\bigl(x,y(x),y'(x)\bigr)
\\
&\qquad\qquad
-
\frac{\partial F}{\partial y'}\bigl(x,y(x),y'(x)\bigr)\cdot y'(x)
\biggr)
\end{align*}
von $y(x)$.
Damit ist der Satz bewiesen.
\end{proof}

In der Theorie der zweiten Variation von Kapitel~\ref{buch:chapter:variation2}
wird die nach Legendre notwendige Bedingung
\[
\frac{\partial^2 F}{\partial y^{\prime 2}}\bigl(x,y(x),y'(x)\bigr) \ne 0
\ge 0
\]
für ein Extremum hergeleitet.
Falls in dieser Bedingung sogar das strikte Zeichen gilt, kann man
somit folgern, dass das Maximum sogar eine zweimal stetig differenzierbare
Funktion ist.




