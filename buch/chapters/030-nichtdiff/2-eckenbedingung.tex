%
% 2-eckenbedingung.tex
%
% (c) 2024 Prof Dr Andreas Müller
%
\section{Die weierstrass-erdmannsche Eckenbedingung
\label{buch:nichtdiff:section:ecken}}
\kopfrechts{Die weierstrass-erdmannsche Eckenbedingung}
Die Lösungen der Euler-Lagrange-Differentialgleichungen sind überall
differenzierbar, aber nicht jedes Variationsproblem hat überhaupt
eine differenzierbare Lösung.
Nicht allzu pathologische Variationsprobleme werden aber mindestens
stückweise differenzierbare Lösungen haben.
Zwischen endlichen vielen Punkten, wo die Ableitung der Lösungsfunktion
Sprungstellen haben kann, erfüllt die Lösung die
Euler-Lagrange-Differentialgleichung.
Die Punkte, in denen die Lösungskurve ``geknickt'' ist oder ``Ecken''
hat, sind aber nicht beliebig.
Die in Abschnitt~\ref{buch:nichtdiff:eckenbedingung:subsection:eckenbedingung}
wird die weierstrass-erdmannsche Eckenbedingung hergeleitet, die
den Steigungen an den Knickstellen Einschränkungen auferlegt.

%
% Ein Variationsproblem mit nicht differenzierbarer Lösung
%
\subsection{Ein Variationsproblem mit nicht differenzierbarer Lösung}

%
% Die Eckenbedingung
%
\subsection{Die Eckenbedingung
\label{buch:nichtdiff:eckenbedingung:subsection:eckenbedingung}}
Wir betrachten eine Lösung $y(x)$ eines Variationsproblems 
\[
\delta
\int_{x_0}^{x_1}
F(x,y(x),y'(x))
\,dx
=
0
\]
auf dem Intervall $[x_0,x_1]$ mit vorgegebenen Randwerten 
$y(x_0)=y_0$ und $y(x_1)=y_1$.
Wir verlangen nicht, dass die Funktion $y(x)$ überall differenzierbar
ist, der Einfachheit halber nehmen wir aber an, dass es nur eine
Stelle $x_*$ gibt, an der die Funktion $y(x)$ nicht differenzierbar ist.
Auf den Teilintervallen $[x_0,x_*]$ und $[x_*,x_1]$ ist $y(x)$ eine
Lösung der Euler-Lagrange-Differentialgleichung.
Es bleibt also nur noch der Punkt $(x_*,y_*)$ gewählt werden, so dass
die Variation verschwindet.

Die Lösung $y(x)$ setzt sich zusammen aus den Lösungen $y|_{[x_0,x_*]}$
und $y|_{[x_*,x_1]}$ mit den Randbedingungen
\begin{equation*}
\begin{aligned}
y(x_0)&=y_0\\
y(x_*)&=y_*
\end{aligned}
\qquad\text{und}\qquad
\begin{aligned}
y(x_*)&=y_*\\
y(x_1)&=y_1.
\end{aligned}
\end{equation*}
Auch das Funktional $I(y)$ setzt sich aus zwei Teilen
\[
I(y)
=
\underbrace{
\int_{x_0}^{x_*}
F(x,y(x),y'(x))
\,dx
}_{\displaystyle I_0}
+
\underbrace{
\int_{x_*}^{x_1}
F(x,y(x),y'(x))
\,dx
}_{\displaystyle I_1}
\]
zusammen.
Die Variation von $\delta I(y)=\delta I_0+\delta I_1$ ist die Summe 
der Variationen der Teilfunktionale.
Der gemeinsame Endpunkt $(x_*,y_*)$ ist frei beweglich,
die Variation der Teile muss diese Schwankung berücksichtigen.
Sei $\vec{r}$ der Richtungsvektor der Bewegung des Punktes während
einer Variation der Funktion $y(x)$.
Da der Punkt den beiden Teilen gemeinsam ist, muss $\vec{r}$
mit dem Richtungsvektor $\vec{r}_1(0)$ der zugehörigen Variation
der Lösung auf dem linken Teil des Intervalls und dem
Richtungsvektor $\vec{r}_0(0)$ der Variation der Lösung dem rechtn Teil
des Intervalls übereinstimmen.

Nach Satz~\ref{buch:variation:allgemein:satz:allgemeinvariation1} 
lässt sich die Variation der beiden Teile durch 
\begin{align*}
\delta I(y)
&=
\vec{f}_1\cdot \vec{r}
+
\int_{x_0}^{x_*}
\biggl(
\frac{\partial F}{\partial y}(x,y(x),y'(x))
-
\frac{d}{dx}
\frac{\partial F}{\partial y'}(x,y(x),y'(x))
\biggr)
\frac{\partial y}{\partial\varepsilon}(x,0)\,dx
\,dx
\\
&\qquad
-
\vec{f}_0\cdot \vec{r}
+
\int_{x_*}^{x_0}
\biggl(
\frac{\partial F}{\partial y}(x,y(x),y'(x))
-
\frac{d}{dx}
\frac{\partial F}{\partial y'}(x,y(x),y'(x))
\biggr)
\frac{\partial y}{\partial\varepsilon}(x,0)\,dx
\,dx
\end{align*}
ausdrücken.
Da in den Teilintervallen die Lösung die Euler-Lagrange-Differentialgleichung
erfüllt, fallen die Integrale weg.
Die verbleibenden Terme sind
\begin{align*}
\delta I(y)
&=
(\vec{f}_1-\vec{f}_0)\cdot \vec{r}.
\end{align*}
Da dies für alle Vektoren $\vec{r}$ in der $x$-$y$-Ebene verschwinden muss,
folgt
\begin{equation}
\vec{f}_1-\vec{f}_0
=
0
\qquad\Rightarrow\qquad
\vec{f}_1=\vec{f}_2.
\label{buch:nichtdiff:eckenbedingung:eqn:e2}
\end{equation}
Darin gehört der $\vec{f}_1$ zur Lösung auf dem Teilintervall $[x_0,x_*]$,
wird also mit dem Funktionswert $y(x_*)$ und dem linksseitigen Grenzwert
$y'(x_*-)$ der Ableitung berechnet, während $\vec{f}_0$ zum Teilintervall
$[x_*,x_1]$ gehört und mit $y(x_*)$ und dem rechtsseitigen Grenzwert
$y'(x_*+)$ berechnet wird.
Da $\vec{f}_i$ Vektoren sind, 
Ausgeschrieben lauten die Bedingungen, entsprechen der Gleichung
\eqref{buch:nichtdiff:eckenbedingung:eqn:e2}
die zwei Komponentengleichungen
\begin{align*}
F(x_*,y_*,y'(x_*-))
- 
\frac{\partial F}{\partial y'}\bigl(x_*,y_*, y'(x_*-)\bigr)
\frac{\partial y}{\partial\varepsilon}(x,0)
&=
F(x_*,y_*,y'(x_*-))
- 
\frac{\partial F}{\partial y'}\bigl(x_*,y_*, y'(x_*-)\bigr)
\frac{\partial y}{\partial\varepsilon}(x,0)
\\
\frac{\partial F}{\partial y'}\bigl(x_*,y_*, y'(x_*-)\bigr)
&=
\frac{\partial F}{\partial y'}\bigl(x_*,y_*, y'(x_*+)\bigr)
\end{align*}
Die zweite Bedingung bedeutet, dass die Ableitungen nach $y'$ auch
in der ersten Bedingung gleich sind.
Die erste Bedingung vereinfacht sich daher zu
\[
F(x_*,y_*,y'(x_*-))
=
F(x_*,y_*,y'(x_*-)).
\]
Die links- und rechtsseitigen Steigungen in Punkt $(x_*,y_*)$ sind
also nicht beliebig, sie müssen zwei zusätzliche Bedingungen erfüllen.
Für den Fall $y'(x_*+)=y'(x_*-)$ sind beide Bedingungen erfüllt.
Diese Bedinung heisst auch die weierstrass-erdmansche Eckenbedingung.

\begin{satz}[Weierstrass-Erdmann]
Sei $y(x)$ eine Funktion auf dem Intervall $[x_0,x_1]$ mit Randwerten
$y(x_0)=y_0$ und $y(x_1)=y_1$, die mit Ausnahme von endlich
vielen Stellen $\overline{x}_k$ zweimal
stetig differenzierbar ist und das Funktional
\[
I(y)
=
\int_{x_0}^{x_1}
F(x,y(x),y'(x))
\,dx
\]
stationär macht.
Dann erfüllen rechts- und linksseitigen Steigungen
\[
y'(\overline{x}_k-) = \lim_{x\to \overline{x}_k-} y'(x)
\qquad\text{und}\qquad
y'(\overline{x}_k+) = \lim_{x\to \overline{x}_k+} y'(x)
\]
der Funktion in jedem Punkt $\overline{x}_k$ die Bedingung
\begin{equation}
\begin{aligned}
F\bigl(
\overline{x}_k,
y(\overline{x}_k),
y'(\overline{x}_k-)
\bigr)
&=
F\bigl(
\overline{x}_k,
y(\overline{x}_k),
y'(\overline{x}_k+)
\bigr)
\\
\frac{\partial F}{\partial y'}\bigl(
\overline{x}_k,
y(\overline{x}_k),
y'(\overline{x}_k-)
\bigr)
&=
\frac{\partial F}{\partial y'}\bigl(
\overline{x}_k,
y(\overline{x}_k),
y'(\overline{x}_k+)
\bigr)
\end{aligned}
\label{buch:nichtdiff:eckenedingung:eqn:we}
\end{equation}
\end{satz}

Ein interessanter Spezialfall ist, wenn der Sprung der Steigung
an der Stelle $x_*$ durch eine Unstetigkeit der Funktion $f(x,y,y')$
an dieser Stelle verursacht wird.
In diesem Fall ist klar, dass die Ecke $(x_*,y_*)$ gar nicht in
$x$-Richtung bewegt werden kann, der Richtungsvektor $\vec{r}$ ist
vertikal.
Damit folgt aus $(\vec{f}_1-\vec{f}_0)\cdot\vec{r}$ nicht mehr,
dass die beiden Vektoren gleich sein müssen.
Es folgt nur noch, dass sie die gleich $y$-Komponenten haben,
dass also
\begin{equation}
\frac{\partial F}{\partial y'}\bigl(x_*-,y(x_*),y'(x_*-)\bigr)
=
\frac{\partial F}{\partial y'}\bigl(x_*+,y(x_*),y'(x_*+)\bigr).
\label{buch:nichtdiff:eckenbedingung:eqn:we0}
\end{equation}
Das folgende Beispiel illustriert diesen Spezialfall anhand des
Brechungsgesetztes.

\begin{beispiel}
%
% brechung.tex -- Herleitung des Brechungsgesetztes
%
% (c) 2021 Prof Dr Andreas Müller, OST Ostschweizer Fachhochschule
%
\documentclass[tikz]{standalone}
\usepackage{amsmath}
\usepackage{times}
\usepackage{txfonts}
\usepackage{pgfplots}
\usepackage{csvsimple}
\definecolor{darkred}{rgb}{0.8,0,0}
\usetikzlibrary{arrows,intersections,math,calc}
\begin{document}
\def\skala{1}
\begin{tikzpicture}[>=latex,thick,scale=\skala]

\def\xnull{-3.5}
\def\ynull{1}
\def\xeins{4.0}
\def\yeins{5}
\def\ystern{3.5}

\pgfmathparse{atan2(\yeins-\ystern,\xeins)}
\xdef\alphaeins{\pgfmathresult}
\pgfmathparse{atan2(\ynull-\ystern,\xnull)}
\xdef\alphanull{\pgfmathresult}

\coordinate (A) at (\xnull,\ynull);
\coordinate (B) at (0,\ystern);
\coordinate (C) at (\xeins,\yeins);

\fill[color=blue!10] (-5,-0.1) rectangle (0,6.1);
\fill[color=blue!30] (0,-0.1) rectangle (5,6.1);

\fill[color=gray!40,opacity=0.5]
	(B) -- +(0:2) arc (0:\alphaeins:2) -- cycle;
\node at ($(B)+({\alphaeins/2}:1.5)$) {$\alpha(0+)$};
\fill[color=gray!40,opacity=0.5]
	(B) -- +(-180:2) arc (-180:\alphanull:2) -- cycle;
\node at ($(B)+({-180+(180+\alphanull)/2}:{1.5})$) {$\alpha(0-)$};

\draw[->] (0,-0.1) -- (0,6.5) coordinate[label={right:$y$}];
\draw[->] (-5.1,0) -- (5.5,0) coordinate[label={$x$}];

\draw[line width=0.3pt] (-3,\ystern) -- (3,\ystern);

\draw[line width=0.3pt] (\xnull,0) -- (A);
\draw (\xnull,0.05) -- (\xnull,-0.05);
\node at (\xnull,-0.05) [below] {$x_0\mathstrut$};
\draw[line width=0.3pt] (\xeins,0) -- (C);
\draw (\xeins,0.05) -- (\xeins,-0.05);
\node at (\xeins,-0.05) [below] {$x_1\mathstrut$};
\node at (0,-0.05) [below] {$0\mathstrut$};

\draw[color=darkred,line width=1.4pt] (A) -- (B) -- (C);
\fill[color=darkred] (A) circle[radius=0.08];
\node at (A) [left] {$(x_0,y_0)$};
\fill[color=darkred] (C) circle[radius=0.08];
\node at (C) [above] {$(x_1,y_1)$};

\node[color=blue] at (-4.5,5.5) {$n_-$};
\node[color=blue] at (4.5,0.5) {$n_+$};

\node[color=darkred] at (B) [below right] {$(0,y_*)$};
\fill[color=darkred] (B) circle[radius=0.08];

\end{tikzpicture}
\end{document}


Das Brechungsgesetz beschreibt die Ausbreitung eines Lichtstrahls
in homogenen Medien, die entlang einer Kurve zusammentreffen.
In der $x$-$y$-Ebene sei die Halbebene $x<0$ mit einem Medium
mit Brechungsindex $n_-$ gefüllt, während die rechte Halbebene
den Brechungsindex $n_+$ hat.
Die Lagrange-Funktion ist
\[
F(x,y,y')
=
\begin{cases}
\displaystyle
\frac{1}{n_-}\sqrt{1+y^{\prime 2}} &\qquad \text{für $x<0$}\\
\\[-8pt]
\displaystyle
\frac{1}{n_+}\sqrt{1+y^{\prime 2}} &\qquad \text{für $x>0$.}
\end{cases}
\]
Gesucht ist eine Funktion $y(x)$, die $y(x_0)=y_0$ und $y(x_1)=y_1$ 
erfüllt, wobei $x_0<0$ und $x_1>0$ ist.
In jeder Halbebene ist die Lagrange-Funktion im Wesentlichen
die Lagrange-Funktion von
Beispiel~\ref{buch:variation:eulerlagrange:beispiel:gerade}.
Dort wurde gefunden, dass die Lösungen Geraden sind.
Es folgt, dass die Lösung im vorliegenden Problem sich aus
Geraden zusammensetzen muss, nur bei $x=0$ könnte ein Stelle
vorliegen, wo die Funktion $y(x)$ nicht differenzierbar ist.
Dort muss daher die weierstrass-erdmannsche Eckenbedingung
\eqref{buch:nichtdiff:eckenbedingung:eqn:we0}
gelten.

Die Ableitung der Lagrange-Funktion nach $y'$ ist
\[
\frac{\partial F}{\partial y'}(x,y,y')
=
\frac{1}{n_\pm}
\frac{y'}{\sqrt{1+y^{\prime 2}}}.
\]
Die weierstrass-erdmannsche Eckenbedingung ist jetzt
\begin{align}
\frac{\partial F}{\partial y'}\bigl(0-,y_*,y'(0-)\bigr)
&=
\frac{\partial F}{\partial y'}\bigl(0+,y_*,y'(0+)\bigr)
\notag
\intertext{oder nach Einsetzen der Ableitung}
\frac{1}{n_-}
\frac{y'(0-)}{\sqrt{1+y'(0-)^2}}
&=
\frac{1}{n_+}
\frac{y'(0+)}{\sqrt{1+y'(0+)^2}}
\label{buch:nichtdiff:eckenbedingung:eqn:steigung}
\end{align}
Die Steigung $y'(x)$ ist der Tangens des Winkels $\alpha(x)$ zwischen der 
Horizontalen und dem Lichtstrahl.
Der Bruch in \eqref{buch:nichtdiff:eckenbedingung:eqn:steigung}
ist der Sinus
\[
\sin\alpha(x)
=
\frac{y'(x)}{\sqrt{1+y'(x)^2}}
\]
dieses Winkels.
Die Bedingung~\eqref{buch:nichtdiff:eckenbedingung:eqn:steigung}
wird damit zu
\[
\frac{1}{n_-}\sin\alpha(0-)
=
\frac{1}{n_+}\sin\alpha(0+)
\qquad\text{oder}\qquad
\frac{\sin\alpha(0-)}{\sin\alpha(0+)}
=
\frac{n_+}{n_-},
\]
dies ist genau das Brechungsgesetz von Snellius.
\end{beispiel}


