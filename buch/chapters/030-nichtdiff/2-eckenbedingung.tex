%
% 2-eckenbedingung.tex
%
% (c) 2024 Prof Dr Andreas Müller
%
\section{Weierstrass-Erdmannsche Eckenbedingung
\label{buch:nichtdiff:section:ecken}}
Die Lösungen der Euler-Lagrange-Differentialgleichungen sind überall
differenzierbar, aber nicht jedes Variationsproblem hat überhaupt
eine differenzierbare Lösung.
Nicht allzu pathologische Variationsprobleme werden aber mindestens
stückweise differenzierbare Lösungen haben.
Zwischen endlichen vielen Punkten, wo die Ableitung der Lösungsfunktion
Sprungstellen haben kann, erfüllt die Lösung die
Euler-Lagrange-Differentialgleichung.
Die Punkte, in denen die Lösungskurve ``geknickt'' ist oder ``Ecken''
hat, sind aber nicht beliebig.
Die in Abschnitt~\ref{buch:nichtdiff:eckenbedingung:subsection:eckenbedingung}
wird die weierstrass-erdmannsche Eckenbedingung hergeleitet, die
den Steigungen an den Knickstellen Einschränkungen auferlegt.

%
% Ein Variationsproblem mit nicht differenzierbarer Lösung
%
\subsection{Ein Variationsproblem mit nicht differenzierbarer Lösung}

%
% Die Eckenbedingung
%
\subsection{Die Eckenbedingung
\label{buch:nichtdiff:eckenbedingung:subsection:eckenbedingung}}


