%
% 4-splines.tex
%
% (c) 2023 Prof Dr Andreas Müller
%
\section{Spline-Interpolation
\label{buch:nichtdiff:section:splines}}
\kopfrechts{Spline-Interpolation}
In diesem Abschnitt soll eine Anwendung diskutiert werden, die grossen
Nutzen in der numerischen Mathematik bringt.
Es unterscheidet sich von früheren Beispielen dadurch, dass die
gesuchte Funktion nicht nur an den Intervallenden festgehalten wird,
sondern auch die Werte in Zwischenpunkten vorgeschrieben sind.

%
% Problemstellung
%
\subsection{Problemstellung}
Von einer Funktion sind die Werte 
$f_i=f(x_i)$ in den Stellen $x_0=a<x_1<\dots<x_n=b$ bekannt.
Man finde eine Interpolationsfunktion $g(x)$ derart, dass
\[
f(x_i) = g(x_i)
\quad
\forall i=0,\dots,n.
\]
Ausserdem soll der Funktionsgraph von $g(x)$ möglichst wenig Krümmung haben.
Da die Krümmung eines Graphen ist im Wesentlichen abhängig von der zweiten
Ableitung, wir verlangen daher, dass das Integral
\begin{equation}
I(g)
=
\int_a^b g''(x)^2\,dx
\label{buch:nichtdiff:spline:eqn:funktional}
\end{equation}
möglichst klein wird.
Eine solche Funktion heisst {\em Spline-Interpolationsfunktion}.

%
% Euler-Lagrange-Differentialgleichung
%
\subsection{Euler-Lagrange-Differentialgleichung}
Die Lagrange-Funktion des
Funktionals~\eqref{buch:nichtdiff:spline:eqn:funktional} ist
\[
L(x,y,y',y'') = y^{\prime\prime 2}.
\]
Für die Herleitung der Euler-Lagrange-Differentialgleichungen
dürfen jetzt aber nur Funktionen $\eta$ verwendet werden, die
in den Stellen $x_i$ verschwinden.
Die Variation ergibt
\begin{align*}
\delta I
&=
\frac{d}{dt}
\int_a^b (y''(x)+t\eta''(x))^2\,dx
\bigg|_{t=0}
\\
&=
\int_a^b 2y''(x)\eta''(x) \,dx.
\end{align*}
Da wir nur wissen, dass $y''(x)$ in den Teilintervallen $[x_i,x_{i+1}]$
differenzierbar ist, muss die partielle Integration auf diese
Teilintervalle beschränkt werden:
\begin{align*}
\delta I
&=
2
\sum_{i=0}^{n-1}
\int_{x_i}^{x_{i+1}} y''(x) \eta'(x)\,dx
=
2\sum_{i=0}^{n-1}\biggl[y''(x)\eta(x)\biggr]_{x_i}^{x_{i+1}}
-
2\sum_{i=0}^{n-1}\int_{x_i}^{x_{i+1}} y'''(x)\eta'(x)\,dx
\\
&=
-2y''(a)\eta'(a)
+\sum_{i=0}^{n-1}\bigl(y''(x_i-) -y''(x_i+)\bigr)\eta'(x_i)
+
2y''(b)\eta'(b)
-
2
\sum_{i=0}^{n-1}
\biggl[
y'''(x_i)\eta(x_i)
\biggr]_{x_i}^{x_{i+1}}
\\
&\qquad
+2
\sum_{i=0}^{n-1}
\int_{x_i}^{x_{i+1}} y''''(x)\eta(x)\,dx.
\end{align*}
Da $\eta$ in den Stützstellen verschwindet, ist diezweite Summe auf der
rechten Seite $=0$.
Die Variation kann nur für alle $\eta$ verschwinden, wenn die
Koeffizienten in der ersten Summe verschwinden, es folgt daher, dass
\[
y''(x_i-) = y''(x_i+) \quad\text{für alle $i$},
\]
die Funktion $y(x)$ ist also zweimal stetig differenzierbar auf dem
ganzen Intervall.
Aus dem Fundamentallemma angewandt auf die letzte Summe folgt, dass
$y''''(x)=0$ ist in jedem Teilintervall $[x_i,x_{i+1}]$.
Die Funktion $y(x)$ ist daher auf jedem solchen Teilintervall ein
kubisches Polynom.

\begin{verbatim}
- Problemstellung
- Euler-Lagrange-Differentialgleichung
- Lösung der Differntialgleichung
- Fehlerformel
\end{verbatim}
