%
% 1-gradient.tex
%
% (c) 2024 Prof Dr Andreas Müller
%
\section{Direkte Lösung endlichdimensionaler Optimierungsprobleme
\label{buch:direkt:section:gradient}}
\kopfrechts{Direkte Lösungen endlichdimensionaler Optimierungsprobleme}
Die direkten Methoden der Variationsrechnung beruhen darauf, dass
sie sich auf endlichdimensionale Extremalprobleme reduzieren oder
sich durch solche approximieren lassen, die sich viel einfacher
mit dem Computer lösen lassen.
Auch das Training von künstlichen neuronalen Netzwerken verlangt nach
der Lösung eines nichtlinearen, endlichdimensionalen Optimierungsproblems,
wenngleich die Dimension sehr hoch sein kann.
In diesem Abschnitt soll daher Methoden zur numerischen Lösung von
endlichdimensionalen Extremalproblemen gezeigt werden.

%
% Aufgabenstellung
%
\subsection{Aufgabenstellung
\label{buch:direkt:gradient:subsection:aufgabenstellung}}

\begin{definition}
Ein {\em relatives Maximum} einer Funktion $f\colon\Omega\to\mathbb{R}$
ist ein Punkt $x\in\Omega$ mit einer $\varepsilon$-Umgebung
$U$ von $x$ derart, $f(x)\ge f(x')$ für $x'\in U$.
Ein {\em absolutes Maximum} von $f$ ist ein Punkt $x\in\Omega$ derart,
dass $f(x)\ge f(x')$ für alle $x'\in\Omega$.
Sinngemäss werden {\em relatives Minimum} und {\em absolutes Minimum}
definiert.
\index{Maximum!relativ}%
\index{Maximum!absolut}%
\index{Minimum!relativ}%
\index{Minimum!absolut}%
\index{absolutes Maximum}%
\index{relatives Maximum}%
\index{absolutes Minimum}%
\index{relatives Minimum}%
\end{definition}

\begin{aufgabe}
\label{buch:direkt:gradient:aufgabe:extremal}
Sei $\Omega\subset\mathbb{R}^n$ ein Gebiet in $\mathbb{R}$ und
$f\colon\Omega\to\mathbb{R}$ eine stetig differenzierbare
Funktion.
\end{aufgabe}

%
% Gradientabstieg
%
\subsection{Gradientabstieg
\label{buch:direkt:gradient:subdsection:gradientabstieg}}
Wir gehen in diesem Abschnitt davon aus, dass ein Minimum der Funktion
$f(x)$ gefunden werden soll.
Die Überlegungen sind natürlich auf den Fall eines Maximums direkt
übertragbar.

Da die Richtungsableitung der Funktion $f$ an der Stelle $x\in\mathbb{R}^n$
in Richtung $v\in\mathbb{R}^n$ durch
\[
D_vf(x)
=
v\cdot \grad f(x)
\]
gebeben ist, wird die grösste Änderungsrate von $f$ in der Richtung
$v=\grad f(x)$ erreicht.
Für nicht allzu grosse Werte von $t$ kann man daher erwarten, dass
\(
f(x-t\grad f(x))
<
f(x)
\)
ist.
Entlang der Geraden mit Richtungsvektor $\grad f(x)$ kann man also
dem Minimum der Funktion $f$ näher kommen.

\begin{definition}[Gradientabstieg]
Ist $f\colon\mathbb{R}^n\to\mathbb{R}$ eine stetig differenzierbare
Funktion, $x_0\in\mathbb{R}$ und $\alpha\in\mathbb{R}$, dann heisst
die Folge $(x_n)_{n\mathbb{N}}$ definiert durch
\[
x_n = x_{n-1} - \alpha \grad f(x_{n-1})
\]
die zugehörige Gradientabstiegsfolge.
\end{definition}

Die Gradientabstiegsfolge nähert sich einer Nullstelle des Gradient
an, es ist aber unklar, wie schnell die Konvergenz sein wird.
Es ist auch durchaus möglich, dass die Nullstelle, gegen die die
Gradientabstiegsfolge strebt, gar kein Extremum ist.

\begin{beispiel}
Die Gradientabstiegsfolge der Funktion $f(x)=x^2$ zu $x_0$ und $\alpha$ ist
definiert durch
\[
x_n
=
x_{n-1}  -\alpha f'(x_{n-1})
=
x_{n-1}  -2\alpha x_{n-1}
=
(1-2\alpha)x_{n-1}
=
\dots
=
(1-2\alpha)^n x_0.
\]
Für $|1-2\alpha|< 1$ konvergiert die Folge tatsächlich gegen das
Minimum bei $x=0$.
Die Bedingung ist gleichbedeutend damit, dass $2\alpha\in(0,2)$.
Die Anzahl der korrekten Stellen von $x_n$ kann durch den
Logarithmus abgeschätzt werden:
\[
\log x_{n-1}
=
\log x_0 + n \log|1-2\alpha|.
\]
Ausser im Fall $1-2\alpha=0$ ist die Konvergenz also immer linear:
die Anzahl korrekter Stellen vergrössert sich in jedem Iterationsschritt
um den gleichen Betrag.
Die Wahl $\alpha = \frac12$ ist optimal, sie liefert das Minimum in
einer einzigen Iteration.
\end{beispiel}

\begin{beispiel}
Die Gradientabstiegsfolge der Funktion $f(x)=x^3$ zu $x_0=1$ und $\alpha$
ist definiert durch
\[
x_n
=
x_{n-1} - \alpha f'(x_{n-1})
=
x_{n-1} - \alpha (3x_{n-1}^2)
=
x_{n-1}(1-3\alpha x_{n-1}).
\]
Auch in diesem Fall konvergiert die Folge für $\alpha < \frac13$ immer
noch gegen die Nullstelle $x=0$ des Gradienten.
Unabhängig von der Wahl von $\alpha$ strebt der Faktor
$(1-3\alpha x_{n-1})$ immer gegen 1, die Konvergenz der Gradientabstiegsfolge
wird also immer langsamer.
Falls $\alpha > \frac13$ gewählt wird, ist $x_1=x_0-3\alpha x_0^2=1-3\alpha<0$.
Da $f'(x)>0$ ist für $x<0$, gilt für alle folgenden Glieder der
Gradientabstiegsfolge
\[
x_{n} = x_n-3\alpha x_{n-1}^2 < x_{n-1} < \dots < x_1 < 0,
\]
die Folge divergiert also.
\end{beispiel}

\begin{beispiel}
Sei schliesslich $f(x)=x^4$ und $x_0=1$, dann  ist die Gradientabstiegsfolge
definiert durch
\[
x_n
=
x_n - 4\alpha x_{n-1}^3
=
x_n(1-4\alpha x_{n-1}^2).
\]
Der Faktor $1-4\alpha x_{n-1}^2$ strebt (ausser im Fall $\alpha=\frac14$) 
gegen $1$, entsprechend langsam konvergiert $x_n$ gegen das Minimum.
\end{beispiel}

Die Wahl von $\alpha$ hat offenbar einen grossen Einfluss auf die
Konvergenzgeschwindigkeit.
Doch auch die Funktion selbst beeinflusst,
wie schnell die Gradientabstiegsfolge konvergiert.
Falls beim Extremum nicht nur die ersten Ableitungen versschwinden,
sondern auch noch höhere Ableitungen, wird die Konvergenz besonders
langsam.
Da man von der Funktion $f(x)$ nur weiss, dass sie differenzierbar ist,
aber nichts über höhere Ableitungen und ihre Grösse bekannt ist, kann
man auch keine allgemeingültigen Aussagen darüber machen, welcher Wert
von $\alpha$ gewählt werden soll.
Oft sind numerische Experimente die einzige erfolgversprechende
Strategie, geeignete Werte von $\alpha$ zu finden.

Für die Funktion $f(x)=ax^2+bx+c$ mit $a>0$ hat das Minimum $x=-b/2a$.
Die Gradientabstiegsfolge
\[
x_n 
=
x_{n-1} - 2\alpha x_{n-1}
=
(1-2\alpha)x_{n-1}
\]
kann das Minimum in einem einzigen Schritt erreichen, wenn $\alpha$
so gewählt wird, dass
\[
-\frac{b}{2a}
=
(1-2\alpha)x_0
\qquad
\Rightarrow
\qquad
\alpha
=
\frac12\biggl(
1+\frac{b}{2ax_0}
\biggr).
\]
Im Fall einer quadratischen Funktion ist es also möglich, $\alpha$ so
zu wählen, dass die Konvergenz unmittelbar ist.
Um diese Idee auch auf Funktionen mehrere Variablen zu übertragen,
ist zunächst notwendig, quadratische Minimalproblem genauer zu
studieren, was im nächsten Abschnitt geschehen soll.

%
% Quadratische Extremalprobleme
%
\subsection{Quadratische Extremalprobleme
\label{buch:direkt:gradient:subdsection:quadratisch}}
Eine besonders einfach Situation entsteht, wenn die Funktion
$f\colon\mathbb{R}^n\to\mathbb{R}$ höchstens vom Grad~2 ist.
Die Funktion lässt sich dann in der Form
\[
f(x)
=
\sum_{i,k=1}^n a_{ik}x_ix_k
+
\sum_{i=1}^n b_ix_i
+ c
=
\transpose{x}Ax + \transpose{b}x + c
\]
schreiben, wobei $A$ eine symmetrische Matrix ist.

%
% Die Matrix $A$ ist definit
%
\subsubsection{Die Matrix $A$ ist definit}
Damit eine solche Funktion ein Minimum hat, muss für jeden beliebigen
Richtungsvektor $r$, $|r|=1$, der quadratische Term gegenüber dem
linearen Term dominant sein, für $t\to\pm\infty$ muss also
\[
f(tr)
=
t^2
\transpose{r}Ar
+
t
\transpose{b}r
+
c
\to \infty
\]
streben.
Dies funktioniert aber nur, wenn der Koeffizient $\transpose{r}Ar$ von
$t^2$ postiv ist.
Dies ist aber genau die Definition einer positiv definiten Matrix.
Ein Minimum der Funktion ist also nur zu erwarten, $A$ eine positiv
definite Matrix ist.
Ein positiv definite Matrix ist auch invertierbar, wie zum Beispiel
sofort aus der Cholesky-Zerlegung $A=L\transpose{L}$ folgt, da $L$
eine Dreiecksmatrix ist, die auf der Diagonalen keine Nullen enthält
und damit invertierbar ist.

Analog kann argumentiert werden, dass die Funktion $f$ nur dann
ein Maximum hat, wenn $\transpose{r}Ar<0$ ist für jede Richtung
$r\in\mathbb{R}^n$.
In diesem Fall ist die Matrix $A$ also negativ definit und damit
ebenfalls invertierbar.


%
% Nullstellen des Gradienten
%
\subsubsection{Nullstellen des Gradienten}
In diesem Fall lässt sich der Gradient von $f$ direkt
berechnen:
\begin{align*}
\frac{\partial f}{\partial x_l}
&=
\frac{\partial}{\partial x_l}
\sum_{i,k=1}^n a_{ik}x_ix_k
+
\frac{\partial}{\partial x_l}
\sum_{i=1}^n b_ix_i
\\
&=
\sum_{i,k=1}^n a_{ik}
\frac{\partial x_i}{\partial x_l}
x_k
+
\sum_{i,k=1}^n a_{ik}
x_i
\frac{\partial x_k}{\partial x_l}
+
\sum_{i=1}^n b_i
\frac{\partial x_i}{\partial x_l}.
\end{align*}
Mit den partiellen Ableitungen
\[
\frac{\partial x_i}{\partial x_l}
=
\delta_{il}
=
\begin{cases}
1&\qquad\text{für $i=l$}\\
0&\qquad\text{sonst}
\end{cases}
\]
wird die partielle Ableitung von $f$ zu
\begin{align*}
\frac{\partial f}{\partial x_l}
&=
\sum_{i,k=1}^n a_{ik}
\delta_{il}
x_k
+
\sum_{i,k=1}^n a_{ik}
x_i
\delta_{kl}
\frac{\partial x_k}{\partial x_l}
+
\sum_{i=1}^n b_i
\delta_{il}
\frac{\partial x_i}{\partial x_l}
\\
&=
\sum_{k=1}^n a_{lk}x_k
+
\sum_{i=1}^n a_{il}x_i
+
b_l.
\intertext{Wir benennen die Summationsvariable $i$ in der zweiten Summe
in $k$ um, und nützen die Symmetrie von $A$ aus um}
&=
\sum_{k=1}^n a_{lk}x_k
+
\sum_{k=1}^n a_{kl}x_i
+
b_l
\\
&=
\sum_{k=1}^n a_{lk}x_k
+
\sum_{k=1}^n a_{lk}x_k
+
b_l
\\
&=
2\sum_{k=1}^n a_{lk}x_k
+
b_l
\end{align*}
zu erhalten.
Der Gradient ist daher
\[
\grad f(x) = 2Ax + b.
\]
Damit lassen sich aber auch sofort eventuelle Extrema finden, indem
Nullstellen des Gradienten gesucht werden.
Es ist nämlich
\begin{equation}
0=\grad f(x) = 2Ax + b
\qquad\Rightarrow\qquad
Ax
=
-\frac12 b
\qquad\Rightarrow\qquad
x
=
-\frac12 A^{-1}b,
\label{buch:direkt:gradient:quadratisch:eqn:loesung}
\end{equation}
Da die Matrix $A$ definit ist, ist sie auch invertierbar, die
Lösung~\eqref{buch:direkt:gradient:quadratisch:eqn:loesung} existiert
also tatsächlich.

%
% Vergleich mit dem eindimensionalen Fall
%
\subsubsection{Vergleich mit dem eindimensionalen Fall}
Ein quadratisches Polynom $y = ax^2+bx+c$ von nur einer Variablen ist
ein Spezialfall, der schon im elementaren Analysisunterricht behandelt
wird.
Die Ableitung ist
\[
y' = 2ax+b
\]
mit der Nullstelle
\[
x = -\frac{b}{2a} = -\frac12 a^{-1} b,
\]
also dem eindimensionalen Fall der Formel
\eqref{buch:direkt:gradient:quadratisch:eqn:loesung}

%
% Nebenbedingungen
%
\subsubsection{Nebenbedingungen}
Für ein Extremalproblem mit Nebenbedingungen $g_i(x)=0$ liefert die
Methode der Lagrange-Multiplikatoren eine Lösung.
Falls auch die Funktionen
\[
g_i(x)
=
\transpose{x}A^{(i)}x + b^{(i)}x + c^{(i)}
\]
quadratisch sind, lässt sich auch der Gradient von $g_i$ als
\[
2A^{(i)}x + b^{i}
\]
berechnen.
Das Extremum von $f(x)$ unter den Nebenbedingungen $g_i(x)=0$ erfüllt
daher die Gleichung
\begin{equation}
2Ax + b
=
\sum_{i=1}^m {\color{darkred}\lambda_i} (2A^{(i)}x+b^{(i)}).
\label{buch:direkt:gradient:quadratisch:lagrange}
\end{equation}
Zusätzlich müssen auch noch die Nebenbedingungen $g_i(x)=0$ erfüllt sein.
Die Vektorgleichung \eqref{buch:direkt:gradient:quadratisch:lagrange}
umfasst $n$ Komponenten, dazu kommen die $m$ Gleichungen $g_i(x)$,
was genügen sollte, die $n+m$ Unbekannten $x$ und ${\color{darkred}\lambda_i}$
zu bestimmen.

Man beachte, dass für die Existenz eines Extremums nicht mehr gefordert
werden muss, dass die Matrix $A$ definit ist.
Zum Beispiel hat die Funktion $f(x) = x_1^2 - x_2^2$, die ganz offensichtlich
nicht definit ist,
unter der Nebenbedingung $g(x)=x_2-1=0$ das Minimum $-1$ im Punkt $(0,1)$,
unter der Nebenbedingung $g(x)=x_1-1=0$ das Maximum $1$ im Punkt $(1,0)$ und
unter der Nebenbedingung $g(x)=x_1^2+x_2^2-1=0$ Minima in den Punkten
$(0,\pm1)$ und Maxima in den Punkten $(\pm1,0)$.

Besonders interessant ist der Spezialfall einer einzelnen Nebenbedingung,
wenn gleichzeitig die beiden Vektoren $b$ und $b^{(1)}$ verschwinden.
In diesem Fall erfüllt $x$ die Bedingung
\[
2Ax=2{\color{darkred}\lambda_1}A^{(i)}x.
\]
Ein solcher Vektor heisst ein verallgemeinerter Eigenvektor im Sinne
der folgenden Definition.

\begin{definition}[Verallgmeinerter Eigenvektor]
Seine $A$ und $B$ $n\times n$-Matrizen.
Ein Vektor $x\in\mathbb{R}$ heisst ein verallgemeinerter Eigenvektor,
wenn es $\lambda\in\mathbb{R}\setminus \{0\}$ gibt derart, dass
$Ax=\lambda Bx$.
\end{definition}

Falls $B$ invertierbar ist, kann ein verallgemeinerter Eigenvektor von $A$
und $B$ wegen
\[
Ax=\lambda Bx
\qquad\Rightarrow\qquad
=
B^{-1}Ax=\lambda x
\]
als gewöhnlicher Eigenvektor von $B^{-1}A$ gefunden werden.

%
% Gradientabstieg für annähernd quadratische Funktionen
%
\subsubsection{Gradientabstieg für annähernd quadratische Funktionen}
Am Ende von Abschnitt~\ref{buch:direkt:gradient:subdsection:gradientabstieg}
wurde darauf hingewiesen, dass sich für den Gradientabstieg bei einer
quadratischen Funktion einer Variablen ein optimales $\alpha$ finden lässt.
In diesem Abschnitt soll dies auf quadratische Extremalprobleme von
mehreren Variablen erweitert werden.

Sei als 
\[
f(x)
=
\transpose{x}Ax+\transpose{b}x+c
\]
mit $A\in M_{n\times n}(\mathbb{R})$ symmetrisch und $b\in\mathbb{R}^n$.
Das Extremum findet man bei 
\[
x = -\frac12 A^{-1} b.
\]
Der Gradient der Funktion $f$ an der Stelle $x$ ist
\[
\grad f(x)
=
2Ax + b
\]
und die Gradientabstiegsfolge ist definiert durch
\[
x_n
=
x_{n-1} -\alpha \grad f(x_{n-1})
=
x_{n-1} - \alpha (2A x_{n-1} + b).
\]
Jetzt soll $\alpha$ so gewöhlt werden, dass das Extremum ausgehend von 
$x_0$ in einem Schritt erreicht wird, also
\[
-\frac12 A^{-1}b
=
x_1
=
x_0 - \alpha (2A x_0 + b)
\]
Dies ist eine $n$-dimensionale Vektorgleichung für die einzige Unbekannt
$\alpha$, man kann also keine Lösung erwarten.
Dies ist auch aus geometrischen Gründen naheliegen: dazu müsste nämlich
die Gerade durch $x_0$ mit dem Richtungsvektor $\grad f(x_0)$ durch das
Minimum verlaufen, was unrealistisch ist.

Da nur die eine Variable $\alpha$ zu bestimmen ist, muss das Problem auf
ein eindimensionales Problem reduziert werden.
Es ist naheliegend anzunehmen, dass man dem Minimum besonders nahe
kommen kann, wenn man das Minimum der Funktion
\[
\alpha
\mapsto
g(\alpha)
=
f(x_0 - \alpha\grad f(x_0))
\]
sucht.
Wir schreiben zur Abkürzung $v=\grad f(x_0)$ und erhalten
die in $\alpha$ quadratische Funktion
\begin{align*}
g(\alpha)
&=
\transpose{(x_0 - \alpha v)}A(x_0-\alpha v) + \transpose b(x_0-\alpha v) + c.
\\
&=
\alpha^2
\transpose{v}Av
+
\alpha
(-\transpose{x_0}Av -\transpose{v}Ax_0 -\alpha \transpose{b}x_0)
+
\transpose{x_0}Ax_0+\transpose{b}x_0 + c.
\end{align*}
Das Minimum wird erreicht für
\[
\alpha
=
-\frac{
(-\transpose{x_0}Av -\transpose{v}Ax_0 -\alpha \transpose{b}x_0)
}{
\transpose{v}Av
}
=
\frac{2\transpose{x_0}Av -\transpose{b}x_0}{\transpose{v}Av}
\]
Wenn eine Funktion ausreichend gut durch eine quadratische Funktion
beschreibbar ist, kann man also dieser Methode versuchen, einen gut
geeigneten Wert von $\alpha$ zu finden.



