%
% fem.tex -- Das FEM-Verfahren
%
% (c) 2024 Prof Dr Andreas Müller
%
\section{Diskretisation nach Euler
\label{buch:direkt:section:euler}}
\kopfrechts{Diskretisation nach Euler}
Leonhard Euler
\index{Euler, Leonhard}%
\index{Leonhard Euler}%
hat die Euler-Lagrange auf eine Weise hergeleitet, die eher an die 
Vorgehensweise direkter Methoden erinnert.
In diesem Abschnitt soll das Integral
\begin{equation}
I(y)
=
\int_a^b
F(x,y(x),y'(x))
\,dx
\label{buch:direkt:euler:eqn:aufgabe}
\end{equation}
mit den Randbedingungen $y(a)=y_a$ und $y(b)=y_b$ minimiert werden.

%
% Polygon-Approximation
%
\subsection{Polygon-Approximation
\label{buch:direkt:euler:subsection:polygon}}
%
% euler.tex -- Herleitung der Euler-Lagrange-DGL
%
% (c) 2021 Prof Dr Andreas Müller, OST Ostschweizer Fachhochschule
%
\documentclass[tikz]{standalone}
\usepackage{amsmath}
\usepackage{times}
\usepackage{txfonts}
\usepackage{pgfplots}
\usepackage{csvsimple}
\usetikzlibrary{arrows,intersections,math}
\begin{document}
\def\skala{1}
\definecolor{darkred}{rgb}{0.8,0,0}
\def\punkt#1#2{
	\fill[color=white] #1 circle[radius=0.08];
	\draw[color=#2] #1 circle[radius=0.07];
}
\begin{tikzpicture}[>=latex,thick,scale=\skala,
declare function={
	f(\t) = -0.5-0.12*(\t-6)*(\t-6)-0.02*(\t-7)*(\t-7)*(\t-7)+0.8*\t;
}]

\draw[color=darkred,line width=1.4pt,smooth]
	plot[domain=0.5:10.5] ({\x},{f(\x)});

\draw (1,-0.05) -- (1,0.05);
\node at (1,-0.05) [below] {$a\mathstrut=x_0$};
\draw[line width=0.3pt] (1,0) -- (1,{f(1)});
\punkt{(1,{f(1)})}{darkred}
\node at (1,{0.7*f(1)}) [left] {$y_a=y_0\mathstrut$};

\foreach \k in {1,...,3}{
	\draw ({1+\k},-0.05) -- ({1+\k},0.05);
	\node at ({1+\k},-0.05) [below] {$x_\k\mathstrut$};
	\draw[line width=0.3pt] ({1+\k},0) -- ({1+\k},{f(1+\k)});
	\punkt{({1+\k},{f(1+\k)})}{darkred}
	\node at ({1+\k},{0.7*f(1+\k)}) [left] {$y_\k\mathstrut$};
}

\foreach \k in {6,7,8}{
	\draw (\k,-0.05) -- (\k,0.05);
	\draw[line width=0.3pt] (\k,0) -- (\k,{f(\k)});
}
\fill[color=blue!20,opacity=0.5]
	(6,{f(6)}) -- (7,{f(6)}) -- (7,{f(7)}) -- cycle;
\fill[color=blue!20,opacity=0.5]
	(7,{f(7)}) -- (8,{f(7)}) -- (8,{f(8)}) -- cycle;
\draw[color=blue,line width=1.4pt] (6,{f(6)}) -- (7,{f(7)});
\draw[color=blue,line width=1.4pt] (7,{f(7)}) -- (8,{f(8)});
\node[color=blue] at (6.5,{f(6.5)})
	[above,rotate={atan(f(7)-f(6))}] {$y'_{k-1}$};
\node[color=blue] at (7.5,{f(7.5)})
	[above,rotate={atan(f(8)-f(7))}] {$y'_{k}$};
\node at (6,-0.05) [below] {$x_{k-1}$};
\node at (7,-0.05) [below] {$x_{k}$};
\node at (8,-0.05) [below] {$x_{k+1}$};
\punkt{(6,{f(6)})}{darkred}
\punkt{(7,{f(7)})}{darkred}
\punkt{(8,{f(8)})}{darkred}
\node at (6,{0.7*f(6)}) [left] {$y_{k-1}\mathstrut$};
\node at (7,{0.7*f(7)}) [left] {$y_{k}\mathstrut$};
\node at (8,{0.7*f(8)}) [right] {$y_{k+1}\mathstrut$};

\node at (1.5,0) [above] {$h\mathstrut$};
\node at (2.5,0) [above] {$h\mathstrut$};
\node at (3.5,0) [above] {$h\mathstrut$};
\node at (6.5,0) [above] {$h\mathstrut$};
\node at (7.5,0) [above] {$h\mathstrut$};

\node at (5,{0.4*f(5)}) {$\dots$};
\node at (9,{0.4*f(9)}) {$\dots$};

\draw (10,-0.05) -- (10,0.05);
\node at (10,-0.05) [below] {$b\mathstrut=x_n$};
\draw[line width=0.3pt] (10,0) -- (10,{f(10)});
\punkt{(10,{f(10)})}{darkred}

\node at (10,{0.7*f(10)}) [right] {$y_b=y_n\mathstrut$};

\draw[->] (-0.6,0) -- (11.5,0) coordinate[label={$x$}];
\draw[->] (-0.5,-0.1) -- (-0.5,6.5) coordinate[label={right:$y$}];

\end{tikzpicture}
\end{document}


Die gesuchte Funktion $y\colon[a,b]\mathbb{R}$ wird durch eine stückweise
lineare Funktion approximiert (Abbildung~\ref{buch:direkt:euler:fig:euler}).
Sie ist definiert durch die Werte an den $n-1$ Zwischenpunkten
$x_k=a+kh$ mit $h=(b-a)/n$.
Die Endpunkte des Intervalls sind $x_0=a$ und $x_n=b$.
Zu bestimmen sind die Funktionswerte $y_k = y(x_k)$ für $k=1,\dots,n-1$.
Die Funktionswerte $y_0=y(x_0)=y(a)=y_a$ und $y_n=y(x_n)=y(b)=y_b$ 
sind durch die Randbedingung vorgegeben.

Das Integral $I(y)$ von \eqref{buch:direkt:euler:eqn:aufgabe} muss
jetzt durch die unbekannten Werte $y_1,\dots,y_{n-1}$ approximiert
werden.
Für die Ableitung an der Stelle $x_k$ kann der Differenzenquotient
\[
y'(x_k) 
\approx
y'_k
=
\frac{y_{k+1\mathstrut}-y_{k\mathstrut}}{h}
\]
verwendet werden.
Der Wert des Integrals 
\begin{equation}
I(y)
\approx
I(y_1,\dots,y_{n-1})
=
\sum_{k=1}^{n-1}
F(x_k, y_k, y'_k)\cdot h
\label{buch:direkt:euler:eqn:summe}
\end{equation}
Die Aufgabenstellung wird damit zum folgenden endlichdimensionalen
Problem.

\begin{aufgabe}
Finde die Werte $(y_1,\dots,y_{n-1})\in\mathbb{R}^{n-1}$, für die
die Summe~\eqref{buch:direkt:euler:eqn:summe} extremal wird.
\end{aufgabe}

%
% Notwendig Bedingung
%
\subsection{Euler-Lagrange-Differentialgleichung
\label{buch:direkt:euler:subsection:eldgl}}
Eine notwendige Bedingung dafür, dass die Funktion $I(y_1,\dots,y_{n-1})$
ein Extremum annimmt, ist das Verschwinden der partiellen Ableitungen
\begin{equation}
\frac{\partial I}{\partial y_i} (y_1,\dots,y_{n-1})
=
0
\end{equation}
für $i=1,\dots,n-1$.
In der Summe~\eqref{buch:direkt:euler:eqn:summe} kommt die Variable
$y_i$ zunächst im Term mit $k=i$ vor.
Da aber $y'_k$ aus $y_{k+1}$ und $y_k$ berechnet wird, kommt $y_i$
in der Summe auch im Term mit $k=i-1$ vor.
Die Ableitung der Summe nach $y_k$ ist daher
\begin{align}
\frac{\partial I}{\partial y_i}(y_1,\dots,y_{n-1})
&=
\frac{\partial}{\partial y_i}
\sum_{k=1}^{n-1}
F(x_k, y_k, y'_k)\cdot h
\notag
\\
&=
\frac{\partial}{\partial y_i}
F(x_i, {\color{darkred}y_i}, {\color{darkred}y'_i})\cdot h
+
\frac{\partial}{\partial y_i}
F(x_{i-1}, y_{i-1}, {\color{darkred}y'_{i-1}})\cdot h
\notag
\\
&=
\frac{\partial}{\partial y_i}
F\biggl(x_i, {\color{darkred}y_i}, \frac{y_{i+1}-{\color{darkred}y_i}}{h}\biggr)\cdot h
+
\frac{\partial}{\partial y_i}
F\biggl(x_{i-1}, y_{i-1}, \frac{{\color{darkred}y_i}-y_{i-1}}{h}\biggr)\cdot h.
\notag
\intertext{Die Ableitung kann mit der Kettenregel berechnet werden
und ergibt}
&=
\frac{\partial F}{\partial y}\biggl(x_i, y_i,\frac{y_{i+1}-y_i}{h}\biggr)
\cdot h
\\
&\qquad\qquad
-
\frac{\partial F}{\partial y'}\biggl(x_i,y_i,\frac{y_{i+1}-y_i}{h}\biggr)
+
\frac{\partial F}{\partial y'}\biggl(x_{i-1},y_{i-1},\frac{y_i-y_{i-1}}h\biggr).
\intertext{Im zweiten und dritten Term sind die Faktoren $h$ verschwunden,
weil sie gegen einen von der inneren Ableitung herstammenden Faktor $1/h$ 
gekürzt wurden.
Die letzten beiden Terme können noch zusammengefasst werden:
}
&=
h\cdot \biggl(
\frac{\partial F}{\partial y}(x_i,y_i,y'_i)
-
\frac1h\biggl(
\frac{\partial F}{\partial y'}(x_i,y_i,y'_i)
-
\frac{\partial F}{\partial y'}(x_{i-1},y_{i-1},y'_{i-1})
\biggr)\biggr).
\notag
\intertext{Die innere Klammer ist ein Differenzenquotient, der die
Ableitung nach $x$ approximiert.
Für genügend feine Diskretisation ist daher }
\frac{\partial I}{\partial y_i}(y_1,\dots,y_{n-1})
&\approx
h\cdot\biggl(
\frac{\partial F}{\partial y}(x_i,y_i,y'_i)
-
\frac{d}{dx}\frac{\partial F}{\partial y'}(x_{i-1}, y_{i-1}, y'_{i-1})
\biggr)
=
0.
\label{buch:direkt:euler:eqn:approx}
\end{align}
Die Ableitungen $\partial I/\partial y_i$ verschwinden also, wenn 
die grosse Klammer den Wert $0$ annimmt.
Eine notwendige Bedingung dafür, dass die Lösungsfunktion $y(x)$, die in den
Stützstellen $x_i$ die Werte $y(x_i) = y_i$ annimmt, das
Integral $I(y)$ von \eqref{buch:direkt:euler:eqn:aufgabe} extremal macht,
ist also, dass $y(x)$ die Euler-Lagrange-Differentialgleichung
\[
\frac{\partial F}{\partial y}\bigl(x,y(x),y'(x)\bigr)
-
\frac{d}{dx}
\frac{\partial F}{\partial y'}\bigl(x,y(x),y'(x)\bigr)
=
0
\]
erfüllt.

Das Argument passt zum Stil, wie solche Beweise zu Eulers Zeit 
formuliert wurden.
Es ist aber nicht ganz vollständig.
Zum Beispiel ist die rechte Seite von \eqref{buch:direkt:euler:eqn:approx}
auch dann klein, wenn die grosse Klammer nur beschränkt ist.
Die Begründung, dass sich im Grenzwert $n\to\infty$ oder $h\to 0$ die
Klammer verschwinden müsse, ist also nicht ganz schlüssig.
Gerade solche Schwächen oder Lücken in Beweisen haben später im
im 19.~Jahrhundert eine Weiterentwicklung der Definitionen und
Beweistechniken der Analysis vor allem durch Weierstrass provoziert.
Auch die Begründung der Euler-Lagrange-Differentialgleichung als Grundlage
der Variationsrechnung sah sich im 19.~Jahrhundert dieser methodischen
Kritik durch Weierstrass und andere ausgesetzt.
Die sich daraus entwickelnde Diskussion hat zu den heute akzeptierten
Beweisen, dem Fundamentallemma oder auch dem in 
Abschnitt~\ref{buch:nichtdiff:section:duboisreymond} 
erklärten Lemma von Dubois-Reymond geführt.


