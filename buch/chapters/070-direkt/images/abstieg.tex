%
% abstieg.tex -- Abstieg 
%
% (c) 2021 Prof Dr Andreas Müller, OST Ostschweizer Fachhochschule
%
\documentclass[tikz]{standalone}
\usepackage{amsmath}
\usepackage{times}
\usepackage{txfonts}
\usepackage{pgfplots}
\usepackage{csvsimple}
\usetikzlibrary{arrows,intersections,math}
\definecolor{darkred}{rgb}{0.8,0,0}
\begin{document}
\def\skala{1}
\begin{tikzpicture}[>=latex,thick,scale=\skala]

\draw[color=gray] (-6.1,0) -- (6.5,0) coordinate[label={$x$}];
\draw[color=gray] (0,-4.1) -- (0,4.5) coordinate[label={right:$y$}];

\begin{scope}
	\clip (-6,-4.05) rectangle (6,4.05);
	\draw (0,0) ellipse (5cm and 0.5cm);
	\draw (0,0) ellipse (10cm and 1cm);
	\draw (0,0) ellipse (15cm and 1.5cm);
	\draw (0,0) ellipse (20cm and 2.0cm);
	\draw (0,0) ellipse (25cm and 2.5cm);
	\draw (0,0) ellipse (30cm and 3.0cm);
	\draw (0,0) ellipse (35cm and 3.5cm);
	\draw (0,0) ellipse (40cm and 4.0cm);
\end{scope}

\def\x{3.5}
\def\y{3.5}
\node at (\x,\y) [above] {$(x_0,y_0)$};
\fill[color=darkred] (\x,\y) circle[radius=0.05];
\def\a{0.099}
\foreach \k in {1,...,20}{
	\pgfmathparse{\x-2*\a*\x}
	\xdef\xnew{\pgfmathresult}
	\pgfmathparse{\y-20*\a*\y}
	\xdef\ynew{\pgfmathresult}
	\draw[color=darkred] (\x,\y) -- (\xnew,\ynew);
	\fill[color=darkred] (\xnew,\ynew) circle[radius=0.05];
	\xdef\x{\xnew}
	\xdef\y{\ynew}
}

\fill[color=blue] (0,0) circle[radius=0.05];

\end{tikzpicture}
\end{document}

