%
% tikztemplate.tex -- template for standalon tikz images
%
% (c) 2021 Prof Dr Andreas Müller, OST Ostschweizer Fachhochschule
%
\documentclass[tikz]{standalone}
\usepackage{amsmath}
\usepackage{times}
\usepackage{txfonts}
\usepackage{pgfplots}
\usepackage{csvsimple}
\usetikzlibrary{arrows,intersections,math}
\begin{document}
\def\dglsol{
	({0.0000*\dx},{0.0000*\dy})
	-- ({0.0317*\dx},{-0.0106*\dy})
	-- ({0.0635*\dx},{-0.0210*\dy})
	-- ({0.0952*\dx},{-0.0311*\dy})
	-- ({0.1269*\dx},{-0.0411*\dy})
	-- ({0.1587*\dx},{-0.0508*\dy})
	-- ({0.1904*\dx},{-0.0603*\dy})
	-- ({0.2221*\dx},{-0.0695*\dy})
	-- ({0.2539*\dx},{-0.0786*\dy})
	-- ({0.2856*\dx},{-0.0874*\dy})
	-- ({0.3173*\dx},{-0.0960*\dy})
	-- ({0.3491*\dx},{-0.1044*\dy})
	-- ({0.3808*\dx},{-0.1126*\dy})
	-- ({0.4125*\dx},{-0.1205*\dy})
	-- ({0.4443*\dx},{-0.1282*\dy})
	-- ({0.4760*\dx},{-0.1358*\dy})
	-- ({0.5077*\dx},{-0.1430*\dy})
	-- ({0.5395*\dx},{-0.1501*\dy})
	-- ({0.5712*\dx},{-0.1570*\dy})
	-- ({0.6029*\dx},{-0.1636*\dy})
	-- ({0.6347*\dx},{-0.1700*\dy})
	-- ({0.6664*\dx},{-0.1762*\dy})
	-- ({0.6981*\dx},{-0.1822*\dy})
	-- ({0.7299*\dx},{-0.1880*\dy})
	-- ({0.7616*\dx},{-0.1935*\dy})
	-- ({0.7933*\dx},{-0.1989*\dy})
	-- ({0.8251*\dx},{-0.2040*\dy})
	-- ({0.8568*\dx},{-0.2089*\dy})
	-- ({0.8885*\dx},{-0.2136*\dy})
	-- ({0.9203*\dx},{-0.2181*\dy})
	-- ({0.9520*\dx},{-0.2224*\dy})
	-- ({0.9837*\dx},{-0.2264*\dy})
	-- ({1.0155*\dx},{-0.2302*\dy})
	-- ({1.0472*\dx},{-0.2339*\dy})
	-- ({1.0789*\dx},{-0.2373*\dy})
	-- ({1.1107*\dx},{-0.2405*\dy})
	-- ({1.1424*\dx},{-0.2434*\dy})
	-- ({1.1741*\dx},{-0.2462*\dy})
	-- ({1.2059*\dx},{-0.2488*\dy})
	-- ({1.2376*\dx},{-0.2511*\dy})
	-- ({1.2693*\dx},{-0.2532*\dy})
	-- ({1.3011*\dx},{-0.2551*\dy})
	-- ({1.3328*\dx},{-0.2568*\dy})
	-- ({1.3645*\dx},{-0.2583*\dy})
	-- ({1.3963*\dx},{-0.2596*\dy})
	-- ({1.4280*\dx},{-0.2607*\dy})
	-- ({1.4597*\dx},{-0.2615*\dy})
	-- ({1.4915*\dx},{-0.2622*\dy})
	-- ({1.5232*\dx},{-0.2626*\dy})
	-- ({1.5549*\dx},{-0.2628*\dy})
	-- ({1.5867*\dx},{-0.2628*\dy})
	-- ({1.6184*\dx},{-0.2626*\dy})
	-- ({1.6501*\dx},{-0.2622*\dy})
	-- ({1.6819*\dx},{-0.2615*\dy})
	-- ({1.7136*\dx},{-0.2607*\dy})
	-- ({1.7453*\dx},{-0.2596*\dy})
	-- ({1.7771*\dx},{-0.2583*\dy})
	-- ({1.8088*\dx},{-0.2568*\dy})
	-- ({1.8405*\dx},{-0.2551*\dy})
	-- ({1.8723*\dx},{-0.2532*\dy})
	-- ({1.9040*\dx},{-0.2511*\dy})
	-- ({1.9357*\dx},{-0.2488*\dy})
	-- ({1.9675*\dx},{-0.2462*\dy})
	-- ({1.9992*\dx},{-0.2434*\dy})
	-- ({2.0309*\dx},{-0.2405*\dy})
	-- ({2.0627*\dx},{-0.2373*\dy})
	-- ({2.0944*\dx},{-0.2339*\dy})
	-- ({2.1261*\dx},{-0.2302*\dy})
	-- ({2.1579*\dx},{-0.2264*\dy})
	-- ({2.1896*\dx},{-0.2224*\dy})
	-- ({2.2213*\dx},{-0.2181*\dy})
	-- ({2.2531*\dx},{-0.2136*\dy})
	-- ({2.2848*\dx},{-0.2089*\dy})
	-- ({2.3165*\dx},{-0.2040*\dy})
	-- ({2.3483*\dx},{-0.1989*\dy})
	-- ({2.3800*\dx},{-0.1935*\dy})
	-- ({2.4117*\dx},{-0.1880*\dy})
	-- ({2.4435*\dx},{-0.1822*\dy})
	-- ({2.4752*\dx},{-0.1762*\dy})
	-- ({2.5069*\dx},{-0.1700*\dy})
	-- ({2.5387*\dx},{-0.1636*\dy})
	-- ({2.5704*\dx},{-0.1570*\dy})
	-- ({2.6021*\dx},{-0.1501*\dy})
	-- ({2.6339*\dx},{-0.1430*\dy})
	-- ({2.6656*\dx},{-0.1358*\dy})
	-- ({2.6973*\dx},{-0.1282*\dy})
	-- ({2.7291*\dx},{-0.1205*\dy})
	-- ({2.7608*\dx},{-0.1126*\dy})
	-- ({2.7925*\dx},{-0.1044*\dy})
	-- ({2.8243*\dx},{-0.0960*\dy})
	-- ({2.8560*\dx},{-0.0874*\dy})
	-- ({2.8877*\dx},{-0.0786*\dy})
	-- ({2.9195*\dx},{-0.0695*\dy})
	-- ({2.9512*\dx},{-0.0603*\dy})
	-- ({2.9829*\dx},{-0.0508*\dy})
	-- ({3.0147*\dx},{-0.0411*\dy})
	-- ({3.0464*\dx},{-0.0311*\dy})
	-- ({3.0781*\dx},{-0.0210*\dy})
	-- ({3.1099*\dx},{-0.0106*\dy})
	-- ({3.1416*\dx},{-0.0000*\dy})
}
\def\lone{
	({0.00000*\dx},{0.00000*\dy})
	-- ({0.03173*\dx},{-0.00860*\dy})
	-- ({0.06347*\dx},{-0.01719*\dy})
	-- ({0.09520*\dx},{-0.02576*\dy})
	-- ({0.12693*\dx},{-0.03430*\dy})
	-- ({0.15867*\dx},{-0.04281*\dy})
	-- ({0.19040*\dx},{-0.05128*\dy})
	-- ({0.22213*\dx},{-0.05970*\dy})
	-- ({0.25387*\dx},{-0.06805*\dy})
	-- ({0.28560*\dx},{-0.07634*\dy})
	-- ({0.31733*\dx},{-0.08455*\dy})
	-- ({0.34907*\dx},{-0.09268*\dy})
	-- ({0.38080*\dx},{-0.10071*\dy})
	-- ({0.41253*\dx},{-0.10864*\dy})
	-- ({0.44427*\dx},{-0.11646*\dy})
	-- ({0.47600*\dx},{-0.12416*\dy})
	-- ({0.50773*\dx},{-0.13174*\dy})
	-- ({0.53947*\dx},{-0.13919*\dy})
	-- ({0.57120*\dx},{-0.14650*\dy})
	-- ({0.60293*\dx},{-0.15365*\dy})
	-- ({0.63467*\dx},{-0.16066*\dy})
	-- ({0.66640*\dx},{-0.16750*\dy})
	-- ({0.69813*\dx},{-0.17417*\dy})
	-- ({0.72986*\dx},{-0.18067*\dy})
	-- ({0.76160*\dx},{-0.18699*\dy})
	-- ({0.79333*\dx},{-0.19312*\dy})
	-- ({0.82506*\dx},{-0.19905*\dy})
	-- ({0.85680*\dx},{-0.20478*\dy})
	-- ({0.88853*\dx},{-0.21031*\dy})
	-- ({0.92026*\dx},{-0.21562*\dy})
	-- ({0.95200*\dx},{-0.22072*\dy})
	-- ({0.98373*\dx},{-0.22560*\dy})
	-- ({1.01546*\dx},{-0.23025*\dy})
	-- ({1.04720*\dx},{-0.23466*\dy})
	-- ({1.07893*\dx},{-0.23884*\dy})
	-- ({1.11066*\dx},{-0.24278*\dy})
	-- ({1.14240*\dx},{-0.24648*\dy})
	-- ({1.17413*\dx},{-0.24993*\dy})
	-- ({1.20586*\dx},{-0.25312*\dy})
	-- ({1.23760*\dx},{-0.25606*\dy})
	-- ({1.26933*\dx},{-0.25875*\dy})
	-- ({1.30106*\dx},{-0.26117*\dy})
	-- ({1.33280*\dx},{-0.26333*\dy})
	-- ({1.36453*\dx},{-0.26522*\dy})
	-- ({1.39626*\dx},{-0.26685*\dy})
	-- ({1.42800*\dx},{-0.26821*\dy})
	-- ({1.45973*\dx},{-0.26930*\dy})
	-- ({1.49146*\dx},{-0.27011*\dy})
	-- ({1.52320*\dx},{-0.27066*\dy})
	-- ({1.55493*\dx},{-0.27093*\dy})
	-- ({1.58666*\dx},{-0.27093*\dy})
	-- ({1.61840*\dx},{-0.27066*\dy})
	-- ({1.65013*\dx},{-0.27011*\dy})
	-- ({1.68186*\dx},{-0.26930*\dy})
	-- ({1.71360*\dx},{-0.26821*\dy})
	-- ({1.74533*\dx},{-0.26685*\dy})
	-- ({1.77706*\dx},{-0.26522*\dy})
	-- ({1.80880*\dx},{-0.26333*\dy})
	-- ({1.84053*\dx},{-0.26117*\dy})
	-- ({1.87226*\dx},{-0.25875*\dy})
	-- ({1.90400*\dx},{-0.25606*\dy})
	-- ({1.93573*\dx},{-0.25312*\dy})
	-- ({1.96746*\dx},{-0.24993*\dy})
	-- ({1.99920*\dx},{-0.24648*\dy})
	-- ({2.03093*\dx},{-0.24278*\dy})
	-- ({2.06266*\dx},{-0.23884*\dy})
	-- ({2.09440*\dx},{-0.23466*\dy})
	-- ({2.12613*\dx},{-0.23025*\dy})
	-- ({2.15786*\dx},{-0.22560*\dy})
	-- ({2.18959*\dx},{-0.22072*\dy})
	-- ({2.22133*\dx},{-0.21562*\dy})
	-- ({2.25306*\dx},{-0.21031*\dy})
	-- ({2.28479*\dx},{-0.20478*\dy})
	-- ({2.31653*\dx},{-0.19905*\dy})
	-- ({2.34826*\dx},{-0.19312*\dy})
	-- ({2.37999*\dx},{-0.18699*\dy})
	-- ({2.41173*\dx},{-0.18067*\dy})
	-- ({2.44346*\dx},{-0.17417*\dy})
	-- ({2.47519*\dx},{-0.16750*\dy})
	-- ({2.50693*\dx},{-0.16066*\dy})
	-- ({2.53866*\dx},{-0.15365*\dy})
	-- ({2.57039*\dx},{-0.14650*\dy})
	-- ({2.60213*\dx},{-0.13919*\dy})
	-- ({2.63386*\dx},{-0.13174*\dy})
	-- ({2.66559*\dx},{-0.12416*\dy})
	-- ({2.69733*\dx},{-0.11646*\dy})
	-- ({2.72906*\dx},{-0.10864*\dy})
	-- ({2.76079*\dx},{-0.10071*\dy})
	-- ({2.79253*\dx},{-0.09268*\dy})
	-- ({2.82426*\dx},{-0.08455*\dy})
	-- ({2.85599*\dx},{-0.07634*\dy})
	-- ({2.88773*\dx},{-0.06805*\dy})
	-- ({2.91946*\dx},{-0.05970*\dy})
	-- ({2.95119*\dx},{-0.05128*\dy})
	-- ({2.98293*\dx},{-0.04281*\dy})
	-- ({3.01466*\dx},{-0.03430*\dy})
	-- ({3.04639*\dx},{-0.02576*\dy})
	-- ({3.07813*\dx},{-0.01719*\dy})
	-- ({3.10986*\dx},{-0.00860*\dy})
	-- ({3.14159*\dx},{-0.00000*\dy})
}
\def\eone{
	({0.00000*\dx},{0.0000*\dy})
	-- ({0.03173*\dx},{0.20070*\dy})
	-- ({0.06347*\dx},{0.37988*\dy})
	-- ({0.09520*\dx},{0.53844*\dy})
	-- ({0.12693*\dx},{0.67731*\dy})
	-- ({0.15867*\dx},{0.79737*\dy})
	-- ({0.19040*\dx},{0.89953*\dy})
	-- ({0.22213*\dx},{0.98469*\dy})
	-- ({0.25387*\dx},{1.05374*\dy})
	-- ({0.28560*\dx},{1.10756*\dy})
	-- ({0.31733*\dx},{1.14702*\dy})
	-- ({0.34907*\dx},{1.17298*\dy})
	-- ({0.38080*\dx},{1.18631*\dy})
	-- ({0.41253*\dx},{1.18785*\dy})
	-- ({0.44427*\dx},{1.17843*\dy})
	-- ({0.47600*\dx},{1.15888*\dy})
	-- ({0.50773*\dx},{1.13000*\dy})
	-- ({0.53947*\dx},{1.09259*\dy})
	-- ({0.57120*\dx},{1.04743*\dy})
	-- ({0.60293*\dx},{0.99529*\dy})
	-- ({0.63467*\dx},{0.93692*\dy})
	-- ({0.66640*\dx},{0.87306*\dy})
	-- ({0.69813*\dx},{0.80441*\dy})
	-- ({0.72986*\dx},{0.73168*\dy})
	-- ({0.76160*\dx},{0.65555*\dy})
	-- ({0.79333*\dx},{0.57668*\dy})
	-- ({0.82506*\dx},{0.49571*\dy})
	-- ({0.85680*\dx},{0.41326*\dy})
	-- ({0.88853*\dx},{0.32994*\dy})
	-- ({0.92026*\dx},{0.24631*\dy})
	-- ({0.95200*\dx},{0.16293*\dy})
	-- ({0.98373*\dx},{0.08034*\dy})
	-- ({1.01546*\dx},{-0.00096*\dy})
	-- ({1.04720*\dx},{-0.08047*\dy})
	-- ({1.07893*\dx},{-0.15775*\dy})
	-- ({1.11066*\dx},{-0.23234*\dy})
	-- ({1.14240*\dx},{-0.30385*\dy})
	-- ({1.17413*\dx},{-0.37189*\dy})
	-- ({1.20586*\dx},{-0.43609*\dy})
	-- ({1.23760*\dx},{-0.49612*\dy})
	-- ({1.26933*\dx},{-0.55168*\dy})
	-- ({1.30106*\dx},{-0.60249*\dy})
	-- ({1.33280*\dx},{-0.64829*\dy})
	-- ({1.36453*\dx},{-0.68887*\dy})
	-- ({1.39626*\dx},{-0.72401*\dy})
	-- ({1.42800*\dx},{-0.75356*\dy})
	-- ({1.45973*\dx},{-0.77736*\dy})
	-- ({1.49146*\dx},{-0.79532*\dy})
	-- ({1.52320*\dx},{-0.80734*\dy})
	-- ({1.55493*\dx},{-0.81336*\dy})
	-- ({1.58666*\dx},{-0.81336*\dy})
	-- ({1.61840*\dx},{-0.80734*\dy})
	-- ({1.65013*\dx},{-0.79532*\dy})
	-- ({1.68186*\dx},{-0.77736*\dy})
	-- ({1.71360*\dx},{-0.75355*\dy})
	-- ({1.74533*\dx},{-0.72401*\dy})
	-- ({1.77706*\dx},{-0.68886*\dy})
	-- ({1.80880*\dx},{-0.64829*\dy})
	-- ({1.84053*\dx},{-0.60249*\dy})
	-- ({1.87226*\dx},{-0.55168*\dy})
	-- ({1.90400*\dx},{-0.49612*\dy})
	-- ({1.93573*\dx},{-0.43608*\dy})
	-- ({1.96746*\dx},{-0.37188*\dy})
	-- ({1.99920*\dx},{-0.30384*\dy})
	-- ({2.03093*\dx},{-0.23233*\dy})
	-- ({2.06266*\dx},{-0.15774*\dy})
	-- ({2.09440*\dx},{-0.08046*\dy})
	-- ({2.12613*\dx},{-0.00095*\dy})
	-- ({2.15786*\dx},{0.08035*\dy})
	-- ({2.18959*\dx},{0.16294*\dy})
	-- ({2.22133*\dx},{0.24632*\dy})
	-- ({2.25306*\dx},{0.32995*\dy})
	-- ({2.28479*\dx},{0.41328*\dy})
	-- ({2.31653*\dx},{0.49573*\dy})
	-- ({2.34826*\dx},{0.57670*\dy})
	-- ({2.37999*\dx},{0.65557*\dy})
	-- ({2.41173*\dx},{0.73170*\dy})
	-- ({2.44346*\dx},{0.80443*\dy})
	-- ({2.47519*\dx},{0.87307*\dy})
	-- ({2.50693*\dx},{0.93694*\dy})
	-- ({2.53866*\dx},{0.99531*\dy})
	-- ({2.57039*\dx},{1.04745*\dy})
	-- ({2.60213*\dx},{1.09261*\dy})
	-- ({2.63386*\dx},{1.13002*\dy})
	-- ({2.66559*\dx},{1.15890*\dy})
	-- ({2.69733*\dx},{1.17845*\dy})
	-- ({2.72906*\dx},{1.18787*\dy})
	-- ({2.76079*\dx},{1.18634*\dy})
	-- ({2.79253*\dx},{1.17301*\dy})
	-- ({2.82426*\dx},{1.14704*\dy})
	-- ({2.85599*\dx},{1.10758*\dy})
	-- ({2.88773*\dx},{1.05377*\dy})
	-- ({2.91946*\dx},{0.98472*\dy})
	-- ({2.95119*\dx},{0.89956*\dy})
	-- ({2.98293*\dx},{0.79739*\dy})
	-- ({3.01466*\dx},{0.67733*\dy})
	-- ({3.04639*\dx},{0.53847*\dy})
	-- ({3.07813*\dx},{0.37991*\dy})
	-- ({3.10986*\dx},{0.20073*\dy})
	-- ({3.14159*\dx},{0.00003*\dy})
}
\def\ltwo{
	({0.00000*\dx},{0.00000*\dy})
	-- ({0.03173*\dx},{-0.00960*\dy})
	-- ({0.06347*\dx},{-0.01918*\dy})
	-- ({0.09520*\dx},{-0.02873*\dy})
	-- ({0.12693*\dx},{-0.03822*\dy})
	-- ({0.15867*\dx},{-0.04765*\dy})
	-- ({0.19040*\dx},{-0.05699*\dy})
	-- ({0.22213*\dx},{-0.06622*\dy})
	-- ({0.25387*\dx},{-0.07534*\dy})
	-- ({0.28560*\dx},{-0.08433*\dy})
	-- ({0.31733*\dx},{-0.09316*\dy})
	-- ({0.34907*\dx},{-0.10184*\dy})
	-- ({0.38080*\dx},{-0.11034*\dy})
	-- ({0.41253*\dx},{-0.11865*\dy})
	-- ({0.44427*\dx},{-0.12677*\dy})
	-- ({0.47600*\dx},{-0.13468*\dy})
	-- ({0.50773*\dx},{-0.14236*\dy})
	-- ({0.53947*\dx},{-0.14982*\dy})
	-- ({0.57120*\dx},{-0.15705*\dy})
	-- ({0.60293*\dx},{-0.16404*\dy})
	-- ({0.63467*\dx},{-0.17078*\dy})
	-- ({0.66640*\dx},{-0.17727*\dy})
	-- ({0.69813*\dx},{-0.18350*\dy})
	-- ({0.72986*\dx},{-0.18948*\dy})
	-- ({0.76160*\dx},{-0.19520*\dy})
	-- ({0.79333*\dx},{-0.20066*\dy})
	-- ({0.82506*\dx},{-0.20586*\dy})
	-- ({0.85680*\dx},{-0.21080*\dy})
	-- ({0.88853*\dx},{-0.21549*\dy})
	-- ({0.92026*\dx},{-0.21992*\dy})
	-- ({0.95200*\dx},{-0.22409*\dy})
	-- ({0.98373*\dx},{-0.22802*\dy})
	-- ({1.01546*\dx},{-0.23170*\dy})
	-- ({1.04720*\dx},{-0.23514*\dy})
	-- ({1.07893*\dx},{-0.23835*\dy})
	-- ({1.11066*\dx},{-0.24132*\dy})
	-- ({1.14240*\dx},{-0.24406*\dy})
	-- ({1.17413*\dx},{-0.24659*\dy})
	-- ({1.20586*\dx},{-0.24889*\dy})
	-- ({1.23760*\dx},{-0.25098*\dy})
	-- ({1.26933*\dx},{-0.25287*\dy})
	-- ({1.30106*\dx},{-0.25455*\dy})
	-- ({1.33280*\dx},{-0.25603*\dy})
	-- ({1.36453*\dx},{-0.25732*\dy})
	-- ({1.39626*\dx},{-0.25842*\dy})
	-- ({1.42800*\dx},{-0.25933*\dy})
	-- ({1.45973*\dx},{-0.26005*\dy})
	-- ({1.49146*\dx},{-0.26060*\dy})
	-- ({1.52320*\dx},{-0.26096*\dy})
	-- ({1.55493*\dx},{-0.26114*\dy})
	-- ({1.58666*\dx},{-0.26114*\dy})
	-- ({1.61840*\dx},{-0.26096*\dy})
	-- ({1.65013*\dx},{-0.26060*\dy})
	-- ({1.68186*\dx},{-0.26005*\dy})
	-- ({1.71360*\dx},{-0.25933*\dy})
	-- ({1.74533*\dx},{-0.25842*\dy})
	-- ({1.77706*\dx},{-0.25732*\dy})
	-- ({1.80880*\dx},{-0.25603*\dy})
	-- ({1.84053*\dx},{-0.25455*\dy})
	-- ({1.87226*\dx},{-0.25287*\dy})
	-- ({1.90400*\dx},{-0.25098*\dy})
	-- ({1.93573*\dx},{-0.24889*\dy})
	-- ({1.96746*\dx},{-0.24659*\dy})
	-- ({1.99920*\dx},{-0.24406*\dy})
	-- ({2.03093*\dx},{-0.24132*\dy})
	-- ({2.06266*\dx},{-0.23835*\dy})
	-- ({2.09440*\dx},{-0.23514*\dy})
	-- ({2.12613*\dx},{-0.23170*\dy})
	-- ({2.15786*\dx},{-0.22802*\dy})
	-- ({2.18959*\dx},{-0.22409*\dy})
	-- ({2.22133*\dx},{-0.21992*\dy})
	-- ({2.25306*\dx},{-0.21549*\dy})
	-- ({2.28479*\dx},{-0.21080*\dy})
	-- ({2.31653*\dx},{-0.20586*\dy})
	-- ({2.34826*\dx},{-0.20066*\dy})
	-- ({2.37999*\dx},{-0.19520*\dy})
	-- ({2.41173*\dx},{-0.18948*\dy})
	-- ({2.44346*\dx},{-0.18350*\dy})
	-- ({2.47519*\dx},{-0.17727*\dy})
	-- ({2.50693*\dx},{-0.17078*\dy})
	-- ({2.53866*\dx},{-0.16404*\dy})
	-- ({2.57039*\dx},{-0.15705*\dy})
	-- ({2.60213*\dx},{-0.14982*\dy})
	-- ({2.63386*\dx},{-0.14236*\dy})
	-- ({2.66559*\dx},{-0.13468*\dy})
	-- ({2.69733*\dx},{-0.12677*\dy})
	-- ({2.72906*\dx},{-0.11865*\dy})
	-- ({2.76079*\dx},{-0.11034*\dy})
	-- ({2.79253*\dx},{-0.10184*\dy})
	-- ({2.82426*\dx},{-0.09316*\dy})
	-- ({2.85599*\dx},{-0.08433*\dy})
	-- ({2.88773*\dx},{-0.07534*\dy})
	-- ({2.91946*\dx},{-0.06622*\dy})
	-- ({2.95119*\dx},{-0.05699*\dy})
	-- ({2.98293*\dx},{-0.04765*\dy})
	-- ({3.01466*\dx},{-0.03822*\dy})
	-- ({3.04639*\dx},{-0.02873*\dy})
	-- ({3.07813*\dx},{-0.01918*\dy})
	-- ({3.10986*\dx},{-0.00960*\dy})
	-- ({3.14159*\dx},{-0.00000*\dy})
}
\def\etwo{
	({0.00000*\dx},{0.0000*\dy})
	-- ({0.03173*\dx},{0.10044*\dy})
	-- ({0.06347*\dx},{0.18025*\dy})
	-- ({0.09520*\dx},{0.24122*\dy})
	-- ({0.12693*\dx},{0.28514*\dy})
	-- ({0.15867*\dx},{0.31376*\dy})
	-- ({0.19040*\dx},{0.32879*\dy})
	-- ({0.22213*\dx},{0.33190*\dy})
	-- ({0.25387*\dx},{0.32471*\dy})
	-- ({0.28560*\dx},{0.30877*\dy})
	-- ({0.31733*\dx},{0.28559*\dy})
	-- ({0.34907*\dx},{0.25658*\dy})
	-- ({0.38080*\dx},{0.22308*\dy})
	-- ({0.41253*\dx},{0.18634*\dy})
	-- ({0.44427*\dx},{0.14754*\dy})
	-- ({0.47600*\dx},{0.10775*\dy})
	-- ({0.50773*\dx},{0.06794*\dy})
	-- ({0.53947*\dx},{0.02901*\dy})
	-- ({0.57120*\dx},{-0.00827*\dy})
	-- ({0.60293*\dx},{-0.04321*\dy})
	-- ({0.63467*\dx},{-0.07523*\dy})
	-- ({0.66640*\dx},{-0.10384*\dy})
	-- ({0.69813*\dx},{-0.12867*\dy})
	-- ({0.72986*\dx},{-0.14941*\dy})
	-- ({0.76160*\dx},{-0.16587*\dy})
	-- ({0.79333*\dx},{-0.17794*\dy})
	-- ({0.82506*\dx},{-0.18560*\dy})
	-- ({0.85680*\dx},{-0.18889*\dy})
	-- ({0.88853*\dx},{-0.18794*\dy})
	-- ({0.92026*\dx},{-0.18296*\dy})
	-- ({0.95200*\dx},{-0.17419*\dy})
	-- ({0.98373*\dx},{-0.16194*\dy})
	-- ({1.01546*\dx},{-0.14657*\dy})
	-- ({1.04720*\dx},{-0.12849*\dy})
	-- ({1.07893*\dx},{-0.10811*\dy})
	-- ({1.11066*\dx},{-0.08590*\dy})
	-- ({1.14240*\dx},{-0.06233*\dy})
	-- ({1.17413*\dx},{-0.03788*\dy})
	-- ({1.20586*\dx},{-0.01303*\dy})
	-- ({1.23760*\dx},{0.01173*\dy})
	-- ({1.26933*\dx},{0.03595*\dy})
	-- ({1.30106*\dx},{0.05918*\dy})
	-- ({1.33280*\dx},{0.08099*\dy})
	-- ({1.36453*\dx},{0.10099*\dy})
	-- ({1.39626*\dx},{0.11883*\dy})
	-- ({1.42800*\dx},{0.13419*\dy})
	-- ({1.45973*\dx},{0.14681*\dy})
	-- ({1.49146*\dx},{0.15647*\dy})
	-- ({1.52320*\dx},{0.16301*\dy})
	-- ({1.55493*\dx},{0.16630*\dy})
	-- ({1.58666*\dx},{0.16630*\dy})
	-- ({1.61840*\dx},{0.16301*\dy})
	-- ({1.65013*\dx},{0.15647*\dy})
	-- ({1.68186*\dx},{0.14681*\dy})
	-- ({1.71360*\dx},{0.13419*\dy})
	-- ({1.74533*\dx},{0.11883*\dy})
	-- ({1.77706*\dx},{0.10099*\dy})
	-- ({1.80880*\dx},{0.08099*\dy})
	-- ({1.84053*\dx},{0.05918*\dy})
	-- ({1.87226*\dx},{0.03596*\dy})
	-- ({1.90400*\dx},{0.01174*\dy})
	-- ({1.93573*\dx},{-0.01303*\dy})
	-- ({1.96746*\dx},{-0.03787*\dy})
	-- ({1.99920*\dx},{-0.06233*\dy})
	-- ({2.03093*\dx},{-0.08590*\dy})
	-- ({2.06266*\dx},{-0.10810*\dy})
	-- ({2.09440*\dx},{-0.12848*\dy})
	-- ({2.12613*\dx},{-0.14656*\dy})
	-- ({2.15786*\dx},{-0.16193*\dy})
	-- ({2.18959*\dx},{-0.17417*\dy})
	-- ({2.22133*\dx},{-0.18294*\dy})
	-- ({2.25306*\dx},{-0.18793*\dy})
	-- ({2.28479*\dx},{-0.18888*\dy})
	-- ({2.31653*\dx},{-0.18558*\dy})
	-- ({2.34826*\dx},{-0.17793*\dy})
	-- ({2.37999*\dx},{-0.16586*\dy})
	-- ({2.41173*\dx},{-0.14939*\dy})
	-- ({2.44346*\dx},{-0.12865*\dy})
	-- ({2.47519*\dx},{-0.10383*\dy})
	-- ({2.50693*\dx},{-0.07521*\dy})
	-- ({2.53866*\dx},{-0.04319*\dy})
	-- ({2.57039*\dx},{-0.00825*\dy})
	-- ({2.60213*\dx},{0.02903*\dy})
	-- ({2.63386*\dx},{0.06796*\dy})
	-- ({2.66559*\dx},{0.10777*\dy})
	-- ({2.69733*\dx},{0.14756*\dy})
	-- ({2.72906*\dx},{0.18637*\dy})
	-- ({2.76079*\dx},{0.22310*\dy})
	-- ({2.79253*\dx},{0.25661*\dy})
	-- ({2.82426*\dx},{0.28562*\dy})
	-- ({2.85599*\dx},{0.30880*\dy})
	-- ({2.88773*\dx},{0.32473*\dy})
	-- ({2.91946*\dx},{0.33192*\dy})
	-- ({2.95119*\dx},{0.32881*\dy})
	-- ({2.98293*\dx},{0.31379*\dy})
	-- ({3.01466*\dx},{0.28517*\dy})
	-- ({3.04639*\dx},{0.24125*\dy})
	-- ({3.07813*\dx},{0.18028*\dy})
	-- ({3.10986*\dx},{0.10047*\dy})
	-- ({3.14159*\dx},{0.00003*\dy})
}
\def\lthree{
	({0.00000*\dx},{0.00000*\dy})
	-- ({0.03173*\dx},{-0.00996*\dy})
	-- ({0.06347*\dx},{-0.01990*\dy})
	-- ({0.09520*\dx},{-0.02978*\dy})
	-- ({0.12693*\dx},{-0.03958*\dy})
	-- ({0.15867*\dx},{-0.04929*\dy})
	-- ({0.19040*\dx},{-0.05886*\dy})
	-- ({0.22213*\dx},{-0.06829*\dy})
	-- ({0.25387*\dx},{-0.07754*\dy})
	-- ({0.28560*\dx},{-0.08661*\dy})
	-- ({0.31733*\dx},{-0.09548*\dy})
	-- ({0.34907*\dx},{-0.10413*\dy})
	-- ({0.38080*\dx},{-0.11254*\dy})
	-- ({0.41253*\dx},{-0.12072*\dy})
	-- ({0.44427*\dx},{-0.12864*\dy})
	-- ({0.47600*\dx},{-0.13632*\dy})
	-- ({0.50773*\dx},{-0.14373*\dy})
	-- ({0.53947*\dx},{-0.15089*\dy})
	-- ({0.57120*\dx},{-0.15779*\dy})
	-- ({0.60293*\dx},{-0.16443*\dy})
	-- ({0.63467*\dx},{-0.17081*\dy})
	-- ({0.66640*\dx},{-0.17695*\dy})
	-- ({0.69813*\dx},{-0.18284*\dy})
	-- ({0.72986*\dx},{-0.18849*\dy})
	-- ({0.76160*\dx},{-0.19391*\dy})
	-- ({0.79333*\dx},{-0.19911*\dy})
	-- ({0.82506*\dx},{-0.20408*\dy})
	-- ({0.85680*\dx},{-0.20885*\dy})
	-- ({0.88853*\dx},{-0.21341*\dy})
	-- ({0.92026*\dx},{-0.21776*\dy})
	-- ({0.95200*\dx},{-0.22192*\dy})
	-- ({0.98373*\dx},{-0.22589*\dy})
	-- ({1.01546*\dx},{-0.22966*\dy})
	-- ({1.04720*\dx},{-0.23325*\dy})
	-- ({1.07893*\dx},{-0.23665*\dy})
	-- ({1.11066*\dx},{-0.23986*\dy})
	-- ({1.14240*\dx},{-0.24288*\dy})
	-- ({1.17413*\dx},{-0.24571*\dy})
	-- ({1.20586*\dx},{-0.24834*\dy})
	-- ({1.23760*\dx},{-0.25077*\dy})
	-- ({1.26933*\dx},{-0.25301*\dy})
	-- ({1.30106*\dx},{-0.25504*\dy})
	-- ({1.33280*\dx},{-0.25685*\dy})
	-- ({1.36453*\dx},{-0.25845*\dy})
	-- ({1.39626*\dx},{-0.25984*\dy})
	-- ({1.42800*\dx},{-0.26100*\dy})
	-- ({1.45973*\dx},{-0.26193*\dy})
	-- ({1.49146*\dx},{-0.26263*\dy})
	-- ({1.52320*\dx},{-0.26310*\dy})
	-- ({1.55493*\dx},{-0.26333*\dy})
	-- ({1.58666*\dx},{-0.26333*\dy})
	-- ({1.61840*\dx},{-0.26310*\dy})
	-- ({1.65013*\dx},{-0.26263*\dy})
	-- ({1.68186*\dx},{-0.26193*\dy})
	-- ({1.71360*\dx},{-0.26100*\dy})
	-- ({1.74533*\dx},{-0.25984*\dy})
	-- ({1.77706*\dx},{-0.25845*\dy})
	-- ({1.80880*\dx},{-0.25685*\dy})
	-- ({1.84053*\dx},{-0.25504*\dy})
	-- ({1.87226*\dx},{-0.25301*\dy})
	-- ({1.90400*\dx},{-0.25077*\dy})
	-- ({1.93573*\dx},{-0.24834*\dy})
	-- ({1.96746*\dx},{-0.24571*\dy})
	-- ({1.99920*\dx},{-0.24288*\dy})
	-- ({2.03093*\dx},{-0.23986*\dy})
	-- ({2.06266*\dx},{-0.23665*\dy})
	-- ({2.09440*\dx},{-0.23325*\dy})
	-- ({2.12613*\dx},{-0.22966*\dy})
	-- ({2.15786*\dx},{-0.22589*\dy})
	-- ({2.18959*\dx},{-0.22192*\dy})
	-- ({2.22133*\dx},{-0.21776*\dy})
	-- ({2.25306*\dx},{-0.21341*\dy})
	-- ({2.28479*\dx},{-0.20885*\dy})
	-- ({2.31653*\dx},{-0.20408*\dy})
	-- ({2.34826*\dx},{-0.19911*\dy})
	-- ({2.37999*\dx},{-0.19391*\dy})
	-- ({2.41173*\dx},{-0.18849*\dy})
	-- ({2.44346*\dx},{-0.18284*\dy})
	-- ({2.47519*\dx},{-0.17695*\dy})
	-- ({2.50693*\dx},{-0.17081*\dy})
	-- ({2.53866*\dx},{-0.16443*\dy})
	-- ({2.57039*\dx},{-0.15779*\dy})
	-- ({2.60213*\dx},{-0.15089*\dy})
	-- ({2.63386*\dx},{-0.14373*\dy})
	-- ({2.66559*\dx},{-0.13632*\dy})
	-- ({2.69733*\dx},{-0.12864*\dy})
	-- ({2.72906*\dx},{-0.12072*\dy})
	-- ({2.76079*\dx},{-0.11254*\dy})
	-- ({2.79253*\dx},{-0.10413*\dy})
	-- ({2.82426*\dx},{-0.09548*\dy})
	-- ({2.85599*\dx},{-0.08661*\dy})
	-- ({2.88773*\dx},{-0.07754*\dy})
	-- ({2.91946*\dx},{-0.06829*\dy})
	-- ({2.95119*\dx},{-0.05886*\dy})
	-- ({2.98293*\dx},{-0.04929*\dy})
	-- ({3.01466*\dx},{-0.03958*\dy})
	-- ({3.04639*\dx},{-0.02978*\dy})
	-- ({3.07813*\dx},{-0.01990*\dy})
	-- ({3.10986*\dx},{-0.00996*\dy})
	-- ({3.14159*\dx},{-0.00000*\dy})
}
\def\ethree{
	({0.00000*\dx},{0.0000*\dy})
	-- ({0.03173*\dx},{0.06423*\dy})
	-- ({0.06347*\dx},{0.10871*\dy})
	-- ({0.09520*\dx},{0.13613*\dy})
	-- ({0.12693*\dx},{0.14909*\dy})
	-- ({0.15867*\dx},{0.15011*\dy})
	-- ({0.19040*\dx},{0.14157*\dy})
	-- ({0.22213*\dx},{0.12572*\dy})
	-- ({0.25387*\dx},{0.10465*\dy})
	-- ({0.28560*\dx},{0.08025*\dy})
	-- ({0.31733*\dx},{0.05422*\dy})
	-- ({0.34907*\dx},{0.02803*\dy})
	-- ({0.38080*\dx},{0.00294*\dy})
	-- ({0.41253*\dx},{-0.02001*\dy})
	-- ({0.44427*\dx},{-0.04001*\dy})
	-- ({0.47600*\dx},{-0.05646*\dy})
	-- ({0.50773*\dx},{-0.06897*\dy})
	-- ({0.53947*\dx},{-0.07735*\dy})
	-- ({0.57120*\dx},{-0.08159*\dy})
	-- ({0.60293*\dx},{-0.08184*\dy})
	-- ({0.63467*\dx},{-0.07838*\dy})
	-- ({0.66640*\dx},{-0.07163*\dy})
	-- ({0.69813*\dx},{-0.06208*\dy})
	-- ({0.72986*\dx},{-0.05031*\dy})
	-- ({0.76160*\dx},{-0.03692*\dy})
	-- ({0.79333*\dx},{-0.02256*\dy})
	-- ({0.82506*\dx},{-0.00786*\dy})
	-- ({0.85680*\dx},{0.00658*\dy})
	-- ({0.88853*\dx},{0.02020*\dy})
	-- ({0.92026*\dx},{0.03247*\dy})
	-- ({0.95200*\dx},{0.04299*\dy})
	-- ({0.98373*\dx},{0.05139*\dy})
	-- ({1.01546*\dx},{0.05743*\dy})
	-- ({1.04720*\dx},{0.06095*\dy})
	-- ({1.07893*\dx},{0.06191*\dy})
	-- ({1.11066*\dx},{0.06034*\dy})
	-- ({1.14240*\dx},{0.05638*\dy})
	-- ({1.17413*\dx},{0.05025*\dy})
	-- ({1.20586*\dx},{0.04224*\dy})
	-- ({1.23760*\dx},{0.03271*\dy})
	-- ({1.26933*\dx},{0.02207*\dy})
	-- ({1.30106*\dx},{0.01075*\dy})
	-- ({1.33280*\dx},{-0.00080*\dy})
	-- ({1.36453*\dx},{-0.01212*\dy})
	-- ({1.39626*\dx},{-0.02276*\dy})
	-- ({1.42800*\dx},{-0.03233*\dy})
	-- ({1.45973*\dx},{-0.04046*\dy})
	-- ({1.49146*\dx},{-0.04683*\dy})
	-- ({1.52320*\dx},{-0.05122*\dy})
	-- ({1.55493*\dx},{-0.05346*\dy})
	-- ({1.58666*\dx},{-0.05346*\dy})
	-- ({1.61840*\dx},{-0.05122*\dy})
	-- ({1.65013*\dx},{-0.04683*\dy})
	-- ({1.68186*\dx},{-0.04045*\dy})
	-- ({1.71360*\dx},{-0.03233*\dy})
	-- ({1.74533*\dx},{-0.02276*\dy})
	-- ({1.77706*\dx},{-0.01211*\dy})
	-- ({1.80880*\dx},{-0.00080*\dy})
	-- ({1.84053*\dx},{0.01075*\dy})
	-- ({1.87226*\dx},{0.02207*\dy})
	-- ({1.90400*\dx},{0.03272*\dy})
	-- ({1.93573*\dx},{0.04225*\dy})
	-- ({1.96746*\dx},{0.05025*\dy})
	-- ({1.99920*\dx},{0.05639*\dy})
	-- ({2.03093*\dx},{0.06035*\dy})
	-- ({2.06266*\dx},{0.06192*\dy})
	-- ({2.09440*\dx},{0.06096*\dy})
	-- ({2.12613*\dx},{0.05744*\dy})
	-- ({2.15786*\dx},{0.05140*\dy})
	-- ({2.18959*\dx},{0.04300*\dy})
	-- ({2.22133*\dx},{0.03249*\dy})
	-- ({2.25306*\dx},{0.02021*\dy})
	-- ({2.28479*\dx},{0.00660*\dy})
	-- ({2.31653*\dx},{-0.00784*\dy})
	-- ({2.34826*\dx},{-0.02255*\dy})
	-- ({2.37999*\dx},{-0.03691*\dy})
	-- ({2.41173*\dx},{-0.05029*\dy})
	-- ({2.44346*\dx},{-0.06206*\dy})
	-- ({2.47519*\dx},{-0.07161*\dy})
	-- ({2.50693*\dx},{-0.07836*\dy})
	-- ({2.53866*\dx},{-0.08182*\dy})
	-- ({2.57039*\dx},{-0.08157*\dy})
	-- ({2.60213*\dx},{-0.07733*\dy})
	-- ({2.63386*\dx},{-0.06895*\dy})
	-- ({2.66559*\dx},{-0.05643*\dy})
	-- ({2.69733*\dx},{-0.03999*\dy})
	-- ({2.72906*\dx},{-0.01999*\dy})
	-- ({2.76079*\dx},{0.00296*\dy})
	-- ({2.79253*\dx},{0.02805*\dy})
	-- ({2.82426*\dx},{0.05424*\dy})
	-- ({2.85599*\dx},{0.08028*\dy})
	-- ({2.88773*\dx},{0.10468*\dy})
	-- ({2.91946*\dx},{0.12575*\dy})
	-- ({2.95119*\dx},{0.14159*\dy})
	-- ({2.98293*\dx},{0.15013*\dy})
	-- ({3.01466*\dx},{0.14912*\dy})
	-- ({3.04639*\dx},{0.13616*\dy})
	-- ({3.07813*\dx},{0.10874*\dy})
	-- ({3.10986*\dx},{0.06425*\dy})
	-- ({3.14159*\dx},{0.00003*\dy})
}
\def\lfour{
	({0.00000*\dx},{0.00000*\dy})
	-- ({0.03173*\dx},{-0.01015*\dy})
	-- ({0.06347*\dx},{-0.02026*\dy})
	-- ({0.09520*\dx},{-0.03030*\dy})
	-- ({0.12693*\dx},{-0.04023*\dy})
	-- ({0.15867*\dx},{-0.05004*\dy})
	-- ({0.19040*\dx},{-0.05968*\dy})
	-- ({0.22213*\dx},{-0.06913*\dy})
	-- ({0.25387*\dx},{-0.07837*\dy})
	-- ({0.28560*\dx},{-0.08739*\dy})
	-- ({0.31733*\dx},{-0.09616*\dy})
	-- ({0.34907*\dx},{-0.10468*\dy})
	-- ({0.38080*\dx},{-0.11295*\dy})
	-- ({0.41253*\dx},{-0.12096*\dy})
	-- ({0.44427*\dx},{-0.12870*\dy})
	-- ({0.47600*\dx},{-0.13619*\dy})
	-- ({0.50773*\dx},{-0.14344*\dy})
	-- ({0.53947*\dx},{-0.15043*\dy})
	-- ({0.57120*\dx},{-0.15719*\dy})
	-- ({0.60293*\dx},{-0.16373*\dy})
	-- ({0.63467*\dx},{-0.17004*\dy})
	-- ({0.66640*\dx},{-0.17615*\dy})
	-- ({0.69813*\dx},{-0.18204*\dy})
	-- ({0.72986*\dx},{-0.18774*\dy})
	-- ({0.76160*\dx},{-0.19325*\dy})
	-- ({0.79333*\dx},{-0.19856*\dy})
	-- ({0.82506*\dx},{-0.20368*\dy})
	-- ({0.85680*\dx},{-0.20861*\dy})
	-- ({0.88853*\dx},{-0.21334*\dy})
	-- ({0.92026*\dx},{-0.21788*\dy})
	-- ({0.95200*\dx},{-0.22221*\dy})
	-- ({0.98373*\dx},{-0.22633*\dy})
	-- ({1.01546*\dx},{-0.23025*\dy})
	-- ({1.04720*\dx},{-0.23394*\dy})
	-- ({1.07893*\dx},{-0.23741*\dy})
	-- ({1.11066*\dx},{-0.24066*\dy})
	-- ({1.14240*\dx},{-0.24367*\dy})
	-- ({1.17413*\dx},{-0.24646*\dy})
	-- ({1.20586*\dx},{-0.24901*\dy})
	-- ({1.23760*\dx},{-0.25134*\dy})
	-- ({1.26933*\dx},{-0.25343*\dy})
	-- ({1.30106*\dx},{-0.25530*\dy})
	-- ({1.33280*\dx},{-0.25694*\dy})
	-- ({1.36453*\dx},{-0.25837*\dy})
	-- ({1.39626*\dx},{-0.25958*\dy})
	-- ({1.42800*\dx},{-0.26058*\dy})
	-- ({1.45973*\dx},{-0.26137*\dy})
	-- ({1.49146*\dx},{-0.26196*\dy})
	-- ({1.52320*\dx},{-0.26235*\dy})
	-- ({1.55493*\dx},{-0.26255*\dy})
	-- ({1.58666*\dx},{-0.26255*\dy})
	-- ({1.61840*\dx},{-0.26235*\dy})
	-- ({1.65013*\dx},{-0.26196*\dy})
	-- ({1.68186*\dx},{-0.26137*\dy})
	-- ({1.71360*\dx},{-0.26058*\dy})
	-- ({1.74533*\dx},{-0.25958*\dy})
	-- ({1.77706*\dx},{-0.25837*\dy})
	-- ({1.80880*\dx},{-0.25694*\dy})
	-- ({1.84053*\dx},{-0.25530*\dy})
	-- ({1.87226*\dx},{-0.25343*\dy})
	-- ({1.90400*\dx},{-0.25134*\dy})
	-- ({1.93573*\dx},{-0.24901*\dy})
	-- ({1.96746*\dx},{-0.24646*\dy})
	-- ({1.99920*\dx},{-0.24367*\dy})
	-- ({2.03093*\dx},{-0.24066*\dy})
	-- ({2.06266*\dx},{-0.23741*\dy})
	-- ({2.09440*\dx},{-0.23394*\dy})
	-- ({2.12613*\dx},{-0.23025*\dy})
	-- ({2.15786*\dx},{-0.22633*\dy})
	-- ({2.18959*\dx},{-0.22221*\dy})
	-- ({2.22133*\dx},{-0.21788*\dy})
	-- ({2.25306*\dx},{-0.21334*\dy})
	-- ({2.28479*\dx},{-0.20861*\dy})
	-- ({2.31653*\dx},{-0.20368*\dy})
	-- ({2.34826*\dx},{-0.19856*\dy})
	-- ({2.37999*\dx},{-0.19325*\dy})
	-- ({2.41173*\dx},{-0.18774*\dy})
	-- ({2.44346*\dx},{-0.18204*\dy})
	-- ({2.47519*\dx},{-0.17615*\dy})
	-- ({2.50693*\dx},{-0.17004*\dy})
	-- ({2.53866*\dx},{-0.16373*\dy})
	-- ({2.57039*\dx},{-0.15719*\dy})
	-- ({2.60213*\dx},{-0.15043*\dy})
	-- ({2.63386*\dx},{-0.14344*\dy})
	-- ({2.66559*\dx},{-0.13619*\dy})
	-- ({2.69733*\dx},{-0.12870*\dy})
	-- ({2.72906*\dx},{-0.12096*\dy})
	-- ({2.76079*\dx},{-0.11295*\dy})
	-- ({2.79253*\dx},{-0.10468*\dy})
	-- ({2.82426*\dx},{-0.09616*\dy})
	-- ({2.85599*\dx},{-0.08739*\dy})
	-- ({2.88773*\dx},{-0.07837*\dy})
	-- ({2.91946*\dx},{-0.06913*\dy})
	-- ({2.95119*\dx},{-0.05968*\dy})
	-- ({2.98293*\dx},{-0.05004*\dy})
	-- ({3.01466*\dx},{-0.04023*\dy})
	-- ({3.04639*\dx},{-0.03030*\dy})
	-- ({3.07813*\dx},{-0.02026*\dy})
	-- ({3.10986*\dx},{-0.01015*\dy})
	-- ({3.14159*\dx},{-0.00000*\dy})
}
\def\efour{
	({0.00000*\dx},{0.0000*\dy})
	-- ({0.03173*\dx},{0.04581*\dy})
	-- ({0.06347*\dx},{0.07276*\dy})
	-- ({0.09520*\dx},{0.08439*\dy})
	-- ({0.12693*\dx},{0.08406*\dy})
	-- ({0.15867*\dx},{0.07492*\dy})
	-- ({0.19040*\dx},{0.05986*\dy})
	-- ({0.22213*\dx},{0.04142*\dy})
	-- ({0.25387*\dx},{0.02182*\dy})
	-- ({0.28560*\dx},{0.00286*\dy})
	-- ({0.31733*\dx},{-0.01403*\dy})
	-- ({0.34907*\dx},{-0.02783*\dy})
	-- ({0.38080*\dx},{-0.03790*\dy})
	-- ({0.41253*\dx},{-0.04392*\dy})
	-- ({0.44427*\dx},{-0.04593*\dy})
	-- ({0.47600*\dx},{-0.04421*\dy})
	-- ({0.50773*\dx},{-0.03925*\dy})
	-- ({0.53947*\dx},{-0.03172*\dy})
	-- ({0.57120*\dx},{-0.02238*\dy})
	-- ({0.60293*\dx},{-0.01204*\dy})
	-- ({0.63467*\dx},{-0.00150*\dy})
	-- ({0.66640*\dx},{0.00849*\dy})
	-- ({0.69813*\dx},{0.01729*\dy})
	-- ({0.72986*\dx},{0.02437*\dy})
	-- ({0.76160*\dx},{0.02935*\dy})
	-- ({0.79333*\dx},{0.03202*\dy})
	-- ({0.82506*\dx},{0.03234*\dy})
	-- ({0.85680*\dx},{0.03041*\dy})
	-- ({0.88853*\dx},{0.02647*\dy})
	-- ({0.92026*\dx},{0.02089*\dy})
	-- ({0.95200*\dx},{0.01411*\dy})
	-- ({0.98373*\dx},{0.00666*\dy})
	-- ({1.01546*\dx},{-0.00096*\dy})
	-- ({1.04720*\dx},{-0.00821*\dy})
	-- ({1.07893*\dx},{-0.01462*\dy})
	-- ({1.11066*\dx},{-0.01979*\dy})
	-- ({1.14240*\dx},{-0.02341*\dy})
	-- ({1.17413*\dx},{-0.02529*\dy})
	-- ({1.20586*\dx},{-0.02534*\dy})
	-- ({1.23760*\dx},{-0.02360*\dy})
	-- ({1.26933*\dx},{-0.02023*\dy})
	-- ({1.30106*\dx},{-0.01549*\dy})
	-- ({1.33280*\dx},{-0.00971*\dy})
	-- ({1.36453*\dx},{-0.00331*\dy})
	-- ({1.39626*\dx},{0.00328*\dy})
	-- ({1.42800*\dx},{0.00961*\dy})
	-- ({1.45973*\dx},{0.01527*\dy})
	-- ({1.49146*\dx},{0.01987*\dy})
	-- ({1.52320*\dx},{0.02312*\dy})
	-- ({1.55493*\dx},{0.02480*\dy})
	-- ({1.58666*\dx},{0.02480*\dy})
	-- ({1.61840*\dx},{0.02312*\dy})
	-- ({1.65013*\dx},{0.01987*\dy})
	-- ({1.68186*\dx},{0.01527*\dy})
	-- ({1.71360*\dx},{0.00961*\dy})
	-- ({1.74533*\dx},{0.00328*\dy})
	-- ({1.77706*\dx},{-0.00331*\dy})
	-- ({1.80880*\dx},{-0.00971*\dy})
	-- ({1.84053*\dx},{-0.01548*\dy})
	-- ({1.87226*\dx},{-0.02022*\dy})
	-- ({1.90400*\dx},{-0.02359*\dy})
	-- ({1.93573*\dx},{-0.02533*\dy})
	-- ({1.96746*\dx},{-0.02528*\dy})
	-- ({1.99920*\dx},{-0.02340*\dy})
	-- ({2.03093*\dx},{-0.01978*\dy})
	-- ({2.06266*\dx},{-0.01461*\dy})
	-- ({2.09440*\dx},{-0.00819*\dy})
	-- ({2.12613*\dx},{-0.00094*\dy})
	-- ({2.15786*\dx},{0.00667*\dy})
	-- ({2.18959*\dx},{0.01413*\dy})
	-- ({2.22133*\dx},{0.02090*\dy})
	-- ({2.25306*\dx},{0.02648*\dy})
	-- ({2.28479*\dx},{0.03042*\dy})
	-- ({2.31653*\dx},{0.03236*\dy})
	-- ({2.34826*\dx},{0.03204*\dy})
	-- ({2.37999*\dx},{0.02937*\dy})
	-- ({2.41173*\dx},{0.02439*\dy})
	-- ({2.44346*\dx},{0.01731*\dy})
	-- ({2.47519*\dx},{0.00851*\dy})
	-- ({2.50693*\dx},{-0.00148*\dy})
	-- ({2.53866*\dx},{-0.01202*\dy})
	-- ({2.57039*\dx},{-0.02236*\dy})
	-- ({2.60213*\dx},{-0.03170*\dy})
	-- ({2.63386*\dx},{-0.03923*\dy})
	-- ({2.66559*\dx},{-0.04419*\dy})
	-- ({2.69733*\dx},{-0.04591*\dy})
	-- ({2.72906*\dx},{-0.04390*\dy})
	-- ({2.76079*\dx},{-0.03787*\dy})
	-- ({2.79253*\dx},{-0.02781*\dy})
	-- ({2.82426*\dx},{-0.01401*\dy})
	-- ({2.85599*\dx},{0.00289*\dy})
	-- ({2.88773*\dx},{0.02185*\dy})
	-- ({2.91946*\dx},{0.04145*\dy})
	-- ({2.95119*\dx},{0.05988*\dy})
	-- ({2.98293*\dx},{0.07495*\dy})
	-- ({3.01466*\dx},{0.08409*\dy})
	-- ({3.04639*\dx},{0.08442*\dy})
	-- ({3.07813*\dx},{0.07279*\dy})
	-- ({3.10986*\dx},{0.04583*\dy})
	-- ({3.14159*\dx},{0.00003*\dy})
}
\def\lfive{
	({0.00000*\dx},{0.00000*\dy})
	-- ({0.03173*\dx},{-0.01026*\dy})
	-- ({0.06347*\dx},{-0.02047*\dy})
	-- ({0.09520*\dx},{-0.03060*\dy})
	-- ({0.12693*\dx},{-0.04059*\dy})
	-- ({0.15867*\dx},{-0.05043*\dy})
	-- ({0.19040*\dx},{-0.06007*\dy})
	-- ({0.22213*\dx},{-0.06949*\dy})
	-- ({0.25387*\dx},{-0.07868*\dy})
	-- ({0.28560*\dx},{-0.08761*\dy})
	-- ({0.31733*\dx},{-0.09628*\dy})
	-- ({0.34907*\dx},{-0.10470*\dy})
	-- ({0.38080*\dx},{-0.11285*\dy})
	-- ({0.41253*\dx},{-0.12076*\dy})
	-- ({0.44427*\dx},{-0.12842*\dy})
	-- ({0.47600*\dx},{-0.13585*\dy})
	-- ({0.50773*\dx},{-0.14306*\dy})
	-- ({0.53947*\dx},{-0.15005*\dy})
	-- ({0.57120*\dx},{-0.15684*\dy})
	-- ({0.60293*\dx},{-0.16343*\dy})
	-- ({0.63467*\dx},{-0.16983*\dy})
	-- ({0.66640*\dx},{-0.17603*\dy})
	-- ({0.69813*\dx},{-0.18204*\dy})
	-- ({0.72986*\dx},{-0.18784*\dy})
	-- ({0.76160*\dx},{-0.19345*\dy})
	-- ({0.79333*\dx},{-0.19885*\dy})
	-- ({0.82506*\dx},{-0.20403*\dy})
	-- ({0.85680*\dx},{-0.20899*\dy})
	-- ({0.88853*\dx},{-0.21372*\dy})
	-- ({0.92026*\dx},{-0.21823*\dy})
	-- ({0.95200*\dx},{-0.22250*\dy})
	-- ({0.98373*\dx},{-0.22655*\dy})
	-- ({1.01546*\dx},{-0.23036*\dy})
	-- ({1.04720*\dx},{-0.23395*\dy})
	-- ({1.07893*\dx},{-0.23732*\dy})
	-- ({1.11066*\dx},{-0.24046*\dy})
	-- ({1.14240*\dx},{-0.24340*\dy})
	-- ({1.17413*\dx},{-0.24612*\dy})
	-- ({1.20586*\dx},{-0.24865*\dy})
	-- ({1.23760*\dx},{-0.25097*\dy})
	-- ({1.26933*\dx},{-0.25309*\dy})
	-- ({1.30106*\dx},{-0.25502*\dy})
	-- ({1.33280*\dx},{-0.25674*\dy})
	-- ({1.36453*\dx},{-0.25826*\dy})
	-- ({1.39626*\dx},{-0.25958*\dy})
	-- ({1.42800*\dx},{-0.26068*\dy})
	-- ({1.45973*\dx},{-0.26157*\dy})
	-- ({1.49146*\dx},{-0.26225*\dy})
	-- ({1.52320*\dx},{-0.26270*\dy})
	-- ({1.55493*\dx},{-0.26292*\dy})
	-- ({1.58666*\dx},{-0.26292*\dy})
	-- ({1.61840*\dx},{-0.26270*\dy})
	-- ({1.65013*\dx},{-0.26225*\dy})
	-- ({1.68186*\dx},{-0.26157*\dy})
	-- ({1.71360*\dx},{-0.26068*\dy})
	-- ({1.74533*\dx},{-0.25958*\dy})
	-- ({1.77706*\dx},{-0.25826*\dy})
	-- ({1.80880*\dx},{-0.25674*\dy})
	-- ({1.84053*\dx},{-0.25502*\dy})
	-- ({1.87226*\dx},{-0.25309*\dy})
	-- ({1.90400*\dx},{-0.25097*\dy})
	-- ({1.93573*\dx},{-0.24865*\dy})
	-- ({1.96746*\dx},{-0.24612*\dy})
	-- ({1.99920*\dx},{-0.24340*\dy})
	-- ({2.03093*\dx},{-0.24046*\dy})
	-- ({2.06266*\dx},{-0.23732*\dy})
	-- ({2.09440*\dx},{-0.23395*\dy})
	-- ({2.12613*\dx},{-0.23036*\dy})
	-- ({2.15786*\dx},{-0.22655*\dy})
	-- ({2.18959*\dx},{-0.22250*\dy})
	-- ({2.22133*\dx},{-0.21823*\dy})
	-- ({2.25306*\dx},{-0.21372*\dy})
	-- ({2.28479*\dx},{-0.20899*\dy})
	-- ({2.31653*\dx},{-0.20403*\dy})
	-- ({2.34826*\dx},{-0.19885*\dy})
	-- ({2.37999*\dx},{-0.19345*\dy})
	-- ({2.41173*\dx},{-0.18784*\dy})
	-- ({2.44346*\dx},{-0.18204*\dy})
	-- ({2.47519*\dx},{-0.17603*\dy})
	-- ({2.50693*\dx},{-0.16983*\dy})
	-- ({2.53866*\dx},{-0.16343*\dy})
	-- ({2.57039*\dx},{-0.15684*\dy})
	-- ({2.60213*\dx},{-0.15005*\dy})
	-- ({2.63386*\dx},{-0.14306*\dy})
	-- ({2.66559*\dx},{-0.13585*\dy})
	-- ({2.69733*\dx},{-0.12842*\dy})
	-- ({2.72906*\dx},{-0.12076*\dy})
	-- ({2.76079*\dx},{-0.11285*\dy})
	-- ({2.79253*\dx},{-0.10470*\dy})
	-- ({2.82426*\dx},{-0.09628*\dy})
	-- ({2.85599*\dx},{-0.08761*\dy})
	-- ({2.88773*\dx},{-0.07868*\dy})
	-- ({2.91946*\dx},{-0.06949*\dy})
	-- ({2.95119*\dx},{-0.06007*\dy})
	-- ({2.98293*\dx},{-0.05043*\dy})
	-- ({3.01466*\dx},{-0.04059*\dy})
	-- ({3.04639*\dx},{-0.03060*\dy})
	-- ({3.07813*\dx},{-0.02047*\dy})
	-- ({3.10986*\dx},{-0.01026*\dy})
	-- ({3.14159*\dx},{-0.00000*\dy})
}
\def\efive{
	({0.00000*\dx},{0.0000*\dy})
	-- ({0.03173*\dx},{0.03472*\dy})
	-- ({0.06347*\dx},{0.05149*\dy})
	-- ({0.09520*\dx},{0.05462*\dy})
	-- ({0.12693*\dx},{0.04816*\dy})
	-- ({0.15867*\dx},{0.03576*\dy})
	-- ({0.19040*\dx},{0.02054*\dy})
	-- ({0.22213*\dx},{0.00508*\dy})
	-- ({0.25387*\dx},{-0.00868*\dy})
	-- ({0.28560*\dx},{-0.01940*\dy})
	-- ({0.31733*\dx},{-0.02631*\dy})
	-- ({0.34907*\dx},{-0.02922*\dy})
	-- ({0.38080*\dx},{-0.02834*\dy})
	-- ({0.41253*\dx},{-0.02426*\dy})
	-- ({0.44427*\dx},{-0.01780*\dy})
	-- ({0.47600*\dx},{-0.00994*\dy})
	-- ({0.50773*\dx},{-0.00167*\dy})
	-- ({0.53947*\dx},{0.00611*\dy})
	-- ({0.57120*\dx},{0.01261*\dy})
	-- ({0.60293*\dx},{0.01726*\dy})
	-- ({0.63467*\dx},{0.01974*\dy})
	-- ({0.66640*\dx},{0.01995*\dy})
	-- ({0.69813*\dx},{0.01803*\dy})
	-- ({0.72986*\dx},{0.01435*\dy})
	-- ({0.76160*\dx},{0.00939*\dy})
	-- ({0.79333*\dx},{0.00375*\dy})
	-- ({0.82506*\dx},{-0.00196*\dy})
	-- ({0.85680*\dx},{-0.00713*\dy})
	-- ({0.88853*\dx},{-0.01127*\dy})
	-- ({0.92026*\dx},{-0.01400*\dy})
	-- ({0.95200*\dx},{-0.01511*\dy})
	-- ({0.98373*\dx},{-0.01456*\dy})
	-- ({1.01546*\dx},{-0.01246*\dy})
	-- ({1.04720*\dx},{-0.00911*\dy})
	-- ({1.07893*\dx},{-0.00488*\dy})
	-- ({1.11066*\dx},{-0.00023*\dy})
	-- ({1.14240*\dx},{0.00433*\dy})
	-- ({1.17413*\dx},{0.00836*\dy})
	-- ({1.20586*\dx},{0.01144*\dy})
	-- ({1.23760*\dx},{0.01329*\dy})
	-- ({1.26933*\dx},{0.01374*\dy})
	-- ({1.30106*\dx},{0.01278*\dy})
	-- ({1.33280*\dx},{0.01054*\dy})
	-- ({1.36453*\dx},{0.00725*\dy})
	-- ({1.39626*\dx},{0.00328*\dy})
	-- ({1.42800*\dx},{-0.00096*\dy})
	-- ({1.45973*\dx},{-0.00503*\dy})
	-- ({1.49146*\dx},{-0.00851*\dy})
	-- ({1.52320*\dx},{-0.01104*\dy})
	-- ({1.55493*\dx},{-0.01238*\dy})
	-- ({1.58666*\dx},{-0.01238*\dy})
	-- ({1.61840*\dx},{-0.01104*\dy})
	-- ({1.65013*\dx},{-0.00851*\dy})
	-- ({1.68186*\dx},{-0.00502*\dy})
	-- ({1.71360*\dx},{-0.00096*\dy})
	-- ({1.74533*\dx},{0.00328*\dy})
	-- ({1.77706*\dx},{0.00725*\dy})
	-- ({1.80880*\dx},{0.01054*\dy})
	-- ({1.84053*\dx},{0.01279*\dy})
	-- ({1.87226*\dx},{0.01375*\dy})
	-- ({1.90400*\dx},{0.01329*\dy})
	-- ({1.93573*\dx},{0.01145*\dy})
	-- ({1.96746*\dx},{0.00836*\dy})
	-- ({1.99920*\dx},{0.00434*\dy})
	-- ({2.03093*\dx},{-0.00022*\dy})
	-- ({2.06266*\dx},{-0.00487*\dy})
	-- ({2.09440*\dx},{-0.00910*\dy})
	-- ({2.12613*\dx},{-0.01245*\dy})
	-- ({2.15786*\dx},{-0.01454*\dy})
	-- ({2.18959*\dx},{-0.01510*\dy})
	-- ({2.22133*\dx},{-0.01399*\dy})
	-- ({2.25306*\dx},{-0.01125*\dy})
	-- ({2.28479*\dx},{-0.00711*\dy})
	-- ({2.31653*\dx},{-0.00194*\dy})
	-- ({2.34826*\dx},{0.00376*\dy})
	-- ({2.37999*\dx},{0.00941*\dy})
	-- ({2.41173*\dx},{0.01437*\dy})
	-- ({2.44346*\dx},{0.01805*\dy})
	-- ({2.47519*\dx},{0.01996*\dy})
	-- ({2.50693*\dx},{0.01976*\dy})
	-- ({2.53866*\dx},{0.01728*\dy})
	-- ({2.57039*\dx},{0.01263*\dy})
	-- ({2.60213*\dx},{0.00613*\dy})
	-- ({2.63386*\dx},{-0.00165*\dy})
	-- ({2.66559*\dx},{-0.00992*\dy})
	-- ({2.69733*\dx},{-0.01778*\dy})
	-- ({2.72906*\dx},{-0.02423*\dy})
	-- ({2.76079*\dx},{-0.02831*\dy})
	-- ({2.79253*\dx},{-0.02919*\dy})
	-- ({2.82426*\dx},{-0.02629*\dy})
	-- ({2.85599*\dx},{-0.01937*\dy})
	-- ({2.88773*\dx},{-0.00866*\dy})
	-- ({2.91946*\dx},{0.00510*\dy})
	-- ({2.95119*\dx},{0.02057*\dy})
	-- ({2.98293*\dx},{0.03578*\dy})
	-- ({3.01466*\dx},{0.04819*\dy})
	-- ({3.04639*\dx},{0.05464*\dy})
	-- ({3.07813*\dx},{0.05151*\dy})
	-- ({3.10986*\dx},{0.03475*\dy})
	-- ({3.14159*\dx},{0.00003*\dy})
}
\def\lsix{
	({0.00000*\dx},{0.00000*\dy})
	-- ({0.03173*\dx},{-0.01033*\dy})
	-- ({0.06347*\dx},{-0.02061*\dy})
	-- ({0.09520*\dx},{-0.03078*\dy})
	-- ({0.12693*\dx},{-0.04081*\dy})
	-- ({0.15867*\dx},{-0.05064*\dy})
	-- ({0.19040*\dx},{-0.06026*\dy})
	-- ({0.22213*\dx},{-0.06964*\dy})
	-- ({0.25387*\dx},{-0.07876*\dy})
	-- ({0.28560*\dx},{-0.08762*\dy})
	-- ({0.31733*\dx},{-0.09622*\dy})
	-- ({0.34907*\dx},{-0.10457*\dy})
	-- ({0.38080*\dx},{-0.11268*\dy})
	-- ({0.41253*\dx},{-0.12055*\dy})
	-- ({0.44427*\dx},{-0.12821*\dy})
	-- ({0.47600*\dx},{-0.13567*\dy})
	-- ({0.50773*\dx},{-0.14292*\dy})
	-- ({0.53947*\dx},{-0.14998*\dy})
	-- ({0.57120*\dx},{-0.15684*\dy})
	-- ({0.60293*\dx},{-0.16350*\dy})
	-- ({0.63467*\dx},{-0.16996*\dy})
	-- ({0.66640*\dx},{-0.17621*\dy})
	-- ({0.69813*\dx},{-0.18224*\dy})
	-- ({0.72986*\dx},{-0.18805*\dy})
	-- ({0.76160*\dx},{-0.19363*\dy})
	-- ({0.79333*\dx},{-0.19898*\dy})
	-- ({0.82506*\dx},{-0.20410*\dy})
	-- ({0.85680*\dx},{-0.20899*\dy})
	-- ({0.88853*\dx},{-0.21366*\dy})
	-- ({0.92026*\dx},{-0.21810*\dy})
	-- ({0.95200*\dx},{-0.22233*\dy})
	-- ({0.98373*\dx},{-0.22635*\dy})
	-- ({1.01546*\dx},{-0.23016*\dy})
	-- ({1.04720*\dx},{-0.23377*\dy})
	-- ({1.07893*\dx},{-0.23718*\dy})
	-- ({1.11066*\dx},{-0.24039*\dy})
	-- ({1.14240*\dx},{-0.24340*\dy})
	-- ({1.17413*\dx},{-0.24619*\dy})
	-- ({1.20586*\dx},{-0.24878*\dy})
	-- ({1.23760*\dx},{-0.25115*\dy})
	-- ({1.26933*\dx},{-0.25329*\dy})
	-- ({1.30106*\dx},{-0.25522*\dy})
	-- ({1.33280*\dx},{-0.25692*\dy})
	-- ({1.36453*\dx},{-0.25839*\dy})
	-- ({1.39626*\dx},{-0.25965*\dy})
	-- ({1.42800*\dx},{-0.26068*\dy})
	-- ({1.45973*\dx},{-0.26150*\dy})
	-- ({1.49146*\dx},{-0.26212*\dy})
	-- ({1.52320*\dx},{-0.26252*\dy})
	-- ({1.55493*\dx},{-0.26272*\dy})
	-- ({1.58666*\dx},{-0.26272*\dy})
	-- ({1.61840*\dx},{-0.26252*\dy})
	-- ({1.65013*\dx},{-0.26212*\dy})
	-- ({1.68186*\dx},{-0.26150*\dy})
	-- ({1.71360*\dx},{-0.26068*\dy})
	-- ({1.74533*\dx},{-0.25965*\dy})
	-- ({1.77706*\dx},{-0.25839*\dy})
	-- ({1.80880*\dx},{-0.25692*\dy})
	-- ({1.84053*\dx},{-0.25522*\dy})
	-- ({1.87226*\dx},{-0.25329*\dy})
	-- ({1.90400*\dx},{-0.25115*\dy})
	-- ({1.93573*\dx},{-0.24878*\dy})
	-- ({1.96746*\dx},{-0.24619*\dy})
	-- ({1.99920*\dx},{-0.24340*\dy})
	-- ({2.03093*\dx},{-0.24039*\dy})
	-- ({2.06266*\dx},{-0.23718*\dy})
	-- ({2.09440*\dx},{-0.23377*\dy})
	-- ({2.12613*\dx},{-0.23016*\dy})
	-- ({2.15786*\dx},{-0.22635*\dy})
	-- ({2.18959*\dx},{-0.22233*\dy})
	-- ({2.22133*\dx},{-0.21810*\dy})
	-- ({2.25306*\dx},{-0.21366*\dy})
	-- ({2.28479*\dx},{-0.20899*\dy})
	-- ({2.31653*\dx},{-0.20410*\dy})
	-- ({2.34826*\dx},{-0.19898*\dy})
	-- ({2.37999*\dx},{-0.19363*\dy})
	-- ({2.41173*\dx},{-0.18805*\dy})
	-- ({2.44346*\dx},{-0.18224*\dy})
	-- ({2.47519*\dx},{-0.17621*\dy})
	-- ({2.50693*\dx},{-0.16996*\dy})
	-- ({2.53866*\dx},{-0.16350*\dy})
	-- ({2.57039*\dx},{-0.15684*\dy})
	-- ({2.60213*\dx},{-0.14998*\dy})
	-- ({2.63386*\dx},{-0.14292*\dy})
	-- ({2.66559*\dx},{-0.13567*\dy})
	-- ({2.69733*\dx},{-0.12821*\dy})
	-- ({2.72906*\dx},{-0.12055*\dy})
	-- ({2.76079*\dx},{-0.11268*\dy})
	-- ({2.79253*\dx},{-0.10457*\dy})
	-- ({2.82426*\dx},{-0.09622*\dy})
	-- ({2.85599*\dx},{-0.08762*\dy})
	-- ({2.88773*\dx},{-0.07876*\dy})
	-- ({2.91946*\dx},{-0.06964*\dy})
	-- ({2.95119*\dx},{-0.06026*\dy})
	-- ({2.98293*\dx},{-0.05064*\dy})
	-- ({3.01466*\dx},{-0.04081*\dy})
	-- ({3.04639*\dx},{-0.03078*\dy})
	-- ({3.07813*\dx},{-0.02061*\dy})
	-- ({3.10986*\dx},{-0.01033*\dy})
	-- ({3.14159*\dx},{-0.00000*\dy})
}
\def\esix{
	({0.00000*\dx},{0.0000*\dy})
	-- ({0.03173*\dx},{0.02736*\dy})
	-- ({0.06347*\dx},{0.03763*\dy})
	-- ({0.09520*\dx},{0.03592*\dy})
	-- ({0.12693*\dx},{0.02684*\dy})
	-- ({0.15867*\dx},{0.01435*\dy})
	-- ({0.19040*\dx},{0.00157*\dy})
	-- ({0.22213*\dx},{-0.00921*\dy})
	-- ({0.25387*\dx},{-0.01662*\dy})
	-- ({0.28560*\dx},{-0.02007*\dy})
	-- ({0.31733*\dx},{-0.01970*\dy})
	-- ({0.34907*\dx},{-0.01615*\dy})
	-- ({0.38080*\dx},{-0.01042*\dy})
	-- ({0.41253*\dx},{-0.00368*\dy})
	-- ({0.44427*\dx},{0.00293*\dy})
	-- ({0.47600*\dx},{0.00844*\dy})
	-- ({0.50773*\dx},{0.01215*\dy})
	-- ({0.53947*\dx},{0.01370*\dy})
	-- ({0.57120*\dx},{0.01306*\dy})
	-- ({0.60293*\dx},{0.01052*\dy})
	-- ({0.63467*\dx},{0.00663*\dy})
	-- ({0.66640*\dx},{0.00206*\dy})
	-- ({0.69813*\dx},{-0.00248*\dy})
	-- ({0.72986*\dx},{-0.00631*\dy})
	-- ({0.76160*\dx},{-0.00894*\dy})
	-- ({0.79333*\dx},{-0.01006*\dy})
	-- ({0.82506*\dx},{-0.00959*\dy})
	-- ({0.85680*\dx},{-0.00769*\dy})
	-- ({0.88853*\dx},{-0.00472*\dy})
	-- ({0.92026*\dx},{-0.00116*\dy})
	-- ({0.95200*\dx},{0.00244*\dy})
	-- ({0.98373*\dx},{0.00556*\dy})
	-- ({1.01546*\dx},{0.00777*\dy})
	-- ({1.04720*\dx},{0.00878*\dy})
	-- ({1.07893*\dx},{0.00849*\dy})
	-- ({1.11066*\dx},{0.00699*\dy})
	-- ({1.14240*\dx},{0.00453*\dy})
	-- ({1.17413*\dx},{0.00150*\dy})
	-- ({1.20586*\dx},{-0.00164*\dy})
	-- ({1.23760*\dx},{-0.00444*\dy})
	-- ({1.26933*\dx},{-0.00649*\dy})
	-- ({1.30106*\dx},{-0.00751*\dy})
	-- ({1.33280*\dx},{-0.00737*\dy})
	-- ({1.36453*\dx},{-0.00611*\dy})
	-- ({1.39626*\dx},{-0.00393*\dy})
	-- ({1.42800*\dx},{-0.00116*\dy})
	-- ({1.45973*\dx},{0.00179*\dy})
	-- ({1.49146*\dx},{0.00449*\dy})
	-- ({1.52320*\dx},{0.00654*\dy})
	-- ({1.55493*\dx},{0.00764*\dy})
	-- ({1.58666*\dx},{0.00764*\dy})
	-- ({1.61840*\dx},{0.00654*\dy})
	-- ({1.65013*\dx},{0.00449*\dy})
	-- ({1.68186*\dx},{0.00179*\dy})
	-- ({1.71360*\dx},{-0.00116*\dy})
	-- ({1.74533*\dx},{-0.00392*\dy})
	-- ({1.77706*\dx},{-0.00610*\dy})
	-- ({1.80880*\dx},{-0.00736*\dy})
	-- ({1.84053*\dx},{-0.00750*\dy})
	-- ({1.87226*\dx},{-0.00648*\dy})
	-- ({1.90400*\dx},{-0.00443*\dy})
	-- ({1.93573*\dx},{-0.00163*\dy})
	-- ({1.96746*\dx},{0.00151*\dy})
	-- ({1.99920*\dx},{0.00454*\dy})
	-- ({2.03093*\dx},{0.00700*\dy})
	-- ({2.06266*\dx},{0.00850*\dy})
	-- ({2.09440*\dx},{0.00879*\dy})
	-- ({2.12613*\dx},{0.00778*\dy})
	-- ({2.15786*\dx},{0.00558*\dy})
	-- ({2.18959*\dx},{0.00246*\dy})
	-- ({2.22133*\dx},{-0.00115*\dy})
	-- ({2.25306*\dx},{-0.00470*\dy})
	-- ({2.28479*\dx},{-0.00767*\dy})
	-- ({2.31653*\dx},{-0.00957*\dy})
	-- ({2.34826*\dx},{-0.01004*\dy})
	-- ({2.37999*\dx},{-0.00893*\dy})
	-- ({2.41173*\dx},{-0.00630*\dy})
	-- ({2.44346*\dx},{-0.00246*\dy})
	-- ({2.47519*\dx},{0.00207*\dy})
	-- ({2.50693*\dx},{0.00665*\dy})
	-- ({2.53866*\dx},{0.01054*\dy})
	-- ({2.57039*\dx},{0.01308*\dy})
	-- ({2.60213*\dx},{0.01372*\dy})
	-- ({2.63386*\dx},{0.01218*\dy})
	-- ({2.66559*\dx},{0.00847*\dy})
	-- ({2.69733*\dx},{0.00296*\dy})
	-- ({2.72906*\dx},{-0.00366*\dy})
	-- ({2.76079*\dx},{-0.01040*\dy})
	-- ({2.79253*\dx},{-0.01612*\dy})
	-- ({2.82426*\dx},{-0.01967*\dy})
	-- ({2.85599*\dx},{-0.02005*\dy})
	-- ({2.88773*\dx},{-0.01659*\dy})
	-- ({2.91946*\dx},{-0.00918*\dy})
	-- ({2.95119*\dx},{0.00160*\dy})
	-- ({2.98293*\dx},{0.01437*\dy})
	-- ({3.01466*\dx},{0.02687*\dy})
	-- ({3.04639*\dx},{0.03595*\dy})
	-- ({3.07813*\dx},{0.03766*\dy})
	-- ({3.10986*\dx},{0.02739*\dy})
	-- ({3.14159*\dx},{0.00003*\dy})
}
\def\tabelleninhalt{
1& -0.270966& -0.271520& -0.271568& -0.271578& -0.271582& -0.271583\mathstrut\rlap{\raisebox{3pt}{\strut}}\\
2&           & -0.010363& -0.010441& -0.010451& -0.010453& -0.010454\mathstrut\\
3&           &           & -0.002235& -0.002260& -0.002264& -0.002265\mathstrut\\
4&           &           &           & -0.000813& -0.000823& -0.000825\mathstrut\\
5&           &           &           &           & -0.000382& -0.000387\mathstrut\\
6&           &           &           &           &           & -0.000209\mathstrut\\[3pt]
}
\def\steigungen{
1 &    -0.270966 \rlap{\raisebox{3pt}{\strut}}\\
2 &    -0.302608 \\
3 &    -0.314068 \\
4 &    -0.319920 \\
5 &    -0.323459 \\
6 &    -0.325827 \\[2pt]
\hline
\infty &    -0.337699 \rlap{\raisebox{3pt}{\strut}}\\[2pt]
}

\def\skala{1}
\def\dx{1.6}
\def\dy{2}
\def\achsen#1{
	\draw[->] (-0.1,0) -- ({3.1415*\dx+0.4},0) coordinate[label={$x$}];
	\draw[->] (0,-1.7) -- (0,2.8) coordinate[label={right:$#1$}];
	\draw ({0.5*3.1415*\dx},-0.05) -- ({0.5*3.1415*\dx},0.05);
	\node at ({0.5*3.1415*\dx},-0.05) [below]
		{$\displaystyle\frac{\pi}2\mathstrut$};
	\draw ({3.1415*\dx},-0.05) -- ({3.1415*\dx},0.05);
	\node at ({3.1415*\dx},-0.05) [below]
		{$\displaystyle\pi\mathstrut$};
}
\begin{tikzpicture}[>=latex,thick,scale=\skala]

% add image content here


\begin{scope}
\def\dy{5}
%\draw[color=red!10,line width=1.2pt] \lone;
%\draw[color=red!10,line width=1.2pt] \ltwo;
%\draw[color=red!10,line width=1.2pt] \lthree;
%\draw[color=red!10,line width=1.2pt] \lfour;
%\draw[color=red!10,line width=1.2pt] \lfive;
\draw[color=red,line width=1.4pt] \lsix;
\draw[color=blue,line width=1.4pt,dashed] \dglsol;
\draw (-0.05,{0.1*\dy}) -- (0.05,{0.1*\dy});
\draw (-0.05,{0.2*\dy}) -- (0.05,{0.2*\dy});
\draw (-0.05,{0.3*\dy}) -- (0.05,{0.3*\dy});
\draw (-0.05,{0.4*\dy}) -- (0.05,{0.4*\dy});
\draw (-0.05,{0.5*\dy}) -- (0.05,{0.5*\dy});
\draw (-0.05,{-0.1*\dy}) -- (0.05,{-0.1*\dy});
\draw (-0.05,{-0.2*\dy}) -- (0.05,{-0.2*\dy});
\draw (-0.05,{-0.3*\dy}) -- (0.05,{-0.3*\dy});
\node at (-0.05,{-0.2*\dy}) [left] {$-0.2$};
\achsen{y}
\end{scope}

\begin{scope}[xshift=6.3cm]
\def\dy{2}
\draw[color=red] \eone;
\draw[color=red] \etwo;
\draw[color=red] \ethree;
\draw[color=red] \efour;
\draw[color=red] \efive;
\draw[color=red] \esix;
\achsen{e}
\draw (-0.05,{1*\dy}) -- (0.05,{1*\dy});
\node at (-0.05,{1*\dy}) [left] {$10^{-2}$};
\end{scope}

\end{tikzpicture}
\end{document}

