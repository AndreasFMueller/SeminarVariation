%
% vorwort.tex -- Vorwort zum Buch zum Seminar
%
% (c) 2019 Prof Dr Andreas Mueller, Hochschule Rapperswil
%
\chapter*{Vorwort}


Dieses Buch entstand im Rahmen des Mathematischen Seminars
im Frühjahrssemester 2024 an der Ostschweizer Fachhochschule in Rapperswil.
Die Teilnehmer, Studierende der Studiengänge für Elektrotechnik, Informatik,
Erneuerbare Energien und Umwelttechnik und Bauingenieurwesen
der OST, erarbeiteten nach einer Einführung in das Themengebiet jeweils
einzelne Aspekte des Gebietes in Form einer Seminararbeit, über
deren Resultate sie auch in einem Vortrag informierten. 

Im Frühjahr 2024 waren Variationsprinzipien das Thema des Seminars.
Die Variationsrechnung als Aufgabenstellung wurde berühmt durch
das Brachistochronenproblem, welches Johann Bernoulli 1696 in der
Zeitschrift {\em Acta eruditorum} der mathematischen Öffentlichkeit
stellte.
Newton beteiligte sich unter einem Pseudonym, seine Lösung war aber nicht
seine einzige Beschäftigung mit einem Variationsproblem.
Man findet bei ihm auch eine Lösung des Problems, einen Rotationskörper
minimalen Luftwiderstandes zu finden.
Sie geht allerdings von unrealistischen Voraussetzungen aus und ist
nur für stark verdünnte Gase sinnvoll.
Dieses Beispiel wird neben vielen technischen Anwendungen und 
der Geschichte der Variationsrechnung in dem ausführlichen Lehrbuch
\cite{buch:funk} von Paul Funk dargestellt, welche dem ersten Teil
des Buches oft als Inspiration diente.

Eine systematische Lösungsmethode für Variationsprobleme entsteht unter
den Händen von Euler und Lagrange.
Eulers Herleitung basiert auf einer anschaulichen Diskretisierung,
die die Methoden der Theorie der Extrema von Funktionen mehrerer
Variablen anwendet.
Diese Methoden welche in Kapitel~\ref{buch:chapter:fuvar}
zusammengefasst.

Eulers Herleitung erfüllt heutige Standards an mathematische
Strenge nicht mehr.
Daher wird die Euler-Lagrange-Differentialgleichung in
Kapitel~\ref{buch:chapter:variation} mithilfe des
Fundamentallemmas~\ref{buch:variation:fundamentallemma:satz:fundamentallemma}
bewiesen.
Es wird aber auch die Kritik von Weierstrass ernstgenommen und
in Kapitel~\ref{buch:chapter:nichtdiff} die
Herleitung für nicht differenzierbare Lösungen nach Dubois-Reymond
dargestellt und die weierstrass-ermannsche Eckenbedingung abgeleitet.

Von besonderer wissenschaftshistorischer Bedeutung war die Beobachtung
von Maupertuis, dass physikalischen Gesetzen oft Extremalprinzipien zu
Grunde liegen.
Die Lagrange- und die Hamilton-Mechanik kanonisieren diese Sicht und geben
der klassische Mechanik die heutige Form, sie werden in
Kapitel~\ref{buch:chapter:hamiltonjacobi}
und
Kapitel~\ref{buch:chapter:mechanik}
dargestellt.
Das Buch \cite{buch:levi} entwickelt die klassische Mechanik systematisch
aus Variationsideen.

Im zweiten Teil des Buches werden einzelne Probleme der Naturwissenschaften
und der Technik von den Seminarteilnehmern bearbeitet.
Darunter findet man sowohl klassische Probleme, wie das schon von
Euler gelöste Problem, wie man einen Fluss am effizientesten durchschwimmen
kann, aber auch ganz moderne Probleme wie die Frage, wie ein
Raumschiff mit dem kleinsten Treibstoffbudget eine Erdumlaufbahn
erreichen kann.
Detailliertere Angaben zu den Seminararbeiten findet der Leser in der
Einleitung zum zweiten Teil auf Seite~\pageref{buch:part2:anwendungen}.

In einigen Arbeiten wurde auch Code zur Demonstration der 
besprochenen Methoden und Resultate geschrieben, soweit
möglich und sinnvoll wurde dieser Code im Github-Repository
\index{Github-Repository}%
dieses Kurses%
\footnote{\url{https://github.com/AndreasFMueller/SeminarVariation.git}}
\cite{buch:repo}
abgelegt.
Im genannten Repository findet sich auch der Source-Code dieses
Skriptes, es wird hier unter einer Creative Commons Lizenz
zur Verfügung gestellt.

