%
% offen.tex -- Offene Menge und Rand
%
% (c) 2021 Prof Dr Andreas Müller, OST Ostschweizer Fachhochschule
%
\documentclass[tikz]{standalone}
\usepackage{amsmath}
\usepackage{times}
\usepackage{txfonts}
\usepackage{pgfplots}
\usepackage{csvsimple}
\usetikzlibrary{arrows,intersections,math}
\begin{document}
\def\skala{1}
\begin{tikzpicture}[>=latex,thick,scale=\skala]

\def\qx{1.11}
\def\qy{3.18}

\def\px{2}
\def\py{-2}

\draw[color=red,line width=0.1pt] (4,3)--(5.2,3.5);

\fill[color=blue!20] (\qx,\qy) circle[radius=0.5];

\fill[color=red!10]
	(-2,2)--(-2,-4)--(2,-4)--(3,-3)
		arc (-80:30:1)
		arc (-80:30:1)
		arc (-60:50:1)
		arc (-40:70:1)
		arc (60:120:3)
		arc (0:160:0.8)
		--(-1,2)
		--cycle;

\draw[color=red,line width=1pt]
	(2,-4)--(3,-3)
		arc (-80:30:1)
		arc (-80:30:1)
		arc (-60:50:1)
		arc (-40:70:1)
		arc (60:120:3)
		arc (0:160:0.8)
		--(-1,2)
		--(-2,2);

%\draw[color=blue] (\qx,\qy) circle[radius=0.1];

\draw[->,line width=0.7pt] (-2.0,0)--(6.2,0) coordinate[label={$x_1$}];
\draw[->,line width=0.7pt] (0,-4.0)--(0,4.2) coordinate[label={right:$x_n$}];

\node at ({\qx+0.12},{\qy+0.1}) [below left] {$Q$};
\fill[color=blue] (\qx,\qy) circle[radius=0.05];
\fill[color=blue] (\px,\py) circle[radius=0.05];

\node at ({\px-0.1},{\py+0.05}) [below right] {$P$};

\def\a{10}

\draw[->,color=blue,line width=0.1pt]
	(\qx,\qy)--({\qx+0.5*cos(\a)},{\qy+0.5*sin(-\a)});
\node[color=blue] at ({\qx+0.25*cos(\a)},{\qy+0.25*sin(\a)-0.08})
	[below] {$\varepsilon$};
\draw[color=blue,line width=0.7pt] (\qx,\qy) circle[radius=0.5];

\draw[->,color=blue,line width=0.1pt]
	(\px,\py)--({\px+0.5*cos(\a)},{\py+0.5*sin(\a)});
\node[color=blue] at ({\px+0.25*cos(\a)},{\py+0.25*sin(\a)-0.08})
	[above] {$\varepsilon$};
\draw[color=blue,line width=0.7pt] (\px,\py) circle[radius=0.5];

\node at (0.5,-1.1) [right] {innerer Punkt};

\node at (0.5,2.3) [right] {Randpunkt};

\node at (-1,-2) {$\Omega$};

\node[color=red] at (5.1,3.5) [right] {$\partial\Omega$};


\end{tikzpicture}
\end{document}

