%
% 14-fundamtenallemma.tex
%
% (c) 2023 Prof Dr Andreas Müller
%

%
% Das Fundamentallemma für mehrere Variablen
%
\subsection{Das Fundamentallemma für mehrere Variablen}
Das Fundamentallemma war instrumental bei der Herleitung der
Euler-Lagrange-Dif\-fe\-ren\-tial\-glei\-chung.
Es besagt, dass eine stetige Funktion von einer Variablen auf einem Intervall
verschwindet, wenn alle ihre Skalarprodukte mit beliebig of stetig
differenzierbaren Funktionen verschwinden.
Diese Eigenschaft gilt auch für Funktionen mehrerer Variablen und kann
ganz analog bewiesen werden.

\begin{satz}[Fundamentallemma]
Sei $\Omega\subset\mathbb{R}^n$ ein Gebiet und $f\colon \Omega\to\mathbb{F}$
eine stetige Funktion.
Ausserdem gelte für jede glatte Funktion $g\colon\Omega\to\mathbb{R}$ 
\begin{equation}
\int_{\Omega} f(x)g(x)\,dx = 0.
\label{buch:felder:fundamentallemma:eqn:bedfundamentallemma}
\end{equation}
Dann folgt $f(x)=0$.
\end{satz}

\begin{proof}
Wie im eindimensionalen Fall kann der Beweis mit einem Widerspruch
geführt werden.
Sei $(x^*_1,\dots,x^*_n)$ ein Punkt im Inneren des Gebietes $\Omega$,
für den $f(x^*_1,\dots,x^*_n)=a\ne 0$ ist.
Da die Funktion $f$ stetig ist, gibt es ein $\delta>0$ derart,
dass sogar
\[
f(x_1,\dots,x_n) > \frac{a}2
\qquad\text{für alle $x_i$ mit $|x_i-x^*_i|<\delta$.}
\]
Da $(x^*_1,\dots,x^*_n)$ im Inneren des Gebietes liegt, kann 
$\delta$ sogar so klein gewählt werden, dass alle Punkte
$x=(x_1,\dots,x_n)$ mit $|x^*_i-x_i|<\delta$ immer noch in
$\Omega$ liegen.

Nach Satz~\ref{buch:felder:fundamentallemma:glatt:satz:glatt}
gibt es eine beliebig oft stetig differenzierbare Funktion $g(x)$,
die nur $x\in\Omega$ mit $|x^*_i-x_i|<\delta$ von $0$ verschieden ist,
und deren Integral $1$ ist.
Damit wird das 
Integral
\begin{align*}
\int_{\Omega} f(x)\,g(x)\,dx
&>
\int_{\operatorname{supp}g}
\frac{a}2
\cdot
g(x)
\,dx
=
\frac{a}2
\int_{\operatorname{supp}g}
g(x)
\,dx
=
\frac{a}2>0.
\end{align*}
Mit $g(x)$ ist also eine Funktion gefunden, die
Bedingung~\eqref{buch:felder:fundamentallemma:eqn:bedfundamentallemma}
des Fundamentallemmas nicht erfüllt.
Dieser Widerspruch zeigt, dass die Annahme, $f$ verschwinde nicht überall,
falsch ist.
Damit folgt, dass $f(x)=0$ ist.
\end{proof}



