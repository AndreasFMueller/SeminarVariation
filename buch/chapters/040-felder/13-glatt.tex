%
% 13-glatt.tex
%
% (c) 2023 Prof Dr Andreas Müller
%

%
% Glatte Funktionen mit kompaktem Träger
%
\subsection{Glatte Funktionen mit kompaktem Träger}
Für das Fundamtentallemma müssen glatte Funktionen konstruiert werden, die
höchstens innerhalb einer vorgegenen offenen Menge $>0$ sind.
Solche Funktionen können leicht mit den Funktion $g_{a,b}(x)$
konstruiert werden, die in Satz~\ref{buch:variation:fundamentallemma:satz:gab}
eingeführt worden sind.

\begin{satz}
\label{buch:felder:fundamentallemma:glatt:satz:glatt}
Ist $U\subset \mathbb{R}^n$ eine offene Umgebung des Punktes
$x\in\mathbb{R}^n$, dann gibt es eine beliebig oft stetig
differenzierbare, nicht negative Funktion $g\colon \mathbb{R}^n\to\mathbb{R}$,
deren Träger $\operatorname{supp}g$ in $U$ enthalten ist,
und deren Integral $1$ ist.
\end{satz}

\begin{proof}
Da $U$ eine offene Umgebung des Punktes ist, gibt es ein $\varepsilon>0$
derart, dass $y\in U$ für alle $y\in\mathbb{R}^n$ mit $|y-x|<\varepsilon$.
Dann gilt auch für alle Punkte $y\in\mathbb{R}^n$ mit
\[
|y_i-x_i|< \frac{\varepsilon}{n}
\qquad
\forall i=1,\dots,n,
\]
dass 
\begin{equation}
|y-x|
=
\sqrt{
(y_1-x_1)^2
+
\dots
+
(y_n-x_n)^2
}
\le
|y_1-x_1|+\dots+|y_n-x_n|
<
n\cdot \frac{\varepsilon}n
=
\varepsilon.
\label{buch:felder:fundamentallemma:glatt:eqn:eps}
\end{equation}
Der Punkt $y$ ist also wieder in $U$.
Wir definieren jetzt die Funktion $g(y)$ 
\[
g_0(y)
=
g_{x_1-\frac{\varepsilon}n,x_1+\frac{\varepsilon}n}(y_1)
\cdot
\ldots
\cdot
g_{x_n-\frac{\varepsilon}n,x_n+\frac{\varepsilon}n}(y_n).
\]
Sie ist beliebig oft stetig differenzierbar und nicht negativ,
weil die Faktoren $g_{x_i-\frac{\varepsilon}n,x_i+\frac{\varepsilon}n}(y_i)$
beliebig oft stetig differenzierbar und nicht negativ sind.
Der Funktionswert $g_0(y)$ ist nur dann von $0$ verschieden, wenn
$|y_i-x_i|\le \frac{\varepsilon}n$ für alle $i=1,\dots,n$ gilt.
Nach
\eqref{buch:felder:fundamentallemma:glatt:eqn:eps}
impliziert dies, dass $y\in U$, als folgt
$\operatorname{supp}g_0\subset U$.
Die normierte Funktion
\[
g(x) = \frac{1}{N}g_0(x),\quad\text{mit }
N=\int_{\mathbb{R}^n} g_0(x)\,dx,
\]
hat alle im Satz versprochenen Eigenschaften.
\end{proof}





