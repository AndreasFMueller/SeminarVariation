Auf dem Gebiet
$\Omega = \{ (r,\varphi) \mid 1<r<2\wedge 0<\varphi<\frac{3\pi}{2} \}$
von Abbildung~\ref{buch:401:fig:domain} soll das Funktional
\begin{equation}
I(u)
=
\int_1^2 \int_0^\frac{3\pi}2
a(r)
\biggl(\frac{\partial u}{\partial r}\biggr)^2
+
b(r)
\biggl(\frac{\partial u}{\partial \varphi}\biggr)^2
\,d\varphi\,dr
\label{buch:401:eqn:funktional}
\end{equation}
durch die Funktion $u(r,\varphi)$
unter den
Randbedingungen 
\begin{equation*}
\begin{aligned}
u(1,\varphi)&=u_1(\varphi)&&\text{und}&
u(2,\varphi)&=u_2(\varphi)&&\text{für alle }\varphi\in\biggl[0,\frac{3\pi}2\biggr]
\\
u(r,0)&=u_0(r)&&\text{und}&
u\biggl(r,\frac{3\pi}2\biggr)&=u_3(r)&&\text{für alle }r\in[1,2]
\end{aligned}
\label{buch:401:eqn:rb}
\end{equation*}
extremal gemacht werden,
wobei $u_1$, $u_2$, $u_0$ und $u_3$ beliebige Funktionen sind.
\begin{teilaufgaben}
\item
Finden Sie die Lagrange-Funktion des Funktionals
\eqref{buch:401:eqn:funktional}.
\item
Finden Sie die zugehörige Euler-Ostrogradski-Differentialgleichung.
\item
Zeigen Sie, dass sich in der Euler-Ostrogradski-Differentialgleichung
für $a(r)=r$ und $b(r)=1/r$ bis auf einen Faktor der Laplace-Operator
\[
\Delta
=
\frac{\partial^2}{\partial r^2}
+
\frac{1}{r}\frac{\partial u}{\partial r}
+
\frac{1}{r^2}\frac{\partial u}{\partial\varphi^2}
\]
in Polarkoordinaten ergibt.
\end{teilaufgaben}
\begin{figure}[h]
\centering
\def\r{3}
\def\w{55}
\def\wmax{135}
\definecolor{darkred}{rgb}{0.8,0,0}
\begin{tikzpicture}[>=latex,thick]

\fill[color=darkred!20] ({\r/2},0) -- (\r,0) arc(0:\wmax:\r)
	-- (\wmax:{\r/2}) arc(\wmax:0:{\r/2}) -- cycle;
\draw[color=darkred] ({\r/2},0) -- (\r,0) arc(0:\wmax:\r)
	-- (\wmax:{\r/2}) arc(\wmax:0:{\r/2}) -- cycle;
\draw[color=darkred,line width=0.3pt] (0,0) -- (\wmax:\r);

\fill[color=blue!10] (0,0) -- (0:{0.4*\r}) arc(0:55:{0.4*\r}) -- cycle;
\draw[color=blue] (0,0) -- (\w:{0.7*\r});
\fill[color=blue] (\w:{0.7*\r}) circle[radius=0.08];
\node[color=blue] at (\w:{0.7*\r}) [below right] {$r$};
\node[color=blue] at ({\w/2}:{0.23*\r}) {$\varphi$};

\fill (0,0) circle[radius=0.08];
\draw[->] (0,0) -- ({\r+0.5},0) coordinate[label={$r$}];

\node[color=darkred] at ({0.5*(\wmax+\w)}:{0.75*\r}) {$\Omega$};
\node at (0:{\r/2}) [below] {$1\mathstrut$};
\node at (0:\r) [below] {$2\mathstrut$};
\node[color=darkred] at (\wmax:{\r/2}) [below,rotate={\wmax-180}]
	{$\displaystyle\frac{3\pi}{2}$};

\end{tikzpicture}
\caption{Gebiet für das Funktional von Aufgabe~\ref{401}.
\label{buch:401:fig:domain}}
\end{figure}

\begin{loesung}
\begin{teilaufgaben}
\item
Die Lagrange-Funktion des Funktionals ist
\[
L(r,\varphi,u,u_r,u_\varphi)
=
a(r)
u_r^2
+
b(r)u_\varphi^2.
\]
\item
Für die Euler-Ostrogradski-Differentialgleichung brauchen wir die
partiellen Ableitungen
\begin{align*}
\frac{\partial L}{\partial r}
&=
0,
&
\frac{\partial L}{\partial u_r}
&=
2
a(r)
u_r
&&\text{und}
&
\frac{\partial L}{\partial u_\varphi}
&=
2b(r)
u_\varphi
\end{align*}
von $L$,
woraus sich die Euler-Ostrogradski-Differentialgleichung 
\begin{align}
0
&=
\frac{\partial L}{\partial u}
\biggl(u,\frac{\partial u}{\partial r},\frac{\partial u}{\partial\varphi}\biggr)
-
\frac{\partial}{\partial r}
\frac{\partial L}{\partial u_r}
\biggl(u,\frac{\partial u}{\partial r},\frac{\partial u}{\partial\varphi}\biggr)
-
\frac{\partial}{\partial \varphi}
\frac{\partial L}{\partial u_\varphi}
\biggl(u,\frac{\partial u}{\partial r},\frac{\partial u}{\partial\varphi}\biggr)
\notag
\\
&=
-\frac{\partial}{\partial r}
\biggl(
2a(r)\frac{\partial u}{\partial r}
\biggr)
-\frac{\partial}{\partial\varphi}
\biggl(
2b(r)\frac{\partial u}{\partial\varphi}
\biggr)
\notag
\\
&=
-
2a(r)\frac{\partial^2 u}{\partial r^2}
-
2a'(r)\frac{\partial u}{\partial r}
-
2b(r)\frac{\partial^2 u}{\partial\varphi^2}
\notag
\\
&=
-2\biggl(
a(r)
\frac{\partial^2u}{\partial r^2}
+
a'(r)
\frac{\partial u}{\partial r}
+
b(r)
\frac{\partial^2u}{\partial\varphi^2}
\biggr)
\label{buch:401:eqn:eodgl}
\end{align}
ergibt.
\item
Mit der Ableitung $a'(r) = 1$
ergibt sich für die Euler-Ostrogradski-Differentialgleichung
aus
\eqref{buch:401:eqn:eodgl}
\begin{align*}
0
&=
-2\biggl(
r
\frac{\partial^2u}{\partial r^2}
+
\frac{\partial u}{\partial r}
+
\frac{1}{r}
\frac{\partial u}{\partial\varphi^2}
\biggr).
\intertext{Durch Ausklammern des gemeinsamen Faktors $r$ wird daraus}
0
&=
-2r
\biggl(
\frac{\partial^2u}{\partial r^2}
+
\frac{1}{r}\frac{\partial u}{\partial r}
+
\frac{1}{r^2}\frac{\partial u}{\partial\varphi^2}
\biggr)
=
-2r\Delta u.
\qedhere
\end{align*}
\end{teilaufgaben}
\end{loesung}

\begin{diskussion}
Für physikalische Anwendungen beschränken Dimensionsüberlegungen die
mögliche Wahl der Funktionen $a(r)$ und $b(r)$.
Wir bezeichnen die Masseinheit von $u$ mit $[u]$.
Da $\partial^2u/\partial r^2$ die Masseinheit $[u]/[\text{Länge}]^2$
und $\partial^2u/\partial\varphi^2$ die Masseinheit $[u]$, muss auch
$a(r)/b(r)$ die Masseinheit $[\text{Länge}]^2$ haben.
Für die Wahl $a(r)=r$ und $b(r)=1/r$ realisiert $a(r)/b(r)=r/(1/r)=r^2$
dieses Verhältnis.
\end{diskussion}
