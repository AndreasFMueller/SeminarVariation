%
% 3-vektorfelder.tex
%
% (c) 2023 Prof Dr Andreas Müller
%
\section{Varationsproblem für Vektorfelder
\label{buch:felder:section:vektorfelder}}
\kopfrechts{Variationsprobleme für Vektorfelder}
In Abschnitt~\ref{buch:felder:section:euler-ostrogradski} wurde gezeigt,
wie ein Variationsprinzip für eine einzelne Funktion von mehreren Variablen
formuliert werden kann.
Wie im Fall einer Funktion einer Variablen ist die Erweiterung auf
mehrere Funktion vor allem die Herausforderung, eine geeignete Notation
für die vielen involvierten Funktionen zu finden.

%
% Lagrange-Funktion
%
\subsection{Lagrange-Funktion
\label{buch:felder:vektorfeilder:subsetion:lagrange-funktion}}
Statt einer skalaren Funktion $u(x_1,\dots,x_n)$ von $n$ Variablen
wird jetzt eine vektorwertige Funktion
\[
u
\colon
\mathbb{R}^n \to \mathbb{R}^m
:
(x_1,\dots,x_n)
\mapsto
(
u_1(x_1,\dots,x_n)
,\dots,
u_m(x_1,\dots,x_n)
)
\]
gesucht.
Jede der $m$ Komponenten $u_i(x_1,\dots,x_n)$ ist Funktion von
$n$ Variablen.

Im Folgenden werden Funktionen betrachtet, die ein- oder mehrmals
stetig differenzierbar sind.
Dies bedeutet, dass die partiellen Ableitungen nach jeder beliebigen
Variablen stetige Funktionen sind.
Die $m$ Funktionen $u_1,\dots,u_m$ haben $nm$ partielle Ableitungen
nach den $n$ Variablen $x_1,\dots,x_n$.
Wir schreiben wieder $u_x$ für die die ersten Ableitungen.
$u_x$ besteht aus den $nm$ Ableitungen 
\[
u_{ik}
=
u_{i,k}
=
\frac{\partial u_i}{\partial x_k}.
\]
Die Notation $x_{i,k}$ ist in der Tensoranalysis üblich als Bezeichnung
für die partiellen Ableitung der Komponenten eines Tensors.
Wir werden auch die noch kompaktere Notation $\grad\varphi$ dafür 
verwenden.

Für die Formulierung von Variationsproblemen wird wieder eine
Lagrange-Funktion benötigt, die sowohl von den Funktionswerten
$u_i$ wie auch von allen ersten Ableitungen $u_{ik}$ abhängt.
Die Lagrange-Funktion ist also eine Funktion
\begin{equation*}
L
\colon
\mathbb{R}^n\times \mathbb{R}^m \times \mathbb{R}^{nm}\to\mathbb{R}
:
(x,u,u_x)
\mapsto
L(x,u,u_x).
\end{equation*}
Wenn wir die Komponenten explizit machen wollen, schreiben wir
\[
L(x_1,\dots,x_n,u_1,\dots,u_m,u_{11},\dots,u_{1n},\dots,u_{mn}).
\]
Das zugehörige Funktion ist dann wieder das Integral
\begin{equation}
I(\varphi)
=
\int_{\Omega}
L\biggl(x_1,\dots,x_n,
\varphi_1(x),\dots,\varphi_m(x),
\frac{\partial\varphi_1}{\partial x_1}(x),
\dots,
\frac{\partial\varphi_1}{\partial x_k}(x),
\dots
\frac{\partial\varphi_m}{\partial x_n}(x)
\biggr)
\,dx
\label{buch:felder:vektorfelder:eqn:lagrange-funktional}
\end{equation}
für eine auf dem Gebiet $\Omega$ definierte vektorwertige Funktion
$\varphi$ mit den Komponenten $\varphi_i(x_1,\dots,x_n)$.

In der Euler-Ostrogradski-Gleichung treten die Ableitungen nach
den $u_i$ und den Variablen $u_{ik}$, die für die partiellen
Ableitungen stehen.
Die partiellen Ableitungen nach den $u_i$ bilden einen Vektor
mit Komponenten
\begin{align*}
\frac{\partial L}{\partial u}
&=
\biggl(
\frac{\partial L}{\partial u_1}
,\dots,
\frac{\partial L}{\partial u_m}
\biggr)
\end{align*}
mit $m$ Komponenten.
Die $mn$ partiellen Ableitungen der Komponenten $i$ nach der Variablen $x_k$
sind von Paaren $(i,k)$ indiziert, wir schreiben Komponenten als
\[
\biggl(
\frac{\partial L}{\partial u_x}
\biggr)_{ik}
=
\frac{\partial L}{\partial u_{ik}}
\]
mit $i=1,\dots,n$ und $k=1,\dots,m$.

%
% Euler-Ostrogradski-Differentialgleichung
%
\subsection{Euler-Ostrogradski-Differentialgleichung
\label{buch:felder:vektorfelder:subsection:euler-ostrogradski-dgl}}
Die Euler-Ostrogradski-Differentialgleichung für eine skalare
Funktion lässt sich zusammen mit ihrer Herleitung auf beliebige 
Variationsproblem für eine vektorielle Funktion und ein Funktional
der Form~\eqref{buch:felder:vektorfelder:eqn:lagrange-funktional}
verallgemeinern.

Sei also $\varphi(x)$ eine zweimal stetig differenzierbare, vektorwertige
Funktion, die das Funktional $I(\varphi)$
von~\eqref{buch:felder:vektorfelder:eqn:lagrange-funktional}
zu einem Extremum macht.
Sei $\eta(x)$ eine vektorwertige, einmal stetig differenzierbare
Funktion mit verschwindenden Werten auf dem Rand von $\Omega$.
Dann ist die Richtungsableitung in Richtung $\eta$ an der Stelle
$\varepsilon=0$
\begin{align}
\frac{d}{d\varepsilon}
I(\varphi+\varepsilon\eta)
\bigg|_{\varepsilon=0}
&=
\int_{\Omega}
\sum_{i=1}^n
\frac{\partial L}{\partial u_i}
\eta_i
+
\sum_{i=1}^n
\sum_{k=1}^m
\frac{\partial L}{\partial u_{ik}} \frac{\partial \eta_i}{\partial x_k}
\,dx.
\label{buch:felder:vektorfelder:eqn:richtungsableitung}
\end{align}
Im Interesse der Übersichtlichkeit der Formeln haben wir die Argumente
der Funktionen $\varphi$ und $\eta$ weggelassen.
Da $\varphi$ extremal ist, muss die Richtungsableitung für jede 
zulässige Funktion $\eta$ verschwinden.

Indem man jede bis auf eine Komponenten von $\eta$ gleich $0$
wählt, folgen aus
\eqref{buch:felder:vektorfelder:eqn:richtungsableitung}
die Gleichungen
\begin{equation*}
0
=
\int_{\Omega}
\frac{\partial L}{\partial u_i}
\eta_i
+
\sum_{k=1}^n
\frac{\partial L}{\partial u_{ik}}
\frac{\partial \eta_i}{\partial x_k}
\,dx
\end{equation*}
für $i=1,\dots,m$.
Partielle Integration ermöglicht, das Integral über den zweiten
Term in ein Integral mit $\eta_i$ als zweitem Faktor umzuwandeln,
nämlich
\begin{equation*}
0
=
\int_{\Omega}
\frac{\partial L}{\partial u_i}
(x,\varphi(x),\grad\varphi(x))
+
\sum_{k=1}^n
\frac{\partial}{\partial x_k}
\frac{\partial L}{\partial u_{ik}}
(x,\varphi(x),\grad\varphi(x))
\,dx
\end{equation*}
mit $i=1,\dots m$.
Nach dem Fundamentallemma
(Satz~\ref{buch:felder:fundamentallemma:satz:fundamentallemma})
folgt jetzt wieder, dass
\begin{equation}
\frac{\partial L}{\partial u_i}
(x,\varphi(x),\grad\varphi(x))
+
\sum_{k=1}^n
\frac{\partial}{\partial x_k}
\frac{\partial L}{\partial u_{ik}}
(x,\varphi(x),\grad\varphi(x))
=
0
\label{buch:felder:vektorrelder:eqn:euler-ostrogradski}
\end{equation}
für $i=1,\dots m$ ist.
Man beachte, dass die ``äussere'' partielle Ableitung im zweiten Term
von
\eqref{buch:felder:vektorrelder:eqn:euler-ostrogradski}
wieder als die Ableitung der Funktion
\[
x_k
\mapsto
L(x,\varphi(x),\grad\varphi(x))
\]
nach $x_k$ zu interpretieren ist.
Es werden also erst die Funktion $\varphi_i(x)$ und die Ableitungen
in $L$ eingesetzt, erst dann wird die Ableitung gebildet.
Sie hat nichts mit der Ableitung $\partial L/\partial x_k$ zu tun, welche
die partielle Ableitung der Funktion nach dem Argument $x_k$ ist.


Die Gleichungen
\eqref{buch:felder:vektorrelder:eqn:euler-ostrogradski}
bilden ein System von $m$ partiellen Differentialgleichungen
für die $m$ Funktionen $\varphi_i(x)$.
Sie heissen die {\em Euler-Ostrogradski-Differentialgleichungen}
für das Funktional
\eqref{buch:felder:vektorfelder:eqn:lagrange-funktional}.



