Wir betrachten eine Lagrange-Funktion $L(x,\dot{x})$ im
dreidimensionalen Raum $x\in\mathbb{R}^3$, die unter der
Schraubensymmetrie
\[
h^s(x)
=
x
\mapsto 
\begin{pmatrix}
\cos s & - \sin s & 0 \\
\sin s &\phantom{-}\cos s& 0 \\
 0 & 0 & 1
\end{pmatrix}
x
+
\begin{pmatrix}
0\\
0\\
us
\end{pmatrix}
\]
invariant ist.
Nehmen Sie an, dass
\[
\frac{\partial L}{\partial\dot{x}}=p
\]
der Impuls ist.
Finden Sie die Grösse, die nach dem
Satz~\ref{buch:symmetrien:noether:satz:noether} 
entlang einer Bahnkurve des durch $L$ definierten mechanischen
Systems erhalten bleibt.

\begin{loesung}
Nach Satz~\ref{buch:symmetrien:noether:satz:noether} ist die Erhaltungsgrösse
\[
I(x,\dot{x})
=
\frac{\partial L}{\partial\dot{x}}
\cdot
\frac{d}{ds}h^s(x)\bigg|_{s=0}.
\]
Die Ableitung nach $s$ ist
\[
\frac{d}{ds} h^s(x)\bigg|_{s=0}
=
\begin{pmatrix}
0&-1&0\\
1& 0&0\\
0& 0&0
\end{pmatrix}
x
+
\begin{pmatrix}
0\\
0\\
u
\end{pmatrix}
=
\begin{pmatrix}
-x_2\\
 x_1\\
 u
\end{pmatrix}.
\]
Die Erhaltungsgrösse ist jetzt
\begin{align*}
\frac{\partial L}{\partial\dot{x}}]\cdot \frac{d}{ds}h^s(x)\bigg|_{s=0}
=
p\cdot \begin{pmatrix}-x_2\\x_1\\ux_3\end{pmatrix}
&=
p_2x_1-x_2p_1+up_3.
\intertext{Die ersten zwei Terme bilden die $3$-Komponente $l_3$ des
Drehimpulses.
Somit ist die Erhaltungsgrösse}
&=l_3+up_3.
\end{align*}
Weder die Drehimpulskomponenten in $3$-Richtung noch die Impulskomponennte
sind also für sich genommen erhalten, aber die Linearkombination erhalten.
Während der Bewegung kann also Drehimpuls in Impuls in der $3$-Richtung
umgewandelt werden und umgekehrt.
\end{loesung}


