%
% chapter.tex
%
% (c) 2023 Prof Dr Andreas Müller
%
\chapter{Symmetrien
\label{buch:chapter:symmetrien}}
\kopflinks{Symmetrien}
Umgangsprachlich versteht man unter einer Symmetrie oft einfach  nur
eine Abbildung wie die Spiegelung
\(
(x,y)\mapsto (-x,y)
\)
der Ebene $\mathbb{R}^2$ an der $y$-Achse.
In diesem Abschnitt soll aber eine Art von Symmetrie betrachtet
werden, die stetig von einem Parameter abhängt, wie zum Beispiel
die Drehung der Ebene um den Nullpunkt um den Winkel $\alpha$.
In Abschnitt~\ref{buch:symmetrien:section:symmetrie} wird daher
erst ein Begriff differenzierbarer Symmetrien definiert.
In Abschnitt~\ref{buch:symmetrien:section:noether} wird dann gezeigt,
wie solche Symmetrien automatisch auf Erhaltungssätze führen.
Dies ist der Inhalt des Satzes von Emmy Noether
(Satz~\ref{buch:symmetrien:noether:satz:noether}).

%
% 1-symmetrie.tex
%
% (c) 2023 Prof Dr Andreas Müller
%
\section{Symmetrien und Funktionale
\label{buch:symmetrien:section:symmetrie}}
\kopfrechts{Symmetrie und Funktionale}
In diesem Abschnitt betrachten wir immer eine Lagrange-Funktion
der Form $L(t,x,\dot{x})$ und damit ein Funktional der Form
\[
I(x)
=
\int_{t_0}^{t_1}
L\bigl(t,x(t),\dot{x}(t)\bigr)\,dt,
\]
abhängig von einer Funktion $x\colon[t_0,t_1]\to\mathbb{R}$.
In diesem Abschnitt soll definiert werden, was unter einer
differenzierbare Symmetrie der Lagrange-Funktion $L$ zu verstehen
ist.

%
% Definition
%
\subsection{Definition}
Eine Symmetrie der Lagrange-Funktion soll ihren Wert nicht ändern.
Da sie aber auch von $\dot{x}$ abhängt, muss erklärt werden, wie
der Vektor $\dot{x}$ abgebildet werden muss.
Dazu beachtet man, dass $\dot{x}$ als die Geschwindigkeit eines
Punktes interpretiert werden, der sich entlang einer Kurve $x(t)$
bewegt.
Eine Symmetrieabbildung $\varphi\colon\mathbb{R}^n\to\mathbb{R}^n$ 
bildet die Kurve $t\mapsto x(t)$ auf die Kurve $t\mapsto \varphi(x(t))$
ab. 
Die Geschwindigkeit im Punkt $x(t)$ wird dabei nach der Kettenregel
auf 
\[
\frac{d}{dt}\varphi(x(t))
=
D_{x(t)}\varphi\cdot \dot{x}(t)
\]
abgebildet.
Die lineare Abbildung, die auf den Geschwindigkeitsvektor $\dot{x}(t)$
anzuwenden ist, hängt daher vom Punkt $x(t)$ ab, an dem sich der
Punkt gerade befindet.
Insbesondere ist es nur für differenzierbare Abbildungen $\varphi$
sinnvoll, von der Invarianz einer Lagrange-Funktion zu sprechen.

\begin{definition}[Invarianz einer Lagrange-Funktion]
Eine Lagrange-Funktion heisst {\em invariant} unter einer differenzierbaren
\index{Invarianz der Lagrange-Funktion}%
Symmetrieabbildung $\varphi\colon\mathbb{R}^n\to\mathbb{R}^n$, wenn
\[
L\bigl(t,\varphi(x), D_x\varphi(\dot{x})\bigr)
=
L(t,x,\dot{x})
=
I(x)
\]
für alle $t$, $x$ und $\dot{x}$.
\end{definition}

%
% Extremalen
%
\subsection{Extremalen}
Die Invarianz der Lagrange-Funktion hat auch zur Folge, dass der
Wert des Funktionals nicht ändert, wenn man eine Funktion $t\mapsto x(t)$
der Transformation $\varphi$ unterwirft, es gilt also für alle
solchen Funktionen
\[
I(\varphi\circ x)
=
\int_{t_0}^{t_1}
L\bigl(t,\varphi(x(t)),D_{x(t)}\varphi(\dot{x}(t))\bigr)
\,dt
=
\int_{t_0}^{t_1}
L\bigl(t,x(t),\dot{x}(t)\bigr)
\,dt.
\]
Daraus kann man allerdings noch nicht schliessen, dass $x(t)$ genau
dann eine extremale ist, wenn auch $\varphi(x(t))$ eine Extremale ist.
Dazu muss man zusätzlich verlangen, dass die Abbildung $\varphi$
invertierbar ist.

\begin{satz}
\label{buch:symmetrien:symmetrie:satz:invarianz}
Ist $\varphi$ eine invertierbare Symmetrieabbildung, unter der die
Lagrange-Funktion invariant ist.
Zudem sei auch die Umkehrabbildung $\varphi^{-1}$ differenzierbar.
Dann ist $\varphi(x(t))$ eine Extremale genau dann, wenn $x(t)$ eine
Extremale ist.
\end{satz}

\begin{proof}
Sei $x(t)$ eine Extremale, also eine Funktion, für die die Ableitung
\[
\frac{d}{d\varepsilon}I(x(t,\varepsilon))
\bigg|_{\varepsilon=0}
=
0
\]
für jede Variation $x(t,\varepsilon)$ verschwindet.
Da $\varphi$ invertierbar ist, lässt sich jede Variation von
$\varphi(x(t))$ in der Form $\varphi(x(t,\varepsilon))$ ausdrücken.
Ist nämlich $y(t,\varepsilon)$ ein Variation von $\varphi(x(t))$
mit $y(t,0) = \varphi(x(t))$, dann ist $\varphi^{-1}(y(t,\varepsilon))$
eine Variation der Funktion $x(t)$.
Die Abbildung $\varphi$ ändert aber den Wert des Funktionals nicht,
also ändert sich dadurch auch die Ableitung nach $\varepsilon$ nicht.
Verschwindet die Ableitung von $I(x(t,\varepsilon))$ nach $\varepsilon$
für alle Variationen $x(t,\varepsilon)$ von $x(t)$, dann verschwindet
auch die Ableitung von $I(y(t,\varepsilon))$ für alle Variationen
$y(t,\varepsilon)$ von $\varphi(x(t))$.
\end{proof}

Nach Satz~\ref{buch:symmetrien:symmetrie:satz:invarianz} gehen 
also bei einer unter $\varphi$ invarianten Lagrange-Funktion die
Extremalen durch $\varphi$ in Extremalen über.

%
% Stetige Symmetrien
%
\subsection{Stetige Symmetrien}
Drehungen oder Translationen sind invertierbare Symmetrieoperationen,
die zusätzlich stetig oder sogar differenzierbar von einem Parameter
abhängen.

\begin{definition}[Stetige Symmetrie]
\index{stetige Symmetrie}%
\index{Symmetrie!stetig}%
Sei $h^s\colon \mathbb{R}^n\to\mathbb{R}^n$ für jeden Wert $s$
eine invertierbare und differenzierbare Abbildung.
Sie heisst eine {\em stetige Symmetrie} von $L$, wenn
\[
L\bigl(t, h^s(x), Dh^s(\dot{x})\bigr)
=
L(t,x,\dot{x})
\]
und $h^s$ hängt stetig von $s$ ab.
\end{definition}

\begin{definition}[Differenzierbare Symmetrie]
Eine stetige Symmetrie 
$h^s$ heisst {\em differenzierbar}, wenn $h^s(x)$
\index{Symmetrie!differenzierbar}%
\index{differenzierbare Symmetrie}%
eine differenzierbare Funktion von $s$ ist.
\end{definition}

\begin{beispiel}
\label{buch:symmetrien:symmetrie:beispiel:homogen}
Wir betrachten die Transformation
\[
h^s
\colon
\mathbb{R}^3 \to \mathbb{R}^3
:
x \mapsto x + su
\]
mit einem Vektor $u\in\mathbb{R}$.
$h^s$ ist invertierbar, denn $(h^s)^{-1}=h^{-s}$.
Ausserdem ist sie differenzierbar, denn
\[
Dh^s(v)
=
sv.
\]
In der Lagrange-Funktion
\[
L
=
\frac12m\dot{x}^2 - mgx_3
\]
ist $D\varphi(\dot{x})=\dot{x}$, der erste Term ändert also nicht.
Der zweite Term dagegen wird bei der Symmetrieabbildung zu
$mgx_3\mapsto mg(x_3+u_3)$.
Die Lagrange-Funktion ist also nur für Translationen mit Vektoren $u$
invariant, die senkrecht auf $\vec{e}_3$ sind oder $u_3=0$ haben.
\end{beispiel}

\begin{beispiel}
\label{buch:symmetrien:symmetrie:beispiel:drehung}
Drehungen des dreidimensionalen Raumes $\mathbb{R}^3$ können durch
orthogonale Matrizen $O$ beschrieben werden.
In der Lagrange-Funktion $L=\frac12m\dot{x}^2-V(x)$ ist der erste
Term wegen
\[
\frac12m \transpose{(O\dot{x})}(O\dot{x})
=
\frac12m\dot{x}^2
\]
invariant.
Der zweite Term ist genau dann invariant, wenn das Potential $V(x)$
invariant ist, wenn also $V(O(x))=V(x)$ ist.
Das Potential muss rotationsinvariant sein.
\end{beispiel}

%
% Zeitunabhängige Lagrange-Funktion
%
\subsection{Zeitinvarianz}
\index{Zeitinvarianz}%
Die bisher betrachteten Symmetrien betrafen nur die abhängigen
Variablen $x$ und $\dot{x}$, nicht die unabhängige Variable $t$.
Die unabhängige Variable spielt eine spezielle Rolle, die nicht
direkt mit den abhängigen Variablen vergleichbar ist.
Eine Transformation der unabhängigen Variablen ändert auch den
Definitionsbereich, so dass eine erweiterte Theorie nötig wird, 
die in Abschnitt~\ref{buch:symmetrien:felder:subsection:divergenzform}
besprochen wird.
Ein einfacher Spezialfall ist der Fall, wo $L$ nicht von der
Zeit abhängt, also $L(t,x,\dot{x})=L(x,\dot{x})$ geschrieben werden
kann.
In diesem Fall ergibt der Hamilton-Formalismus die Hamilton-Funktion
$H=pq-L$, die ebenfalls nicht explizit von der Zeit abhängt, also
auch in der Form $H(p,q)$ geschrieben werden kann.
Nach den hamiltonschen Differentialgleichungen ist dann
\begin{align*}
\frac{d}{dt}H(p,q)
=
\sum_{k=1}^n
\frac{\partial H}{\partial q_k}\dot{q}_k
+
\sum_{k=1}^n
\frac{\partial H}{\partial p_k}\dot{p}_k
=
\sum_{k=1}^n
\bigl(
-
\dot{p}_k\dot{q}_k
+
\dot{q}_k\dot{p}_k
\bigr)
=
0.
\end{align*}
Somit ist $H(p,q)$ eine Erhaltungsgrösse.
Im Falle einer zeitunabhängigen Lagrange-Funk\-tion ist also die
Energie eine Erhaltungsgrösse.
\index{Energie}%
Auch in diesem Fall ergibt sich also ein Erhaltungssatz, wie ihn
der Satz von Emmy Noether im nächsten Abschnitt für jede differenzierbare
Symmetrie einer Lagrange-Funktion liefern wird.

%
% Lie-Gruppen
%
\subsection{Lie-Gruppen}
Der allgemeine Rahmen der differenzierbaren Symmetrien ist die
Theorie der Lie-Gruppen.
\index{Lie-Gruppe}%
Sie ergeben sich auf natürliche Art und Weise als die Gruppen der
Matrizen, die gewisse Funktionen von Vektoren invariant lassen.
die Gruppe $\operatorname{SO}(3)$ der Drehungen des Raumes
zum Beispiel ist die Gruppe der Matrizen, die das Skalarprodukt
invariant lassen.
In der relativistischen Mechanik
\index{relativistische Mechanik}%
sind nur Transformationen zulässig, die die Grösse
\[
l^2
=
x^2+y^2+z^2-c^2t^2
\]
invariant lassen.
Sie bilden die Lie-Gruppe der Lorentz-Transformationen.

Die Transformationen in einer Lie-Gruppe können wie die Drehungen
durch die Matrixexponentialfunktion $\exp(sA)$ aus einer Matrix $A$
gefunden werden, die als die Ableitung der Funktion $s\to\exp(sA)$
erhalten werden können.
Sie sind damit automatisch differenzierbare Symmetrien.
Die Ableitungen haben die zusätzliche Struktur einer Lie-Algebra
\cite[p.~434]{buch:linalg}, die für die weiteren Untersuchungen
nützlich sein kann.


%
% 2-noether.tex
%
% (c) 2023 Prof Dr Andreas Müller
%
\section{Der Satz von Emmy Noether
\label{buch:symmetrien:section:noether}}
\kopfrechts{Der Satz von Noether}
Die Invarianz einer Lagrange-Funktion unter einer differenzierbaren
Symmetrie hat die überraschende Konsequenz, dass es dazu eine
Erhaltungsgrösse gibt.

\begin{satz}[Emmy Noether]
\label{buch:symmetrien:noether:satz:noether}
Sei $L(x,\dot{x})$ eine Lagrange-Funktion, die nicht von der Zeit
abhängt und $h^s\colon\mathbb{R}^n\to\mathbb{R}^n$ eine differenzierbare
Symmetrie, unter der $L$ invariant bleibt.
Dann ist die Grösse 
\begin{equation}
I(x,\dot{x})
=
\frac{\partial L}{\partial\dot{x}} (x,\dot{x})
\cdot 
\frac{d}{ds} h^s(x)\bigg|_{s=0}
\end{equation}
eine Erhaltungsgrösse.
\end{satz}

\begin{proof}
Sei $t\mapsto x(t)$ eine Lösung der Euler-Lagrange-Differentialgleichung
zur Lagrange-Funktion $L$.
Die Funktion $\Phi(s,t)$ definiert durch
\[
\Phi(s,t) = h^s(x(t))
\]
kombiniert die Symmetrietransformation mit der Zeitentwicklung.
Für jeden Wert von $s$ ist $t\mapsto \Phi(s,t)=h^s(x(t))$ eine 
Kurve.
Ihr Geschwindigkeitsvektor ist die Ableitung nach $t$, also
\[
\frac{d}{dt} \Phi(s,t)
=
\dot{\Phi}(s,t).
\]
Da die Lagrange-Funktionen unter der Symmetrietransformation $h^s$ 
invariant ist, muss die Ableitung von $L(\Phi(s,t), \dot{\Phi}(s,t))$
von $s$ unabhängig sein.
Die Ableitung
\begin{equation}
\frac{\partial}{\partial s} L(\Phi(s,t), \dot{\Phi}(s,t))
=
\frac{\partial L}{\partial x}(\Phi(s,t),\dot{\Phi}(s,t))
\cdot
\frac{\partial \Phi}{\partial s}
+
\frac{\partial L}{\partial \dot{x}}(\Phi(s,t),\dot{\Phi}(s,t))
\cdot
\frac{\partial \dot{\Phi}}{\partial s}
=0
\label{buch:symmetrioen:noether:eqn:sabl}
\end{equation}
muss daher verschwinden.

Ist $x(t)$ eine Lösungskurve der Euler-Lagrange-Differentialgleichung,
dann ist auch $t\mapsto h^s(x(t)) = \Phi(s,t)$ für jeden Wert von $s$
eine Lösungskurve der Euler-Lagrange-Differentialgleichung.
Die Euler-Lagrange-Differentialgleichung ist
\begin{equation}
\frac{\partial L}{\partial x}(\Phi(s,t),\dot{\Phi}(s,t))
=
\frac{\partial t}{\partial t}
\frac{\partial L}{\partial \dot{x}}(\Phi(s,t),\dot{\Phi}(s,t)).
\end{equation}
Dies kann dazu verwendet werden, den ersten Term auf der rechten
Seite von \eqref{buch:symmetrioen:noether:eqn:sabl} umzuformen,
wir erhalten
\begin{align}
0
&=
\frac{\partial}{\partial t}
\biggl(
\frac{\partial L}{\partial \dot{x}}(\Phi(s,t),\dot{\Phi}(s,t))
\biggr)
\cdot
\frac{\partial \Phi}{\partial s}
+
\frac{\partial L}{\partial\dot{x}}(\Phi(s,t),\dot{\Phi}(s,t))
\cdot
\frac{\partial \dot{\Phi}}{\partial s}(s,t)
\notag
\intertext{im zweiten Term kann man die Ableitung nach $s$ mit der Ableitung
nach $t$ vertauschen:}
&=
\frac{\partial}{\partial t}
\biggl(
\frac{\partial L}{\partial \dot{x}}(\Phi(s,t),\dot{\Phi}(s,t))
\biggr)
\cdot
\frac{\partial \Phi}{\partial s}
+
\frac{\partial L}{\partial\dot{x}}(\Phi(s,t),\dot{\Phi}(s,t))
\cdot
\frac{\partial^2\Phi}{\partial s\,\partial t}(s,t)
\notag
\\
&=
\frac{\partial}{\partial t}
\biggl(
\frac{\partial L}{\partial \dot{x}}(\Phi(s,t),\dot{\Phi}(s,t))
\biggr)
\cdot
\frac{\partial \Phi}{\partial s}
+
\frac{\partial L}{\partial\dot{x}}(\Phi(s,t),\dot{\Phi}(s,t))
\cdot
\frac{\partial}{\partial t}
\biggl(
\frac{\partial\Phi}{\partial s}(s,t)
\biggr).
\label{buch:symmetrien:noether:eqn:tindep}
\intertext{Auf der rechten Seite kann man die Kettenregel erkennen,
sie ist die Ableitung}
0
&=
\frac{d}{dt}\biggl(
\frac{\partial L}{\partial \dot{x}}(\Phi(s,t),\dot{\Phi}(s,t))
\cdot
\frac{\partial\Phi}{\partial s}(s,t)
\biggr).
\notag
\end{align}
Da die Ableitung verschwindet, ist der Klammerausdruck eine Konstante,
er hängt also nicht von $s$ und $t$ ab.
Setzt man $s=0$ folgt, dass
\[
\frac{\partial L}{\partial \dot{x}}(x,\dot{x})
\cdot
\frac{d}{ds}h^s(x)\bigg|_{s=0}
=
I(x,\dot{x})
\]
eine Erhaltungsgrösse ist.
\end{proof}

\begin{beispiel}
\label{buch:symmetrien:noether:beispiel:homogen}
Die Lagrange-Funktion $L(x,\dot{x}) = \frac12m\dot{x}^2-mgx_3$ beschreibt
einen Massepunkt in einem homogenen vertikalen Gravitationsfeld.
Die Lagrange-Funktion ist unter horizontalen Translationen
\[
h^s(x) = x + su\qquad u\perp \vec{e}_3
\]
invariant,
wie in Beispiel~\ref{buch:symmetrien:symmetrie:beispiel:homogen} 
gezeigt wurde.
Nach Satz~\ref{buch:symmetrien:noether:satz:noether} ist 
\[
I(x,\dot{x})
=
\frac{\partial L}{\partial \dot{x}}\cdot \frac{d}{ds} h^s(x)
\bigg|_{s=0}
=
m\dot{x}
\cdot
u
=
p\cdot u
\]
für jeden horizontalten Einheitsvektor.
Das Skalarprodukt $p\cdot u$ ist die horizontale Komponente des Impulses,
die horizontalen Komponenten des Impulses sind also erhalten, die
vertikale aber natürlich nicht.
\end{beispiel}

\begin{beispiel}
\label{buch:symmetrien:noether:beispiel:drehimpuls}
Sie $L(x) = \frac12m\dot{x}^2 - V(x)$ die Lagrange-Funktion eines Teilchens
der Masse $m$ in einem rotationssymmetrischen Potential $V(x)$.
Für jede beliebige Drehmatrix $A$ gilt daher $V(A(x))=V(x)$.
Die Lagrange-Funktion ist invariant unter Drehungen, wie in Beispiel
\ref{buch:symmetrien:symmetrie:beispiel:drehung} gezeigt wurde.
Nach Satz~\ref{buch:symmetrien:noether:satz:noether} gehört zu jeder 
Drehmatrix eine passende Erhaltungsgrösse.

Um die Erhaltungsgrösse zu finden, betrachten wir zunächst die Drehmatrix
\[
D_{z,s}
=
\begin{pmatrix}
\cos s &          - \sin s & 0 \\
\sin s & \phantom{-}\cos s & 0 \\
   0   &            0      & 1
\end{pmatrix}
\qquad\text{mit}\qquad
h^s(x)
=
D_{z,s}x.
\]
Für die Ableitung an der Stelle $s=0$ finden wir
\[
\frac{d}{ds} D_{z,s}x
=
\begin{pmatrix}
          -\sin s & -\cos s & 0 \\
\phantom{-}\cos s & -\sin s & 0 \\
        0         &    0    & 0
\end{pmatrix}
x
\qquad\Rightarrow\qquad
\frac{d}{ds} D_{z,s}x
\bigg|_{s=0}
=
\begin{pmatrix}
 0 & -1 & 0 \\
 1 &  0 & 0 \\
 0 &  0 & 0
\end{pmatrix}
x
=
\begin{pmatrix}
-x_2\\
\phantom{-}x_1\\
0
\end{pmatrix}.
\]
Daraus ergibt sich die Erhaltungsgrösse als
\[
\frac{\partial L}{\partial\dot{x}}
\cdot
\frac{d}{ds}h^s(x)\bigg|_{s=0}
=
m\dot{x}\cdot
\begin{pmatrix}
-x_2\\
\phantom{-}x_1\\
0
\end{pmatrix}
=
m(x_1\dot{x}_2 - x_2\dot{x}_1).
\]
Dies ist die dritte Komponente des Drehimpulses.

Eine allgemeine Drehung um die Achse $u$ kann geschrieben werden also
\[
D_s
=
\exp s\Omega_u
\qquad\text{mit}\qquad
\Omega_u
=
\begin{pmatrix}
\phantom{-}0   &         - u_3 & \phantom{-}u_2\\
\phantom{-}u_3 &\phantom{-}0   &          - u_1\\
         - u_2 &\phantom{-}u_1 & \phantom{-}0
\end{pmatrix}.
\]
Die Ableitung von $D_s$ ist
\[
\frac{d}{ds} D_s
=
\frac{d}{ds} \Omega D_s
\qquad\Rightarrow\qquad
\frac{d}{ds}D_s\bigg|_{s=0}
=
\Omega_u.
\]
Dann folgt für die Erhaltungsgrösse
von Satz~\ref{buch:symmetrien:noether:satz:noether} 
\begin{align*}
I(x,\dot{x})
&=
\frac{\partial L}{\partial\dot{x}}
\cdot
\frac{d}{ds}D_sx\bigg|_{s=0}
\\
&=
m\dot{x}
\cdot
\Omega x
\\
&=
m
\begin{pmatrix}
\dot{x}_1\\
\dot{x}_2\\
\dot{x}_3
\end{pmatrix}
\cdot
\begin{pmatrix}
\phantom{-}0   &         - u_3 & \phantom{-}u_2\\
\phantom{-}u_3 &\phantom{-}0   &          - u_1\\
         - u_2 &\phantom{-}u_1 & \phantom{-}0
\end{pmatrix}
\begin{pmatrix}
x_1\\x_2\\x_3
\end{pmatrix}
\\
&=
m
\begin{pmatrix}
\dot{x}_1\\
\dot{x}_2\\
\dot{x}_3
\end{pmatrix}
\cdot
\begin{pmatrix}
          -u_3x_2+u_2x_3\\
\phantom{-}u_3x_1       -u_1x_3\\
          -u_2x_1+u_1x_2    
\end{pmatrix}
\\
&=
m(
- u_3 \dot{x}_1 x_2
+ u_2 \dot{x}_1 x_3
+ u_3 \dot{x}_2 x_1
- u_1 \dot{x}_2 x_3
- u_2 \dot{x}_3 x_1
+ u_1 \dot{x}_3 x_2
)
\\
&=
m\bigl(
u_1(x_2\dot{x}_3 - x_3\dot{x}_2)
+
u_2(x_3\dot{x}_1-x_1\dot{x}_3)
+
u_3(x_1\dot{x}_2-x_2\dot{x}_1)
\bigr)
\\
&=
\begin{pmatrix}u_1\\u_2\\u_3\end{pmatrix}
\cdot
(x\times m\dot{x})
=
u\cdot \vec{L},
\end{align*}
die Erhaltungsgrösse ist also die Projektion des Drehimpulses $\vec{L}$
auf die Achsrichtung $u$.
\end{beispiel}

Der Satz von Emmy Noether setzt voraus, dass die Lagrange-Funktion
nicht von der Zeit abhängt.
Man kann im Allgemeinen nicht erwarten, dass es eine ähnlich einfache
Erhaltungsgrösse gibt, wenn $L$ auch noch von der Zeit abhängt.
Die Zeitunabhängigkeit wird in 
\eqref{buch:symmetrien:noether:eqn:tindep}
verwendet.
Dieser Ausdruck wäre nicht mehr das Resultat der Kettenregel,
es müsste noch einen Term $\partial L/\partial t$ vorhanden sein.
In der Tat kann die Zeitabhängigkeit bedeuten, dass der Massepunkt
äusseren Kräften unterworfen ist, die Energie und Impuls des Massepunktes
verändern können, wodurch sie ganz offensichtlich nicht länger
Erhaltungsgrössen sind.



%
% 3-felder.te
%
% (c) 2024 Prof Dr Andreas Müller
%
\section{Erhaltungssätze für Felder
\label{buch:symmetrien:felder}}
\kopfrechts{Erhaltungssätze für Felder}



\uebungsabschnitt

\aufgabetoplevel{chapters/100-symmetrien/uebungsaufgaben}
\begin{uebungsaufgaben}
\uebungsaufgabe{1001}
\end{uebungsaufgaben}
\enduebungsabschnitt

