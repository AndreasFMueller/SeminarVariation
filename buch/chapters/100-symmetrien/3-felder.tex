%
% 3-felder.te
%
% (c) 2024 Prof Dr Andreas Müller
%
\section{Erhaltungssätze für Felder
\label{buch:symmetrien:felder}}
\kopfrechts{Erhaltungssätze für Felder}
Für Felder im Gegensatz zu Funktionen einer Variablen muss der
Begriff einer Symmetrie neu formuliert werden und auch der Begriff
eines Erhaltungssatzes 

%
% Divergenzform
%
\subsection{Divergenzform}
Wir versuchen erst zu ermitteln, welche Art von Gleichung wir anstelle
einer Konstanten als Erhaltungssatz erwarten dürfen.

%
% Kontinuitätsgleichung
%
\subsubsection{Kontinuitätsgleichung}
Wir betrachten die Dichteverteilung $\varrho(x,y,z,t)$ eines Mediums
im dreidimensionalen Raum zusammen mit dem Vektorfeld $\vec{v}(x,y,z,t)$
der Strämungsgeschwindigkeit.
Dann können wir in einem kleinen Quader mit Seitenlägen
$\Delta x$,
$\Delta y$
und
$\Delta z$
die Veränderung der darin enthaltenen Masse auf zwei Arten berechnen.
Einerseits wird sie approximiert durch die Änderung der Dichte
\[
\Delta m
=
(\varrho(x,y,z,t+\Delta t)-\varrho(x,y,z,t))
\Delta x
\,
\Delta y
\,
\Delta z.
\]
Andererseits kann man die durch die Seitenflächen fliessenden Masse
ermitteln.
Der Massestrom ist das Produkt $\vec{\jmath}=\varrho\vec{v}$.
Damit wird
\begin{align*}
\Delta m
&=
(
j_x(x+\Delta x,y,z,t)
-
j_x(x,y,z,t)
)
\Delta y\,\Delta z \, \Delta t
\\
&\quad
+
(
j_y(x,y+\Delta y,z,t)
-
j_y(x,y,z,t)
)
\Delta x\,\Delta z \, \Delta t
\\
&\quad
+
(
j_z(x,y,z+\Delta z,t)
-
j_z(x,y,z,t)
)
\Delta x\,\Delta y \, \Delta t
\end{align*}
Nach Division durch $\Delta x\, \Delta y\, \Delta z\,\Delta t$ ergeben
die Terme die Gleichung
\begin{align*}
\frac{\Delta m}{\Delta x\,\Delta y\,\Delta z\,\Delta t}
&=
\frac{\varrho(x,y,z,z+\Delta t)-\varrho(x,y,z,t)}{\Delta t}
\\
&=
\frac{
j_x(x+\Delta x,y,z,t)
-
j_x(x,y,z,t)
}{
\Delta x
}
\\
&\quad
+
\frac{
j_y(x,y+\Delta y,z,t)
-
j_y(x,y,z,t)
}{
\Delta y
}
\\
&\quad
+
\frac{
j_z(x,y,z+\Delta z,t)
-
j_z(x,y,z,t)
}{
\Delta z
}.
\end{align*}
Nach dem Grenzübergang
$\Delta x\to 0$,
$\Delta y\to 0$,
$\Delta z\to 0$
und
$\Delta t\to 0$
wird daraus die Kontinuitätsgleichung
\begin{equation}
\frac{\partial \varrho}{\partial t}(x,y,z,t)
=
\frac{\partial j_x}{\partial x}(x,y,z,t)
+
\frac{\partial j_y}{\partial y}(x,y,z,t)
+
\frac{\partial j_z}{\partial 7}(x,y,z,t)
=
\operatorname{div}\vec{\jmath}(x,y,z,t).
\end{equation}
Die Kontinuitätsgleichung ist der prototypische Erhaltungssatz für
Felder.
Beispielsweise ist die Wärmeleitungsgleichung nur die Kontinuitätsgleichung
für die Wärmeenergiedichte und den Wärmeenergiestrom.

%
% Divergenzform
%
\subsubsection{Divergenzform}
Wenn eine skalare Grösse wie die Dichte $\varrho$ erhalten ist, dann bleibt
in der zugehörigen Kontinuitätsgleichung mit der Stromdichte
$\vec{\jmath}$ nur die rechte Seite stehen, es gilt dann 
\[
\frac{\partial\varrho}{\partial t}
=
0
=
\operatorname{div} \vec{\jmath}.
\]
Anstelle einer Grösse, die konstant bleibt, kann man daher nur erwarten,
dass es ein Vektorfeld gibt, dessen Divergenz verschwindet.

%
% Gebietsvariation und Multiplikator
%
\subsection{Gebietsvariation und Multiplikator}

\begin{definition}
Eine differenzierbare Abbildung
\[
\varphi^s
\colon
\mathbb{R}^n\times \mathbb{R}
:
(x,s)
\mapsto
\varphi^s(x),
\qquad
\varphi^0(x)=x
\]
heisst eine {\em Gebietsvariation}.
\index{Gebietsvariation}%
Die Ableitung
\[
\frac{\partial \varphi^s(x)}{\partial s}\bigg|_{s=0}
=
\vec{v}(x)
\]
nach $s$ an der Stelle $s=0$ ist ein $n$-dimensionales Vektorfeld.
\end{definition}

\begin{satz}
\label{buch:symmetrien:felder:satz:gebietsintegral}
Für eine $f\colon \mathbb{R}^n\to\mathbb{R}$ differenzierbare Funktion
gilt
\[
\frac{d}{ds}
\int_{\varphi^s(U)}
f(x)\,dx
\bigg|_{s=0}
=
\int_{\partial U} f(x)\vec{v}(x)\cdot d\vec{o}.
\]
\end{satz}

\begin{proof}
TODO
\end{proof}

\begin{definition}
Die allgemeine Variation einer Funktion
\(
u
\colon
\mathbb{R}^n\to \mathbb{R}
\)
ist eine Funktion
\[
w
\colon
\mathbb{R}^n \times\mathbb{R}
\to
\mathbb{R}
:
(x,s)
\mapsto
w(x,s)
\]
mit $w(x,0)=u(x)$.
Die Ableitung
\[
\frac{\partial w}{\partial s}(x,0) = m(x)
\]
nach $s$ an der Stelle $s=0$ heisst ein Multiplikator.
\end{definition}

\begin{definition}
Ein Funktional mit Lagrange-Funktioin 
$L(x,u,u_x)$ heisst unter der Gebietsvariation $\varphi^s$ und 
der allgemeinen Variation $w$ von $u$ invariant, wenn
\begin{equation}
\int_U
L(x,w(x,s),\nabla w(x,s))\,dx
=
\int_{\varphi^s(U)}
L(x,u(x),\nabla u(x))\,dx
\label{buch:symmetrien:felder:eqn:invarianz}
\end{equation}
gilt.
\end{definition}

%
% Der Satz von Noether für Felder
%
\subsection{Der Satz von Noether für Felder}
Wenn die Lagrange-Funktion eines Funktionals unter einer Gebietsvariation
und einer allgemeine Variation von $u$ invariant ist, dann gibt es einen
Stromvektor, dessen Divergenz verschwindet.

\begin{satz}
Ist das Funktional $I$ mit Lagrange-Funktion $L$ invariant unter $\varphi^s$
und $w$.
Der Vektor $\vec{j}$ habe die Komponenten
\[
j_k
=
L(x,u(x),\nabla u(x))\,v_k(x)
-
m(x)
\frac{\partial L}{\partial u_{x_k}}(x,u(x),\nabla u(x)).
\]
Dann gilt
\begin{align}
\operatorname{div}\vec{\jmath}
=
\sum_{k=1}^n \frac{\partial j_k}{\partial x_k}
&=
\sum_{k=1}^n
\frac{\partial }{\partial x_k}
\biggl(
L(x,u(x),\nabla u(x))\,v_k(x)
-
m(x)
\frac{\partial L}{\partial u_{x_k}}
(x,u(x),\nabla u(x))
\biggr)
\notag
\\
&=
m(x)\biggl(
\frac{\partial L}{\partial u}(x,u(x),\nabla u(x))
-
\sum_{k=1}^n\frac{\partial L}{\partial u_{x_k}}(x,u(x),\nabla u(x))
\biggr)
\label{buch:symmetrien:felder:eqn:rhs}
\end{align}
Falls $u$ ein kritischer Punkt des Funktionals $I$ ist, gilt
\[
\operatorname{div}\vec{\jmath}=0.
\]
\end{satz}

\begin{proof}
Die Klammer auf der rechten Seite von \eqref{buch:symmetrien:felder:eqn:rhs} 
ist
\[
E(x)
=
\frac{\partial L}{\partial u}(x,u(x),\nabla u(x))
-
\sum_{k=1}^n\frac{\partial L}{\partial u_{x_k}}(x,u(x),\nabla u(x)).
\]
Dies ist die linke Seite der Euler-Ostrogradski-Diffe\-ren\-tial\-gleichung
für die Funktion $u$.
Für einen kritischen Punkt $u$ des Funktionals ist die rechte Seite
von \eqref{buch:symmetrien:felder:eqn:rhs} $=0$,
die letzte Behauptung folgt daher aus~\eqref{buch:symmetrien:felder:eqn:rhs}.

Wir berechnen die Ableitung der
Identität~\eqref{buch:symmetrien:felder:eqn:invarianz}
an der Stelle $s=0$.
Für die linke Seite erhalten wir wegen $w(x,s)=u(x)$
\begin{align*}
\text{LHS}
&=
\int_U
\frac{\partial L}{\partial u}(x,u(x),\nabla u(x))
\frac{\partial w}{\partial s}(x,0)
+
\sum_{k=1}^n
\frac{\partial L}{\partial u_{x_k}}(x,u(x),\nabla u(x))
\cdot
\frac{\partial}{\partial s}
\frac{\partial w}{\partial x_k}(x,0)
\,dx.
\end{align*}
Die Ableitung von $w(x,s)$ nach $s$ ergibt den Multiplikator $m(x)$.
Die beiden Ableitungen von $w$ in der Summe können vertauscht werden
und ebenfalls durch $m(x)$ ausgedrückt werden, nämlich als
\[
\frac{\partial s}{\partial w}{\partial x_k}(x,0)
=
\frac{\partial }{\partial x_k}\frac{\partial w}{\partial s}(x,0)
=
\frac{\partial m}{\partial x_k}(x)
\]
Somit ist
\begin{align*}
\text{LHS}
&=
\int_U
\frac{\partial L}{\partial u}(x,u(x),\nabla u(x))
m(x)
+
\sum_{k=1}^n
\frac{\partial L}{\partial u_{x_k}}(x,u(x),\nabla u(x))
\cdot
\frac{\partial m}{\partial x_k}(x)
\,dx.
\intertext{Partielle Integration der Summe ergibt}
&=
\int_U
\frac{\partial L}{\partial u}(x,u(x),\nabla u(x))
m(x)
\,dx
+
\int_{\partial U}
m(x)
\frac{\partial L}{\partial u_{x}}(x,u(x),\nabla u(x))
\cdot
d\vec{o}
\\
&\qquad
-
\int_{U}
m(x)
\sum_{k=1}^n
\frac{\partial}{\partial x_k}
\frac{\partial L}{\partial u_{x_k}}(x,u(x),\nabla u(x))
\,dx.
\\
&=
\int_U
m(x)
\biggl(
\frac{\partial L}{\partial u}(x,u(x),\nabla u(x))
-
\sum_{k=1}^n
\frac{\partial}{\partial x_k}
\frac{\partial L}{\partial u_{x_k}}(x,u(x),\nabla u(x))
\biggr)\,dx
\\
&\qquad
+
\int_{\partial U}
m(x)
\frac{\partial L}{\partial u_{x}}(x,u(x),\nabla u(x))
\cdot
d\vec{o}
\intertext{oder noch einfacher}
&=
\int_U
m(x)
E(x)
\,dx
+
\int_{\partial U}
m(x)
\frac{\partial L}{\partial u_{x}}(x,u(x),\nabla u(x))
\cdot
d\vec{o}.
\end{align*}

Die rechte Seite der Ableitung der
Identität~\eqref{buch:symmetrien:felder:eqn:invarianz}
wendet die Formel für die Ableitung eines Integrals über ein
von einem Parameter abhängendes Gebiet von
Satz~\ref{buch:symmetrien:felder:satz:gebietsintegral}
auf die Funktion $f(x) = L(x,u(x),\nabla u(x))$ an, die 
\[
\frac{d}{ds}
\int_{\varphi^s(U)} L(x,u(x),\nabla u(x))\,dx
\bigg|_{s=0}
=
\int_U L(x,u(x),\nabla u(x))\,\vec{v}\cdot d\vec{o}.
\]
ergibt.

Die bisherigen Rechnungen haben ergeben, dass
\begin{align*}
\int_U m(x) E(x)\,dx
+
\int_{\partial U}
m(x)\frac{\partial L}{\partial u_x}(x,u(x),\nabla u(x))\cdot d\vec{o}
&=
\int_{\partial U}L(x,u(x),\nabla u(x))\,\vec{v}(x)\cdot d\vec{o}.
\end{align*}
Bringen wir das zweite Integral links auf die rechte Seite, ergibt
sich
\begin{align*}
\int_U m(x) E(x)\,dx
&=
\int_{\partial U}
\biggl(
-
m(x)\frac{\partial L}{\partial u_x}(x,u(x),\nabla u(x))
+
\int_{\partial U}L(x,u(x),\nabla u(x))\,\vec{v}(x)
\biggr)
\cdot d\vec{o}.
\intertext{Mit dem Integralsatz von Gauss kann das Oberflächenintegral
auf der rechten Seite in das Gebietsintegral über $U$}
&=
\int_U
\operatorname{div}
\biggl(
L(x,u(x),\nabla u(x))\,\vec{v}(x)
-m(x)
\frac{\partial L}{\partial u_x}(x,u(x),\nabla u(x))
\biggr)
\,dx
\intertext{umgewandelt werden.
In Komponenten ist dies}
&=
\int_U
\sum_{k=1}^n
\frac{\partial}{\partial x_k}
\biggl(
L(x,u(x),\nabla u(x))\,v_k(x)
-m(x)
\frac{\partial L}{\partial u_{x_k}}(x,u(x),\nabla u(x))
\biggr)
\,dx
\intertext{oder}
\int_U m(x) E(x)\,dx
&=
\int_U \sum_{k=1}^n \frac{\partial j_k}{\partial x_k}(x) \,dx
=
\int_U \operatorname{div}\vec{\jmath}\,dx.
\end{align*}
Da dies für jedes beliebige Gebiet $U$ gilt, müssen auch die Integranden
\begin{equation}
m(x)
E(x)
=
\operatorname{div}\vec{\jmath}
\end{equation}
übereinstimmen.
Dies ist die Gleichung \eqref{buch:symmetrien:felder:eqn:rhs}.
Damit ist der Satz bewiesen.
\end{proof}


