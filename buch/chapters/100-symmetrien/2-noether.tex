%
% 2-noether.tex
%
% (c) 2023 Prof Dr Andreas Müller
%
\section{Der Satz von Emmy Noether
\label{buch:symmetrien:section:noether}}
\kopfrechts{Der Satz von Noether}
Die Invarianz einer Lagrange-Funktion unter einer differenzierbaren
Symmetrie hat die überraschende Konsequenz, dass es dazu eine
Erhaltungsgrösse gibt.

\begin{satz}[Emmy Noether]
\label{buch:symmetrien:noether:satz:noether}
Sei $L(x,\dot{x})$ eine Lagrange-Funktion, die nicht von der Zeit
abhängt und $h^s\colon\mathbb{R}^n\to\mathbb{R}^n$ eine differenzierbare
Symmetrie, unter der $L$ invariant bleibt.
Dann ist die Grösse 
\begin{equation}
I(x,\dot{x})
=
\frac{\partial L}{\partial\dot{x}} (x,\dot{x})
\cdot 
\frac{d}{ds} h^s(x)\bigg|_{s=0}
\end{equation}
eine Erhaltungsgrösse.
\end{satz}

\begin{proof}
Sei $t\mapsto x(t)$ eine Lösung der Euler-Lagrange-Differentialgleichung
zur Lagrange-Funktion $L$.
Die Funktion $\Phi(s,t)$ definiert durch
\[
\Phi(s,t) = h^s(x(t))
\]
kombiniert die Symmetrietransformation mit der Zeitentwicklung.
Für jeden Wert von $s$ ist $t\mapsto \Phi(s,t)=h^s(x(t))$ eine 
Kurve.
Ihr Geschwindigkeitsvektor ist die Ableitung nach $t$, also
\[
\frac{d}{dt} \Phi(s,t)
=
\dot{\Phi}(s,t).
\]
Da die Lagrange-Funktionen unter der Symmetrietransformation $h^s$ 
invariant ist, muss die Ableitung von $L(\Phi(s,t), \dot{\Phi}(s,t))$
von $s$ unabhängig sein.
Die Ableitung
\begin{equation}
\frac{\partial}{\partial s} L(\Phi(s,t), \dot{\Phi}(s,t))
=
\frac{\partial L}{\partial x}(\Phi(s,t),\dot{\Phi}(s,t))
\cdot
\frac{\partial \Phi}{\partial s}
+
\frac{\partial L}{\partial \dot{x}}(\Phi(s,t),\dot{\Phi}(s,t))
\cdot
\frac{\partial \dot{\Phi}}{\partial s}
=0
\label{buch:symmetrioen:noether:eqn:sabl}
\end{equation}
muss daher verschwinden.

Ist $x(t)$ eine Lösungskurve der Euler-Lagrange-Differentialgleichung,
dann ist auch $t\mapsto h^s(x(t)) = \Phi(s,t)$ für jeden Wert von $s$
eine Lösungskurve der Euler-Lagrange-Differentialgleichung.
Die Euler-Lagrange-Differentialgleichung ist
\begin{equation}
\frac{\partial L}{\partial x}(\Phi(s,t),\dot{\Phi}(s,t))
=
\frac{\partial t}{\partial t}
\frac{\partial L}{\partial \dot{x}}(\Phi(s,t),\dot{\Phi}(s,t)).
\end{equation}
Dies kann dazu verwendet werden, den ersten Term auf der rechten
Seite von \eqref{buch:symmetrioen:noether:eqn:sabl} umzuformen,
wir erhalten
\begin{align}
0
&=
\frac{\partial}{\partial t}
\biggl(
\frac{\partial L}{\partial \dot{x}}(\Phi(s,t),\dot{\Phi}(s,t))
\biggr)
\cdot
\frac{\partial \Phi}{\partial s}
+
\frac{\partial L}{\partial\dot{x}}(\Phi(s,t),\dot{\Phi}(s,t))
\cdot
\frac{\partial \dot{\Phi}}{\partial s}(s,t)
\notag
\intertext{im zweiten Term kann man die Ableitung nach $s$ mit der Ableitung
nach $t$ vertauschen:}
&=
\frac{\partial}{\partial t}
\biggl(
\frac{\partial L}{\partial \dot{x}}(\Phi(s,t),\dot{\Phi}(s,t))
\biggr)
\cdot
\frac{\partial \Phi}{\partial s}
+
\frac{\partial L}{\partial\dot{x}}(\Phi(s,t),\dot{\Phi}(s,t))
\cdot
\frac{\partial^2\Phi}{\partial s\,\partial t}(s,t)
\notag
\\
&=
\frac{\partial}{\partial t}
\biggl(
\frac{\partial L}{\partial \dot{x}}(\Phi(s,t),\dot{\Phi}(s,t))
\biggr)
\cdot
\frac{\partial \Phi}{\partial s}
+
\frac{\partial L}{\partial\dot{x}}(\Phi(s,t),\dot{\Phi}(s,t))
\cdot
\frac{\partial}{\partial t}
\biggl(
\frac{\partial\Phi}{\partial s}(s,t)
\biggr).
\label{buch:symmetrien:noether:eqn:tindep}
\intertext{Auf der rechten Seite kann man die Kettenregel erkennen,
sie ist die Ableitung}
0
&=
\frac{d}{dt}\biggl(
\frac{\partial L}{\partial \dot{x}}(\Phi(s,t),\dot{\Phi}(s,t))
\cdot
\frac{\partial\Phi}{\partial s}(s,t)
\biggr).
\notag
\end{align}
Da die Ableitung verschwindet, ist der Klammerausdruck eine Konstante,
er hängt also nicht von $s$ und $t$ ab.
Setzt man $s=0$ folgt, dass
\[
\frac{\partial L}{\partial \dot{x}}(x,\dot{x})
\cdot
\frac{d}{ds}h^s(x)\bigg|_{s=0}
=
I(x,\dot{x})
\]
eine Erhaltungsgrösse ist.
\end{proof}

\begin{beispiel}
\label{buch:symmetrien:noether:beispiel:homogen}
Die Lagrange-Funktion $L(x,\dot{x}) = \frac12m\dot{x}^2-mgx_3$ beschreibt
einen Massepunkt in einem homogenen vertikalen Gravitationsfeld.
Die Lagrange-Funktion ist unter horizontalen Translationen
\[
h^s(x) = x + su\qquad u\perp \vec{e}_3
\]
invariant,
wie in Beispiel~\ref{buch:symmetrien:symmetrie:beispiel:homogen} 
gezeigt wurde.
Nach Satz~\ref{buch:symmetrien:noether:satz:noether} ist 
\[
I(x,\dot{x})
=
\frac{\partial L}{\partial \dot{x}}\cdot \frac{d}{ds} h^s(x)
\bigg|_{s=0}
=
m\dot{x}
\cdot
u
=
p\cdot u
\]
für jeden horizontalten Einheitsvektor.
Das Skalarprodukt $p\cdot u$ ist die horizontale Komponente des Impulses,
die horizontalen Komponenten des Impulses sind also erhalten, die
vertikale aber natürlich nicht.
\end{beispiel}

\begin{beispiel}
\label{buch:symmetrien:noether:beispiel:drehimpuls}
Sie $L(x) = \frac12m\dot{x}^2 - V(x)$ die Lagrange-Funktion eines Teilchens
der Masse $m$ in einem rotationssymmetrischen Potential $V(x)$.
Für jede beliebige Drehmatrix $A$ gilt daher $V(A(x))=V(x)$.
Die Lagrange-Funktion ist invariant unter Drehungen, wie in Beispiel
\ref{buch:symmetrien:symmetrie:beispiel:drehung} gezeigt wurde.
Nach Satz~\ref{buch:symmetrien:noether:satz:noether} gehört zu jeder 
Drehmatrix eine passende Erhaltungsgrösse.

Um die Erhaltungsgrösse zu finden, betrachten wir zunächst die Drehmatrix
\[
D_{z,s}
=
\begin{pmatrix}
\cos s &          - \sin s & 0 \\
\sin s & \phantom{-}\cos s & 0 \\
   0   &            0      & 1
\end{pmatrix}
\qquad\text{mit}\qquad
h^s(x)
=
D_{z,s}x.
\]
Für die Ableitung an der Stelle $s=0$ finden wir
\[
\frac{d}{ds} D_{z,s}x
=
\begin{pmatrix}
          -\sin s & -\cos s & 0 \\
\phantom{-}\cos s & -\sin s & 0 \\
        0         &    0    & 0
\end{pmatrix}
x
\qquad\Rightarrow\qquad
\frac{d}{ds} D_{z,s}x
\bigg|_{s=0}
=
\begin{pmatrix}
 0 & -1 & 0 \\
 1 &  0 & 0 \\
 0 &  0 & 0
\end{pmatrix}
x
=
\begin{pmatrix}
-x_2\\
\phantom{-}x_1\\
0
\end{pmatrix}.
\]
Daraus ergibt sich die Erhaltungsgrösse als
\[
\frac{\partial L}{\partial\dot{x}}
\cdot
\frac{d}{ds}h^s(x)\bigg|_{s=0}
=
m\dot{x}\cdot
\begin{pmatrix}
-x_2\\
\phantom{-}x_1\\
0
\end{pmatrix}
=
m(x_1\dot{x}_2 - x_2\dot{x}_1).
\]
Dies ist die dritte Komponente des Drehimpulses.

Eine allgemeine Drehung um die Achse $u$ kann geschrieben werden also
\[
D_s
=
\exp s\Omega_u
\qquad\text{mit}\qquad
\Omega_u
=
\begin{pmatrix}
\phantom{-}0   &         - u_3 & \phantom{-}u_2\\
\phantom{-}u_3 &\phantom{-}0   &          - u_1\\
         - u_2 &\phantom{-}u_1 & \phantom{-}0
\end{pmatrix}.
\]
Die Ableitung von $D_s$ ist
\[
\frac{d}{ds} D_s
=
\frac{d}{ds} \Omega D_s
\qquad\Rightarrow\qquad
\frac{d}{ds}D_s\bigg|_{s=0}
=
\Omega_u.
\]
Dann folgt für die Erhaltungsgrösse
von Satz~\ref{buch:symmetrien:noether:satz:noether} 
\begin{align*}
I(x,\dot{x})
&=
\frac{\partial L}{\partial\dot{x}}
\cdot
\frac{d}{ds}D_sx\bigg|_{s=0}
\\
&=
m\dot{x}
\cdot
\Omega x
\\
&=
m
\begin{pmatrix}
\dot{x}_1\\
\dot{x}_2\\
\dot{x}_3
\end{pmatrix}
\cdot
\begin{pmatrix}
\phantom{-}0   &         - u_3 & \phantom{-}u_2\\
\phantom{-}u_3 &\phantom{-}0   &          - u_1\\
         - u_2 &\phantom{-}u_1 & \phantom{-}0
\end{pmatrix}
\begin{pmatrix}
x_1\\x_2\\x_3
\end{pmatrix}
\\
&=
m
\begin{pmatrix}
\dot{x}_1\\
\dot{x}_2\\
\dot{x}_3
\end{pmatrix}
\cdot
\begin{pmatrix}
          -u_3x_2+u_2x_3\\
\phantom{-}u_3x_1       -u_1x_3\\
          -u_2x_1+u_1x_2    
\end{pmatrix}
\\
&=
m(
- u_3 \dot{x}_1 x_2
+ u_2 \dot{x}_1 x_3
+ u_3 \dot{x}_2 x_1
- u_1 \dot{x}_2 x_3
- u_2 \dot{x}_3 x_1
+ u_1 \dot{x}_3 x_2
)
\\
&=
m\bigl(
u_1(x_2\dot{x}_3 - x_3\dot{x}_2)
+
u_2(x_3\dot{x}_1-x_1\dot{x}_3)
+
u_3(x_1\dot{x}_2-x_2\dot{x}_1)
\bigr)
\\
&=
\begin{pmatrix}u_1\\u_2\\u_3\end{pmatrix}
\cdot
(x\times m\dot{x})
=
u\cdot \vec{L},
\end{align*}
die Erhaltungsgrösse ist also die Projektion des Drehimpulses $\vec{L}$
auf die Achsrichtung $u$.
\end{beispiel}

Der Satz von Emmy Noether setzt voraus, dass die Lagrange-Funktion
nicht von der Zeit abhängt.
Man kann im Allgemeinen nicht erwarten, dass es eine ähnlich einfache
Erhaltungsgrösse gibt, wenn $L$ auch noch von der Zeit abhängt.
Die Zeitunabhängigkeit wird in 
\eqref{buch:symmetrien:noether:eqn:tindep}
verwendet.
Dieser Ausdruck wäre nicht mehr das Resultat der Kettenregel,
es müsste noch einen Term $\partial L/\partial t$ vorhanden sein.
In der Tat kann die Zeitabhängigkeit bedeuten, dass der Massepunkt
äusseren Kräften unterworfen ist, die Energie und Impuls des Massepunktes
verändern können, wodurch sie ganz offensichtlich nicht länger
Erhaltungsgrössen sind.


