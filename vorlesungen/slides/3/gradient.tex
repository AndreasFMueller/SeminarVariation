%
% gradient.tex -- slide template
%
% (c) 2021 Prof Dr Andreas Müller, OST Ostschweizer Fachhochschule
%
\bgroup
\begin{frame}[t]
\setlength{\abovedisplayskip}{5pt}
\setlength{\belowdisplayskip}{5pt}
\frametitle{Gradient}
\vspace{-20pt}
\begin{columns}[t,onlytextwidth]
\begin{column}{0.44\textwidth}
\begin{block}{$n$ Variablen}
Richtungsableitung
\begin{align*}
D_{\vec{v}}f(x)
&=
\vec{v}\cdot
\operatorname{grad} f(x) 
\\
&=
v_1\frac{\partial f}{\partial x_1}(x)
+
v_n\frac{\partial f}{\partial x_n}(x)
\end{align*}
\end{block}
\begin{block}{``Gradient'' für Variationsprobleme}
\begin{gather*}
\frac{\partial L}{\partial y}(x,y(x),y'(x))\qquad\qquad
\\
-
\frac{d}{dx}
\frac{\partial L}{\partial y'}(x,y(x),y'(x))
=
0
\end{gather*}
\end{block}
\end{column}
\begin{column}{0.52\textwidth}
\begin{block}{Variationsproblem}
\begin{align*}
\delta I(y)
&=
\int_{x_1}^{x_2}
\biggl(
\frac{\partial L}{\partial y}(x,y(x),y'(x))
\\
&\quad
-
\frac{d}{dx}
\frac{\partial L}{\partial y'}(x,y(x),y'(x))
\biggr)\eta(x)\,dx
\end{align*}
\end{block}
\begin{block}{$L^2$-Skalarprodukt}
Skalarprodukt von Funktionen
\begin{align*}
\langle f,g\rangle
&=
\int_{x_1}^{x_2}
f(x)\,g(x)\,dx
\\
\delta I(y)
&=
\biggl\langle
\frac{\partial L}{\partial y}(x,y(x),y'(x))
\\
&\qquad
-
\frac{d}{dx}
\frac{\partial L}{\partial y'}(x,y(x),y'(x)),
\eta(x)
\biggr\rangle
\end{align*}
\end{block}
\end{column}
\end{columns}
\end{frame}
\egroup
