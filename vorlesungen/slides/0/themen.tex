%
% themen.tex -- slide template
%
% (c) 2021 Prof Dr Andreas Müller, OST Ostschweizer Fachhochschule
%
\bgroup
\begin{frame}[t]
\setlength{\abovedisplayskip}{5pt}
\setlength{\belowdisplayskip}{5pt}
\frametitle{Themen}
\vspace{-20pt}
\begin{columns}[t,onlytextwidth]
\only<1-20>{
\begin{column}{0.48\textwidth}
\begin{enumerate}
\item<2-> Doppelpendel
\item<3-> Brachistochrone mit Reibung
\item<4-> Kettenlinie als Variationsproblem
\item<5-> Relativistische Mechanik
\item<6-> Maxwell-Gleichungen
\item<7-> LEO
\item<8-> Balkengleichung und Variationsprinzip
\item<9-> Variation und das Prinzip der virtuellen Arbeit
\item<10-> Minimaler Luftwiderstand
\item<11-> Minimalflächen
\end{enumerate}
\end{column}
}
\begin{column}{0.48\textwidth}
\begin{enumerate}
\setcounter{enumi}{10}
\item<12-> Geodäten
\item<13-> Variationsprinzip und elektrische Schaltungen
\item<14-> Finite Elemente
\item<15-> Wasserwellen
\item<16-> Antennen und Variationsprinzipien
\item<17-> Dissipative Systeme
\item<18-> Cahn-Hilliard-Gleichung (MSE)
\item<19-> Timshenko-Ehrenfest-Balkentheorie
\item<20-> Wie schwimmt man am besten durch den Fluss?
\end{enumerate}
\end{column}
\only<21->{%
\begin{column}{0.48\textwidth}
\begin{enumerate}
\setcounter{enumi}{19}
\item<21-> Die stärkste Säule
\item<22-> Melan-Gleichung
\item<23-> Form eines Planeten
\item<24-> Deformation eines Rings
\end{enumerate}
\end{column}
}
\end{columns}
\end{frame}
\egroup
